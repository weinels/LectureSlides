\documentclass{beamer}

\usepackage[english]{babel}
\usepackage[utf8x]{inputenc}
\usepackage{slide_helper}

\title[MATH 2250 - Section 3.2]{Systems of Linear Equations}

\begin{document}

\begin{frame}
  \titlepage
\end{frame}

\begin{frame}
\begin{block}{System of Linear Equations}
A $m\times n$ \textbf{system of linear equations} is a set of $m$ equations in $n$ variables $x_1,x_2,\dots,x_n$ of the form
\begin{center}
\begin{tabular}{ccccccccc}
$a_{11}x_1$&$+$&$a_{12}x_2$&$+$&$\dots$&$+$&$a_{1n}x_n$&$=$&$b_1$\\
$a_{21}x_1$&$+$&$a_{22}x_2$&$+$&$\dots$&$+$&$a_{2n}x_n$&$=$&$b_2$\\
$a_{31}x_1$&$+$&$a_{32}x_2$&$+$&$\dots$&$+$&$a_{3n}x_n$&$=$&$b_3$\\
$\vdots$&&$\vdots$&&&&$\vdots$&&$\vdots$\\
$a_{m1}x_1$&$+$&$a_{m2}x_2$&$+$&$\dots$&$+$&$a_{mn}x_n$&$=$&$b_m$\\
\end{tabular}
\end{center}
\end{block}
\end{frame}

\begin{frame}
\begin{block}{}
There are two primary ways of writing a linear systems using matrices.
\end{block}\pause
\begin{block}{An Augmented Matrix}
\begin{equation*}
\begin{bmatrix}[cccc|c]
a_{11}&a_{12}&\cdots&a_{1n}&b_1\\
\vdots&\vdots&\ddots&\vdots&\\
a_{m1}&a_{m2}&\cdots&a_{mn}&b_m
\end{bmatrix}
\end{equation*}
\end{block}\pause
\begin{block}{A Matrix Equation (We will look at these in section 3.3)}
As the matrix equation $A\vect{x}=\vect{b}$, where:
\begin{equation*}
\underbrace{\begin{bmatrix}
a_{11}&a_{12}&\cdots&a_{1n}\\
\vdots&\vdots&\ddots&\vdots\\
a_{m1}&a_{m2}&\cdots&a_{mn}
\end{bmatrix}}_A
\underbrace{\begin{bmatrix}
x_1\\\vdots\\x_n
\end{bmatrix}}_\vect{x}
=
\underbrace{\begin{bmatrix}
b_1\\\vdots\\b_m
\end{bmatrix}}_\vect{b}
\end{equation*}
\end{block}
\end{frame}

\begin{frame}
\begin{block}{Row Operation Notation} 
\begin{itemize}
\item $r_i$ denotes row $i$ \emph{before} the row operation is applied
\item $R_i$ denotes row $i$ \emph{after} the row operation is applied
\end{itemize}
\end{block}\pause

\begin{block}{Elementary Row Operations} 
\begin{itemize}[<+- | alert@+>]
\item Swap row $i$ and row $j$:
\begin{equation*}
R_i\leftrightarrow R_j \quad\left(\text{or}\ R_i=r_j\text{, } R_j=r_i\right)
\end{equation*}
\item Multiply row $i$ by a nonzero constant:
\begin{equation*}
R_i=c\cdot r_i
\end{equation*}
\item Add row $j$ to row $i$ (leaving row $j$ unchanged):
\begin{equation*}
R_i=r_i+r_j
\end{equation*}
\end{itemize}
\end{block}
\end{frame}

\begin{frame}
\begin{block}{Gaussian Elimination}
\small
Use row operations until the augmented matrix is in \textbf{R}ow \textbf{E}chelon \textbf{F}orm\@:
\begin{equation*}
\begin{bmatrix}[ccccc|c]
1         & c_{12} & c_{13} & \cdots & c_{1n} & d_1\\
0         & 1         & c_{23} & \cdots & c_{2n} & d_2\\
0         & 0         & 1         & \cdots & c_{3n} & d_2\\
\vdots & \vdots  & \vdots &\ddots & \vdots & \vdots\\
0         & 0         & 0         & \cdots  & 1         & d_m
\end{bmatrix}
\end{equation*}\pause
Then back solve the system:
\begin{equation*}
\begin{aligned}
x_1 + c_{12}x_2+c_{13}x_3+\cdots+c_{1n}x_n  &= d_1\\
                    x_2+c_{23}x_3+\cdots+c_{2n}x_n  &= d_2\\
                                                                          &\vdots\\
                                                                   x_n &= d_m          
\end{aligned}
\end{equation*}
\end{block}
\end{frame}

\begin{frame}
\begin{example}
\begin{overprint}
\onslide<1-2>
Consider the system
\begin{center}
\begin{tabular}{rcrcrcr}
$x$&$+$&$y$&$+$&$z$&$=$&$3$\\
$2x$&$-$&$3y$&$-$&$z$&$=$&$-8$\\
$-x$&$+$&$2y$&$+$&$2z$&$=$&$3$
\end{tabular}
\end{center}
\visible<2>{
We can write this as the augmented matrix:
\begin{equation*}
\begin{bmatrix}[rrr|r]
 1 &  1 &  1 &  3\\
 2 & -3 & -1 & -8\\
-1 &  2 &  2 &  3
\end{bmatrix}
\end{equation*}
We now want to use row operations to transform this augmented matrix into Row Echelon Form.}
\onslide<3-5>%
\LARGE
\begin{equation*}
	\begin{aligned}
		&	\begin{bmatrix}[rrr|r]
				 \+1 & \+1 & \+1 & \+3\\
				 \+2 &  -3 &  -1 &  -8\\
				  -1 & \+2 & \+2 & \+3
			\end{bmatrix}
			\visible<4-5>{\begin{aligned}
				& \phantom{R_1}\\
				& R_2=r_2+2r_3\\
				& \phantom{R_2}
			\end{aligned}}\\
		\visible<5>{\Rightarrow
		&	\begin{bmatrix}[rrr|r]
				 \+1 & \+1 & \+1 & \+3\\
				 \+0 & \+1 & \+3 &  -2\\
				  -1 & \+2 & \+2 & \+3
			\end{bmatrix}}
	\end{aligned}
\end{equation*}
\onslide<6-8>%
\LARGE
\begin{equation*}
	\begin{aligned}
		&	\begin{bmatrix}[rrr|r]
				 \+1 & \+1 & \+1 & \+3\\
				 \+0 & \+1 & \+3 & -2\\
				   -1 & \+2 & \+2 & \+3
			\end{bmatrix}
			\visible<7-8>{\begin{aligned}
				& \phantom{R_1}\\
				& \phantom{R_2}\\
				& R_3=r_1+r_3
			\end{aligned}}\\
		\visible<8>{\Rightarrow
		&	\begin{bmatrix}[rrr|r]
				 \+1 & \+1 & \+1 & \+3\\
				 \+0 & \+1 & \+3 &  -2\\
				 \+0 & \+3 & \+3 & \+6
			\end{bmatrix}}
	\end{aligned}
\end{equation*}
\onslide<9-11>%
\LARGE
\begin{equation*}
	\begin{aligned}
		&	\begin{bmatrix}[rrr|r]
				 \+1 & \+1 & \+1 & \+\phantom{1}3\\
				 \+0 & \+1 & \+3 &  -2\\
				 \+0 & \+3 & \+3 & \+6
			\end{bmatrix}
			\visible<10-11>{\begin{aligned}
				& \phantom{R_1}\\
				& \phantom{R_2}\\
				& R_3=r_3-3r_2
			\end{aligned}}\\
		\visible<11>{\Rightarrow
		&	\begin{bmatrix}[rrr|r]
				 \+1 & \+1 & \+1 & \+3\\
				 \+0 & \+1 & \+3 &  -2\\
				 \+0 & \+0 &  -6 & \+12
			\end{bmatrix}}
	\end{aligned}
\end{equation*}
\onslide<12-14>%
\LARGE
\begin{equation*}
	\begin{aligned}
		&	\begin{bmatrix}[rrr|r]
				 \+1 & \+1 & \+1 & \+3\\
				 \+0 & \+1 & \+3 &  -2\\
				 \+0 & \+0 &  -6 & \+12
			\end{bmatrix}
			\visible<13-14>{\begin{aligned}
				& \phantom{R}\\
				& \phantom{R}\\
				& R_3=-\tfrac{1}{6}r_3
			\end{aligned}}\\
		\visible<14>{\Rightarrow
		&	\begin{bmatrix}[rrr|r]
				 \+1 & \+1 & \+1 & \+\phantom{1}3\\
				 \+0 & \+1 & \+3 &  -2\\
				 \+0 & \+0 &  1   &  -2
			\end{bmatrix}}
	\end{aligned}
\end{equation*}
\onslide<15->
Now, back solve the system
\begin{center}
\begin{tabular}{rcrcrcr}
$x$&$+$&$y$&$+$&$  z$&$=$&$3$\\
      &      &$y$&$+$&$3z$&$=$&$-2$\\
      &      &      &      &$  z$&$=$&$-2$
\end{tabular}
\end{center}
\visible<16->{Start with the third equation: $z=-2$}

\visible<17->{Plug it into the second equation and solve for $y$:
\begin{equation*}
y+3(-2)=-2\quad\Rightarrow\quad y=4
\end{equation*}}
\visible<18->{Plug both into the first equation and solve for $x$:
\begin{equation*}
x+(4)+(-2)=3\quad\Rightarrow\quad x=1
\end{equation*}}
\end{overprint}
\end{example}
\end{frame}

\begin{frame}
\begin{block}{Existence and Uniqueness of Solutions}
During Gaussian Elimination:
\begin{itemize}
\item<+->If a row of the form
\begin{equation*}
\begin{bmatrix}[ccc|c]
0 & \cdots & 0 & k\neq 0
\end{bmatrix}
\end{equation*}
is encountered, then the system has \emph{no solutions}.
\item<+->
If a row of the form
\begin{equation*}
\begin{bmatrix}[ccc|c]
0 & \cdots & 0 & 0
\end{bmatrix}
\end{equation*}
is encountered, then the system has \emph{infinitely many solutions}.
\end{itemize}
\onslide<+->
Some vocabulary:
\begin{itemize}
\item<.-> If a system has no solutions, it is called \textbf{inconsistent}.
\item<+-> If a system has at least one solution, it is called \textbf{consistent}.
\begin{itemize}
\item<+-> A system with exactly one solution is called \textbf{independent}.
\item<+-> A system with more than one solution is called \textbf{dependent}.
\end{itemize}
\end{itemize}
\end{block}
\end{frame}

\begin{frame}
\begin{block}{Reduced Row Echelon Form}
An augmented matrix is said to be in \textbf{R}educed \textbf{R}ow \textbf{E}chelon \textbf{F}orm if:
\begin{equation*}
\begin{bmatrix}[ccc|c]
1&\cdots&0&k_1\\
\vdots&\ddots&\vdots&\vdots\\
0&\cdots&1&k_m
\end{bmatrix}
\end{equation*}
\end{block}\pause
\begin{block}{Rank}
The \textbf{rank} $r$ of a matrix is equal to how many 1's are in the diagonal of it's Reduced Row Echelon Form.
\begin{itemize}
\item If $r$ equals the number of variables, there is a unique solution.
\item If $r$ is less than the number of variables, the solutions are not unique.
\end{itemize}
\end{block}
\end{frame}
\end{document}
