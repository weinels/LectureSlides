\documentclass{beamer}

\usepackage[english]{babel}
\usepackage[utf8x]{inputenc}
\usepackage[inline]{asymptote}
\usepackage{slide_helper}
\usepackage{caption}
\usepackage{subcaption}

%\hideasymptote% uncomment to not compile asymptote graphs

\title[MATH 2250 - Section 4.2]{Real Characteristic Roots}

\begin{asydef}
import graph;
import slopefield;
import fontsize;
defaultpen(fontsize(9pt));
ngraph=1000;

int pen_pos=-1;
pen[] pens={blue, red, heavycyan, heavymagenta, lightolive};
pens.cyclic=true;

pen next_color() {return pens[++pen_pos];}
\end{asydef}

\begin{document}

\begin{frame}
  \titlepage
\end{frame}

\begin{frame}
\begin{block}{Constant Coefficient Second-Order DE}
Consider the special case of a linear second-order homogeneous DE where all the coefficients are constant.
\begin{equation*}
ay^{\prime\prime}+by^{\prime}+cy=0,
\quad a,b,c\in\R
\quad\text{and}\quad a\neq 0
\end{equation*}
\begin{overprint}
\onslide<1>
\onslide<2-8>
Let us try an approach similar to that for first-order linear DEs.

\visible<3->{\vspace{2mm}
If we let $y=e^{rt}$}\visible<4->{, then $y^\prime=re^{rt}$}\visible<5->{ and $y^{\prime\prime}=r^2e^{rt}$.}

\visible<6->{\vspace{2mm}
Thus, the DE becomes:
\begin{equation*}
0=ar^2e^{rt}+bre^{rt}+ce^{rt}\visible<7->{=e^{rt}(ar^2+br+c)}
\end{equation*}}

\visible<8->{\vspace{-5mm}
Because the range of $e^{rt}$ is $\interval{\open{0}}{\open{\infty}}$ this will be satisfied only when
\begin{equation*}
ar^2+br+c=0
\end{equation*}
We call this the \textbf{characteristic equation} of the DE and is key to finding the solutions that form a basis of the solution space.}
\onslide<9->
We can solve the characteristic equation for r using the quadratic formula.
\begin{equation*}
ar^2+br+c=0
\quad\Rightarrow\quad
r=\dfrac{-b\pm\sqrt{b^2-4ac}}{2a}
\end{equation*}
\visible<10->{Recall that the \emph{discriminant} ($\Delta=b^2-4ac$) tells us which of the possibilities we have for the solutions:
\begin{itemize}
\item<10- | alert@10> $\Delta>0$: Two distinct real roots.
\item<11- | alert@11> $\Delta=0$: One real root.
\item<12- | alert@12> $\Delta<0$: Two conjugate complex roots. (Section 4.3.)
\end{itemize}}

\visible<13->{These roots are called \textbf{characteristic roots} or \textbf{eigenvalues}. 

(The term \emph{eigenvalue} is from Linear Algebra and will be talked about later.)}
\end{overprint}
\end{block}
\end{frame}

\begin{frame}
\begin{block}{Solution for Distinct Real Characteristic Roots}
For $\Delta>0$, the characteristic roots of the DE
\begin{equation*}
ay^{\prime\prime}+by^{\prime}+cy=0
\end{equation*}
are
\begin{equation*}
r_1=\dfrac{-b+\sqrt{b^2-4ac}}{2a}
,\quad\text{and}\quad
r_2=\dfrac{-b-\sqrt{b^2-4ac}}{2a}
\end{equation*}\pause
The functions $e^{r_1t}$ and $e^{r_2t}$ are linearly independent solutions, and the general solution is given by
\begin{equation*}
y(t)=c_1e^{r_1t}+c_2e^{r_2t}
\end{equation*}
where $c_1$ and $c_2$ are arbitrary constants determined by the initial conditions.\pause

\vspace{2mm}
The set $\left\{e^{r_1t},e^{r_2t}\right\}$ forms a basis for the solution space $\Sol$.
\end{block}
\end{frame}

\begin{frame}[fragile]
\begin{example}
\begin{overprint}
\onslide<1-6>
Let us find the general solution of
\begin{equation*}
y^{\prime\prime}+5y^{\prime}+6y=0
\end{equation*}
\visible<2->{First, let us write the characteristic equation:
\begin{equation*}
0=r^2+5r+6
\visible<3->{=(r+2)(r+3)}
\end{equation*}}

\vspace{-5mm}
\visible<4->{which has solutions $r_1=-2$ and $r_2=-3$.}
\visible<5->{

\vspace{2mm}
Thus, the general solution is 
\begin{equation*}
y(t)=c_1e^{-2t}+c_2e^{-3t}
\end{equation*}}

\visible<6->{\vspace{-5mm}
The set $\left\{e^{-2t},e^{-3t}\right\}$ is a basis of the solution space $\Sol$, and $\dim\Sol=2$.}
\onslide<7->
\begin{figure}
\centering
\begin{subfigure}[b]{0.49\textwidth}
\begin{asy}
size(160,0);
real min_x=-4, max_x=4;
real min_y=-4, max_y=4;

real r_1=-2;
real r_2=-3;
pair[] curves = {	( 3.6, 0.0), 
					( 2.4, 0.0), 
					( 0.0, 2.0),
					(-1.0, 1.4),
					(-2.9, 3.4)};
					
for (pair k : curves)
{
	real A=k.x;
	real B=k.y;
	real c_1=3*A+B;
	real c_2=-B-2A;
	real f(real t) {return c_1*exp(r_1*t)+c_2*exp(r_2*t);}
	draw(graph(f,min_x, max_x),next_color()+0.75bp);
}
limits((min_x,min_y),(max_x,max_y),Crop);
xaxis("$t$",YEquals(0),min_x,max_x,Ticks(NoZero));
yaxis("$y$",XEquals(0),min_y,max_y,Ticks(NoZero));
\end{asy}
\caption{Time Series}
\end{subfigure}
\begin{subfigure}[b]{0.49\textwidth}
\begin{asy}
size(160);
real min_x=-6, max_x=6;
real min_y=-6, max_y=6;

pair start=(min_x,min_y);
pair end=(max_x,max_y);

real length(pair z) {return (z.x == 0) && (z.y == 0) ? 0.0001 : sqrt(z.x*z.x+z.y*z.y);}

// phase plot for ay''+by'+cy=0
real a=1;
real b=5;
real c=6;
real F(pair z) {return (-b*z.y-c*z.x)/a;}
path vector(pair z) {return (0,0)--1/(2*length((z.y,F(z))))*(z.y, F(z));}

add(vectorfield(vector,start,end,arrow=EndArrow(SimpleHead)));

for(real gx=min_x+1; gx<=max_x-1; ++gx)
	draw((gx,min_y)--(gx,max_y),dotted+darkgray);
    
for(real gy=min_y+1; gy<=max_y-1; ++gy)
	draw((min_x,gy)--(max_x,gy),dotted+darkgray); 

// draw trajectories
DefaultHead.size=new real(pen p=currentpen) {return 2.5mm;};

real r_1=-2;
real r_2=-3;
real t_start=-1;
real t_end=2;

triple[] curves = {	( 3.6, 0.0, 0.97), 
					( 2.4, 0.0, 0.965), 
					( 0.0, 2.0, 0.97),
					(-1.0, 1.4, 0.6),
					(-2.9, 3.4, 0.9)};					
for (triple k : curves)
{
	real A=k.x;
	real B=k.y;
	real c_1=3*A+B;
	real c_2=-B-2A;
	real X(real t) {return c_1*exp(r_1*t)+c_2*exp(r_2*t);}
	real Y(real t) {return c_1*r_1*exp(r_1*t)+c_2*r_2*exp(r_2*t);}
	draw(graph(X,Y,t_start,t_end),next_color()+1.0bp,Arrow(Relative(k.z)));
}

limits(start,end,Crop);

xaxis("$y$",YEquals(min_y),min_x,max_x,LeftTicks());
xaxis(YEquals(max_y),min_x,max_x);
yaxis("$y^\prime$",XEquals(min_x),min_y,max_y,LeftTicks());
yaxis(XEquals(max_x),min_y,max_y);
\end{asy}
\caption{Phase Portrait}
\end{subfigure}
\end{figure}
\end{overprint}
\end{example}
\end{frame}

\begin{frame}[fragile]
\begin{example}
\begin{overprint}
\onslide<1-6>
Let us find the general solution of
\begin{equation*}
y^{\prime\prime}-y=0
\end{equation*}
\visible<2->{First, let us write the characteristic equation:
\begin{equation*}
0=r^2-1
\visible<3->{=(r+1)(r-1)}
\end{equation*}}

\visible<4->{\vspace{-5mm}
which has solutions $r_1=1$ and $r_2=-1$.}

\visible<5->{\vspace{2mm}
Thus, the general solution is 
\begin{equation*}
y(t)=c_1e^{t}+c_2e^{-t}
\end{equation*}}

\visible<6->{\vspace{-5mm}
The set $\left\{e^{t},e^{-t}\right\}$ is a basis of the solution space $\Sol$, and $\dim\Sol=2$.}
\onslide<7->
\begin{figure}
\centering
\begin{subfigure}[b]{0.49\textwidth}
\begin{asy}
size(160,0);
real min_x=-6, max_x=6;
real min_y=-6, max_y=6;

real r_1=1;
real r_2=-1;
pair[] curves = {	( 1.0,  0.0), 
					( 0.0,  3.0), 
					( 0.0, -1.9),
					(-4.8,  0.0)};
					
for (pair k : curves)
{
	real A=k.x;
	real B=k.y;
	real c_1=(A+B)/2;
	real c_2=(A-B)/2;
	real f(real t) {return c_1*exp(r_1*t)+c_2*exp(r_2*t);}
	draw(graph(f,min_x, max_x),next_color()+0.75bp);
}
limits((min_x,min_y),(max_x,max_y),Crop);
xaxis("$t$",YEquals(0),min_x,max_x,Ticks(NoZero));
yaxis("$y$",XEquals(0),min_y,max_y,Ticks(NoZero));
\end{asy}
\caption{Time Series}
\end{subfigure}
\begin{subfigure}[b]{0.49\textwidth}
\begin{asy}
size(150);
real min_x=-6, max_x=6;
real min_y=-6, max_y=6;

pair start=(min_x,min_y);
pair end=(max_x,max_y);

real length(pair z) {return (z.x == 0) && (z.y == 0) ? 0.0001 : sqrt(z.x*z.x+z.y*z.y);}

// phase plot for ay''+by'+cy=0
real a=1;
real b=0;
real c=-1;
real F(pair z) {return (-b*z.y-c*z.x)/a;}
path vector(pair z) {return (0,0)--1/(2*length((z.y,F(z))))*(z.y, F(z));}

add(vectorfield(vector,start,end,arrow=EndArrow(SimpleHead)));

for(real gx=min_x+1; gx<=max_x-1; ++gx)
	draw((gx,min_y)--(gx,max_y),dotted+darkgray);
    
for(real gy=min_y+1; gy<=max_y-1; ++gy)
	draw((min_x,gy)--(max_x,gy),dotted+darkgray); 

// draw trajectories
DefaultHead.size=new real(pen p=currentpen) {return 2.5mm;};

real r_1=1;
real r_2=-1;
real t_start=-4;
real t_end=4;

triple[] curves = {	( 1.0,  0.0, 0.48), 
					( 0.0,  3.0, 0.49), 
					( 0.0, -1.9, 0.52),
					(-4.8,  0.0, 0.5)};					
for (triple k : curves)
{
	real A=k.x;
	real B=k.y;
	real c_1=(A+B)/2;
	real c_2=(A-B)/2;
	real X(real t) {return c_1*exp(r_1*t)+c_2*exp(r_2*t);}
	real Y(real t) {return c_1*r_1*exp(r_1*t)+c_2*r_2*exp(r_2*t);}
	draw(graph(X,Y,t_start,t_end),next_color()+1.0bp,Arrow(Relative(k.z)));
}

limits(start,end,Crop);

xaxis("$y$",YEquals(min_y),min_x,max_x,LeftTicks());
xaxis(YEquals(max_y),min_x,max_x);
yaxis("$y^\prime$",XEquals(min_x),min_y,max_y,LeftTicks());
yaxis(XEquals(max_x),min_y,max_y);
\end{asy}
\caption{Phase Portrait}
\end{subfigure}
\end{figure}
\end{overprint}
\end{example}
\end{frame}

\begin{frame}
\begin{block}{Solution for Equal Real Characteristic Roots}
For $\Delta=0$, the characteristic roots of the DE
\begin{equation*}
ay^{\prime\prime}+by^{\prime}+cy=0
\end{equation*}
are
\begin{equation*}
r=-\dfrac{b}{2a}
\end{equation*}\pause
The functions $e^{rt}$ and $te^{rt}$ are linearly independent solutions, and the general solution is given by
\begin{equation*}
y(t)=c_1e^{rt}+c_2te^{t}
\end{equation*}
where $c_1$ and $c_2$ are arbitrary constants determined by the initial conditions.\pause

\vspace{2mm}
The set $\left\{e^{rt},te^{rt}\right\}$ forms a basis for the solution space $\Sol$.
\end{block}
\end{frame}

\begin{frame}[fragile]
\begin{example}
\begin{overprint}
\onslide<1-6>
Let us find the general solution of
\begin{equation*}
y^{\prime\prime}-4y^{\prime}+4y=0
\end{equation*}
\visible<2->{First, let us write the characteristic equation:
\begin{equation*}
0=r^2-4r+4
\visible<3->{={(r-2)}^2}
\end{equation*}}

\visible<4->{\vspace{-5mm}
which has solutions $r=2$.}

\visible<5->{\vspace{2mm}
Thus, the general solution is 
\begin{equation*}
y(t)=c_1e^{2t}+c_2te^{2t}
\end{equation*}}

\visible<6->{\vspace{-5mm}
The set $\left\{e^{2t},te^{2t}\right\}$ is a basis of the solution space $\Sol$, and $\dim\Sol=2$.}
\onslide<7->
\begin{figure}
\centering
\begin{subfigure}[b]{0.49\textwidth}
\begin{asy}
size(160,0);
real min_x=-6, max_x=6;
real min_y=-6, max_y=6;

real r=2;
pair[] curves = {	( 4.0,  0.0), 
					( 1.0,  1.3), 
					(-2.0,  0.0),
					(-5.1,  0.0),
					( 1.3,  0.0)};
					
for (pair k : curves)
{
	real A=k.x;
	real B=k.y;
	real c_1=A;
	real c_2=B-2A;
	real f(real t) {return c_1*exp(r*t)+c_2*t*exp(r*t);}
	draw(graph(f,min_x, max_x),next_color()+0.75bp);
}
limits((min_x,min_y),(max_x,max_y),Crop);
xaxis("$t$",YEquals(0),min_x,max_x,Ticks(NoZero));
yaxis("$y$",XEquals(0),min_y,max_y,Ticks(NoZero));
\end{asy}
\caption{Time Series}
\end{subfigure}
\begin{subfigure}[b]{0.49\textwidth}
\begin{asy}
size(150);
real min_x=-8, max_x=8;
real min_y=-8, max_y=8;

pair start=(min_x,min_y);
pair end=(max_x,max_y);

real length(pair z) {return (z.x == 0) && (z.y == 0) ? 0.0001 : sqrt(z.x*z.x+z.y*z.y);}

// phase plot for ay''+by'+cy=0
real a=1;
real b=-4;
real c=4;
real F(pair z) {return (-b*z.y-c*z.x)/a;}
path vector(pair z) {return (0,0)--1/(2*length((z.y,F(z))))*(z.y, F(z));}

add(vectorfield(vector,start,end,arrow=EndArrow(SimpleHead)));

for(real gx=min_x+1; gx<=max_x-1; ++gx)
	draw((gx,min_y)--(gx,max_y),dotted+darkgray);
    
for(real gy=min_y+1; gy<=max_y-1; ++gy)
	draw((min_x,gy)--(max_x,gy),dotted+darkgray); 

// draw trajectories
DefaultHead.size=new real(pen p=currentpen) {return 2.5mm;};

real r=2;
real t_start=-3.0;
real t_end=1.5;

triple[] curves = {	( 4.0, 0.0, 0.025), 
					( 1.0, 1.3, 0.35), 
					(-2.0, 0.0, 0.02),
					(-5.1, 0.0, 0.02),
					( 1.3, 0.0, 0.02)};				
for (triple k : curves)
{
	real A=k.x;
	real B=k.y;
	real c_1=A;
	real c_2=B-2A;
	real X(real t) {return c_1*exp(r*t)+c_2*t*exp(r*t);}
	real Y(real t) {return c_1*r*exp(r*t)+c_2*r*t*exp(r*t)+c_2*exp(r*t);}
	draw(graph(X,Y,t_start,t_end),next_color()+1.0bp,Arrow(Relative(k.z)));
}

limits(start,end,Crop);

xaxis("$y$",YEquals(min_y),min_x,max_x,LeftTicks());
xaxis(YEquals(max_y),min_x,max_x);
yaxis("$y^\prime$",XEquals(min_x),min_y,max_y,LeftTicks());
yaxis(XEquals(max_x),min_y,max_y);
\end{asy}
\caption{Phase Portrait}
\end{subfigure}
\end{figure}
\end{overprint}
\end{example}
\end{frame}

\begin{frame}
\begin{block}{Overdamped Mass-Spring System}
The motion of a mass-spring system is called \textbf{overdamped} when we have $\Delta>0$. Both characteristic roots are negative and the solutions
\begin{equation*}
x(t)=c_1e^{r_1t}+c_2e^{r_2t}
\end{equation*}
tend towards zero with oscillation, crossing the $t$-axis at most once.
\end{block}
\end{frame}

\begin{frame}[fragile]
\begin{example}
\begin{overprint}
\onslide<1-6>
Let us find the general solution of
\begin{equation*}
\ndot{x}{2}+3\ndot{x}{1}+2x=0
\end{equation*}
\visible<2->{First, let us write the characteristic equation:
\begin{equation*}
0=r^2+3r+2
\visible<3->{=(r+1)(r+2)}
\end{equation*}}

\visible<4->{\vspace{-5mm}
which has solutions $r_1=-1$ and $r_2=-2$.}

\visible<5->{\vspace{2mm}
Thus, this system is overdamped and has general solution 
\begin{equation*}
x(t)=c_1e^{-t}+c_2e^{-2t}
\end{equation*}}

\visible<6->{\vspace{-5mm}
The set $\left\{e^{-t},e^{-2t}\right\}$ is a basis of the solution space $\Sol$, and $\dim\Sol=2$.}
\onslide<7->
\begin{figure}
\centering
\begin{subfigure}[b]{0.49\textwidth}
\begin{asy}
size(160,0);
real min_x=0, max_x=5;
real min_y=-3, max_y=3;

real r_1=-1;
real r_2=-2;
pair[] curves = {	(1.0,  6.0), 
					(1.0,  3.0), 
					(1.0,  0.0),
					(1.0, -10.0),
					(1.0, -14.0)};
					
for (pair k : curves)
{
	real A=k.x;
	real B=k.y;
	real c_1=2A+B;
	real c_2=-A-B;
	real f(real t) {return c_1*exp(r_1*t)+c_2*exp(r_2*t);}
	draw(graph(f,min_x, max_x),next_color()+0.75bp);
}
limits((min_x,min_y),(max_x,max_y),Crop);
xaxis("$t$",YEquals(0),min_x,max_x,Ticks(NoZero));
yaxis("$x$",XEquals(0),min_y,max_y,Ticks(NoZero));
\end{asy}
\caption{Time Series}
\end{subfigure}
\begin{subfigure}[b]{0.49\textwidth}
\begin{asy}
size(150);
real min_x=-5, max_x=5;
real min_y=-5, max_y=5;

pair start=(min_x,min_y);
pair end=(max_x,max_y);

real length(pair z) {return (z.x == 0) && (z.y == 0) ? 0.0001 : sqrt(z.x*z.x+z.y*z.y);}

// phase plot for ay''+by'+cy=0
real a=1;
real b=+3;
real c=2;
real F(pair z) {return (-b*z.y-c*z.x)/a;}
path vector(pair z) {return (0,0)--1/(2*length((z.y,F(z))))*(z.y, F(z));}

add(vectorfield(vector,start,end,arrow=EndArrow(SimpleHead)));

for(real gx=min_x+1; gx<=max_x-1; ++gx)
	draw((gx,min_y)--(gx,max_y),dotted+darkgray);
    
for(real gy=min_y+1; gy<=max_y-1; ++gy)
	draw((min_x,gy)--(max_x,gy),dotted+darkgray); 

// draw trajectories
DefaultHead.size=new real(pen p=currentpen) {return 2.5mm;};

real r_1=-1;
real r_2=-2;
real t_start=-1;
real t_end=10;

triple[] curves = {	(1.0,  6.0,  0.94), 
					(1.0,  3.0,  0.92), 
					(1.0,  0.0,  0.72),
					(1.0, -10.0, 0.97),
					(1.0, -14.0, 0.97)};			
for (triple k : curves)
{
	real A=k.x;
	real B=k.y;
	real c_1=2A+B;
	real c_2=-A-B;
	real X(real t) {return c_1*exp(r_1*t)+c_2*exp(r_2*t);}
	real Y(real t) {return c_1*r_1*exp(r_1*t)+c_2*r_2*exp(r_2*t);}
	draw(graph(X,Y,t_start,t_end),next_color()+1.0bp,Arrow(Relative(k.z)));
}

limits(start,end,Crop);

xaxis("$y$",YEquals(min_y),min_x,max_x,LeftTicks());
xaxis(YEquals(max_y),min_x,max_x);
yaxis("$y^\prime$",XEquals(min_x),min_y,max_y,LeftTicks());
yaxis(XEquals(max_x),min_y,max_y);
\end{asy}
\caption{Phase Portrait}
\end{subfigure}
\end{figure}
\end{overprint}
\end{example}
\end{frame}

\begin{frame}
\begin{block}{Critically Damped Mass-Spring System}
the motion of a mass-spring system is called \textbf{critically damped} when we have $\Delta=0$. The single characteristic root are negative and the solutions
\begin{equation*}
x(t)=c_1e^{rt}+c_2te^{rt}
\end{equation*}
tend towards zero, crossing the $t$-axis at most once.
\end{block}
\end{frame}

\begin{frame}[fragile]
\begin{example}
\begin{overprint}
\onslide<1-6>
Let us find the general solution of
\begin{equation*}
\ndot{x}{2}+6\ndot{x}{1}+9x=0
\end{equation*}
\visible<2->{First, let us write the characteristic equation:
\begin{equation*}
0=r^2+6r+9
\visible<3->{={(r+3)}^2}
\end{equation*}}

\visible<4->{\vspace{-5mm}
which has solution $r=-3$.}

\visible<5->{\vspace{2mm}
Thus, this system is critically damped and has general solution 
\begin{equation*}
x(t)=c_1e^{-3t}+c_2te^{-3t}
\end{equation*}}

\visible<6->{\vspace{-5mm}
The set $\left\{e^{-3t},te^{-3t}\right\}$ is a basis of the solution space $\Sol$, and $\dim\Sol=2$.}
\onslide<7->
\begin{figure}
\centering
\begin{subfigure}[b]{0.49\textwidth}
\begin{asy}
size(160);
real min_x=-1, max_x=4;
real min_y=-2, max_y=3;

real r=-3;
pair[] curves = {	( 1.0, -20.0), 
					( 1.0, -16.0), 
					( 1.0, -12.0),
					( 1.0,   7.0),
					( 1.0,  16.0)};
					
for (pair k : curves)
{
	real A=k.x;
	real B=k.y;
	real c_1=A;
	real c_2=B+3*A;
	real f(real t) {return c_1*exp(r*t)+c_2*t*exp(r*t);}
	draw(graph(f,min_x, max_x),next_color()+0.75bp);
}
limits((min_x,min_y),(max_x,max_y),Crop);
xaxis("$t$",YEquals(0),min_x,max_x,Ticks(NoZero));
yaxis("$y$",XEquals(0),min_y,max_y,Ticks(NoZero));
\end{asy}
\caption{Time Series}
\end{subfigure}
\begin{subfigure}[b]{0.49\textwidth}
\begin{asy}
size(150);
real min_x=-5, max_x=5;
real min_y=-5, max_y=5;

pair start=(min_x,min_y);
pair end=(max_x,max_y);

real length(pair z) {return (z.x == 0) && (z.y == 0) ? 0.0001 : sqrt(z.x*z.x+z.y*z.y);}

// phase plot for ay''+by'+cy=0
real a=1;
real b=6;
real c=9;
real F(pair z) {return (-b*z.y-c*z.x)/a;}
path vector(pair z) {return (0,0)--1/(2*length((z.y,F(z))))*(z.y, F(z));}

add(vectorfield(vector,start,end,arrow=EndArrow(SimpleHead)));

for(real gx=min_x+1; gx<=max_x-1; ++gx)
	draw((gx,min_y)--(gx,max_y),dotted+darkgray);
    
for(real gy=min_y+1; gy<=max_y-1; ++gy)
	draw((min_x,gy)--(max_x,gy),dotted+darkgray); 

// draw trajectories
DefaultHead.size=new real(pen p=currentpen) {return 2.5mm;};

real r=-3;
real t_start=-0.15;
real t_end=3.0;

triple[] curves = {	( 1.0, -20.0, 0.92), 
					( 1.0, -16.0, 0.92), 
					( 1.0, -12.0, 0.92),
					( 1.0,   7.0, 0.4),
					( 1.0,  16.0, 0.8)};			
for (triple k : curves)
{
	real A=k.x;
	real B=k.y;
	real c_1=A;
	real c_2=B-2A;
	real X(real t) {return c_1*exp(r*t)+c_2*t*exp(r*t);}
	real Y(real t) {return c_1*r*exp(r*t)+c_2*r*t*exp(r*t)+c_2*exp(r*t);}
	draw(graph(X,Y,t_start,t_end),next_color()+1.0bp,Arrow(Relative(k.z)));
}

limits(start,end,Crop);

xaxis("$y$",YEquals(min_y),min_x,max_x,LeftTicks());
xaxis(YEquals(max_y),min_x,max_x);
yaxis("$y^\prime$",XEquals(min_x),min_y,max_y,LeftTicks());
yaxis(XEquals(max_x),min_y,max_y);
\end{asy}
\caption{Phase Portrait}
\end{subfigure}
\end{figure}
\end{overprint}
\end{example}
\end{frame}

\begin{frame}
\begin{block}{Existence and Uniqueness Theorem (Second-Order)}
Let $p(t)$ and $q(t)$ be continuous on the open interval $\interval{\open{a}}{\open{b}}$ contaning $t_0$. 

\vspace{2mm}
For \emph{any} $A,B\in\R$, there exists a unique solution $y(t)$ defined on $\interval{\open{a}}{\open{b}}$ to the IVP
\begin{equation*}
y^{\prime\prime}+p(t)y^{\prime}+q(t)y=0,
\quad y(t_0)=A,
\quad y^{\prime}(t_0)=B
\end{equation*}
\end{block}\pause
\begin{proof}
This is an extension of Picard's Theorem.
\end{proof}\pause
\begin{block}{Solution Space Theorem (Second-Order)}
The solution space $\Sol$ for a second-order homogeneous differential equation has dimension 2.
\end{block}\pause
\begin{proof}
See Page 217 in your textbook
\end{proof}
\end{frame}

\begin{frame}
\begin{block}{Solutions of Homogeneous Linear DE (Second-Order)}
For any linear second-order homogeneous DE on $\interval{\open{a}}{\open{b}}$,
\begin{equation*}
y^{\prime\prime}+p(t)y^{\prime}+q(t)y=0
\end{equation*}
for which $p$ and $q$ are continuous on $\interval{\open{a}}{\open{b}}$, \emph{any} two linearly independent solutions $\{y_1,y_2\}$ form a basis of the solution space $\Sol$, and \emph{every} solution $y$ on $\interval{\open{a}}{\open{b}}$ can be written as
\begin{equation*}
y(t)=c_1y_1(t)+c_2y_2(t)
\end{equation*}
for some $c_1,c_2\in\R$.
\end{block}
\end{frame}

\begin{frame}
\begin{block}{}
We can generalize these ideas for $n$th-order DEs.
\end{block}\pause
\begin{block}{Existence and Uniqueness Theorem ($n$th-Order)}
Let $p_1(t), p_2(t),\ldots,p_n(t)$ be continuous on the open interval $\interval{\open{a}}{\open{b}}$ contaning $t_0$. For any initial conditions $A_0,A_1,\ldots,A_{n-1}\in\R$, there exists a unique solution $y(t)$ defined on $\interval{\open{a}}{\open{b}}$ to the IVP
\begin{equation*}
y^{(n)}+p_1(t)y^{(n-1)}+p_2(t)y^{(n-3)}+\cdots+p_n(t)y=0
\end{equation*}
where
\begin{equation*}
y(t_0)=A_0,\quad
y^{\prime}(t_0)=A_1,
\ldots,\quad
y^{(n-1)}(t_0)=A_{n-1}
\end{equation*}
\end{block}\pause
\begin{block}{Solution Space Theorem ($n$th-Order)}
The solution space $\Sol$ for a $n$th-order linear homogeneous differential equation has dimension $n$.
\end{block}
\end{frame}

\begin{frame}
\begin{block}{Solutions of Homogeneous Linear DE ($n$th-Order)}
For any linear $n$th-order homogeneous DE on $\interval{\open{a}}{\open{b}}$,
\begin{equation*}
y^{(n)}+p_1(t)y^{(n-1)}+\cdots+p_n(t)y=0
\end{equation*}
for which $p_1(t), p_2(t),\ldots,p_n(t)$ are continuous on $\interval{\open{a}}{\open{b}}$, \emph{any} $n$ linearly independent solutions $\{y_1,y_2,\ldots,y_2\}$ form a basis of the solution space $\Sol$, and \emph{every} solution $y$ on $\interval{\open{a}}{\open{b}}$ can be written as
\begin{equation*}
y(t)=c_1y_1(t)+c_2y_2(t)+\cdots+c_n y_n(t)
\end{equation*}
for some $c_1,c_2,\ldots,c_n\in\R$.
\end{block}
\end{frame}

\begin{frame}
\begin{block}{Note}
The Solution Space Theorem provides us with the number of solutions in a basis for all $n$-th order linear homogeneous DE\@.
\onslide<+->
If we start with $m$ solutions, then
\begin{itemize}[<+- | alert@+>]
\item if $m>n$, the solutions cannot be linearly independent.
\item if $m=n$, we must test for linear independence.
\item if $m<n$, the set of solutions does not span $\Sol$.
\end{itemize}
\end{block}
\onslide<+->
\begin{block}{}
A Wronskian conveys more information in the test for linear independence when the functions are solutions to the same $n$th-order linear homogeneous DE\@.
\end{block}
\end{frame}

\newcommand{\custdiff}[1]{a_{#1} (t)\dfrac{d^{#1}y}{dt^{#1}}}
\begin{frame}
\begin{block}{The Wronskian Test for Linear Independence of DE Solutions}
Suppose $\{y_1,y_2,\ldots,y_n\}$ is a set of solutions on $\interval{\open{a}}{\open{b}}$ of a $n$th-order linear homogeneous DE\@,
\begin{equation*}
L(y)=\custdiff{n}+\custdiff{n-1}+\cdots+\custdiff{1}+a_0y=0
\end{equation*}

\vspace{-2mm}
\onslide<+->
\begin{enumerate}[<+- | alert@+>]
\item If $W[y_1,y_2,\ldots,y_n]\neq0$ at any point $t\in\interval{\open{a}}{\open{b}}$, the set $\{y_1,y_2,\ldots,y_n\}$ is linearly independent.
\item If $W[y_1,y_2,\ldots,y_n]=0$ on all $t\in\interval{\open{a}}{\open{b}}$, the set is linearly dependent.
\end{enumerate}
\onslide<+->
The Wronskian test works in \quotetext{both directions} only for $n$ solutions to an $n$th-order linear homogeneous DE\@.
\end{block}
\onslide<+->
\begin{proof}
See page 220 in your textbook
\end{proof}
\end{frame}

\begin{frame}
\begin{example}
Consider the set of solutions $A=\{2,t-1,t^2,t^3+t\}$ to $\dfrac{d^4y}{dy^4}=0$ on $\R$.\pause
\begin{equation*}
\begin{aligned}
W &=
\begin{vmatrix}
2 & t-1 & t^2 & t^3+t  \\
0 & 1   & 2t  & 3t^2+1 \\
0 & 0   & 2   & 6t     \\
0 & 0   & 0   & 6      \\
\end{vmatrix}\\\pause
&=2
\begin{vmatrix}
1   & 2t  & 3t^2+1 \\
0   & 2   & 6t     \\
0   & 0   & 6      \\
\end{vmatrix}\\\pause
&=2
\begin{vmatrix}
2   & 6t     \\
0   & 6      \\
\end{vmatrix}\\\pause
&=24\pause
\neq0
\end{aligned}
\end{equation*}

\vspace{-3mm}
So, $A$ is linearly independent and hence a basis of $\Sol$.
\end{example}
\end{frame}

\begin{frame}
\begin{example}
Consider the set of solutions $B=\{t,t+1,t^2-1,t^2\}$ to $\dfrac{d^4y}{dy^4}=0$ on $\R$.\pause
\begin{equation*}
W =
\begin{vmatrix}
t & t+1 & t^2 -1 & t^2  \\
1 & 1   & 2t     & 2t   \\
0 & 0   & 2      & 2    \\
0 & 0   & 0      & 0    \\
\end{vmatrix}\pause
=0
\end{equation*}

\vspace{-3mm}
So, $B$ is linearly dependent.\pause\ (For example, $t=(t+1)+(t^2-1)-(t^2)$.)
\end{example}
\end{frame}

\begin{frame}
\begin{example}
Consider the set of solutions $C=\{1,t^2,t^3\}$ to $\dfrac{d^4y}{dy^4}=0$ on $\R$.\pause
\begin{equation*}
\begin{aligned}
W &=
\begin{vmatrix}
1 & t^2 & t^3  \\
0 & 2t  & 3t^2 \\
0 & 2   & 6t   \\
\end{vmatrix}\\\pause
&=
\begin{vmatrix}
2t  & 3t^2 \\
2   & 6t   \\
\end{vmatrix}\\\pause
&=6t^2\pause
=0\text{ only when }t=0.
\end{aligned}
\end{equation*}

\vspace{-3mm}
%:
Here, $W$ is not identically zero, so we know $C$ is a linearly independent set.\pause

But the strong conclusion of the Wronskian test did not occur here because $C$ contains only three solutions for a fourth-order DE\@.
\end{example}
\end{frame}
\end{document}