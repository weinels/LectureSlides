\documentclass{beamer}
\usepackage[utf8]{inputenc}
\usepackage[english]{babel}
\usepackage[T1]{fontenc}
\usepackage[inline]{asymptote}
\usepackage{slide_helper}

\title[MATH 2250 - Section 2.6]{Systems of Differential Equations: A First Look}

\begin{document}
\begin{frame}
\titlepage
\end{frame}

\begin{frame}
\begin{block}{Systems of Differential Equations}
Often a population study will involve two or more interacting species. This leans two two for more \textbf{coupled} differential equations. (Similar systems arise in electrical and mechanical engineering.)\pause

\vspace{2mm}
For example:
\begin{equation*}
\begin{matrix}[rrrrr]
x^\prime &=& 2x &-& xy \\
y^\prime &=& -3y &+& 0.5y
\end{matrix}
\end{equation*}
where we are looking for two functions $x$ and $y$ that both depend on $t$.\pause

\vspace{2mm}
A simpler case is where the equations in the system are \textbf{uncoupled}:
\begin{equation*}
\begin{matrix}[rrr]
x^\prime &=& 2x \\
y^\prime &=& -3y
\end{matrix}
\end{equation*}
\end{block}\pause

\begin{block}{Analytic Definition of a Solution of a DE System}
A \textbf{solution} of a system of two differential equations is a pair of functions $x(t)$ and $y(t)$ that simultaneously satisfies both equations.
\end{block}
\end{frame}

\end{document}
