\documentclass{beamer}
\usepackage[utf8]{inputenc}
\usepackage[english]{babel}
\usepackage[T1]{fontenc}
\usepackage[inline]{asymptote}
\usepackage{slide_helper}
\usepackage{asy_helper}

\title[MATH 2250 - Section 2.3]{Growth and Decay Phenomena}

\begin{document}
\begin{frame}
\titlepage
\end{frame}

\begin{frame}
\begin{block}{Growth and Decay Equation}
The differential equation
\begin{equation*}
\dfrac{dy}{dt} = ky
\end{equation*}
covers two phenomena, depending on whether $k$ is positive or negative.
\begin{description}[<+(1)- | alert@+>] % chktex 36
\item[Growth:] If $k>0$, $k$ is called the \textbf{growth constant} or \textbf{rate of growth} and the equation is called the \textbf{growth equation}.
\item[Decay:] If $k<0$, $k$ is called the \textbf{decay constant} or \textbf{rate of decay} and the equation is called the \textbf{decay equation}.
\end{description}
\end{block}
\onslide<+(1)-> % chktex 36
\begin{block}{Note}
We last saw this equation in section 1.1 where Thomas Malthus used it to estimate global population growth.
\end{block}
\end{frame}

\begin{frame}
\begin{block}{Solving the Growth and Decay Equation}
Let us start by rewriting the growth and decay equation in a familiar form:
\begin{equation*}
y^\prime - ky = 0
\end{equation*}\pause
We see that in fact it is a linear first-order homogeneous equation.\pause

\vspace{2mm}
We could use the theories discussed last section to solve this equation, but the easiest route is to use Separation of Variables.
\begin{equation*}
\begin{aligned}
\dfrac{dy}{y} &= k~dt \\ \pause
\ln\abs{y} &= kt + C \\ \pause
\abs{y} &= e^{kt+C} \pause = e^C e^{kt}\pause
\end{aligned}
\end{equation*}
Note that $e^C>0$ for all $C\in\R$. Thus, if we replace $e^C$ with an arbitrary constant $A$, which could be negative, we can dispose of the absolute value bars.
\begin{equation*}
y(t) = Ae^{kt},\quad A\in\R
\end{equation*}
\end{block}
\end{frame}

\begin{frame}[fragile]
\begin{block}{Growth and Decay}
For each $k$, the solution of the IVP
\begin{equation*}
\dfrac{dy}{dt} = ky,\quad y(0)=y_0
\end{equation*}
is given by
\begin{equation*}
y(t) = y_0 e^{kt}
\end{equation*}
Solution curves are \textbf{growth curves} for $k>0$ and \textbf{decay curves} for $k<0$.
\begin{center}
\begin{asy}
size(120);

real min_x=0, max_x=3;
real min_y=0, max_y=3;

pair start=(min_x,min_y);
pair end=(max_x,max_y);

//draw_grid_lines(min_x, max_x, min_y, max_y); 
	
real y0 = 1;
real growth(real t) {return y0 * exp(0.5 * t);}
real decay(real t) {return y0 * exp(-1 * t);}

draw(graph(growth, min_x, max_x),next_color()+1bp);
draw(graph(decay, min_x, max_x),next_color()+1bp);

limits(start,end,Crop);

dot((0,y0));
label("$y_0$", (-0.25,y0));

label("$k>0$", (2, 1.9));
label("(growth curve)", (2 ,1.6));
label("$k<0$", (2, 0.75)); 
label("(decay curve)", (2,0.45));

xaxis("$t$",YEquals(min_y),min_x,max_x,NoTicks(), EndArrow(4));
//xaxis(YEquals(max_y),min_x,max_x);
yaxis("$y$",XEquals(min_x),min_y,max_y,NoTicks(), EndArrow(4));
//yaxis(XEquals(max_x),min_y,max_y);
\end{asy}
\end{center}
\end{block}
\end{frame}

\begin{frame}[fragile]
\begin{example}
\textbf{Radiocarbon Dating.} Near Lascaux, France a cave was discovered containing multiple paintings on the walls, as well as the remains of a small fire. By chemical analysis it has been determined that the amount of Carbon-14 remaining in samples of the charcoal was 15\% of the amount such trees would contain when living. The \textbf{half-life} of Carbon-14 is approximately 5600 years. The quantity $Q$ of Carbon-14 in a charcoal sample satisfies the decay equation:
\begin{equation*}
Q^\prime = k Q
\end{equation*}
\begin{overprint}
\onslide<1-3>
What is the value of the decay constant $k$?

\vspace{2mm}
\visible<2->{Let us say that the initial amount of Carbon-14 is $Q(0)=Q_0$. So, after 5600 years, the half-life, there will be $\tfrac{1}{2}Q_0$ remaining. That is,
\begin{equation*}
Q(5600) = Q_0 e^{5600k} = \dfrac{1}{2}Q_0 \visible<3->{\Rightarrow k = -\dfrac{\ln[2]}{5600} \approx -0.00012378}
\end{equation*}}
\onslide<4-5>
What is $Q(t)$ at any time $t$?

\vspace{2mm}
\visible<5>{Since we know $k$, we can plug it into the general solution:
\begin{equation*}
Q(t) = Q_0 e^{-t\ln[2]/5600} \approx Q_0 e^{-0.00012378t} 
\end{equation*}}
\onslide<6-8>
What is the age of the charcoal, and hence the age of the paintings?

\vspace{2mm}
\visible<7->{We want to find the $t$-value that reduces $Q_0$ to 15\% of it's initial amount. This means we need to solve:
\begin{equation*}
Q_0 e^{-t\ln[2]/5600} = 0.15 Q_0 \visible<8->{\Rightarrow t = -\dfrac{5600\ln[0.15]}{\ln[2]} \approx 15,327~\text{years}}
\end{equation*}}
\onslide<9->
\begin{center}
\begin{asy}
size(200,100,IgnoreAspect);

real min_x=0, max_x=3*10^4;
real min_y=0, max_y=1;

pair start=(min_x,min_y);
pair end=(max_x,max_y);

//draw_grid_lines(min_x, max_x, min_y, max_y); 

real k = -0.00012378;
real Q(real t) {return exp(k*t);}

draw(graph(Q,min_x, max_x), black+1bp);

dot((15327, Q(15327)));
limits(start,end,Crop);

label("$Q=e^{"+string(k)+"t}$", (1.5*10^4,0.5));

xaxis("$t$",YEquals(min_y),min_x,max_x,RightTicks());
//xaxis(YEquals(max_y),min_x,max_x);
yaxis("$y$",XEquals(min_x),min_y,max_y,LeftTicks());
//yaxis(XEquals(max_x),min_y,max_y);
\end{asy}
\end{center}
\end{overprint}
\vspace{-32mm}
\end{example}
\end{frame}

\begin{frame}
\begin{block}{Compounded Interest}
If an initial amount of $A_0$ dollars is deposited at an \textbf{annual interest rate} of $r$, \textbf{compounded} $n$ times per year, the \textbf{future value} of the deposit at time $t$ is given by
\begin{equation*}
A(t)=A_0{\left(1+\dfrac{r}{n}\right)}^{nt}
\end{equation*}\pause
To earn the theoretical maximum interest, you would need the account to compounded at every instant of time. That is every hour, every minute, every second, and so on. We can represent this using the following limit:
\begin{equation*}
A(t) = \lim_{n\rightarrow\infty} A_0{\left(a+\dfrac{r}{n}\right)}^{nt}
\end{equation*}\pause
Using the calculus fact that $\lim_{n\rightarrow\infty}{\left(1+\tfrac{r}{n}\right)}^n=e^r$, we see that the limit reduces to:
\begin{equation*}
A(t) = A_0 e^{rt}
\end{equation*}\pause
We call this \textbf{continuously compounded} interest.
\end{block}
\end{frame}

\begin{frame}
\begin{block}{Continuously Compounded Interest}
If an initial amount of $A_0$ dollars is deposited at an annual interest rate of $r$, compounded continuously, the future value $A(t)$ of the deposit at time $t$ satisfies the initial-value problem
\begin{equation*}
\dfrac{dA}{dt} = rA,\quad A(0)=A_0
\end{equation*}
and is therefore given by
\begin{equation*}
A(t) = A_0 e^{rt}
\end{equation*}
\end{block}
\end{frame}

\begin{frame}
\begin{example}
Suppose you came across a time machine and decided to travel to London on 27 July 1694 and deposit \pounds20 (about \$25.78 USD) during the Bank of England's grand opening. Let us suppose you got the founder's day savings account with an annual interest rate of 8\% compounded continuously. How much money would be in the account on the bank's tricentennial in 1994?\pause

\vspace{2mm}
Since it's the tricentennial, we have $t=300$. This allows use to calculate the  future value:
\begin{equation*}
A(300) = 20e^{0.08(300)}\pause \approx 529,782,442,596.869
\end{equation*}\pause
So we would have the queenly sum of \pounds529,782,442,597 pounds (which is about \$682,836,581,952 USD).
\end{example}
\end{frame}

\begin{frame}
\begin{block}{The Savings Problem}
The more interesting type of account is one where in addition to interest being earned, regular deposits are made, such as with retirement accounts and pension funds.\pause

\vspace{2mm}
We now have the case where we have an annual interest rate of $r$ and a annual deposit amount of $a$ dollars. So, if the account starts with $A_0$ dollars, then it satisfies the initial-value problem:
\begin{equation*}
\dfrac{dA}{dt} = rA+a,\quad A(0)=A_0
\end{equation*}\pause

\vspace{-4mm}
It is left as an exercise to the student that we can use the methods from earlier in the chapter solve this nonhomogenous linear IVP, obtaining:
\begin{equation*}
A(t) = A_0 e^{rt} + \dfrac{a}{r}(e^{rt}-1)
\end{equation*}

\vspace{-1mm}
Here the first term represents the accumulation due to the starting deposit, and the second term the accumulation due to the subsequent deposits and the interest that they earn.
\end{block}
\end{frame}

\begin{frame}
\begin{example}
Ravi has just entered college at age 18 and has decided to improve his health and save money by quitting smoking. He figures he can save \$30 per week in this way. If he deposits the amount in an account paying 10\% annual interest compounded continuously how much will he have in the account when he retires at age 65?\pause

\vspace{1.5mm}
We start by noting that \$30 per week represents a rate of \$1,560 per year. \pause

\vspace{1.5mm}
The IVP describing the value of his account is:
\begin{equation*}
\dfrac{dA}{dt} = 0.10 A + 1560,\quad A(0)=0
\end{equation*}\pause
and the solution, by Euler-Lagrange, is
\begin{equation*}
A(t) = 15600(e^{0.1t}-1)
\end{equation*}\pause
After $65-18=47$ years, Ravi will have
\begin{equation*}
A(47) = 15600(e^{0.1(47)}-1) \approx \$1,699,575.89
\end{equation*}
\end{example}
\end{frame}
\end{document}
