\documentclass{beamer}
\usepackage[utf8]{inputenc}
\usepackage[english]{babel}
\usepackage[T1]{fontenc}
\usepackage[inline]{asymptote}
\usepackage{slide_helper}

\title[MATH 2250 - Section 1.3]{Separation of Variables: Quantitative Analysis}

\begin{asydef}
import graph;
import slopefield;
import fontsize;
defaultpen(fontsize(9pt));
ngraph=1000;

int pen_pos=-1;
pen[] pens={blue, red, heavycyan, heavymagenta, lightolive};
pens.cyclic=true;
pen next_color() {return pens[++pen_pos];}
void reset_color() {pen_pos = -1;}

DefaultHead.size=new real(pen p=currentpen) {return 2.5mm;};
\end{asydef}

\begin{document}
\begin{frame}
\titlepage
\end{frame}

\begin{frame}
\begin{block}{Separable Differential Equations}
A \textbf{separable} differential equation is one that can be written $y^\prime=f(t)g(y)$. Constant solutions $y=c$ can be found by solving $g(y)=0$.
\end{block}

\onslide<2->
\begin{example}
Let use consider
\begin{equation*}
\begin{aligned}
\dfrac{dy}{dt} &= 3t^2\cdot (1+y) \\
\visible<3->{\visible<4->{\int}\dfrac{dy}{1+y} &=\visible<4->{\int}3t^2dt \\}
\visible<5->{\ln\abs{1+y} &= t^3+c \\}
\visible<6->{\abs{1+y} &= e^c e^{t^3} \\}
\visible<7->{y &= -1\pm e^c e^{t^3} \\}
\visible<8->{y &= -1 + ke^{t^3}, \quad k\in\R}
\end{aligned}
\end{equation*}
\end{example}
\onslide<9->
\begin{block}{}
The method we just used is called \textbf{Separation of Variables}.
\end{block}
\end{frame}

\begin{frame}
\begin{block}{Why does it work?}

Suppose $G(y)$ and $F(t)$ are antiderivatives of $\dfrac{1}{g(y)}$ and $f(t)$, respectively.
\onslide<2->

\vspace{2mm}
Suppose also that $y=y(t)$ is a function defined \textbf{implicitly} by \begin{equation*}
G(y) = F(t)+c \visible<3->{\quad\Rightarrow\quad G(y(t))=F(t)+c}
\end{equation*}
on some appropriate $t$-interval for some real constant $c$.
\onslide<4->

\vspace{2mm}
Then, if $y$ is differentiable, we can differentiate both sides with respect to $t$.
\begin{equation*}
\begin{aligned}
G^\prime(y(t))y^\prime(t) &= F^\prime(t) \\
\visible<5->{\dfrac{y^\prime(t)}{g(y(t))} &= f(t) \\ }
\visible<6->{y^\prime(t) &= f(t)g(y(t)) \\ }
\end{aligned}
\end{equation*}

\vspace{-8mm}
\visible<7->{This has shown that $y$ is a solution to $y^\prime=f(t)g(y)$ and explains why the previous example works.}
\end{block}
\end{frame}

\begin{frame}
\begin{block}{Method of Separation of Variables}
\begin{description}
\item[\textbf{Step 1:}] Set $g(y)=0$ and solve to find any equilibria.
\item[\textbf{Step 2:}] Now, assume that $g(y)\neq 0$\@. Rewrite the equation in separated form:
\begin{equation*}
\dfrac{dy}{g(y)} = f(t)dt
\end{equation*}
\item[\textbf{Step 3:}] Integrate, if possible, each side:
\begin{equation*}
\int\dfrac{dy}{g(y)} = \int f(t)dt
\end{equation*}
(obtaining the implicit one-parameter family of solutions.)
\item[\textbf{Step 4:}] If possible, solve for $y$ in terms of $t$, getting the explicit solution $y=y(t)$
\item[\textbf{Step 5:}] If you have an IVP, use the initial condition to evaluate $c$\@.
\end{description}
\end{block}
\end{frame}

\begin{frame}
\begin{example}
Are the following differentiable equations are separable?
\begin{enumerate}
\item<1-> $\dfrac{dy}{dt}=-\dfrac{t}{y}
\visible<2->{\quad\Rightarrow\quad y~dy = -t~dt}$
\item<3-> $\dfrac{dy}{dt}=t^2y
\visible<4->{\quad\Rightarrow\quad \dfrac{dy}{y}= t^2dt}$
\item<5-> $\dfrac{dy}{dt}=y+1
\visible<6->{\quad\Rightarrow\quad \dfrac{dy}{y+1} = dt}$
\item<7-> $\dfrac{dy}{dt}=t+y
\visible<8->{\quad\Rightarrow\quad \text{(not separable)}}$
\end{enumerate}
\end{example}
\end{frame}

\begin{frame}[fragile]
\begin{example}
What are the solutions for the following separable differential equation?
\begin{equation*}
y^\prime=\dfrac{t^2}{1-y^2},\quad y\neq\pm 1
\end{equation*}
\begin{overprint}
\onslide<2-5> First we need to rewrite the equation in separated form.
\begin{equation*}
\left(1-y^2\right)dy=t^2~dt
\end{equation*}
\visible<3->{Next, we can integrate both sides to get:
\begin{equation*}
\begin{aligned}
y-\dfrac{y^3}{3} &= \dfrac{t^3}{3} + c \\
\visible<4->{-t^3+3y-y^3 &= k & \text{where $k=3c$}}
\end{aligned}
\end{equation*}}
\visible<5->{%

\vspace{-2mm}
Note that because of the restrictions, each solution curve in the direction field will be a piecewise combination of several functions. A particular solution of an IVP for this DE would only be \emph{one} of these.}
\onslide<6->
\begin{center}
\begin{asy}
size(174);

real min_x=-3, max_x=3;
real min_y=-3, max_y=3;

pair start=(min_x,min_y);
pair end=(max_x,max_y);

for(real gx=min_x+1; gx<=max_x-1; ++gx)
	draw((gx,min_y)--(gx,max_y),dotted+darkgray);
    
for(real gy=min_y+1; gy<=max_y-1; ++gy)
	draw((min_x,gy)--(max_x,gy),dotted+darkgray); 
	
real yp (real t, real y) { return t*t/(1-y*y); }

add(slopefield(yp, start, end, 20));

draw(curve((0,0), yp, (-1.25,-1.1), (1.26,1.1)), next_color()+1bp);
draw(curve((0,1.74), yp, (-3,1), (1.25,3)), next_color()+1bp);
draw(curve((0,-1.74), yp, (-3,-3), (3,-1.1)), next_color()+1bp);

dot((1.26,1),UnFill());
dot((-1.26,-1),UnFill());

label("$k=0$", (-2,1.5), UnFill());

limits(start,end,Crop);

xaxis("$t$",YEquals(min_y),min_x,max_x,LeftTicks());
xaxis(YEquals(max_y),min_x,max_x);
yaxis("$y$",XEquals(min_x),min_y,max_y,LeftTicks());
yaxis(XEquals(max_x),min_y,max_y);
\end{asy}
\end{center}
\end{overprint}
\vspace{-54mm}
\end{example}
\end{frame}

\begin{frame}[fragile]
\begin{example}
Solve the following IVP\@.
\begin{equation*}
\dfrac{dy}{dt} = \dfrac{3t^2+1}{1+2y},\quad y(0)=1
\end{equation*}

\begin{overprint}
\onslide<2-5>
This separable equation has no equilibrium solutions. Moreover, $y\neq-\tfrac{1}{2}$. 
\visible<3->{%

\vspace{2mm}
To find any other solutions we need to separate the DE\@.
\begin{equation*}
\begin{aligned}
(1+2y)dy &= (3t^2+1)dt \\
\visible<4->{\int(1+2y)dy &= \int(3t^2+1)dt \\ }
\visible<5->{y+y^2 &= t^3+t+c \\ }
\end{aligned}
\end{equation*}}
\onslide<6-9>
To satisfy the initial conditions we must have $y=1$ when $t=0$.
\begin{equation*}
1+1^2 = 0^3+0+c \visible<7->{\quad\rightarrow\quad c=2}
\end{equation*}
\visible<8->{Which give the solution:
\begin{equation*}
\begin{aligned}
y+y^2 &= t^3+t+2 \\
\visible<9->{y &= \dfrac{1}{2}\left(-1\pm\sqrt{4t^3+4t+9}\right)}
\end{aligned}
\end{equation*}}
\visible<9->{Again, the solution curve is made up of multiple parts.}
\onslide<10->
\begin{center}
\begin{asy}
size(174);

real min_x=-3, max_x=3;
real min_y=-3, max_y=3;

pair start=(min_x,min_y);
pair end=(max_x,max_y);

for(real gx=min_x+1; gx<=max_x-1; ++gx)
	draw((gx,min_y)--(gx,max_y),dotted+darkgray);
    
for(real gy=min_y+1; gy<=max_y-1; ++gy)
	draw((min_x,gy)--(max_x,gy),dotted+darkgray); 
	
real yp (real t, real y) { return (3*t*t+1)/(1+2*y); }

add(slopefield(yp, start, end, 20));

draw(curve((0,1), yp, (-1.06,-0.5), end), next_color()+1bp);
draw(curve((0,-2), yp, start, (max_x,-0.51)), next_color()+1bp);

dot((-1.06,-0.5),UnFill());

//label("$k=0$", (-2,1.5), UnFill());

limits(start,end,Crop);

xaxis("$t$",YEquals(min_y),min_x,max_x,LeftTicks());
xaxis(YEquals(max_y),min_x,max_x);
yaxis("$y$",XEquals(min_x),min_y,max_y,LeftTicks());
yaxis(XEquals(max_x),min_y,max_y);
\end{asy}
\end{center}
\end{overprint}
\vspace{-43mm}
\end{example}
\end{frame}

\begin{frame}[fragile]
\begin{example}
Solve the IVP\@.
\begin{equation*}
y^\prime = -\dfrac{t}{y},\quad y(0)=1
\end{equation*}
\begin{overprint}
\onslide<2-5>
There are no equilibrium solutions and $y\neq 0$. 

\vspace{2mm}
We can separate the DE\@.
\begin{equation*}
\begin{aligned}
y~dy &= -t~dt \\
\visible<3->{\int y~dy &= \int -t~dt \\}
\visible<4->{\dfrac{y^2}{2} &= -\dfrac{t^2}{2} + c}
\end{aligned}
\end{equation*}
\visible<5->{We can then use the initial condition to find $c$.
\begin{equation*}
\dfrac{1}{2} = -\dfrac{0^2}{2} + c 
\quad\Rightarrow\quad
c = \dfrac{1}{2}
\end{equation*}}
\onslide<6-8>
We have two explicit solutions:
\begin{equation*}
\begin{aligned}
\dfrac{y^2}{2} &= -\dfrac{t^2}{2} + \dfrac{1}{2} \\
\visible<7->{y^2 &= -t^2 + 1 \\}
\visible<8->{y &= \pm\sqrt{1-t^2}}
\end{aligned}
\end{equation*}
\onslide<9->
\begin{center}
\begin{asy}
size(174);

real min_x=-2, max_x=2;
real min_y=-2, max_y=2;

pair start=(min_x,min_y);
pair end=(max_x,max_y);

for(real gx=min_x+1; gx<=max_x-1; ++gx)
	draw((gx,min_y)--(gx,max_y),dotted+darkgray);
    
for(real gy=min_y+1; gy<=max_y-1; ++gy)
	draw((min_x,gy)--(max_x,gy),dotted+darkgray); 
	
real yp (real t, real y) { return -t/y; }

add(slopefield(yp, (min_x, min_y), (max_x,-0.01), 20));
add(slopefield(yp, (min_x, 0.01), (max_x, max_x), 20));

draw(curve((0,1), yp, (-1,0.01), (1,1)), next_color()+1bp);
draw(curve((0,-1), yp, (-1,-1), (1,-0.01)), next_color()+1bp);

dot((-1,0),UnFill());
dot((+1,0),UnFill());

//label("$k=0$", (-2,1.5), UnFill());

limits(start,end,Crop);

xaxis("$t$",YEquals(min_y),min_x,max_x,LeftTicks());
xaxis(YEquals(max_y),min_x,max_x);
yaxis("$y$",XEquals(min_x),min_y,max_y,LeftTicks());
yaxis(XEquals(max_x),min_y,max_y);
\end{asy}
\end{center}
\end{overprint}
\vspace{-55mm}
\end{example}
\end{frame}
\end{document}
