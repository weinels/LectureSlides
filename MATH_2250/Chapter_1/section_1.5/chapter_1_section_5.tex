\documentclass{beamer}
\usepackage[utf8]{inputenc}
\usepackage[english]{babel}
\usepackage[T1]{fontenc}
\usepackage[inline]{asymptote}
\usepackage{slide_helper}
\usepackage{asy_helper}

\title[MATH 2250 - Section 1.5]{Picard's Theorem: Theoretical Analysis}

\begin{document}
\begin{frame}
\titlepage
\end{frame}

\begin{frame}
\begin{block}{Why Study Theory?}
When considering a mathematical model two important questions are often considered:
\begin{itemize}
\item Does a solution actually exist? (Existence Theorems.)
\item Is that solution unique? (Uniqueness Theorems.)
\end{itemize}
The theorems we will study can answer both of these questions, without actually needing to find any solutions.
\end{block}\pause
\begin{block}{Note}
Questions of existence and uniqueness are not limited to the study of differential equations. All branches of mathematics ask these questions, as well as many other fields such as philosophy and physics.
\end{block}
\end{frame}

\begin{frame}
\begin{example}
Consider the polynomial
\begin{equation*}
f(x)=x^5+x-1
\end{equation*}\pause
We know, from the Intermediate Value Theorem, that on the interval $\interval{\closed{0}}{\closed{1}}$ at least one root of $f(x)$ exists. Thus, the IVT gives us existence.\pause

Further, given that
\begin{equation*}
f^\prime (x) = 5x^4+1
\end{equation*}
we know, from Rolle's theorem, that there can't exist more than one root in $\interval{\closed{0}}{\closed{1}}$. Thus, Rolle gives us uniqueness.
\end{example}
\end{frame}

\begin{frame}
\begin{block}{What about Differential Equations?}
Many of the differential equations we encounter have solutions. What we will be concerned with solving are Initial Value Problems.\pause

\vspace{2mm}
To this end we will need to check that a family of solutions actually exist, then the we can focus our efforts on finding the particular unique member that solves our IVP\@.\pause

\vspace{2mm}
It will often be useful to look at the direction field first, as some pretty obvious information may be gleamed.
\end{block}
\end{frame}

\begin{frame}[fragile]
\begin{example}
What can we determine from the direction field for $y^\prime = y$?
\begin{overprint}
\onslide<1>
\begin{center}
\begin{asy}
size(185);

real min_x=-3, max_x=3;
real min_y=-3, max_y=3;

pair start=(min_x,min_y);
pair end=(max_x,max_y);

draw_grid_lines(min_x, max_x, min_y, max_y); 
	
real yp (real t, real y) { return y; }

add(slopefield(yp, start, end, 20));

//draw(curve((0,-2.5), yp, start, end), next_color()+1bp);
//draw(curve((0,-1), yp, start, end), next_color()+1bp);
//draw(curve((0,0), yp, start, end), next_color()+1bp);
//draw(curve((0,1), yp, start, end), next_color()+1bp);
//draw(curve((0,2.5), yp, start, end), next_color()+1bp);

limits(start,end,Crop);

xaxis("$t$",YEquals(min_y),min_x,max_x,LeftTicks());
xaxis(YEquals(max_y),min_x,max_x);
yaxis("$y$",XEquals(min_x),min_y,max_y,LeftTicks());
yaxis(XEquals(max_x),min_y,max_y);
\end{asy}
\end{center}
\onslide<2>
\begin{center}
\begin{asy}
size(185);

real min_x=-3, max_x=3;
real min_y=-3, max_y=3;

pair start=(min_x,min_y);
pair end=(max_x,max_y);

draw_grid_lines(min_x, max_x, min_y, max_y); 
	
real yp (real t, real y) { return y; }

add(slopefield(yp, start, end, 20));

draw(curve((0,-2.5), yp, start, end), next_color()+1bp);
draw(curve((0,-1), yp, start, end), next_color()+1bp);
draw(curve((0,0), yp, start, end), next_color()+1bp);
draw(curve((0,1), yp, start, end), next_color()+1bp);
draw(curve((0,2.5), yp, start, end), next_color()+1bp);

limits(start,end,Crop);

xaxis("$t$",YEquals(min_y),min_x,max_x,LeftTicks());
xaxis(YEquals(max_y),min_x,max_x);
yaxis("$y$",XEquals(min_x),min_y,max_y,LeftTicks());
yaxis(XEquals(max_x),min_y,max_y);
\end{asy}
\end{center}

\vspace{-3mm}
The direction field suggests that a unique solution exists for every point, and these solutions are defined for all $t$-values.
\end{overprint}
\end{example}
\end{frame}

\begin{frame}[fragile]
\begin{example}
What can we determine from the direction field for $y^\prime = \sqrt{y}$?
\begin{overprint}
\onslide<1>
\begin{center}
\begin{asy}
size(185);

real min_x=-3, max_x=3;
real min_y=-3, max_y=3;

pair start=(min_x,min_y);
pair end=(max_x,max_y);

draw_grid_lines(min_x, max_x, min_y, max_y); 
	
real yp (real t, real y) { return sqrt(y); }

add(slopefield(yp, (min_x,0), end, 20));

//draw(curve((0,1), yp, (-1.8,0), end), next_color()+1bp);
//draw(curve((0,2), yp, (-2.55,0), end), next_color()+1bp);
//draw(curve((0,3), yp, start, end), next_color()+1bp);
//draw(curve((1,1), yp, (-0.7,0), end), next_color()+1bp);
//draw(curve((0,0), yp, start, end), next_color()+1bp);


//label("?", (-1.8,-0.2));
//label("?", (-2.55,-0.2));
//label("?", (-2.9,-0.2));
//label("?", (-0.7,-0.2));

limits(start,end,Crop);

xaxis("$t$",YEquals(min_y),min_x,max_x,LeftTicks());
xaxis(YEquals(max_y),min_x,max_x);
yaxis("$y$",XEquals(min_x),min_y,max_y,LeftTicks());
yaxis(XEquals(max_x),min_y,max_y);
\end{asy}
\end{center}
\onslide<2>
\begin{center}
\begin{asy}
size(185);

real min_x=-3, max_x=3;
real min_y=-3, max_y=3;

pair start=(min_x,min_y);
pair end=(max_x,max_y);

draw_grid_lines(min_x, max_x, min_y, max_y); 
	
real yp (real t, real y) { return sqrt(y); }

add(slopefield(yp, (min_x,0), end, 20));

draw(curve((0,1), yp, (-1.8,0), end), next_color()+1bp);
draw(curve((0,2), yp, (-2.55,0), end), next_color()+1bp);
draw(curve((0,3), yp, start, end), next_color()+1bp);
draw(curve((1,1), yp, (-0.7,0), end), next_color()+1bp);
//draw(curve((0,0), yp, start, end), next_color()+1bp);


label("?", (-1.8,-0.2));
label("?", (-2.55,-0.2));
//label("?", (-2.95,-0.2));
label("?", (-0.7,-0.2));

limits(start,end,Crop);

xaxis("$t$",YEquals(min_y),min_x,max_x,LeftTicks());
xaxis(YEquals(max_y),min_x,max_x);
yaxis("$y$",XEquals(min_x),min_y,max_y,LeftTicks());
yaxis(XEquals(max_x),min_y,max_y);
\end{asy}
\end{center}

\vspace{-3mm}
The direction field suggests that a unique solution exists for every point in the upper half place, and that initial points on the $t$-axis need a closer look.
\end{overprint}
\end{example}
\end{frame}

\begin{frame}[fragile]
\begin{example}
Let us look at the direction field for $y^\prime = \sqrt{y}+\sqrt{-y}+\dfrac{1}{y}$
\begin{center}
\begin{asy}
size(180);

real min_x=-3, max_x=3;
real min_y=-3, max_y=3;

pair start=(min_x,min_y);
pair end=(max_x,max_y);

draw_grid_lines(min_x, max_x, min_y, max_y); 
	
real yp (real t, real y) { return sqrt(y) + sqrt(-y) + 1/y; }

//add(slopefield(yp, start, end, 20));

//draw(curve((0,-2.5), yp, start, end), next_color()+1bp);
//draw(curve((0,-1), yp, start, end), next_color()+1bp);
//draw(curve((0,0), yp, start, end), next_color()+1bp);
//draw(curve((0,1), yp, start, end), next_color()+1bp);
//draw(curve((0,2.5), yp, start, end), next_color()+1bp);

limits(start,end,Crop);

xaxis("$t$",YEquals(min_y),min_x,max_x,LeftTicks());
xaxis(YEquals(max_y),min_x,max_x);
yaxis("$y$",XEquals(min_x),min_y,max_y,LeftTicks());
yaxis(XEquals(max_x),min_y,max_y);

\end{asy}
\end{center}\pause

\vspace{-3mm}
There is no $y$-value for which this DE is defined, hence no solutions exist.
\end{example}
\end{frame}

\begin{frame}
\begin{example}
Let us look a bit closer at:
\begin{equation*}
y^\prime=\sqrt{y}
,\quad
y(0)=y_0
\end{equation*}\pause
We know that for $y_0<0$, no solutions exist, and for $y_0>0$ things look unique. But what about $y_0=0$?\pause

\vspace{2mm}
Obviously $y(t)=0$ is a solution, but is it the only solution?\pause

\vspace{2mm}
Separation of Variables gives us:

\vspace{-3mm}
\begin{equation*}
y(t)=
\begin{cases}
0 & \text{if}~t\leq 0 \\
\tfrac{1}{4}t^2 & \text{if}~t\geq 0
\end{cases}
\end{equation*}
Thus, $y(t)=0$ is not the only solution that passes through the origin!\pause

\vspace{2mm}
In fact, the situation is even worse. For any $t$-value $c>0$, we have the following solution.

\vspace{-4mm}
\begin{equation*}
y(t)=
\begin{cases}
0 & \text{if}~t\leq c \\
\tfrac{1}{4}{(t-c)}^2 & \text{if}~t\geq c
\end{cases}
\end{equation*}
\end{example}
\end{frame}

\begin{frame}
\begin{block}{}
We can now introduce the theorem we will mainly use, thanks to the French mathematician Charles \'{E}mile Picard (1856 - 1941).
\end{block}

\begin{block}{Picard's Existence and Uniqueness Theorem}
Suppose that the function $f(t,y)$ is continuous on the region
\begin{equation*}
R=\setbuilder{(t,y)}{a<t<b, c<y<d}
\end{equation*}
and $(t_0, y_0)\in\R$. 

\vspace{2mm}
Then there exists a positive number $h$ such that the initial-value problem
\begin{equation*}
y^\prime = f(t,y),\quad y(t_0)=y_0
\end{equation*}
has a solution for $t$ in the interval $\interval{\open{t_0-h}}{\open{t_0+h}}$.

\vspace{2mm}
More over, if $f_y(t,y)$ is also continuous in $R$, that solution is unique.
\end{block}
\end{frame}

\begin{frame}[fragile]
\begin{example}
For the simple initial-value problem
\begin{equation*}
y^\prime = 1 + y^2,\quad y(0)=0
\end{equation*}
\begin{overprint}
\onslide<1>
\onslide<2-4>
we have 
\begin{equation*}
f(t,y) = 1+y^2
\quad\text{and}\quad
f_y(t,y) = 2y
\end{equation*}
\visible<3->{Both of these are polynomials, which are continuous on any open rectangle. 

\vspace{2mm}
This means that Picard's theorem guarantees a unique solution exists for\\ the initial point $(t_0, y_0)=(0,0)$ on \emph{some} interval (though we don't know\\ how large) around $t=0$.}

\visible<4->{\vspace{2mm}
We could, if so desired, use Separation of Variables to solve this IVP\@.\\ Doing so, we find that the solution is $y(t)=\tan[t]$, which is defined on\\ the interval $\interval{\open{-\tfrac{\pi}{2}}}{\open{\tfrac{\pi}{2}}}$.}
\onslide<5->
\begin{center}
\begin{asy}
size(170);

real min_x=-6, max_x=6;
real min_y=-6, max_y=6;

pair start=(min_x,min_y);
pair end=(max_x,max_y);

draw_grid_lines(min_x, max_x, min_y, max_y); 
	
real yp (real t, real y) { return 1+y*y; }

add(slopefield(yp, start, end, 20));

draw(curve((0,0), yp, start, end), next_color()+1bp);
draw(curve((-3.9,0), yp, start, end), next_color()+1bp);
draw(curve((-1.5,0), yp, start, end), next_color()+1bp);
draw(curve((2.1,0), yp, start, end), next_color()+1bp);
draw(curve((4.3,0), yp, start, end), next_color()+1bp);

dot((0,0));
limits(start,end,Crop);

xaxis("$t$",YEquals(min_y),min_x,max_x,LeftTicks());
xaxis(YEquals(max_y),min_x,max_x);
yaxis("$y$",XEquals(min_x),min_y,max_y,LeftTicks());
yaxis(XEquals(max_x),min_y,max_y);
\end{asy}
\end{center}
\end{overprint}
\vspace{-45mm}
\end{example}
\end{frame}

\begin{frame}[fragile]
\begin{example}
For the not-so-simple initial-value problem
\begin{equation*}
y^\prime = t^2 + ty^2,\quad y(-4)=-2
\end{equation*}
\begin{overprint}
\onslide<1>
\onslide<2-3>
we have 
\begin{equation*}
f(t,y) = t^2+ty^2
\quad\text{and}\quad
f_y(t,y) = 2ty
\end{equation*}
\visible<3->{Both of these are polynomials, which are continuous on any open rectangle. 

\vspace{2mm}
This means that Picard's theorem guarantees a unique solution exists for\\ the initial point $(t_0, y_0)=(-4,-2)$ on \emph{some} interval (though again we don't\\ know how large) around $t=0$.}
\onslide<4->
\begin{center}
\begin{asy}
size(180);

real min_x=-6, max_x=6;
real min_y=-6, max_y=6;

pair start=(min_x,min_y);
pair end=(max_x,max_y);

draw_grid_lines(min_x, max_x, min_y, max_y); 
	
real yp (real t, real y) { return t*t+t*y*y; }

add(slopefield(yp, start, end, 20));

//draw(curve((-3.1,-2), yp, start, end), next_color()+1bp);
//draw(curve((-4.6,-2), yp, start, end), next_color()+1bp);
//draw(curve((-4,-2.1), yp, start, end), next_color()+1bp);
//draw(curve((-4,-1.8), yp, start, end), next_color()+1bp);

for(real x = min_x; x <= max_x; x = x + 1.1)
{
	for(real y = min_y; y <= max_y; y = y + 2.1)
	{
		draw(curve((x,y), yp, start, end), grey);
	}
}

draw(curve((-4,-2), yp, start, end), next_color()+1bp);

dot((-4,-2));
limits(start,end,Crop);

xaxis("$t$",YEquals(min_y),min_x,max_x,LeftTicks());
xaxis(YEquals(max_y),min_x,max_x);
yaxis("$y$",XEquals(min_x),min_y,max_y,LeftTicks());
yaxis(XEquals(max_x),min_y,max_y);
\end{asy}
\end{center}
\end{overprint}
\vspace{-26mm}
\end{example}
\end{frame}
\end{document}
