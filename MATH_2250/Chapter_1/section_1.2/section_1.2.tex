\documentclass{beamer}
\usepackage[utf8]{inputenc}
\usepackage[english]{babel}
\usepackage[T1]{fontenc}
\usepackage[inline]{asymptote}
\usepackage{slide_helper}

\begin{asydef}
import graph;
import slopefield;
import fontsize;
defaultpen(fontsize(9pt));
ngraph=1000;

int pen_pos=-1;
pen[] pens={blue, red, heavycyan, heavymagenta, lightolive};
pens.cyclic=true;
pen next_color() {return pens[++pen_pos];}
void reset_color() {pen_pos = -1;}

DefaultHead.size=new real(pen p=currentpen) {return 2.5mm;};
\end{asydef}

\title[Section 1.2]{Solution and Direction Fields: Qualitative Analysis}

\begin{document}
\begin{frame}
\titlepage
\end{frame}

\begin{frame}
\begin{block}{}
A \textbf{differential equation} (or DE) is an equation containing derivatives. The \textbf{order} of the equation refers to the highest-order derivative that occurs.\pause

\vspace{2mm}
In this chapter we will focus on DEs that can be written as:
\begin{equation*}
\dfrac{dy}{dt}=f(t,y)
\quad\text{or}\quad
y^{\prime}=f(t,y)
\end{equation*}
Where the dependent variable $y$ is an unknown function, the \textbf{solution}.
\end{block}\pause

\begin{block}{Note}
There may be more than one solution for a given differential equation.
\end{block}\pause

\begin{block}{Analytic Definition of a Solution}
Analytically, $y(t)$ is a \textbf{solution} of a differential equation if substituting $y(t)$ for $y$ reduced the equation to an identity:
\begin{equation*}
y^{\prime}(t)=f(t, y(t))
\end{equation*}

\vspace{-3mm}
on an appropriate domain for $t$.
\end{block}
\end{frame}

\begin{frame}[fragile]
\begin{example}
Verify that $y(t)$ is a solution to the DE\@.
\begin{equation*}
y^{\prime}(t)=2y,\quad y(t)=e^{2t}
\end{equation*}
\begin{overprint}
\onslide<1>
\onslide<2-6>
Substituting into the DE gives:
\begin{equation*}
\begin{aligned}
y^{\prime}(t) &= \dfrac{d}{dt}e^{2t} \\
\visible<3->{&=2e^{2t} \\}
\visible<4->{&=2y(t) \\}
\visible<5->{&=f(t, y(t))}
\end{aligned}
\end{equation*}
\visible<6->{%
Similarly, we could show that
\begin{equation*}
y(t)=2e^{2t}
\quad\text{and}\quad
y(t)=\dfrac{-3}{2}e^{2t}
\end{equation*}
are also solutions. In fact, any constant multiple of $e^{2t}$ is a solution.}
\onslide<7->
\begin{center}
\begin{asy}
size(180, IgnoreAspect);

real min_x=-2, max_x=2;
real min_y=-15, max_y=15;

pair start=(min_x,min_y);
pair end=(max_x,max_y);

real y_1(real t) {return exp(2t);}
real y_2(real t) {return 2*exp(2t);}
real y_3(real t) {return (-3/2)*exp(2t);}

draw(graph(y_1, min_x, max_x), next_color() + 1bp);
draw(graph(y_2, min_x, max_x), next_color() + 1bp);
draw(graph(y_3, min_x, max_x), next_color() + 1bp);

reset_color();

label("$y=e^{2t}$", (1.3,5), next_color());
label("$y=2 e^{2t}$", (0.45,12), next_color());
label("$y=\dfrac{3}{2} e^{2t}$", (1.5,-8), next_color());

limits(start,end,Crop);

xaxis("$t$",YEquals(0),Ticks(NoZero));
yaxis("$y$",XEquals(0),Ticks(NoZero),autorotate=false);
\end{asy}
\end{center}
\end{overprint}
\vspace{-58mm}
\end{example}
\end{frame}

\end{document}
