\documentclass{beamer}

\usepackage[english]{babel}
\usepackage[utf8x]{inputenc}
\usepackage[inline]{asymptote}
\usepackage{slide_helper}
\usepackage{tabu}

\usepackage{tikz}
\usetikzlibrary{arrows.meta}

\begin{asydef}
import graph;
import slopefield;
import fontsize;
defaultpen(fontsize(9pt));
ngraph=1000;

int pen_pos=-1;
pen[] pens={blue, red, heavycyan, heavymagenta, lightolive};
pens.cyclic=true;
pen next_color() {return pens[++pen_pos];}

DefaultHead.size=new real(pen p=currentpen) {return 2.5mm;};
\end{asydef}

\title[MATH 2250 - Section 8.2]{Solving DEs and IVPs with Laplace Transforms}

\begin{document}

\begin{frame}
  \titlepage
\end{frame}

\begin{frame}
\begin{example}
Consider the second-order IVP\@.
\begin{equation*}
a y^{\prime\prime}+b y^{\prime}+c y=f(t)
\quad
y(0)=y_0,~y^{\prime}(0)=y^\prime_0
\end{equation*}
\begin{overprint}
\onslide<1>
\onslide<2-5>
The Laplace transform of this DE is
\begin{equation*}
a\laplace\{y^{\prime\prime}\}+b\laplace\{y^\prime\}+c\laplace\{y\}=\laplace\{f\}
\end{equation*}
\visible<3->{If we assume that both $f$ and $f^\prime$ have Laplace transforms, then we get
\begin{equation*}
\laplace\{f^\prime(t)\}
\visible<4->{=\int_0^\infty e^{-st} f^\prime(t) dt}
\visible<5->{=\lim_{b\rightarrow\infty} \int_0^b e^{-st} f^\prime(t) dt}
\end{equation*}}
\onslide<6-8>
Integrating by parts gives
\begin{equation*}
\begin{aligned}
\int_0^b \underbracetwoitems{e^{-st}}{u}{f^\prime(t) dt}{dv}
\visible<7->{&={\left[\underbracetwoitems{e^{-st}}{u}{f(t)}{v}\right]}^b_0
-\int_0^b\underbracetwoitems{f(t)}{v}{\left(-s e^{-st} dt\right)}{du}\\}
\visible<8->{&=e^{-sb} f(b) - f(0) + s\int_0^b e^{-st} f(t) dt}
\end{aligned}
\end{equation*}
\onslide<9->
Taking the limit $b \rightarrow \infty$, we get
\begin{equation*}
\begin{aligned}
\laplace\{f^\prime (t)\} &= 
\visible<10->{\lim_{b\rightarrow\infty} \left(e^{-sb} f(b) - f(0) + s\int_0^b e^{-st} f(t) dt\right)\\}
\visible<11->{&= \lim_{b\rightarrow\infty} \left(s\int_0^b e^{-st} f(t) dt-f(0)\right)\\}
\visible<12->{&=s\laplace\{f(t)\}-f(0)}
\end{aligned}
\end{equation*}
\visible<13->{We can easily use this result to calculate
\begin{equation*}
\begin{aligned}
\laplace\{f^{\prime\prime}\}(t)=
\visible<14->{s\laplace\{f^\prime (t)\}-f^\prime (0)}
\visible<15->{&=s\left(s\laplace\{f(t)\}-f(0)\right)-f^\prime (0)\\}
\visible<16->{&=s^2\laplace\{f(t)\} - sf(0) - f^\prime (0)}
\end{aligned}
\end{equation*}}
%This process can be repeated to determine the Laplace transform for any derivative of $f$.
\end{overprint}
\end{example}
\end{frame}

\begin{frame}
\begin{block}{Derivative Theorem for Laplace Transforms}
If $f, f^{\prime},\ldots,f^{(n-1)}$ are continuous on $\interval{\closed{0}}{\open{\infty}}$, $f^{(n)}$ is piecewise continuous on $\interval{\closed{0}}{\open{\infty}}$, and $f,f^\prime,\ldots,f^{(n)}$ are of exponential order $\alpha$, then for $s>a$, and $n=1,2,\ldots$

\vspace{-2mm}
\begin{equation*}
\laplace\{f^{(n)}\}=s^n\laplace\{f\}-s^{n-1}f(0)-s^{n-2}f^\prime(0)-\cdots-f^{(n-1)}(0)
\end{equation*}

\vspace{-1mm}
In particular

\vspace{-4mm}
\begin{equation*}
\begin{aligned}
\laplace\{f^{\prime} (t)\} &= s\laplace\{f(t)\}-f(0)\\
\laplace\{f^{\prime\prime} (t)\} &=s^2\laplace\{f(t)\} - sf(0) - f^\prime (0)\\
\laplace\{f^{\prime\prime\prime} (t)\} &=s^3\laplace\{f(t)\} - s^2 f(0) - s f^\prime(0) - f^{\prime\prime} (0)
\end{aligned}
\end{equation*}
\end{block}\pause
\begin{block}{Strategy to Solve DEs with Laplace Transforms}
\begin{enumerate}[<+- | alert@+>]
\item Using the Laplace transform, transform the IVP with unknown function $y(t)$ into an algebraic problem with unknown function $Y(s)$.
\item Solve the algebraic problem for $Y(s)$.
\item Manipulating $Y(s)$ algebraically if necessary, use the inverse Laplace transform to transform $Y(s)$ into the IVP solution $y(t)$.
\end{enumerate}
\end{block}
\end{frame}

\begin{frame}
\begin{example}
Consider
\begin{equation*}
y^{\prime\prime}-2y^\prime-3y=0
\quad\text{where}\quad
y(0)=2,~y^\prime(0)=-10
\end{equation*}\pause

\vspace{-7mm}
Linearity gives us
\begin{equation*}
\laplace\{y^{\prime\prime}\}-2\laplace\{y^\prime\}-3\laplace\{y\}=0
\end{equation*}\pause
Next, we need to calculate the Laplace transforms of $y^{\prime\prime}$ and $y^\prime$.
\begin{equation*}
\begin{aligned}
\laplace\{y^{\prime\prime}\}&=s^2 \laplace\{y\}-s y(0)-y^\prime(0)=s^2 Y(s) -2s +10\\
\laplace\{y^\prime\}&=s\laplace\{y\}-y(0)=s Y(s)-2
\end{aligned}
\end{equation*}\pause
Substituting into the transformed DE gives an equations we can solve.
\begin{equation*}
\begin{aligned}
0&=\left(s^2 Y(s) -2s +10\right) -2\left(s Y(s)-2\right)-3Y(s)\\\pause
Y(s)&=\dfrac{2s-14}{s^2-2s-3}\pause
=\dfrac{2s-14}{(s+1)(s-3)}\pause
=\dfrac{4}{s+1}-\dfrac{2}{s-3}
\end{aligned}
\end{equation*}\pause

\vspace{-5.5mm}
Which means
\begin{equation*}
y(t)=\laplace^{-1}\{Y(s)\}= 4e^{-t}-2e^{3t}
\end{equation*}
\end{example}
\end{frame}

\begin{frame}
\begin{example}
Consider

\vspace{-5mm}
\begin{equation*}
y^{\prime\prime}+3y^\prime+2y=e^{-3t}
\quad\text{where}\quad
y(0)=0,~y^\prime(0)=1
\end{equation*}\pause

\vspace{-5mm}
Linearity gives us
\begin{equation*}
\laplace\{y^{\prime\prime}\}+3\laplace\{y^\prime\}+2\laplace\{y\}=\laplace\{e^{-3t}\}
\end{equation*}\pause
Next, applying the derivative theorem and and solving for $Y(s)$ gives
\begin{equation*}
\begin{aligned}
\laplace\{e^{-3t}\} &= s^2\laplace\{y\}-sy(0)-y^\prime(0)+3\left(s\laplace\{y\}-y(0)\right) + 2\laplace\{y\}\\\pause
\dfrac{1}{s+3}&=s^2 Y(s)-1+3s Y(s) +2Y(s)\\\pause
 1 + \dfrac{s}{s+3} &= (s^2+3s+2) Y(s)\\\pause
Y(s)&=\dfrac{s+4}{(s^2+3s+2)(s+3)}\pause
=\dfrac{\tfrac{1}{2}}{s+3} - \dfrac{2}{s+2} +\dfrac{\tfrac{3}{2}}{s+1}
\end{aligned}
\end{equation*}\pause

\vspace{-5.5mm}
Which means

\vspace{-5mm}
\begin{equation*}
y(t)=\laplace^{-1}\{Y(s)\}=\dfrac{1}{2}e^{-3t}-2e^{-2t}+\dfrac{3}{2}e^{-t}
\end{equation*}
\end{example}
\end{frame}

\begin{frame}
\begin{example}
Consider

\vspace{-5mm}
\begin{equation*}
y^{\prime\prime}+4y=\sin[t]
\quad\text{where}\quad
y(0)=0,~y^\prime(0)=1
\end{equation*}\pause
Applying the derivative theorem and solving gives
\begin{equation*}
\begin{aligned}
s^2\laplace\{y\}-sy(0)-y^\prime(0)+4\laplace\{y\} &= \laplace\{\sin[t]\}\\\pause
s^2 Y(s)-1+4 Y(s) &= \dfrac{1}{s^2+1}\\\pause
(s^2+1) Y(s) &= \dfrac{s^2+2}{s^2+1}\\\pause
Y(s) &= \dfrac{s^2+2}{(s^2+1)(s^2+4)}\\\pause
&=\dfrac{\tfrac{1}{3}}{s^2+1}+\dfrac{\tfrac{2}{3}}{s^2+4}
\end{aligned}
\end{equation*}\pause

\vspace{-8mm}
Thus, the solution is
\begin{equation*}
y(t)=\dfrac{1}{3}\sin[t]+\dfrac{1}{3}\sin[2t]
\end{equation*}
\end{example}
\end{frame}

\begin{frame}
\begin{example}
Consider
\begin{equation*}
y^{\prime\prime\prime}+y^\prime=e^t
\quad\text{where}\quad
y(0)=0,~y^\prime(0)=0,~y^{\prime\prime}(0)=0
\end{equation*}\pause
Applying the derivative theorem gives us
\begin{equation*}
\begin{aligned}
s^3\laplace\{y\}-s^2\cdot 0-s\cdot 0+s\laplace\{y\}-0&=\laplace\{e^t\}\\\pause
(s^3+s) Y(s) &= \dfrac{1}{s-1}
\end{aligned}
\end{equation*}\pause

\vspace{-8mm}
Then,
\begin{equation*}
\begin{aligned}
Y(s)=\dfrac{1}{(s-1)(s^3+s)}\pause
&=-\dfrac{1}{s}+\dfrac{\tfrac{1}{2}}{s-1}+\dfrac{\tfrac{1}{2}s-\tfrac{1}{2}}{s^2+1}\\\pause
&=-\dfrac{1}{s}+\dfrac{\tfrac{1}{2}}{s-1}+\dfrac{\tfrac{1}{2}s}{s^2+1}-\dfrac{\tfrac{1}{2}}{s^2+1}
\end{aligned}
\end{equation*}\pause

\vspace{-6mm}
Thus, the solution is
\begin{equation*}
y(t)=-1+\dfrac{1}{2}e^{t}+\dfrac{1}{2}\cos[t]-\dfrac{1}{2}\sin[t]
\end{equation*}
\end{example}
\end{frame}

\begin{frame}
\begin{block}{Translation Property for Multiplication by $e^{at}$}
If the Laplace transform $F(s)=\laplace\{f(t)\}$ exists for $s>a$, then
\begin{equation*}
\laplace\{e^{at} f(t)\}=F(s-a)\qquad\text{for}~s>a+\alpha
\end{equation*}
\end{block}\pause
\begin{proof}
The translation property is derived as follows
\begin{equation*}
\laplace\{e^{at} f(t)\}\pause
=\int^\infty_0 e^{-st}e^{at} f(t) dt\pause
=\int^\infty_0 e^{-(s-a)t} f(t) dt\pause
=F(s-a)
\end{equation*}
\end{proof}
\end{frame}

\begin{frame}
\begin{example}
Let us calculate
\begin{equation*}
\laplace\{e^{at}\cos[bt]\}
\end{equation*}\pause
We know that
\begin{equation*}
\laplace\{\cos[bt]\}=F(s)=\dfrac{s}{s^2+b^2}
\end{equation*}\pause
We can use then use the translation property.
\begin{equation*}
\laplace\{e^{at}\cos[bt]\}\pause
=F(s-a)\pause
=\dfrac{s-a}{{(s-a)}^2+b^2}
\end{equation*}
\end{example}\pause
\begin{example}
We can use the inverse of the translation property to calculate
\begin{equation*}
\laplace^{-1}\left\{\dfrac{1}{s^2+6s+10}\right\}\pause
=\laplace^{-1}\left\{\dfrac{1}{{(s+3)}^2+1}\right\}\pause
=e^{-3t}\sin[t]
\end{equation*}
\end{example}
\end{frame}

\begin{frame}
\begin{example}
Similarly, we can calculate
\begin{equation*}
\begin{aligned}
\laplace^{-1}\left\{\dfrac{3s-1}{s^2+2s+5}\right\}\pause
&=\laplace^{-1}\left\{\dfrac{3(s+1)-4}{{(s+1)}^2+4}\right\}\\\pause
&=\laplace^{-1}\left\{\dfrac{3(s+1)}{{(s+1)}^2+4}-\dfrac{4}{{(s+1)}^2+4}\right\}\\\pause
&=3\laplace^{-1}\left\{\dfrac{s+1}{{(s+1)}^2+4}\right\}-2\laplace^{-1}\left\{\dfrac{2}{{(s+1)}^2+4}\right\}\\\pause
&=3e^{-t}\cos[2t]-2e^{-t}\sin[2t]\\\pause
&=e^{-t}\left(3\cos[2t]-2\sin[2t]\right)
\end{aligned}
\end{equation*}
\end{example}
\end{frame}

\begin{frame}
\begin{block}{Multiplication by $t^n$ Rule for the Laplace Transform}
If $f(t)$ is a piecewise continuos function on $\interval{\closed{0}}{\open{\infty}}$ and is of exponential order $\alpha$, then for $s>\alpha$,
\begin{equation*}
\laplace\{t^n f(t)\}={(-1)}^n\dfrac{d^n F}{d s^n}(s)
\quad\text{where}\quad n\in\N^+
\end{equation*}
\end{block}\pause
\begin{proof}
We will prove the result for $n=1$. 
\begin{equation*}
\begin{aligned}
\dfrac{d}{ds} F(s) &= \dfrac{d}{ds}\int_0^\infty e^{-st} f(t) dt \\\pause
&=\int_0^\infty \dfrac{d}{ds} e^{-st} f(t) dt \\\pause
&=-\int_0^\infty t e^{-st} f(t) dt\\\pause
&=-\laplace\{t f(t)\}
\end{aligned}
\end{equation*}

\vspace{-2mm}
This process can be repeated for an arbitrary $n$.
\end{proof}
\end{frame}

\begin{frame}
\begin{example}
Let use 
\begin{equation*}
\laplace\{\cos[bt]\}=F(s)=\dfrac{s}{s^2+b^2}
\end{equation*}\pause
to calculate
\begin{equation*}
\laplace\{t\cos[bt]\}\pause
=-\dfrac{d}{ds}\left(\dfrac{s}{s^2+b^2}\right)\pause
=\dfrac{s^2-b^2}{{(s^2+b^2)}^2}
\end{equation*}
\end{example}
\end{frame}

%\begin{frame}
%\begin{example}
%Let us consider a special DE, called Bessel's equation
%\begin{equation*}
%t y^{\prime\prime}+ y^{\prime} + t y = 0
%\end{equation*}
%with initial conditions $y(0)=1$ and $y^{\prime}(0)=0$.\pause
%
%\vspace{2mm}
%First, we compute
%\begin{equation*}
%\begin{aligned}
%\laplace\{y^{\prime}\}\pause
%&= s\laplace\{y\}-y(0)\pause
%&= s^{\phantom{2}}\laplace\{y\}-1\\\pause
%\laplace\{y^{\prime\prime}\}\pause
%&= s^2 \laplace\{y\} -s y(0) -y^{\prime}(0)\pause
%&= s^2 \laplace\{y\} - s
%\end{aligned}
%\end{equation*}\pause
%We can use the Multiplication Property to get
%\begin{equation*}
%\begin{aligned}
%-\dfrac{d}{ds}(s^2 Y(s) - s)+(sY(s)-1)-\dfrac{d}{ds} Y(s)&=0\\\pause
%(s^2+1) Y^\prime (s) + s Y(s) &= 0
%\end{aligned}
%\end{equation*}
%\end{example}
%\end{frame}
\end{document}