\documentclass{beamer}

\usepackage[english]{babel}
\usepackage[utf8x]{inputenc}
\usepackage[inline]{asymptote}
\usepackage{slide_helper}
\usepackage{caption}
\usepackage{subcaption}

\begin{asydef}
import graph;
import slopefield;
import fontsize;
defaultpen(fontsize(9pt));
ngraph=1000;

int pen_pos=-1;
pen[] pens={blue, red, heavycyan, heavymagenta, lightolive};
pens.cyclic=true;
pen next_color() {return pens[++pen_pos];}

DefaultHead.size=new real(pen p=currentpen) {return 2.5mm;};

void drawgrid(real xmin, real xmax, real ymin, real ymax, pen p=dotted+darkgray)
{
	for(real gx=xmin+1; gx<=xmax-1; ++gx)
		draw((gx,ymin)--(gx,ymax),p);
    
	for(real gy=ymin+1; gy<=ymax-1; ++gy)
		draw((xmin,gy)--(xmax,gy),p); 
}

real length(pair z) {return (z.x == 0) && (z.y == 0) ? 0.0001 : sqrt(z.x*z.x+z.y*z.y);}

path eigenpath(real lambda, pair v)
{
	if (lambda >= 0)
		return (0,0)--v;
	else
		return v--(0,0);
}

void drawequilibrium(pair c, real r, bool stable)
{
	if (stable == true)
		fill(circle(c,r),black);
	else
		unfill(circle(c,r));

	draw(circle(c,r),black);
}
\end{asydef}

\title[MATH 2250 - Section 6.2]{Linear Systems with Real Eigenvalues}

\begin{document}

\begin{frame}
  \titlepage
\end{frame}

\begin{frame}
\begin{example}
\onslide<+->
\usepercentframe{0.85}{%
\only<1-4>{
Let us first look at solutions for a $2\by 2$ system, where $\mat{A}$ is constant
\begin{equation*}
\vect{x^\prime}=\mat{A}\vect{x}
\end{equation*}
\onslide<+->
We can start with what we have learned about the solutions for
\begin{equation*}
ay^{\prime\prime}+by^{\prime}+cy=0
\end{equation*}
\onslide<+->
We know that when the second-order equation is converted to a system, the characteristic roots are the eigenvalues of the system. How can we use this fact?
\onslide<+->

\vspace{2mm}
If $r_1$ and $r_2$ are the characteristic roots, then solutions are build from $e^{r_1 t}$ and $e^{r_2 t}$. So, we need to find similar building blocks for the system.}
\only<1-4>{\vspace{1.5cm}}
\only<5->{Given that solutions to the matrix-vector equation must be vectors:
\begin{equation*}
\vect{x}=e^{\lambda t}\vect{v}
\end{equation*}
\onslide<6->
Which we can substitute into the matrix-vector equation
\begin{equation*}
\begin{aligned}
\vect{x^\prime}&=\mat{A}\vect{x}\\
\visible<7->{\lambda e^{\lambda t}\vect{v}&=\mat{A}e^{\lambda t}\vect{v}\\}
\visible<8->{\vect{0}&=e^{\lambda t}\mat{A}\vect{v}-\lambda e^{\lambda t}\vect{v}\\}
\visible<9->{\vect{0}&=e^{\lambda t}\left(\mat{A}-\lambda\mat{I}\right)\vect{v}}
\end{aligned}
\end{equation*}
\onslide<10->
Given that $e^{\lambda t}>0$, we must find $\lambda$ and $\vect{v}$ that satisify:
\begin{equation*}
\left(\mat{A}-\lambda\mat{I}\right)\vect{v}=\vect{0}
\end{equation*}
\onslide<11->
But, these are just the eigenvalues and eigenvectors of $\mat{A}$!}}
\end{example}
\end{frame}

\begin{frame}
\begin{block}{\small Solving Homogeneous Linear $2\by 2$ DE Systems with Constant Coefficients}
For a two-dimensional system of homogeneous linear differential equations 
\begin{equation*}
\vect{x^\prime}=\mat{A}\vect{x}
\end{equation*}
where $\mat{A}$ is a matrix of constants and has eigenvalues $\lambda_1$ and $\lambda_2$ with corresponding eigenvectors $\vect{v_1}$ and $\vect{v_2}$. We can obtain the two solutions:
\begin{equation*}
e^{\lambda_1 t}\vect{v_1}
\quad\text{and}\quad
e^{\lambda_2 t}\vect{v_2}
\end{equation*}

\vspace{-4mm}
\begin{itemize}
\item If $\lambda_1\neq \lambda_2$, then these two solutions are linearly independent and form a basis for the solutions space. Thus, the general solutions, for $c_1,c_2\in\R$, is
\begin{equation*}
\vect{x}(t)=c_1 e^{\lambda_1 t}\vect{v_1} + c_2 e^{\lambda_2 t}\vect{v_2}
\end{equation*}
\item If $\lambda_1 = \lambda_2$, then there may be only one linearly independent eigenvector. Additional tactics may be required to obtain a basis of two vectors for the solution space.
\end{itemize}
\end{block}\pause
\begin{block}{}
We will first consider some examples where the eigenvalues are distinct.
\end{block}
\end{frame}

\begin{frame}[fragile]
\begin{example}
\begin{overprint}
\onslide<1-3>
Consider the system
\begin{equation*}
\vect{x^\prime}=\mat{A}\vect{x}=
\begin{bmatrix}
1 & 1 \\
4 & 1
\end{bmatrix}
\vect{x}
\end{equation*}
\visible<2->{The matrix $\mat{A}$ has eigenvalues
\begin{equation*}
\lambda_1=-1
\quad\text{and}\quad
\lambda_2=3
\end{equation*}
and eigenvectors
\begin{equation*}
\vect{v_1}=\bvector{\+1,-2}
\quad\text{and}\quad
\vect{v_2}=\bvector{1,2}
\end{equation*}}

\visible<3->{\vspace{-2mm}
Thus, the general solution must be
\begin{equation*}
\vect{x}=c_1 e^{-t}\bvector{\+1,-2}+c_2 e^{3t}\bvector{1,2}
\end{equation*}}
\onslide<4>
\begin{center}
\begin{asy}
size(235);

real min_x=-4, max_x=4;
real min_y=-4, max_y=4;

pair start=(min_x,min_y);
pair end=(max_x,max_y);

// eigenvectors and eigenvalues
real l1=-1;
pair v1=(1,-2);

real l2=3;
pair v2=(1,2);

// draw some trajectories
real t_start=-1.5;
real t_end=0.6;

real D=v1.x*v2.y-v2.x*v1.y;

triple[] curves = {	( -1.2,  0.0, 0.7), 
					(  0.8,  0.0, 0.65), 
					(  0.0,  2.0, 0.6),
					(  0.0, -1.8, 0.65)};					
for (triple k : curves)
{
	real A=k.x;
	real B=k.y;
	real c_1=(v2.y*A-v2.x*B)/D;
	real c_2=(v1.x*B-v1.y*A)/D;
	real X(real t) {return c_1*exp(l1*t)*v1.x+c_2*exp(l2*t)*v2.x;}
	real Y(real t) {return c_1*exp(l1*t)*v1.y+c_2*exp(l2*t)*v2.y;}
	draw(graph(X,Y,t_start,t_end),next_color()+1.0bp,Arrow(Relative(k.z)));
}

// draw eigenspaces
pen pes=black+1.5bp;
draw(scale(2)*eigenpath(l1,  v1), pes, MidArrow());
draw(scale(2)*eigenpath(l1, -v1), pes, MidArrow());

draw(scale(2)*eigenpath(l2,  v2), pes, MidArrow());
draw(scale(2)*eigenpath(l2, -v2), pes, MidArrow());

limits(start,end,Crop);

// draw axes
xaxis("$x$", YEquals(0), min_x, max_x, Ticks(OmitFormat(0)));
yaxis("$y$", XEquals(0), min_y, max_y, Ticks(OmitFormat(0,-1,1,4,-4)));

// draw origin stability
drawequilibrium((0,0),0.1,false);
\end{asy}
\end{center}
\end{overprint}
\vspace{-68mm}
\end{example}
\end{frame}

\begin{frame}[fragile]
\begin{example}
\begin{overprint}
\onslide<1-3>
Consider the system
\begin{equation*}
\vect{x^\prime}=\mat{A}\vect{x}=
\begin{bmatrix}
2 & 2 \\
1 & 3
\end{bmatrix}
\vect{x}
\end{equation*}
\visible<2->{The matrix $\mat{A}$ has eigenvalues
\begin{equation*}
\lambda_1=4
\quad\text{and}\quad
\lambda_2=1
\end{equation*}
and eigenvectors
\begin{equation*}
\vect{v_1}=\bvector{1,1}
\quad\text{and}\quad
\vect{v_2}=\bvector{\+2,-1}
\end{equation*}}

\visible<3->{\vspace{-2mm}
Thus, the general solution must be
\begin{equation*}
\vect{x}=c_1 e^{-4t}\bvector{1,1}+c_2 e^{t}\bvector{\+2,-1}
\end{equation*}}
\onslide<4>
\begin{center}
\begin{asy}
size(235);

real min_x=-4, max_x=4;
real min_y=-4, max_y=4;

pair start=(min_x,min_y);
pair end=(max_x,max_y);

// eigenvectors and eigenvalues
real l1=4;
pair v1=(1,1);

real l2=1;
pair v2=(2,-1);

// draw some trajectories
real t_start=-1.5;
real t_end=0.6;

real D=v1.x*v2.y-v2.x*v1.y;

triple[] curves = {	(  0.0,  2.0, 0.2), 
					(  1.5,  0.0, 0.4), 
					(  0.0, -2.0, 0.2),
					( -1.5,  0.0, 0.4)};					
for (triple k : curves)
{
	real A=k.x;
	real B=k.y;
	real c_1=(v2.y*A-v2.x*B)/D;
	real c_2=(v1.x*B-v1.y*A)/D;
	real X(real t) {return c_1*exp(l1*t)*v1.x+c_2*exp(l2*t)*v2.x;}
	real Y(real t) {return c_1*exp(l1*t)*v1.y+c_2*exp(l2*t)*v2.y;}
	draw(graph(X,Y,t_start,t_end),next_color()+1.0bp,Arrow(Relative(k.z)));
}

// draw eigenspaces
pen pes=black+1.5bp;
draw(scale(4)*eigenpath(l1,  v1), pes, MidArrow());
draw(scale(4)*eigenpath(l1, -v1), pes, MidArrow());

draw(scale(2.1)*eigenpath(l2,  v2), pes, MidArrow());
draw(scale(2.1)*eigenpath(l2, -v2), pes, MidArrow());

limits(start,end,Crop);

// draw axes
xaxis("$x$", YEquals(0), min_x, max_x, Ticks(OmitFormat(0,1)));
yaxis("$y$", XEquals(0), min_y, max_y, Ticks(OmitFormat(0,4,-4)));

// draw origin stability
drawequilibrium((0,0),0.1,false);
\end{asy}
\end{center}
\end{overprint}
\vspace{-68mm}
\end{example}
\end{frame}

\begin{frame}[fragile]
\begin{example}
\begin{overprint}
\onslide<1-3>
Consider the system
\begin{equation*}
\vect{x^\prime}=\mat{A}\vect{x}=
\begin{bmatrix}[rr]
-2 &  1 \\
 1 & -2
\end{bmatrix}
\vect{x}
\end{equation*}
\visible<2->{The matrix $\mat{A}$ has eigenvalues
\begin{equation*}
\lambda_1=-1
\quad\text{and}\quad
\lambda_2=-3
\end{equation*}
and eigenvectors
\begin{equation*}
\vect{v_1}=\bvector{1,1}
\quad\text{and}\quad
\vect{v_2}=\bvector{\+1,-1}
\end{equation*}}

\visible<3->{\vspace{-2mm}
Thus, the general solution must be
\begin{equation*}
\vect{x}=c_1 e^{-t}\bvector{1,1}+c_2 e^{-3t}\bvector{\+1,-1}
\end{equation*}}
\onslide<4>
\begin{center}
\begin{asy}
size(235);

real min_x=-4, max_x=4;
real min_y=-4, max_y=4;

pair start=(min_x,min_y);
pair end=(max_x,max_y);

// eigenvectors and eigenvalues
real l1=-1;
pair v1=(1,1);

real l2=-3;
pair v2=(1,-1);

// draw some trajectories
real t_start=-1.0;
real t_end=2.0;

real D=v1.x*v2.y-v2.x*v1.y;

triple[] curves = {	(  3.0,  1.0, 0.9), 
					(  1.0,  3.0, 0.9), 
					( -2.5, -1.0, 0.89),
					( -1.0, -3.0, 0.9)};					
for (triple k : curves)
{
	real A=k.x;
	real B=k.y;
	real c_1=(v2.y*A-v2.x*B)/D;
	real c_2=(v1.x*B-v1.y*A)/D;
	real X(real t) {return c_1*exp(l1*t)*v1.x+c_2*exp(l2*t)*v2.x;}
	real Y(real t) {return c_1*exp(l1*t)*v1.y+c_2*exp(l2*t)*v2.y;}
	draw(graph(X,Y,t_start,t_end),next_color()+1.0bp,Arrow(Relative(k.z)));
}

// draw eigenspaces
pen pes=black+1.5bp;
draw(scale(4)*eigenpath(l1,  v1), pes, MidArrow());
draw(scale(4)*eigenpath(l1, -v1), pes, MidArrow());

draw(scale(4)*eigenpath(l2,  v2), pes, MidArrow());
draw(scale(4)*eigenpath(l2, -v2), pes, MidArrow());

limits(start,end,Crop);

// draw axes
xaxis("$x$", YEquals(0), min_x, max_x, Ticks(OmitFormat(0,1)));
yaxis("$y$", XEquals(0), min_y, max_y, Ticks(OmitFormat(0,4,-4)));

// draw origin stability
drawequilibrium((0,0),0.1,true);
\end{asy}
\end{center}
\end{overprint}
\vspace{-68mm}
\end{example}
\end{frame}

\begin{frame}
\begin{block}{Behavior of Solutions}
\begin{itemize}[<+- | alert@+>]
\item If all trajectories tend towards the origin as $t\rightarrow\infty$, then the origin is a \textbf{stable} equilibrium. (Both eigenvalues are negative.)
\item If all trajectories tend away from the origin as $t\rightarrow\infty$, then the origin is an \textbf{unstable} equilibrium. (Both eigenvalues are positive.)
\item If only some solutions tend towards the origin as $t\rightarrow\infty$, then the origin is a \textbf{saddle} equilibrium. (Eigenvalues have different signs.)
\end{itemize}
\end{block}
\onslide<+->
\begin{block}{Phase Plane Role of Real Eigenvalues and Eigenvectors}
For an autonomous and homogeneous two-dimensional DE system:
\begin{itemize}[<+- | alert@+>]
\item Trajectories move towards or away from the equilibrium according to the signs of the eigenvalues.
\item Along each eigenvector is a unique trajectory called a \textbf{separatrix} that separates trajectories curving one way from those curving another way.
\item The equilibrium occurs at the origin, and the phase portrait is symmetric about this point.
\end{itemize}
\end{block}
\end{frame}

\begin{frame}
\onslide<+->
\begin{block}{Speed and Shape of Trajectories}
\begin{itemize}[<+- | alert@+>]
\item The \quotetext{speed} along a trajectory in the direction of an eigenvector depends on the relative size of the associated eigenvalue:
\begin{itemize}
\item \quotetext{fast} for the eigenvalue with the largest magnitude.
\item \quotetext{slow} for the eigenvalue with the smallest magnitude.
\end{itemize}
\item Trajectories become parallel to the fast eigenvectors further away from the origin, and tangent to the slow eigenvectors. (closer to the origin for sources or sinks, further from the origin for a saddle.)
\end{itemize}
\end{block}
\end{frame}

\begin{frame}[fragile]
\begin{example}
\begin{overprint}
\onslide<1-3>
Consider the system
\begin{equation*}
\vect{x^\prime}=\mat{A}\vect{x}=
\begin{bmatrix}[rr]
3 & 0 \\
0 & 3
\end{bmatrix}
\vect{x}
\end{equation*}
\visible<2->{The matrix $\mat{A}$ has a (repeated) eigenvalue
\begin{equation*}
\lambda_1=3
\quad\text{and}\quad
\lambda_2=3
\end{equation*}
and two linearly independent eigenvectors
\begin{equation*}
\vect{v_1}=\bvector{1,0}
\quad\text{and}\quad
\vect{v_2}=\bvector{0,1}
\end{equation*}}

\visible<3->{\vspace{-2mm}
Thus, the general solution must be
\begin{equation*}
\vect{x}=c_1 e^{3t}\bvector{1,0}+c_2 e^{3t}\bvector{0,1}
\end{equation*}}
\onslide<4>
\begin{center}
\begin{asy}
size(230);

real min_x=-4, max_x=4;
real min_y=-4, max_y=4;

pair start=(min_x,min_y);
pair end=(max_x,max_y);

// eigenvectors and eigenvalues
real l1=3;
pair v1=(1,0);

real l2=3;
pair v2=(0,1);

// draw some trajectories
real t_start=-1.0;
real t_end=1.0;

real D=v1.x*v2.y-v2.x*v1.y;

triple[] curves = {	(  1.0,  2.0, 0.05), 
					(  1.0, -1.0, 0.08), 
					( -2.0, -1.5, 0.05),
					( -2.0,  1.0, 0.05)};					
for (triple k : curves)
{
	real A=k.x;
	real B=k.y;
	real c_1=(v2.y*A-v2.x*B)/D;
	real c_2=(v1.x*B-v1.y*A)/D;
	real X(real t) {return c_1*exp(l1*t)*v1.x+c_2*exp(l2*t)*v2.x;}
	real Y(real t) {return c_1*exp(l1*t)*v1.y+c_2*exp(l2*t)*v2.y;}
	draw(graph(X,Y,t_start,t_end),next_color()+1.0bp,Arrow(Relative(k.z)));
}

// draw eigenspaces
pen pes=black+1.5bp;
draw(scale(4)*eigenpath(l1,  v1), pes, MidArrow());
draw(scale(4)*eigenpath(l1, -v1), pes, MidArrow());

draw(scale(4)*eigenpath(l2,  v2), pes, MidArrow());
draw(scale(4)*eigenpath(l2, -v2), pes, MidArrow());

limits(start,end,Crop);

// draw axes
xaxis("$x$", YEquals(0), min_x, max_x, Ticks(OmitFormat(0,4,-4)));
yaxis("$y$", XEquals(0), min_y, max_y, Ticks(OmitFormat(0,4,-4)));

// draw origin stability
drawequilibrium((0,0),0.1,false);
\end{asy}
\end{center}
\end{overprint}
\vspace{-74mm}
\end{example}
\end{frame}

\begin{frame}[fragile]
\begin{example}
Consider the system
\begin{equation*}
\vect{x^\prime}=\mat{A}\vect{x}=
\begin{bmatrix}[rr]
2 & -1 \\
4 &  6
\end{bmatrix}
\vect{x}
\end{equation*}
\begin{overprint}
\onslide<1-4>
\visible<2->{The matrix $\mat{A}$ has a (repeated) eigenvalue
\begin{equation*}
\lambda_1=4
\quad\text{and}\quad
\lambda_2=4
\end{equation*}
and only one eigenvector
\begin{equation*}
\vect{v}=\bvector{\+1,-2}
\end{equation*}}

\visible<3->{\vspace{-2mm}
Thus, one solution is
\begin{equation*}
\vect{x_1}=e^{4t}\bvector{\+1,-2}
\end{equation*}}

\visible<4->{\vspace{-2mm}
But, we need two solutions to form a basis of the solution space!}
\onslide<5-8>
Let us try the same trick we used for second-order systems:
\begin{equation*}
\vect{x_2}=te^{4t}\vect{v}
\end{equation*}
\visible<6->{Substituting into the differential equation gives:
\begin{equation*}
\begin{aligned}
{\left(te^{4t}\vect{v}\right)}^\prime&=\mat{A}\left(te^{4t}\vect{v}\right)\\
\visible<7->{e^{4t}\vect{v}+4te^{4t}\vect{v}&=te^{4t}\mat{A}\vect{v}\\}
\end{aligned}
\end{equation*}}

\visible<8->{\vspace{-8mm}
Which is true if and only if $\vect{v}=\vect{0}$. But this contradicts that $\vect{v}$ is an eigenvector. We will need to try something else.}
\onslide<9-14>

\vspace{-2mm}
Let us instead try to introduce a second vector $\vect{u}$, that multiplies the \\ troublesome $e^{4t}$ term:
\begin{equation*}
\vect{x_2}=te^{4t}\vect{v}+e^{4t}\vect{u}
\end{equation*}
\visible<10->{Let us plug $\vect{x_2}$ into the differential equation:
\begin{equation*}
\begin{aligned}
{\left(te^{4t}\vect{v}+e^{4t}\vect{u}\right)}^\prime&=\mat{A}\left(te^{4t}\vect{v}+e^{4t}\vect{u}\right)\\
\visible<11->{e^{4t}\vect{v}+4te^{4t}\vect{v}+4e^{4t}\vect{u}&=\mat{A}\left(te^{4t}\vect{v}+e^{4t}\vect{u}\right)}
\end{aligned}
\end{equation*}}

\visible<12->{\vspace{-2mm}
Equating coefficients gives the system or equations:
\begin{equation*}
\begin{matrix}[lll]
4\vect{v}&=&\mat{A}\vect{v}\\
\phantom{4}\vect{v}+4\vect{u}&=&\mat{A}\vect{u}\\
\end{matrix}
\visible<13->{
\quad\text{or}\quad
\begin{matrix}
(\mat{A}-4\mat{I})\vect{v}&=&\vect{0}\\
(\mat{A}-4\mat{I})\vect{u}&=&\vect{v}
\end{matrix}}
\end{equation*}}

\visible<14->{\vspace{-2mm}
Since $\vect{v}$ is an eigenvector, we only need to find a solution for $\vect{u}$.}
\onslide<15-19>
Which means we need to solve:
\begin{equation*}
\begin{bmatrix}[rr]
-2 & -1 \\
 4 &  2 \\
\end{bmatrix}\bvector{u_1,u_2}
=\bvector{\+1,-2}
\visible<16->{\quad\rightarrow\quad
\begin{bmatrix}[rr|r]
-2 & -1 & 1\\
 4 &  2 & -2\\
\end{bmatrix}}
\visible<17->{\quad\rightarrow\quad
\begin{bmatrix}[rr|r]
1 & \tfrac{1}{2} & -\tfrac{1}{2} \\
0 & 0 & 0 \\
\end{bmatrix}}
\end{equation*}
\visible<18->{So, we have $u_1+\tfrac{1}{2}u_2=-\tfrac{1}{2}$. Letting $u_1=k$ and $u_2=-2k-1$ we get
\begin{equation*}
\vect{u}=\bvector{k,-2k-1}=k\bvector{\+1,-2}+\bvector{\+0,-1}
\end{equation*}}

\visible<19->{\vspace{-5mm}
And,
\begin{equation*}
\vect{x_2}=te^{4t}\bvector{\+1,-2}+ke^{4t}\bvector{\+1,-2}+e^{4t}\bvector{\+0,-1}=te^{4t}\bvector{\+1,-2}+e^{4t}\bvector{\+0,-1}
\end{equation*}}

\onslide<20>
Thus, the general solution is
\begin{equation*}
\vect{x}=c_1 e^{4t}\bvector{\+1,-2}+c_2\left(te^{4t}\bvector{\+1,-2}+e^{4t}\bvector{\+0,-1}\right)
\end{equation*}
\onslide<21>
\begin{center}
\begin{asy}
size(180);

real min_x=-4, max_x=4;
real min_y=-4, max_y=4;

pair start=(min_x,min_y);
pair end=(max_x,max_y);

// eigenvectors and eigenvalues
real l1=4;
pair v1=(1,-2);

// draw some trajectories
real t_start=-1.0;
real t_end=0.5;

real x1(real t) {return exp(4t);}
real x2(real t) {return t*exp(4t);}

real y1(real t) {return -2*exp(4t);}
real y2(real t) {return exp(4t)*(-2*t-1);}

triple[] curves = {	(  0.9,  0.0, 0.15), 
					(  2.0,  0.0, 0.15), 
					( -1.0,  0.0, 0.15),
					( -2.1,  0.0, 0.15)};					
for (triple k : curves)
{
	real A=k.x;
	real B=k.y;
	real c_1=A;
	real c_2=-B-2*A;
	real X(real t) {return c_1*x1(t)+c_2*x2(t);}
	real Y(real t) {return c_1*y1(t)+c_2*y2(t);}
	draw(graph(X,Y,t_start,t_end),next_color()+1.0bp,Arrow(Relative(k.z)));
}

// draw eigenspaces
pen pes=black+1.5bp;
draw(scale(2)*eigenpath(l1,  v1), pes, MidArrow());
draw(scale(2)*eigenpath(l1, -v1), pes, MidArrow());

limits(start,end,Crop);

// draw axes
xaxis("$x$", YEquals(0), min_x, max_x, Ticks(OmitFormat(0,4,-4)));
yaxis("$y$", XEquals(0), min_y, max_y, Ticks(OmitFormat(0,4,-4)));

// draw origin stability
drawequilibrium((0,0),0.1,false);
\end{asy}
\end{center}
\end{overprint}
\vspace{-63mm}
\end{example}
\end{frame}

\begin{frame}
\begin{block}{Creating a Generalized Eigenvector for a System with Insufficient Eigenvectors}
If a homogeneous linear $2\by 2$ system of first-order DEs has repeated eigenvalue $\lambda$ with only a single eigenvector, a second linearly independent solution can be created as follows:
\begin{enumerate}
\item Find an eigenvector $\vect{v}$ corresponding to $\lambda$.
\item Find a nonzero vector $\vect{u}$ such that
\begin{equation*}
(\mat{A}-\lambda\mat{I})\vect{u}=\vect{v}
\end{equation*}
\item Then the general solution is
\begin{equation*}
\vect{x}(t)=c_1 e^{\lambda t}\vect{v}+c_2 e^{\lambda t}(t\vect{v}+\vect{u})
\end{equation*}
\end{enumerate}
The vector $\vect{u}$ is called a \textbf{generalized eigenvector} of $\mat{A}$ corresponding to $\lambda$.
\end{block}
\end{frame}

\begin{frame}
\begin{block}{\small Solving $n\by n$ Homogeneous Linear DE Systems with Constant Coefficients}
For an $n$-dimensional system of homogeneous linear differential equations $\vect{x^\prime}=\mat{A}\vect{x}$ where $\mat{A}$ is a matrix of constants that has eigenvalues $\lambda_1,\lambda_2,\ldots,\lambda_n$ with corresponding eigenvectors $\vect{v_1}, \vect{v_2},\ldots,\vect{v_n}$, we obtain solutions
\begin{equation*}
e^{\lambda_1 t}\vect{v_1}, e^{\lambda_2 t}\vect{v_2},\ldots,e^{\lambda_n t}\vect{v_n}
\end{equation*}\pause
If all eigenvalues are distinct, then these solutions are linearly independent and form a basis of the solution space. Thus, the general solution, for $c_1,c_2,\ldots,c_n\in\R$, is
\begin{equation*}
\vect{x}=c_1 e^{\lambda_1 t}\vect{v_1}+c_2 e^{\lambda_2 t}\vect{v_2}+\cdots+c_n e^{\lambda_n t}\vect{v_n}
\end{equation*}\pause
The case where eigenvalues are repeated will require either independent eigenvectors or generalized eigenvectors.
\end{block}
\end{frame}
\end{document}