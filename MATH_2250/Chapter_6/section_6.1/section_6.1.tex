\documentclass{beamer}

\usepackage[english]{babel}
\usepackage[utf8x]{inputenc}
\usepackage[inline]{asymptote}
\usepackage{slide_helper}
\usepackage{caption}
\usepackage{subcaption}

\title[MATH 2250 - Section 6.1]{Linear Systems of Differential Equations}

\begin{document}

\begin{frame}
  \titlepage
\end{frame}

\begin{frame}
\begin{block}{Linear First-Order DE System}
An $n$-dimensional \textbf{linear first-order DE system} on open interval $I$ is one that can be written as a matrix-vector equation:
\begin{equation*}
\vect{x^\prime}(t)=\mat{A}(t)\vect{x}(t)+\vect{f}(t)
\end{equation*}
\onslide<+->
\begin{itemize}[<+- | alert@+>]
\item $\mat{A}(t)$ is an $n\by n$ matrix of continuous functions on $I$.
\item $\vect{f}(t)$ is an $n\by 1$ vector of continuous functions on $I$.
\item $\vect{x}(t)$ is an $n\by 1$ \textbf{solutions vector} of differentiable functions on $I$ that satisfies the DE\@.
\end{itemize}
\onslide<+->
If $\vect{f}(t)=\vect{0}$, the system is \textbf{homogeneous}
\begin{equation*}
\vect{x^\prime} (t)=\mat{A}(t)\vect{x}(t)
\end{equation*}
\end{block}
\end{frame}

\begin{frame}
\begin{example}
Consider the homogeneous linear first-order system
\begin{equation*}
\begin{aligned}
x^\prime &= 3x - 2y \\
y^\prime &= x \\
z^\prime &= -x + y + 3z \\
\end{aligned}
\end{equation*}\pause

\vspace{-4mm}
Which can be written as:
\begin{equation*}
\vect{x^\prime}=
\begin{bmatrix}[rrr]
3 & -2 & 0 \\
1 & 0 & 0 \\
-1 & 1 & 3 \\
\end{bmatrix}
\vect{x}
\quad\text{where}\quad
\vect{x}=\bvector{x,y,z}
\end{equation*}\pause
and has solution $\vect{x}=\left<2e^{2t}, e^{2t}, e^{2t}\right>$\pause

\vspace{1mm}
We can verify this:
\begin{equation*}
\begin{bmatrix}[rrr]
3 & -2 & 0 \\
1 & 0 & 0 \\
-1 & 1 & 3 \\
\end{bmatrix}
\bvector{2e^{2t},e^{2t},e^{2t}}\pause
=\bvector{6e^{2t}-2e^{2t},2e^{2t},-2e^{2t}+e^{2t}+3e^{2t}}\pause
=\bvector{4e^{2t},2e^{2t},2e^{2t}}\pause
={\bvector{2e^{2t},e^{2t},e^{2t}}}^{\prime}
\end{equation*}

\vspace{-4mm}
\end{example}
\end{frame}

\begin{frame}
\begin{block}{Initial-Value Problem for a Linear DE System}
For a linear DE system, an \textbf{initial-value problem} is the combination of a linear DE system and an initial value vector.
\begin{equation*}
\vect{x^\prime}(t)=\mat{A}(t)\vect{x}(t)+\vect{f}(t)
,\qquad
\vect{x}(t_0)=\bvector{c_1,c_2,\vdots,c_n}
\end{equation*}
\end{block}\pause
\begin{block}{Existence and Uniqueness Theorem for Linear DE Systems}
Given an $n\by n$ matrix function $\mat{A}(t)$ and a $n\by 1$ vector function $\vect{f}(t)$, both continuous on an open interval $I$ containing $t_0$, and a constant $n$-vector $\vect{x_0}$, there exists a unique vector function $\vect{x}(t)$ such that
\begin{equation*}
\vect{x^\prime}=\mat{A}(t)\vect{x}+\vect{f}(t)
\quad\text{and}\quad
\vect{x}(t_0)=\vect{x_0}
\end{equation*}
\end{block}
\end{frame}

\begin{frame}
\begin{block}{The Superposition Principle for Homogeneous Linear DE Systems}
Let $\vect{x_1},\vect{x_2},\ldots,\vect{x_n}$ be solution vectors for the homogenous equation
\begin{equation*}
\vect{x^\prime}=\mat{A}(t)\vect{x}
\quad\text{on $I$}
\end{equation*}
Then, any linear combination of these solution vectors is also a solution vector for the system. 

\vspace{2mm}
That is,
\begin{equation*}
\vect{x}=c_1\vect{x_1}+c_2\vect{x_2}+\cdots+c_n\vect{x_n}
\end{equation*}
is also a solution on $I$ for any $c_1,c_2,\ldots,c_n\in\R$.
\end{block}
\end{frame}

\begin{frame}
\begin{block}{Solution Space Theorem for Homogeneous Linear DE Systems}
If
\begin{equation*}
\vect{x^\prime}=\mat{A}(t)\vect{x}
\end{equation*}
where $\mat{A}$ is an $n\by n$ matrix, then the set of solutions $\vect{x}(t)$ is a vector space of dimension $n$.
\end{block}\pause
\begin{block}{Solution Theorem for Homogenous Linear DE Systems}
For $n$ linearly independent solutions $\vect{x_1},\vect{x_2},\ldots,\vect{x_n}$ of
\begin{equation*}
\vect{x^\prime}=\mat{A}(t)\vect{x}
\end{equation*}
the general solution is
\begin{equation*}
\vect{x}=c_1\vect{x_1}+c_2\vect{x_2}+\cdots+c_n\vect{x_n}
\quad\text{where}\quad
c_1,c_2,\ldots,c_n\in\R
\end{equation*}
\end{block}
\end{frame}

\begin{frame}
\begin{example}
For the system in the last example we have three solutions
\begin{equation*}
\vect{x_1}=\bvector{0,0,e^{3t}}
,\quad
\vect{x_2}=\bvector{2e^{2t},e^{2t},e^{2t}}
,\quad
\vect{x_3}=\bvector{e^t,e^t,0}
\end{equation*}\pause
To show that $\{\vect{x_1}, \vect{x_2}, \vect{x_3}\}$ are linearly independent on $\interval{\open{-\infty}}{\open{\infty}}$ choose a point, say $t_0=0$.\pause

\vspace{2mm}
Calculate $\vect{x_1}(t_0)$, $\vect{x_2}(t_0)$, and $\vect{x_3}(t_0)$. Then construct the column space matrix:
\begin{equation*}
\mat{C}=
\begin{bmatrix}
0&2&1\\
0&1&1\\
1&1&0\\
\end{bmatrix}
\end{equation*}\pause
If the determinant is nonzero, then the solutions are linearly independent.\pause
\begin{equation*}
\abs{\mat{C}}=0(1\cdot 0- 1\cdot 1)-2(0\cdot 0-1\cdot 1)+1(0\cdot 1-1\cdot 1)=1
\end{equation*}\pause

\vspace{-7mm}
So, the general solution is $\vect{x}=c_1\vect{x_1}+c_2\vect{x_2}+c_3\vect{x_3}$.
\end{example}
\end{frame}

\begin{frame}
\begin{block}{}
We have a few ways to express solutions:
\begin{equation*}
\vect{x}=
c_1\bvector{0,0,e^{3t}}+
c_2\bvector{2e^{2t},e^{2t},e^{2t}}+
c_3\bvector{e^t,e^t,0}\pause
=
\begin{bmatrix}
0 & 2e^{2t} & e^t \\
0 & e^{2t} & e^t \\
e^{3t} & e^{2t} & 0 \\
\end{bmatrix}
\bvector{c_1,c_2,c_3}
\end{equation*}
\end{block}\pause
\begin{block}{Fundamental Matrix}
For a basis of $n$ linearly independent solutions of $\vect{x^\prime}=\mat{A}\vect{x}$, the matrix $\mat{X}(t)$ whoose \emph{columns} are the vector solutions $\vect{x_1},\vect{x_2},\ldots,\vect{x_n}$ is called the \textbf{fundamental matrix} for the system.
\begin{equation*}
\vect{x}=
\underbrace{\begin{bmatrix}
\mid & \mid & & \mid \\
\vect{x_1} & \vect{x_2} & \cdots & \vect {x_n} \\
\mid & \mid & & \mid \\
\end{bmatrix}}_{\mat{X}(t)}
\bvector{c_1,c_2,c_3}
\quad c_1,c_2,c_3\in\R
\end{equation*}
\end{block}
\end{frame}

\begin{frame}
\begin{block}{Graphical Views}
\begin{itemize}
\item The $t$-$x$ and $t$-$y$ graphs showing the individual solution functions $x(t)$ and $y(t)$ are called \textbf{component graphs}, \textbf{solution graphs}, or \textbf{time series}.
\item The $x$-$y$ graph is the \textbf{phase plane}. The \textbf{trajectories} in the phase plane are the parametric curves described by $x(t)$ and $y(t)$.
\end{itemize}
Trajectories on a phase plane create a \textbf{phase portrait}.
\end{block}
\end{frame}

\begin{frame}[fragile]
\begin{example}
The familiar equation
\begin{equation*}
x^{\prime\prime}+0.1x=0
\end{equation*}
\begin{overprint}
\onslide<1>
\onslide<2-3>
can be written in the system form as
\begin{equation*}
\begin{matrix}[ll]
x^\prime=y \\
y^\prime=-0.1x
\end{matrix}
\quad\text{or}\quad
{\bvector{x,y}}^\prime=
\begin{bmatrix}
0 & 1 \\
-0.1 & 0 \\
\end{bmatrix}
\bvector{x,y}
\end{equation*}
\visible<3->{Any version of the these equations produces solutions of the form
\begin{equation*}
\begin{aligned}
x(t) &= c_1 \cos[t\sqrt{0.1}] + c_2 \sin[t\sqrt{0.1}]\\
y(t) &= x^\prime(t)=-c_1\sqrt{0.1}\sin[t\sqrt{0.1}]+c_2\sqrt{0.1}\cos[t\sqrt{0.1}]\\
\end{aligned}
\end{equation*}}
\onslide<4->
\begin{figure}[h]
\begin{subfigure}{0.5\textwidth}
\begin{center}
\begin{asy}
import graph;
import slopefield;
import fontsize;
defaultpen(fontsize(9pt));
size(200);
ngraph=1000;
real min_x=-10, max_x=10;
real min_y=-10, max_y=10;

pair start=(min_x,min_y);
pair end=(max_x,max_y);

real length(pair z) {return (z.x == 0) && (z.y == 0) ? 0.0001 : sqrt(z.x*z.x+z.y*z.y);}

// phase plot for ay''+by'+cy=0
real a=1;
real b=0;
real c=0.1;
real F(pair z) {return (-b*z.y-c*z.x)/a;}
path vector(pair z) {return (0,0)--1/(2*length((z.y,F(z))))*(z.y, F(z));}

add(vectorfield(vector,start,end,arrow=EndArrow(SimpleHead)));

for(real gx=min_x+1; gx<=max_x-1; ++gx)
	draw((gx,min_y)--(gx,max_y),dotted+darkgray);
    
for(real gy=min_y+1; gy<=max_y-1; ++gy)
	draw((min_x,gy)--(max_x,gy),dotted+darkgray); 

// draw trajectories
int pen_pos=-1;
pen[] pens={blue, red, heavycyan, heavymagenta, lightolive};
pens.cyclic=true;
pen next_color() {return pens[++pen_pos];}

DefaultHead.size=new real(pen p=currentpen) {return 2.5mm;};

real t_start=0;
real t_end=20;

triple[] curves = {	( 0.0, 1.0, 0.95), 
					( 0.0, 2.0, 0.95), 
					( 0.0, 3.0, 0.95)};					
for (triple k : curves)
{
	real A=k.x;
	real B=k.y;
	real c_1=A;
	real c_2=B/sqrt(0.1);
	real X(real t) {return c_1*cos(sqrt(0.1)*t)+c_2*sin(sqrt(0.1)*t);}
	real Y(real t) {return -sqrt(0.1)*c_1*sin(sqrt(0.1)*t)+sqrt(0.1)*c_2*cos(sqrt(0.1)*t);}
	draw(graph(X,Y,t_start,t_end),next_color()+1.0bp,Arrow(Relative(k.z)));
}

limits(start,end,Crop);

xaxis("$x$",YEquals(min_y),min_x,max_x,LeftTicks());
xaxis(YEquals(max_y),min_x,max_x);
yaxis("$y$",XEquals(min_x),min_y,max_y,LeftTicks());
yaxis(XEquals(max_x),min_y,max_y);
\end{asy}
\end{center}
\end{subfigure}
\begin{subfigure}{0.45\textwidth}
\begin{center}
\begin{asy}
import graph;
import fontsize;
defaultpen(fontsize(9pt));
size(115);
ngraph=1000;
real min_x=0, max_x=20;
real min_y=-10, max_y=10;

int pen_pos=-1;
pen[] pens={blue, red, heavycyan, heavymagenta, lightolive};
pens.cyclic=true;

pen next_color() {return pens[++pen_pos];}

pair[] curves = {	( 0.0, 1.0), 
					( 0.0, 2.0), 
					( 0.0, 3.0)};	
					
for (pair k : curves)
{
	real A=k.x;
	real B=k.y;
	real c_1=A;
	real c_2=B/sqrt(0.1);
	real X(real t) {return c_1*cos(sqrt(0.1)*t)+c_2*sin(sqrt(0.1)*t);}
	real Y(real t) {return -sqrt(0.1)*c_1*sin(sqrt(0.1)*t)+sqrt(0.1)*c_2*cos(sqrt(0.1)*t);}
	draw(graph(Y,min_x, max_x),next_color()+0.75bp);
}
limits((min_x,min_y),(max_x,max_y),Crop);
xaxis("$t$",YEquals(0),min_x,max_x,Ticks(NoZero));
yaxis("$y$",XEquals(0),min_y,max_y,Ticks(NoZero));
\end{asy}
\begin{asy}
import graph;
import fontsize;
defaultpen(fontsize(9pt));
size(115);
ngraph=1000;
real min_x=0, max_x=20;
real min_y=-10, max_y=10;

int pen_pos=-1;
pen[] pens={blue, red, heavycyan, heavymagenta, lightolive};
pens.cyclic=true;

pen next_color() {return pens[++pen_pos];}

pair[] curves = {	( 0.0, 1.0), 
					( 0.0, 2.0), 
					( 0.0, 3.0)};	
					
for (pair k : curves)
{
	real A=k.x;
	real B=k.y;
	real c_1=A;
	real c_2=B/sqrt(0.1);
	real X(real t) {return c_1*cos(sqrt(0.1)*t)+c_2*sin(sqrt(0.1)*t);}
	real Y(real t) {return -sqrt(0.1)*c_1*sin(sqrt(0.1)*t)+sqrt(0.1)*c_2*cos(sqrt(0.1)*t);}
	draw(graph(X,min_x, max_x),next_color()+0.75bp);
}
limits((min_x,min_y),(max_x,max_y),Crop);
xaxis("$t$",YEquals(0),min_x,max_x,Ticks(NoZero));
yaxis("$x$",XEquals(0),min_y,max_y,Ticks(NoZero));
\end{asy}
\end{center}
\end{subfigure}
\end{figure}
\end{overprint}
\end{example}
\end{frame}

\begin{frame}[fragile]
\begin{example}
The second-order DE
\begin{equation*}
x^{\prime\prime}+0.05x^{\prime}+0.1x=0
\end{equation*}
\begin{overprint}
\onslide<1>
\onslide<2-3>
can be written in the system form as
\begin{equation*}
\begin{matrix}[ll]
x^\prime=y \\
y^\prime=-0.1x-0.05y
\end{matrix}
\quad\text{or}\quad
{\bvector{x,y}}^\prime=
\begin{bmatrix}
0 & 1 \\
-0.1 & -0.05 \\
\end{bmatrix}
\bvector{x,y}
\end{equation*}
\visible<3->{With solutions of the (approximate) form
\begin{equation*}
\begin{aligned}
x(t) &\approx e^{-0.025t}\left(c_1\cos[0.32t]+c_2\sin[0.32t]\right)\\
y(t) &\approx e^{-0.025t}\left(-0.32c_1\sin[0.32t]+0.32c_2\cos[0.23t]\right)\\
&\qquad-0.025e^{-0.025t}\left(c_1\cos[0.32t]+c_2\sin[0.32t]\right)\\
\end{aligned}
\end{equation*}}
\onslide<4->
\begin{figure}[h]
\begin{subfigure}{0.5\textwidth}
\begin{center}
\begin{asy}
import graph;
import slopefield;
import fontsize;
defaultpen(fontsize(9pt));
size(190);
ngraph=1000;
real min_x=-8, max_x=10;
real min_y=-9, max_y=9;

pair start=(min_x,min_y);
pair end=(max_x,max_y);

real length(pair z) {return (z.x == 0) && (z.y == 0) ? 0.0001 : sqrt(z.x*z.x+z.y*z.y);}

// phase plot for ay''+by'+cy=0
real a=1;
real b=0.05;
real c=0.1;
real F(pair z) {return (-b*z.y-c*z.x)/a;}
path vector(pair z) {return (0,0)--1/(2*length((z.y,F(z))))*(z.y, F(z));}

add(vectorfield(vector,start,end,arrow=EndArrow(SimpleHead)));

for(real gx=min_x+1; gx<=max_x-1; ++gx)
	draw((gx,min_y)--(gx,max_y),dotted+darkgray);
    
for(real gy=min_y+1; gy<=max_y-1; ++gy)
	draw((min_x,gy)--(max_x,gy),dotted+darkgray); 

// draw trajectories
int pen_pos=-1;
pen[] pens={blue, red, heavycyan, heavymagenta, lightolive};
pens.cyclic=true;
pen next_color() {return pens[++pen_pos];}

DefaultHead.size=new real(pen p=currentpen) {return 2.5mm;};

real t_start=0;
real t_end=100;

triple[] curves = {	( -1.0, 3.5, 0.05), 
					( -0.5, 2.0, 0.05), 
					( 0.0, 1.0, 0.05)};					
for (triple k : curves)
{
	real A=k.x;
	real B=k.y;
	real c_1=A;
	real c_2=(B+0.025*A)/0.32;
	real X(real t) {return exp(-0.025*t)*(c_1*cos(0.32*t)+c_2*sin(0.32*t));}
	real Y(real t) {return exp(-0.025*t)*(0.32*c_1*sin(0.32*t)+0.32*c_2*cos(0.32*t))-0.025*exp(-0.025*t)*(c_1*cos(0.32*t)+c_2*sin(0.32*t));}
	draw(graph(X,Y,t_start,t_end),next_color()+1.0bp,Arrow(Relative(k.z)));
}

limits(start,end,Crop);

xaxis("$x$",YEquals(min_y),min_x,max_x,LeftTicks());
xaxis(YEquals(max_y),min_x,max_x);
yaxis("$y$",XEquals(min_x),min_y,max_y,LeftTicks());
yaxis(XEquals(max_x),min_y,max_y);
\end{asy}
\end{center}
\end{subfigure}
\begin{subfigure}{0.45\textwidth}
\begin{center}
\begin{asy}
import graph;
import fontsize;
defaultpen(fontsize(9pt));
size(115,87,IgnoreAspect);
ngraph=1000;
real min_x=0, max_x=100;
real min_y=-10, max_y=10;

int pen_pos=-1;
pen[] pens={blue, red, heavycyan, heavymagenta, lightolive};
pens.cyclic=true;

pen next_color() {return pens[++pen_pos];}

triple[] curves = {	( -1.0, 3.5, 0.05), 
					( -0.5, 2.0, 0.05), 
					( 0.0, 1.0, 0.05)};					
for (triple k : curves)
{
	real A=k.x;
	real B=k.y;
	real c_1=A;
	real c_2=(B+0.025*A)/0.32;
	real X(real t) {return exp(-0.025*t)*(c_1*cos(0.32*t)+c_2*sin(0.32*t));}
	real Y(real t) {return exp(-0.025*t)*(0.32*c_1*sin(0.32*t)+0.32*c_2*cos(0.32*t))-0.025*exp(-0.025*t)*(c_1*cos(0.32*t)+c_2*sin(0.32*t));}
	draw(graph(Y,min_x, max_x),next_color()+0.75bp);
}
limits((min_x,min_y),(max_x,max_y),Crop);
xaxis("$t$",YEquals(0),min_x,max_x,Ticks(NoZero));
yaxis("$y$",XEquals(0),min_y,max_y,Ticks(NoZero));
\end{asy}
\begin{asy}
import graph;
import fontsize;
defaultpen(fontsize(9pt));
size(115,87,IgnoreAspect);
ngraph=1000;
real min_x=0, max_x=100;
real min_y=-10, max_y=10;

int pen_pos=-1;
pen[] pens={blue, red, heavycyan, heavymagenta, lightolive};
pens.cyclic=true;

pen next_color() {return pens[++pen_pos];}

triple[] curves = {	( -1.0, 3.5, 0.05), 
					( -0.5, 2.0, 0.05), 
					( 0.0, 1.0, 0.05)};					
for (triple k : curves)
{
	real A=k.x;
	real B=k.y;
	real c_1=A;
	real c_2=(B+0.025*A)/0.32;
	real X(real t) {return exp(-0.025*t)*(c_1*cos(0.32*t)+c_2*sin(0.32*t));}
	real Y(real t) {return exp(-0.025*t)*(0.32*c_1*sin(0.32*t)+0.32*c_2*cos(0.32*t))-0.025*exp(-0.025*t)*(c_1*cos(0.32*t)+c_2*sin(0.32*t));}
	draw(graph(X,min_x, max_x),next_color()+0.75bp);
}
limits((min_x,min_y),(max_x,max_y),Crop);
xaxis("$t$",YEquals(0),min_x,max_x,Ticks(NoZero));
yaxis("$x$",XEquals(0),min_y,max_y,Ticks(NoZero));
\end{asy}
\end{center}
\end{subfigure}
\end{figure}
\end{overprint}
\end{example}
\end{frame}

\begin{frame}[fragile]
\begin{example}
Let us consider a nonautonomuous version of
\begin{equation*}
x^{\prime\prime}+0.1x=0.5\cos[t]
\qquad
x(0)=1,\quad x^\prime(0)=0
\end{equation*}
\begin{overprint}
\onslide<1>
\onslide<2-3>
This DE represents a periodically forces harmonic oscillator and has system form:
\begin{equation*}
\begin{matrix}[ll]
x^\prime=y \\
y^\prime=-0.1x+0.5\cos[t]
\end{matrix}
\quad\text{or}\quad
{\bvector{x,y}}^\prime=
\begin{bmatrix}
0 & 1 \\
-0.1 & 0 \\
\end{bmatrix}
\bvector{x,y}+\bvector{0,0.5\cos[t]}
\end{equation*}
\visible<3->{It is not easy to find an analytic solution to this DE, but we can draw the solutions using numerical calculations. 

\vspace{2mm}
We can use Euler's method, which we have seen before.}
\onslide<4>
\vspace{-2mm}
\begin{equation*}
\begin{aligned}
x_{n+1}&=x_n+h\cdot x^\prime(t_n) = x_n+h\cdot y_n\\
y_{n+1}&=y_n+h\cdot y^\prime(t_n) = y_n+h\cdot (-0.1x+0.5\cos[t_n])\\
\end{aligned}
\end{equation*}
With step size $h=0.1$, $x(0)=1$, and $y(0)=0$.

\vspace{-4mm}
\begin{center}\small
\begin{tabular}{ccccc}
$t_n$ & $x_n$ & $y_n$ & $x^\prime$ & $y^\prime$\\\hline
0.0 & 1.0000 & 0.0000 & 0.0000 & 0.4000 \\
0.1 & 1.0000 & 0.0400 & 0.0400 & 0.3975 \\
0.2 & 1.0040 & 0.0798 & 0.0798 & 0.3896 \\
0.3 & 1.0120 & 0.1187 & 0.1187 & 0.3765 \\
0.4 & 1.0238 & 0.1564 & 0.1564 & 0.3581 \\
0.5 & 1.0395 & 0.1922 & 0.1922 & 0.3348 \\
0.6 & 1.0587 & 0.2257 & 0.2257 & 0.3068 \\
0.7 & 1.0813 & 0.2563 & 0.2563 & 0.2743 \\
\vdots&\vdots&\vdots&\vdots&\vdots
\end{tabular}
\end{center}
\onslide<5>
\begin{figure}[h]
\begin{subfigure}{0.5\textwidth}
\begin{center}
\begin{asy}
import graph;
import slopefield;
import fontsize;
defaultpen(fontsize(9pt));
size(180);
ngraph=1000;
real min_x=-3, max_x=3;
real min_y=-3, max_y=3;

pair start=(min_x,min_y);
pair end=(max_x,max_y);

real m=1;
real b=0;
real k=0.1;
real Fo=0.5;
real wf=1;
real wo=sqrt(k/m);
real wdiff=wo*wo-wf*wf;
real A=(m*wdiff*Fo)/(m*m*wdiff*wdiff+b*wf*b*wf);
real B=(b*wf*Fo)/(m*m*wdiff*wdiff+b*wf*b*wf);

for(real gx=min_x+1; gx<=max_x-1; ++gx)
	draw((gx,min_y)--(gx,max_y),dotted+darkgray);
    
for(real gy=min_y+1; gy<=max_y-1; ++gy)
	draw((min_x,gy)--(max_x,gy),dotted+darkgray); 

int pen_pos=-1;
pen[] pens={blue, red, heavycyan, heavymagenta, lightolive};
pens.cyclic=true;
pen next_color() {return pens[++pen_pos];}

DefaultHead.size=new real(pen p=currentpen) {return 2.5mm;};

real t_start=0;
real t_end=1;

triple[] curves = {	( 1.0, 0.0, 0.05)};					
for (triple k : curves)
{
	real c_1=k.x - A;
	real c_2=(k.y - B)/sqrt(0.1);
	real X(real t) {return c_1*cos(sqrt(0.1)*t)+c_2*sin(sqrt(0.1)*t)+A*cos(t)+B*sin(t);}
	real Y(real t) {return -sqrt(0.1)*c_1*sin(sqrt(0.1)*t)+sqrt(0.1)*c_2*cos(sqrt(0.1)*t)-A*sin(t)+B*cos(t);}
	draw(graph(X,Y,t_start,t_end),next_color()+1.0bp);
}

limits(start,end,Crop);

xaxis("$x$",YEquals(min_y),min_x,max_x,LeftTicks());
xaxis(YEquals(max_y),min_x,max_x);
yaxis("$y$",XEquals(min_x),min_y,max_y,LeftTicks());
yaxis(XEquals(max_x),min_y,max_y);
\end{asy}
\end{center}
\end{subfigure}
\begin{subfigure}{0.45\textwidth}
\begin{center}
\begin{asy}
import graph;
import fontsize;
defaultpen(fontsize(9pt));
size(115,87,IgnoreAspect);
ngraph=1000;
real min_x=0, max_x=1;
real min_y=-3, max_y=3;

real m=1;
real b=0;
real k=0.1;
real Fo=0.5;
real wf=1;
real wo=sqrt(k/m);
real wdiff=wo*wo-wf*wf;
real A=(m*wdiff*Fo)/(m*m*wdiff*wdiff+b*wf*b*wf);
real B=(b*wf*Fo)/(m*m*wdiff*wdiff+b*wf*b*wf);

int pen_pos=-1;
pen[] pens={blue, red, heavycyan, heavymagenta, lightolive};
pens.cyclic=true;

pen next_color() {return pens[++pen_pos];}

triple[] curves = {	( 1.0, 0.0, 0.05)};					
for (triple k : curves)
{
	real c_1=k.x - A;
	real c_2=(k.y - B)/sqrt(0.1);
	real X(real t) {return c_1*cos(sqrt(0.1)*t)+c_2*sin(sqrt(0.1)*t)+A*cos(t)+B*sin(t);}
	real Y(real t) {return -sqrt(0.1)*c_1*sin(sqrt(0.1)*t)+sqrt(0.1)*c_2*cos(sqrt(0.1)*t)-A*sin(t)+B*cos(t);}

	draw(graph(Y,min_x, max_x),next_color()+0.75bp);
}
limits((min_x,min_y),(max_x,max_y),Crop);
xaxis("$t$",YEquals(0),min_x,max_x,Ticks(NoZero));
yaxis("$y$",XEquals(0),min_y,max_y,Ticks(NoZero));
\end{asy}
\begin{asy}
import graph;
import fontsize;
defaultpen(fontsize(9pt));
size(115,87,IgnoreAspect);
ngraph=1000;
real min_x=0, max_x=1;
real min_y=-3, max_y=3;

real m=1;
real b=0;
real k=0.1;
real Fo=0.5;
real wf=1;
real wo=sqrt(k/m);
real wdiff=wo*wo-wf*wf;
real A=(m*wdiff*Fo)/(m*m*wdiff*wdiff+b*wf*b*wf);
real B=(b*wf*Fo)/(m*m*wdiff*wdiff+b*wf*b*wf);

int pen_pos=-1;
pen[] pens={blue, red, heavycyan, heavymagenta, lightolive};
pens.cyclic=true;

pen next_color() {return pens[++pen_pos];}

triple[] curves = {	( 1.0, 0.0, 0.05)};					
for (triple k : curves)
{
	real c_1=k.x - A;
	real c_2=(k.y - B)/sqrt(0.1);
	real X(real t) {return c_1*cos(sqrt(0.1)*t)+c_2*sin(sqrt(0.1)*t)+A*cos(t)+B*sin(t);}
	real Y(real t) {return -sqrt(0.1)*c_1*sin(sqrt(0.1)*t)+sqrt(0.1)*c_2*cos(sqrt(0.1)*t)-A*sin(t)+B*cos(t);}

	draw(graph(X,min_x, max_x),next_color()+0.75bp);
}
limits((min_x,min_y),(max_x,max_y),Crop);
xaxis("$t$",YEquals(0),min_x,max_x,Ticks(NoZero));
yaxis("$x$",XEquals(0),min_y,max_y,Ticks(NoZero));
\end{asy}
\end{center}
\end{subfigure}
\end{figure}
\onslide<6>
\begin{figure}[h]
\begin{subfigure}{0.5\textwidth}
\begin{center}
\begin{asy}
import graph;
import slopefield;
import fontsize;
defaultpen(fontsize(9pt));
size(180);
ngraph=1000;
real min_x=-3, max_x=3;
real min_y=-3, max_y=3;

pair start=(min_x,min_y);
pair end=(max_x,max_y);

real m=1;
real b=0;
real k=0.1;
real Fo=0.5;
real wf=1;
real wo=sqrt(k/m);
real wdiff=wo*wo-wf*wf;
real A=(m*wdiff*Fo)/(m*m*wdiff*wdiff+b*wf*b*wf);
real B=(b*wf*Fo)/(m*m*wdiff*wdiff+b*wf*b*wf);

for(real gx=min_x+1; gx<=max_x-1; ++gx)
	draw((gx,min_y)--(gx,max_y),dotted+darkgray);
    
for(real gy=min_y+1; gy<=max_y-1; ++gy)
	draw((min_x,gy)--(max_x,gy),dotted+darkgray); 

int pen_pos=-1;
pen[] pens={blue, red, heavycyan, heavymagenta, lightolive};
pens.cyclic=true;
pen next_color() {return pens[++pen_pos];}

DefaultHead.size=new real(pen p=currentpen) {return 2.5mm;};

real t_start=0;
real t_end=35;

triple[] curves = {	( 1.0, 0.0, 0.05)};					
for (triple k : curves)
{
	real c_1=k.x - A;
	real c_2=(k.y - B)/sqrt(0.1);
	real X(real t) {return c_1*cos(sqrt(0.1)*t)+c_2*sin(sqrt(0.1)*t)+A*cos(t)+B*sin(t);}
	real Y(real t) {return -sqrt(0.1)*c_1*sin(sqrt(0.1)*t)+sqrt(0.1)*c_2*cos(sqrt(0.1)*t)-A*sin(t)+B*cos(t);}
	draw(graph(X,Y,t_start,t_end),next_color()+1.0bp);
}

limits(start,end,Crop);

xaxis("$x$",YEquals(min_y),min_x,max_x,LeftTicks());
xaxis(YEquals(max_y),min_x,max_x);
yaxis("$y$",XEquals(min_x),min_y,max_y,LeftTicks());
yaxis(XEquals(max_x),min_y,max_y);
\end{asy}
\end{center}
\end{subfigure}
\begin{subfigure}{0.45\textwidth}
\begin{center}
\begin{asy}
import graph;
import fontsize;
defaultpen(fontsize(9pt));
size(115,87,IgnoreAspect);
ngraph=1000;
real min_x=0, max_x=35;
real min_y=-3, max_y=3;

real m=1;
real b=0;
real k=0.1;
real Fo=0.5;
real wf=1;
real wo=sqrt(k/m);
real wdiff=wo*wo-wf*wf;
real A=(m*wdiff*Fo)/(m*m*wdiff*wdiff+b*wf*b*wf);
real B=(b*wf*Fo)/(m*m*wdiff*wdiff+b*wf*b*wf);

int pen_pos=-1;
pen[] pens={blue, red, heavycyan, heavymagenta, lightolive};
pens.cyclic=true;

pen next_color() {return pens[++pen_pos];}

triple[] curves = {	( 1.0, 0.0, 0.05)};					
for (triple k : curves)
{
	real c_1=k.x - A;
	real c_2=(k.y - B)/sqrt(0.1);
	real X(real t) {return c_1*cos(sqrt(0.1)*t)+c_2*sin(sqrt(0.1)*t)+A*cos(t)+B*sin(t);}
	real Y(real t) {return -sqrt(0.1)*c_1*sin(sqrt(0.1)*t)+sqrt(0.1)*c_2*cos(sqrt(0.1)*t)-A*sin(t)+B*cos(t);}

	draw(graph(Y,min_x, max_x),next_color()+0.75bp);
}
limits((min_x,min_y),(max_x,max_y),Crop);
xaxis("$t$",YEquals(0),min_x,max_x,Ticks(NoZero));
yaxis("$y$",XEquals(0),min_y,max_y,Ticks(NoZero));
\end{asy}
\begin{asy}
import graph;
import fontsize;
defaultpen(fontsize(9pt));
size(115,87,IgnoreAspect);
ngraph=1000;
real min_x=0, max_x=35;
real min_y=-3, max_y=3;

real m=1;
real b=0;
real k=0.1;
real Fo=0.5;
real wf=1;
real wo=sqrt(k/m);
real wdiff=wo*wo-wf*wf;
real A=(m*wdiff*Fo)/(m*m*wdiff*wdiff+b*wf*b*wf);
real B=(b*wf*Fo)/(m*m*wdiff*wdiff+b*wf*b*wf);

int pen_pos=-1;
pen[] pens={blue, red, heavycyan, heavymagenta, lightolive};
pens.cyclic=true;

pen next_color() {return pens[++pen_pos];}

triple[] curves = {	( 1.0, 0.0, 0.05)};					
for (triple k : curves)
{
	real c_1=k.x - A;
	real c_2=(k.y - B)/sqrt(0.1);
	real X(real t) {return c_1*cos(sqrt(0.1)*t)+c_2*sin(sqrt(0.1)*t)+A*cos(t)+B*sin(t);}
	real Y(real t) {return -sqrt(0.1)*c_1*sin(sqrt(0.1)*t)+sqrt(0.1)*c_2*cos(sqrt(0.1)*t)-A*sin(t)+B*cos(t);}

	draw(graph(X,min_x, max_x),next_color()+0.75bp);
}
limits((min_x,min_y),(max_x,max_y),Crop);
xaxis("$t$",YEquals(0),min_x,max_x,Ticks(NoZero));
yaxis("$x$",XEquals(0),min_y,max_y,Ticks(NoZero));
\end{asy}
\end{center}
\end{subfigure}
\end{figure}
\onslide<7>
\begin{figure}[h]
\begin{subfigure}{0.5\textwidth}
\begin{center}
\begin{asy}
import graph;
import slopefield;
import fontsize;
defaultpen(fontsize(9pt));
size(180);
ngraph=1000;
real min_x=-3, max_x=3;
real min_y=-3, max_y=3;

pair start=(min_x,min_y);
pair end=(max_x,max_y);

real m=1;
real b=0;
real k=0.1;
real Fo=0.5;
real wf=1;
real wo=sqrt(k/m);
real wdiff=wo*wo-wf*wf;
real A=(m*wdiff*Fo)/(m*m*wdiff*wdiff+b*wf*b*wf);
real B=(b*wf*Fo)/(m*m*wdiff*wdiff+b*wf*b*wf);

for(real gx=min_x+1; gx<=max_x-1; ++gx)
	draw((gx,min_y)--(gx,max_y),dotted+darkgray);
    
for(real gy=min_y+1; gy<=max_y-1; ++gy)
	draw((min_x,gy)--(max_x,gy),dotted+darkgray); 

int pen_pos=-1;
pen[] pens={blue, red, heavycyan, heavymagenta, lightolive};
pens.cyclic=true;
pen next_color() {return pens[++pen_pos];}

DefaultHead.size=new real(pen p=currentpen) {return 2.5mm;};

real t_start=0;
real t_end=100;

triple[] curves = {	( 1.0, 0.0, 0.05)};					
for (triple k : curves)
{
	real c_1=k.x - A;
	real c_2=(k.y - B)/sqrt(0.1);
	real X(real t) {return c_1*cos(sqrt(0.1)*t)+c_2*sin(sqrt(0.1)*t)+A*cos(t)+B*sin(t);}
	real Y(real t) {return -sqrt(0.1)*c_1*sin(sqrt(0.1)*t)+sqrt(0.1)*c_2*cos(sqrt(0.1)*t)-A*sin(t)+B*cos(t);}
	draw(graph(X,Y,t_start,t_end),next_color()+1.0bp);
}

limits(start,end,Crop);

xaxis("$x$",YEquals(min_y),min_x,max_x,LeftTicks());
xaxis(YEquals(max_y),min_x,max_x);
yaxis("$y$",XEquals(min_x),min_y,max_y,LeftTicks());
yaxis(XEquals(max_x),min_y,max_y);
\end{asy}
\end{center}
\end{subfigure}
\begin{subfigure}{0.45\textwidth}
\begin{center}
\begin{asy}
import graph;
import fontsize;
defaultpen(fontsize(9pt));
size(115,87,IgnoreAspect);
ngraph=1000;
real min_x=0, max_x=100;
real min_y=-3, max_y=3;

real m=1;
real b=0;
real k=0.1;
real Fo=0.5;
real wf=1;
real wo=sqrt(k/m);
real wdiff=wo*wo-wf*wf;
real A=(m*wdiff*Fo)/(m*m*wdiff*wdiff+b*wf*b*wf);
real B=(b*wf*Fo)/(m*m*wdiff*wdiff+b*wf*b*wf);

int pen_pos=-1;
pen[] pens={blue, red, heavycyan, heavymagenta, lightolive};
pens.cyclic=true;

pen next_color() {return pens[++pen_pos];}

triple[] curves = {	( 1.0, 0.0, 0.05)};					
for (triple k : curves)
{
	real c_1=k.x - A;
	real c_2=(k.y - B)/sqrt(0.1);
	real X(real t) {return c_1*cos(sqrt(0.1)*t)+c_2*sin(sqrt(0.1)*t)+A*cos(t)+B*sin(t);}
	real Y(real t) {return -sqrt(0.1)*c_1*sin(sqrt(0.1)*t)+sqrt(0.1)*c_2*cos(sqrt(0.1)*t)-A*sin(t)+B*cos(t);}

	draw(graph(Y,min_x, max_x),next_color()+0.75bp);
}
limits((min_x,min_y),(max_x,max_y),Crop);
xaxis("$t$",YEquals(0),min_x,max_x,Ticks(NoZero));
yaxis("$y$",XEquals(0),min_y,max_y,Ticks(NoZero));
\end{asy}
\begin{asy}
import graph;
import fontsize;
defaultpen(fontsize(9pt));
size(115,87,IgnoreAspect);
ngraph=1000;
real min_x=0, max_x=100;
real min_y=-3, max_y=3;

real m=1;
real b=0;
real k=0.1;
real Fo=0.5;
real wf=1;
real wo=sqrt(k/m);
real wdiff=wo*wo-wf*wf;
real A=(m*wdiff*Fo)/(m*m*wdiff*wdiff+b*wf*b*wf);
real B=(b*wf*Fo)/(m*m*wdiff*wdiff+b*wf*b*wf);

int pen_pos=-1;
pen[] pens={blue, red, heavycyan, heavymagenta, lightolive};
pens.cyclic=true;

pen next_color() {return pens[++pen_pos];}

triple[] curves = {	( 1.0, 0.0, 0.05)};					
for (triple k : curves)
{
	real c_1=k.x - A;
	real c_2=(k.y - B)/sqrt(0.1);
	real X(real t) {return c_1*cos(sqrt(0.1)*t)+c_2*sin(sqrt(0.1)*t)+A*cos(t)+B*sin(t);}
	real Y(real t) {return -sqrt(0.1)*c_1*sin(sqrt(0.1)*t)+sqrt(0.1)*c_2*cos(sqrt(0.1)*t)-A*sin(t)+B*cos(t);}

	draw(graph(X,min_x, max_x),next_color()+0.75bp);
}
limits((min_x,min_y),(max_x,max_y),Crop);
xaxis("$t$",YEquals(0),min_x,max_x,Ticks(NoZero));
yaxis("$x$",XEquals(0),min_y,max_y,Ticks(NoZero));
\end{asy}
\end{center}
\end{subfigure}
\end{figure}
\end{overprint}
\end{example}
\end{frame}

\begin{frame}
\begin{block}{Graphical Properties of Uniqueness}
\begin{itemize}[<+- | alert@+>]
\item For a linear system of differential equations in $\R^n$, solutions so not cross in $t,x_1,x_2,\ldots,x_n$-space (i.e. $\R^{n+1}$).
\item For an \emph{autonomous} linear system in $\R^n$, trajectories \emph{also} do not cross in $x_1,x_2,\ldots,x_n$-space (i.e. $\R^n$).
\end{itemize}
\end{block}
\end{frame}
\end{document}