\documentclass{beamer}

\usepackage[english]{babel}
\usepackage[utf8x]{inputenc}
\usepackage[inline]{asymptote}
\usepackage{slide_helper}
\usepackage{caption}
\usepackage{subcaption}

\begin{asydef}
import graph;
import slopefield;
import fontsize;
defaultpen(fontsize(9pt));
ngraph=1000;

int pen_pos=-1;
pen[] pens={blue, red, heavycyan, heavymagenta, lightolive};
pens.cyclic=true;
pen next_color() {return pens[++pen_pos];}

DefaultHead.size=new real(pen p=currentpen) {return 2.5mm;};

void drawgrid(real xmin, real xmax, real ymin, real ymax, pen p=dotted+darkgray)
{
	for(real gx=xmin+1; gx<=xmax-1; ++gx)
		draw((gx,ymin)--(gx,ymax),p);
    
	for(real gy=ymin+1; gy<=ymax-1; ++gy)
		draw((xmin,gy)--(xmax,gy),p); 
}

real length(pair z) {return (z.x == 0) && (z.y == 0) ? 0.0001 : sqrt(z.x*z.x+z.y*z.y);}

path eigenpath(real lambda, pair v)
{
	if (lambda >= 0)
		return (0,0)--v;
	else
		return v--(0,0);
}

void drawequilibrium(pair c, real r, bool stable)
{
	if (stable == true)
		fill(circle(c,r),black);
	else
		unfill(circle(c,r));

	draw(circle(c,r),black);
}
\end{asydef}

\newcommand{\xReal}{\vect{x}_{\text{Re}}}
\newcommand{\xImag}{\vect{x}_{\text{Im}}}

\title[MATH 2250 - Section 6.3]{Linear Systems with Nonreal Eigenvalues}

\begin{document}

\begin{frame}
  \titlepage
\end{frame}

\begin{frame}
\begin{block}{Complex Eigenvalues and Eigenvectors}
For a real matrix $\mat{A}$, nonreal eigenvalues come in complex conjugate pairs,
\begin{equation*}
\lambda_1=\alpha+\beta i
\quad\text{and}\quad
\lambda_2=\alpha-\beta i
\end{equation*}
with $\alpha,\beta\in\R$ and $\beta\neq0$.

\vspace{2mm}
The corresponding eigenvectors are also complex conjugate pairs and can be written
\begin{equation*}
\vect{v_1}=\vect{p}+\vect{q}i
\quad\text{and}\quad
\vect{v_2}=\vect{p}-\vect{q}i
\end{equation*}
where $\vect{p}$ and $\vect{q}$ are real vectors.
\end{block}\pause
\begin{block}{Note}
We only need to find one eigenvalue/eigenvector pair.
\end{block}
\end{frame}

\begin{frame}
\begin{example}
Consider the matrix
\begin{equation*}
\mat{A}=
\begin{bmatrix}[rr]
6 & -1 \\
5 &  4 \\
\end{bmatrix}
\end{equation*}\pause
The characteristic equation is:
\begin{equation*}
(6-\lambda)(4-\lambda)+5=0\pause
\quad\rightarrow\quad
\lambda=5\pm 2i
\end{equation*}\pause
The eigenvectors are
\begin{equation*}
\vect{v_1}=\bvector{1,1-2i}
\quad\text{and}\quad
\vect{v_2}=\overline{\vect{v_1}}=\bvector{1,1+2i}
\end{equation*}\pause
Alternately, we can write
\begin{equation*}
\vect{v}=\underbrace{\bvector{1,1}}_{\vect{p}}\pm i\underbrace{\bvector{\+0,-2}}_{\vect{q}}
\end{equation*}
\end{example}
\end{frame}

\begin{frame}
\begin{example}
Consider the DE system:
\begin{equation*}
\vect{x}^\prime=\mat{A}\vect{x}
\end{equation*}
\begin{overprint}
\onslide<1>
\onslide<2-4>
Which has nonreal eigenvalues $\lambda_1,\lambda_2=\alpha\pm\beta i$ and corresponding eigenvectors $\vect{v_1}$ and $\vect{v_2}$. We can then write:
\begin{equation*}
\vect{x}=c_1 e^{\lambda_1 t}\vect{v_1}+c_2 e^{\lambda_2 t}\vect{v_2}.
\end{equation*}
\visible<2->{However, we want this solution in terms of the real vectors $\vect{p}$ and $\vect{q}$.}

\vspace{2mm}
\visible<3->{So, for eigenvalue $\lambda_1=\alpha+\beta i$ and corresponding eigenvector $\vect{v_1}=\vect{p}+\vect{q}i$ we get the solution
\begin{equation*}
\vect{x_1}(t)=e^{\lambda_1 t}\vect{v_1}=e^{\alpha+\beta i}\left(\vect{p}+\vect{q}i\right)
\end{equation*}}

\visible<4->{\vspace{-5mm}
Just like with second-order systems, we shall find that the real and imaginary parts of the complex solution above are both real and linearly independent solutions of the system.}
\onslide<5-11>
Suppose that
\begin{equation*}
\vect{x}(t)=\xReal(t)+\xImag(t)
\end{equation*}
is a complex vector solution to the system, with $\xImag\neq\vect{0}$.

\vspace{2mm}
\visible<6->{Then
\begin{equation*}
\visible<6-9>{\vect{x}^{\prime}}
\visible<7-9>{=}
\visible<7->{\xReal^\prime(t)+i \xImag^\prime(t)}
\visible<8->{=\mat{A}\xReal(t)+i\mat{A}\xImag(t)}
\visible<9>{=\mat{A}\vect{x}(t)}
\end{equation*}}

\visible<10->{\vspace{-2mm}
Separately equating the real and imaginary parts, we get:
\begin{equation*}
\xReal^\prime(t)=\mat{A}\xReal(t)
\quad\text{and}\quad
\xImag^\prime(t)=\mat{A}\xImag(t)
\end{equation*}}

\visible<11->{\vspace{-2mm}
Thus, $\xReal(t)$ and $\xImag(t)$ are separate real solutions to the system.}
\onslide<12-18>
For the complex solution
\begin{equation*}
\vect{x_1}(t)=e^{\lambda_1 t}\vect{v_1}=e^{\alpha+\beta i}\left(\vect{p}+\vect{q}i\right)
\end{equation*}
we can determine the real and imaginary parts by using Euler's formula:
\begin{equation*}
e^{i\theta}=\cos[\theta]+i\sin[\theta]
\end{equation*}

\visible<13->{\vspace{-6mm}
to write:
\begin{equation*}
\begin{aligned}
e^{\lambda_1 t}\vect{v_1}&=e^{\alpha t+\beta t i}\left(\vect{p}+\vect{q}i\right)\\
\visible<14->{&=e^{\alpha t}e^{\beta t i}\left(\vect{p}+\vect{q}i\right)}\\
\visible<15->{&=e^{\alpha t}\left(\cos[\beta t]+i\sin[\beta t]\right)\left(\vect{p}+\vect{q}i\right)}\\
\visible<16->{&=e^{\alpha t}\left(\cos[\beta t]\left(\vect{p}+\vect{q}i\right)+i\sin[\beta t]\left(\vect{p}+\vect{q}i\right)\right)}\\
&\vphantom{=\underbrace{e^{\alpha t}\left(\cos[\beta t]\vect{p}-\sin[\beta t]\vect{q}\right)}_{\xReal(t)}+i \underbrace{e^{\alpha t}\left(\sin[\beta t]\vect{p}+\cos[\beta t]\vect{q} \right)}_{\xImag(t)}}
\temporal<17>{\phantom{{=e^{\alpha t}\left(\cos[\beta t]\vect{p}-\sin[\beta t]\vect{q}\right)+i e^{\alpha t}\left(\sin[\beta t]\vect{p}+\cos[\beta t]\vect{q} \right)}}}%
{=e^{\alpha t}\left(\cos[\beta t]\vect{p}-\sin[\beta t]\vect{q}\right)+i e^{\alpha t}\left(\sin[\beta t]\vect{p}+\cos[\beta t]\vect{q} \right)}%
{=\underbrace{e^{\alpha t}\left(\cos[\beta t]\vect{p}-\sin[\beta t]\vect{q}\right)}_{\xReal(t)}+i \underbrace{e^{\alpha t}\left(\sin[\beta t]\vect{p}+\cos[\beta t]\vect{q} \right)}_{\xImag(t)}}\\
\end{aligned}
\end{equation*}}
\onslide<19->
Since $\xReal(t)$ and $\xImag(t)$ are linearly independent solutions they must span the solution space. Thus, the general solution, for $c_1,c_2\in\R$, is
\begin{equation*}
\vect{x}(t)=c_1\xReal(t)+c_2\xImag(t)
\end{equation*}
\visible<20->{Any solutions derived from $\lambda_2$ and $\vect{v_2}$ will be linear combinations \\ of $\xReal(t)$ and $\xImag(t)$.}
\end{overprint}
\vspace{-5mm}
\end{example}
\end{frame}

\begin{frame}
\begin{block}{Solving $2\by 2$ DE System with Nonreal Eigenvalues}
For the two-dimensional linear homogeneous differential equation $\vect{x}^\prime=\mat{A}\vect{x}$ with real matrix $\mat{A}$, eigenvalues $\lambda_1,\lambda_2=\alpha\pm\beta$ ($\beta\neq 0$) the general solution can be found using the following steps:
\onslide<+->
\begin{enumerate}[<+- | alert@+>]
\item For one eigenvector $\lambda_1$, find it's corresponding eigenvector $\vect{v_1}$. The second eigenvalue $\lambda_2$ and it's eigenvector $\vect{v_2}$ are complex conjugates of the first. The eigenvectors are of the form $\vect{v_1},\vect{v_2}=\vect{p}\pm i\vect{q}$.
\item Construxt the linearly independent real ($\xReal$) and imaginary ($\xImag$) parts of the solutions as follows:
\begin{equation*}
\begin{aligned}
\xReal(t) &= e^{\alpha t}\left(\cos[\beta t]\vect{p}-\sin[\beta t]\vect{q} \right)\\
\xImag(t) &= e^{\alpha t}\left(\sin[\beta t]\vect{p}+\cos[\beta t]\vect{q} \right) \\
\end{aligned}
\end{equation*}
\item The general solution is
\begin{equation*}
\vect{x}(t)=c_1\xReal(t)+c_2\xImag(t)
\end{equation*}
\end{enumerate}
\end{block}
\end{frame}

\begin{frame}[fragile]
\begin{example}
\begin{overprint}
\onslide<1-4>
Consider the system
\begin{equation*}
\vect{x}^\prime=\mat{A}\vect{x}=
\begin{bmatrix}[rr]
6 & -1 \\
5 &  4 \\
\end{bmatrix}\vect{x}
\end{equation*}
\visible<2->{The eigenvalues are $\lambda_1,\lambda_2=5\pm 2i$ and
\begin{equation*}
\vect{v_1}=\bvector{1,1}+i\bvector{\+0,-2}
\end{equation*}}

\vspace{-5mm}
\visible<3->{Thus

\vspace{-8mm}
\begin{equation*}
\begin{aligned}
\xReal(t)&=e^{5t}\cos[2t]\bvector{1,1}-e^{5t}\sin[2t]\bvector{\+0,-2}=e^{5t}\bvector{\cos[2t],\cos[2t]+2\sin[2t]}\\
\xImag(t)&=e^{5t}\sin[2t]\bvector{1,1}+e^{5t}\cos[2t]\bvector{\+0,-2}=e^{5t}\bvector{\sin[2t],\sin[2t]-2\cos[2t]}\\
\end{aligned}
\end{equation*}}

\visible<4->{\vspace{-2mm}
And general solution
\begin{equation*}
\vect{x}(t)=e^{5t}\left(c_1 \bvector{\cos[2t],\cos[2t]+2\sin[2t]}+c_2 \bvector{\sin[2t],\sin[2t]-2\cos[2t]}\right)
\end{equation*}}
\onslide<5->
\begin{center}
\begin{asy}
size(235);

real min_x=-4, max_x=4;
real min_y=-4, max_y=4;

pair start=(min_x,min_y);
pair end=(max_x,max_y);

// eigenvectors and eigenvalues
pair p=(1,1);
pair q=(0,-2);
real alpha=5;
real beta=2;

real Xreal (real t) { return exp(alpha*t)*(cos(beta*t)*p.x-sin(beta*t)*q.x);}
real Yreal (real t) { return exp(alpha*t)*(cos(beta*t)*p.y-sin(beta*t)*q.y);}

real Ximag (real t) { return exp(alpha*t)*(sin(beta*t)*p.x+cos(beta*t)*q.x);}
real Yimag (real t) { return exp(alpha*t)*(sin(beta*t)*p.y+cos(beta*t)*q.y);}

// draw some trajectories
real t_start=-1.5;
real t_end=1.0;

triple[] curves = {	(  0.1,  0.0, 0.08), 
					(  1.5,  0.0, 0.003), 
					(  2.9,  0.0, 0.003), 
					( -0.2,  0.0, 0.01), 
					( -1.5,  0.0, 0.003), 
					( -3.0,  0.0, 0.003), 
					(  0.0,  1.5, 0.01), 
					(  0.0,  2.9, 0.005), 
					(  1.0,  0.1, 0.0101), 
					(  0.0, -0.2, 0.06), 
					(  0.0, -1.5, 0.01), 
					(  0.0, -3.0, 0.005)};
					
for (triple k : curves)
{
	real A=k.x;
	real B=k.y;
	real c_1=A;
	real c_2=A-B/2;
	real X(real t) {return c_1*Xreal(t)+c_2*Ximag(t);}
	real Y(real t) {return c_1*Yreal(t)+c_2*Yimag(t);}
	draw(graph(X,Y,t_start,t_end),next_color()+1.0bp,Arrow(Relative(k.z)));
}

limits(start,end,Crop);

// draw axes
xaxis("$x$", YEquals(0), min_x, max_x, Ticks(OmitFormat(0,1)));
yaxis("$y$", XEquals(0), min_y, max_y, Ticks(OmitFormat(0,4,-4)));

// draw origin stability
drawequilibrium((0,0),0.1,false);
\end{asy}
\end{center}
\end{overprint}
\vspace{-81mm}
\end{example}
\end{frame}

\begin{frame}[fragile]
\begin{example}
\begin{overprint}
\onslide<1-4>
Consider the system
\begin{equation*}
\vect{x}^\prime=\mat{A}\vect{x}=
\begin{bmatrix}[rr]
 0 &  1 \\
-5 & -2 \\
\end{bmatrix}\vect{x}
\end{equation*}
\visible<2->{The eigenvalues are $\lambda_1,\lambda_2=-1\pm 2i$ and
\begin{equation*}
\vect{v_1}=\bvector{\+1,-1}+i\bvector{0,2}
\end{equation*}}

\vspace{-4mm}
\visible<3->{Thus

\vspace{-4mm}
\begin{equation*}
\begin{aligned}
\xReal(t)&=e^{-t}\cos[2t]\bvector{\+1,-1}-e^{-t}\sin[2t]\bvector{0,2}\\
\xImag(t)&=e^{-t}\sin[2t]\bvector{\+1,-1}+e^{-t}\cos[2t]\bvector{0,2}\\
\end{aligned}
\end{equation*}}

\visible<4->{\vspace{-2mm}
And general solution
\begin{equation*}
\vect{x}(t)=e^{-t}\left(c_1 \bvector{\cos[2t],-\cos[2t]-2\sin[2t]}+c_2 \bvector{\sin[2t],-\sin[2t]+2\cos[2t]}\right)
\end{equation*}}
\onslide<5->
\begin{center}
\begin{asy}
size(235);

real min_x=-4, max_x=4;
real min_y=-4, max_y=4;

pair start=(min_x,min_y);
pair end=(max_x,max_y);

// eigenvectors and eigenvalues
pair p=(1,-1);
pair q=(0,2);
real alpha=-1;
real beta=2;

real Xreal (real t) { return exp(alpha*t)*(cos(beta*t)*p.x-sin(beta*t)*q.x);}
real Yreal (real t) { return exp(alpha*t)*(cos(beta*t)*p.y-sin(beta*t)*q.y);}

real Ximag (real t) { return exp(alpha*t)*(sin(beta*t)*p.x+cos(beta*t)*q.x);}
real Yimag (real t) { return exp(alpha*t)*(sin(beta*t)*p.y+cos(beta*t)*q.y);}

// draw some trajectories
real t_start=-1.0;
real t_end=4.0;

triple[] curves = {	(  1.1,  0.0, 0.6), 
					(  2.1,  0.0, 0.6), 
					( -1.0,  0.0, 0.6), 
					( -2.0,  0.0, 0.6)};
					
for (triple k : curves)
{
	real A=k.x;
	real B=k.y;
	real c_1=A;
	real c_2=A-B/2;
	real X(real t) {return c_1*Xreal(t)+c_2*Ximag(t);}
	real Y(real t) {return c_1*Yreal(t)+c_2*Yimag(t);}
	draw(graph(X,Y,t_start,t_end),next_color()+1.0bp,Arrow(Relative(k.z)));
}

limits(start,end,Crop);

// draw axes
xaxis("$x$", YEquals(0), min_x, max_x, Ticks(OmitFormat(0,1)));
yaxis("$y$", XEquals(0), min_y, max_y, Ticks(OmitFormat(0,4,-4)));

// draw origin stability
drawequilibrium((0,0),0.11,true);
\end{asy}
\end{center}
\end{overprint}
\vspace{-81mm}
\end{example}
\end{frame}

\begin{frame}[fragile]
\begin{example}
\begin{overprint}
\onslide<1-4>
Consider the system
\begin{equation*}
\vect{x}^\prime=\mat{A}\vect{x}=
\begin{bmatrix}[rr]
 4 & -5 \\
 5 & -4 \\
\end{bmatrix}\vect{x}
\end{equation*}
\visible<2->{The eigenvalues are $\lambda_1,\lambda_2=0\pm 3i$ and
\begin{equation*}
\vect{v_1}=\bvector{5,4}+i\bvector{\+0,-3}
\end{equation*}}

\vspace{-4mm}
\visible<3->{Thus

\vspace{-4mm}
\begin{equation*}
\begin{aligned}
\xReal(t)&=\cos[3t]\bvector{5,4}-\sin[3t]\bvector{\+0,-3}=\bvector{5\cos[3t],4\cos[3t]+3\sin[3t]}\\
\xImag(t)&=\sin[3t]\bvector{5,4}+\cos[3t]\bvector{\+0,-3}=\bvector{5\sin[3t],4\sin[3t]-3\cos[3t]}\\
\end{aligned}
\end{equation*}}

\visible<4->{\vspace{-2mm}
And general solution
\begin{equation*}
\vect{x}(t)=c_1 \bvector{5\cos[3t],4\cos[3t]+3\sin[3t]}+c_2 \bvector{5\sin[3t],4\sin[3t]-3\cos[3t]}
\end{equation*}}
\onslide<5->
\begin{center}
\begin{asy}
size(235);

real min_x=-4, max_x=4;
real min_y=-4, max_y=4;

pair start=(min_x,min_y);
pair end=(max_x,max_y);

// eigenvectors and eigenvalues
pair p=(5,4);
pair q=(0,-3);
real alpha=0;
real beta=3;

real Xreal (real t) { return exp(alpha*t)*(cos(beta*t)*p.x-sin(beta*t)*q.x);}
real Yreal (real t) { return exp(alpha*t)*(cos(beta*t)*p.y-sin(beta*t)*q.y);}

real Ximag (real t) { return exp(alpha*t)*(sin(beta*t)*p.x+cos(beta*t)*q.x);}
real Yimag (real t) { return exp(alpha*t)*(sin(beta*t)*p.y+cos(beta*t)*q.y);}

// draw some trajectories
real t_start=-1.0;
real t_end=4.0;

triple[] curves = {	(  0.4,  1.6, 0.1), 
					(  0.3,  1.3, 0.25), 
					(  0.2,  1.0, 0.3), 
					(  0.1,  0.5, 0.5)};
					
for (triple k : curves)
{
	real A=k.x;
	real B=k.y;
	real c_1=A;
	real c_2=A-B/2;
	real X(real t) {return c_1*Xreal(t)+c_2*Ximag(t);}
	real Y(real t) {return c_1*Yreal(t)+c_2*Yimag(t);}
	draw(graph(X,Y,t_start,t_end),next_color()+1.0bp,Arrow(Relative(k.z)));
}

limits(start,end,Crop);

// draw axes
xaxis("$x$", YEquals(0), min_x, max_x, Ticks(OmitFormat(0,1)));
yaxis("$y$", XEquals(0), min_y, max_y, Ticks(OmitFormat(0,4,-4)));

// draw origin stability
drawequilibrium((0,0),0.11,true);
\end{asy}
\end{center}
\end{overprint}
\vspace{-81mm}
\end{example}
\end{frame}

\begin{frame}
\begin{block}{Behavior of Solutions}
\begin{itemize}[<+- | alert@+>]
\item An \textbf{unstable equilibrium} is one where all solutions spiral away from the origin. (Since $\alpha>0$.)
\item A \textbf{asymptotically stable equilibrium} is one where all solutions spiral towards the origin. (Since $\alpha<0$.) Technically they never reach zero, because the origin is a separate, fixed-point solution.
\item An \textbf{stable equilibrium} is one where the trajectories neither grow nor decay, they just circle in a periodic motion. (Since $\alpha=0$.)
\end{itemize}
\end{block}
\end{frame}

\begin{frame}
\begin{block}{Nullclines for a DE System}
For a two-dimensional system
\begin{equation*}
\begin{aligned}
x^\prime &= f(x,y) \\
y^\prime &= g (x,y) \\
\end{aligned}
\end{equation*}
\begin{itemize}
\item The \textbf{$v$-nullcline} is the set of all points with vertical slope, which occur on the curve obtained by solving $x^\prime=f (x,y)=0$.
\item The \textbf{$h$-nullcline} is the set of all points with horizontal slope, which occur on the curve obtained by solving $y^\prime=g (x,y)=0$.
\end{itemize}
When an $h$-nullcline and an $v$-nullcline intersect, an \textbf{equilibrium} occurs.
\end{block}
\end{frame}

\begin{frame}
\begin{block}{Interpreting the Solutions}
\onslide<+->
For $\vect{x}^\prime=\mat{A}\vect{x}$ with nonreal eigenvalues $\lambda_1,\lambda_2=\alpha\pm\beta i$ and comple eigenvectors $\vect{v_1},\vect{v_2}=\vect{p}+\vect{q}i$, arrange the components of the solution as
\begin{equation*}
\bvector{\xReal,\xImag}=
\underbrace{%
\vphantom{%
\begin{bmatrix}[rr]
\cos[\beta t] & -\sin[\beta t] \\
\sin[\beta t] &  \cos[\beta t] \\
\end{bmatrix}}
e^{\alpha t}}_{\text{expansion}}
\underbrace{%
\begin{bmatrix}[rr]
\cos[\beta t] & -\sin[\beta t] \\
\sin[\beta t] &  \cos[\beta t] \\
\end{bmatrix}}_{\text{rotation}}
\underbrace{\bvector{\vect{p},\vect{q}}}_{\text{tilt and shape}}
\end{equation*}
\begin{enumerate}[<+- | alert@+>]
\item The first factor $e^{\alpha t}$ determines \emph{expansion} or \emph{contraction}.
\begin{itemize}[<.->]
\item If $\alpha>0$, then trajectories spiral outward, representing unbound growth.
\item If $\alpha<0$, then trajectories spiral inward, decay to zero.
\item If $\alpha=0$, then trajectories are closed loops, representing periodic solutions.
\end{itemize}
\item The second factor is the rotation matrix, which describes the spiral. The angle of rotation $\beta t$ is ever growing.
\item The third factor, containing $\vect{p}$ and $\vect{q}$, determines the \emph{tilt} and \emph{shape} of the \emph{elliptical trajectories} that would result with $\alpha=0$.
\end{enumerate}
\end{block}
\end{frame}
\end{document}