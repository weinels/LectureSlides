\documentclass{beamer}

\usepackage[english]{babel}
\usepackage[utf8x]{inputenc}
\usepackage{slide_helper}
\usepackage{graphicx}

\usepackage[customcolors, beamer]{hf-tikz}
\tikzset{style green/.style={
    set fill color=green!60!lime!50,
    set border color=green!100,
  },
  style cyan/.style={
    set fill color=cyan!50!blue!40,
    set border color=cyan!100,
  },
  style orange/.style={
    set fill color=orange!50!red!60,
    set border color=red!80,
  },
  hor/.style={
    above left offset={-0.15,0.35},
    below right offset={0.15,-0.125},
    #1
  },
  ver/.style={
    above left offset={-0.38,0.35},
    below right offset={0.15,-0.15},
    #1
  },
  ver2/.style={
    above left offset={-0.15,0.35},
    below right offset={0.15,-0.15},
    #1
  }
}

\title[MATH 2250 - Section 3.1]{Matrices: Sum and Products}
\author{Adam Wilson}
\institute{Salt Lake Community College}
\date{ \begin{center}
  \includegraphics[width=0.9\textwidth]{xkcd_matrix_revisted_partial.png}\\
  \tiny{Source: http://xkcd.com/566/}
  \end{center}}

\begin{document}

\begin{frame}
  \titlepage
 
\end{frame}

\begin{frame}{Basic Terminology}
\begin{block}{Matrix}
A \textbf{matrix} is a rectangular array of \textbf{elements} or \textbf{entries} (numbers or functions) arranged in \textbf{rows} (horizontal) and \textbf{columns} (vertical).
\begin{equation*}
A=\begin{bmatrix}
a_{11}&a_{12}&\cdots&a_{1n}\\
a_{21}&a_{22}&\cdots&a_{2n}\\
\vdots&\vdots&\ddots&\vdots\\
a_{m1}&a_{m1}&\cdots&a_{mn}
\end{bmatrix}
\end{equation*}
The \textbf{order} of $A$ is $m\times n$. If $m=n$, we call the matrix \textbf{square}.
\end{block}\pause
\begin{block}{Equal Matrices}
Two matrices of the same order are \textbf{equal} if their corresponding entries are equal. If matrices $A=[a_{ij}]$ and $B=[a_{ij}]$ are both $m\times n$, then
\[A=B \Leftrightarrow a_{ij}=b_{ij},\quad 1\leq i\leq m,\> 1\leq j\leq n\]
\end{block}
\end{frame}

\begin{frame}{Basic Terminology}
\begin{block}{Special Matrices}
\begin{itemize}
\item<+-> The $m\times n$ \textbf{zero matrix}, denoted $\textbf{0}_{mn}$, has all its entries equal to zero.
\item<+->A \textbf{diagonal matrix} is:
\begin{equation*}
D=\begin{bmatrix}
a_{11}&0&\cdots&0\\
0&a_{22}&\cdots&0\\
\vdots&\vdots&\ddots&\vdots\\
0&0&\cdots&a_{mn}
\end{bmatrix}
\end{equation*}
\item<+->The $n\times n$ \textbf{identity matrix}, denoted $\textbf{I}_n$ is:
 \begin{equation*}
 I=\begin{bmatrix}
1&0&\cdots&0\\
0&1&\cdots&0\\
\vdots&\vdots&\ddots&\vdots\\
0&0&\cdots&1
\end{bmatrix}
\end{equation*}
\end{itemize}
\end{block}
\end{frame}

\begin{frame}{Basic Terminology}

\begin{block}{Matrix Addition}
Two matrices of the same order are added (or subtracted) by adding (or subtracting) corresponding entries and recording the results in a matrix of the same size. Using matrix notation, if $A=[a_{ij}]$ and $B=[b_{ij}]$ are both $m\times n$.
\begin{equation*}
\begin{split}
A+B&=[a_{ij}]+[b_{ij}]=[a_{ij}+b_{ij}]\\
A-B&=[a_{ij}]-[b_{ij}]=[a_{ij}-b_{ij}]
\end{split}
\end{equation*}
\end{block}\pause
\begin{block}{Multiplication by a Scalar}
To find the product of a matrix and a scalar (a complex number), multiply each entry of the matrix by that number. This is called \textbf{multiplication by a scalar}. Using matrix notation, if $A=[a_{ij}]$, then
\[c\cdot A=[c\cdot a_{ij}] = [a_{ij}\cdot c] = A\cdot c\]
\end{block}
\end{frame}

\begin{frame}{Basic Properties}
\begin{block}{Properties of Matrix Addition and Scalar Multiplication}
Suppose $A$, $B$, and $C$ are $m\times n$ matrices and $c$ and $k$ are scalars. Then the following properties hold:
\begin{itemize}
\item<+->$A+B=B+A$\hfill(Commutativity)
\item<+->$A+(B+C)=(A+B)+C$\hfill(Associativity)
\item<+->$c(kA)=(ck)A$\hfill(Associativity)
\item<+->$A+\textbf{0}=A$\hfill(Zero Element)
\item<+->$A+(-A)=\textbf{0}$\hfill(Inverse Element)
\item<+->$c(A+B)=cA+cB$\hfill(Distributivity)
\item<+->$(c+k)A=cA+kA$\hfill(Distributivity)
\end{itemize}
\end{block}
\end{frame}

\begin{frame}{Vectors}{(are just tiny matrices)}
A vector $\vect{v}=< v_1,\dots,v_n >$ can be represented by either by a $1\times n$ row matrix, or a $n\times 1$ column matrix.\pause
\begin{block}{Vector addition and Scalar Multiplication}
Let
\[
\vect{x}=\begin{bmatrix}x_1\\\vdots\\x_n\end{bmatrix}
\quad\text{and}\quad
\vect{y}=\begin{bmatrix}y_1\\\vdots\\y_n\end{bmatrix}
\]
be vectors in  $\mathbb{R}^n$ and $c$ be any scalar. Then, we have:
\[
\begin{bmatrix}x_1\\\vdots\\x_n\end{bmatrix}+
\begin{bmatrix}y_1\\\vdots\\y_n\end{bmatrix}=
\begin{bmatrix}x_1+y_1\\\vdots\\x_n+y_n\end{bmatrix}
\quad\text{and}\quad
c\cdot\begin{bmatrix}x_1\\\vdots\\x_n\end{bmatrix}=
\begin{bmatrix} c\cdot x_1\\\vdots\\ c\cdot x_n\end{bmatrix}
\]
\end{block}
\end{frame}

\begin{frame}{Vector Properties}
\begin{block}{Properties of Vector Addition and Multiplication}
For vectors $\vect{u}$, $\vect{v}$, and $\vect{w}$ in $\mathbb{R}^n$ and scalars $c$ and $k$.\begin{itemize}
\item $\vect{u}+\vect{v}=\vect{v}+\vect{u}$\hfill(Commutativity)
\item $\vect{u}+(\vect{v}+\vect{w})=(\vect{u}+\vect{v})+\vect{w}$\hfill(Associativity)
\item $c(k\vect{v})=(ck)\vect{v}$\hfill(Associativity)
\item $\vect{u}+\vect{0}=\vect{u}$\hfill(Zero Element)
\item $\vect{u}+(-\vect{u})=\vect{0}$\hfill(Inverse Element)
\item $c(\vect{u}+\vect{v})=c\vect{u}+c\vect{v}$\hfill(Distributivity)
\item $(c+k)\vect{u}=c\vect{u}+k\vect{u}$\hfill(Distributivity)
\end{itemize}
\end{block}
\end{frame}

\begin{frame}{Dot Product}{(also called the Scalar Product)}
\begin{block}{Dot Product}
The \textbf{dot product} of a row vector $\vect{x}$ and a column vector $\vect{y}$ of equal length $n$ is the result of adding the products of the corresponding entries as follows:
\begin{equation*}
\begin{split}
\vect{x}\pdot\vect{y}&=\begin{bmatrix}x_1&\cdots&x_n\end{bmatrix}\pdot
\begin{bmatrix}y_1\\\vdots\\y_n\end{bmatrix}\\
&=x_1\cdot y_1+x_2\cdot y_2+\cdots +x_n\cdot y_n\\
&=\sum_{k=1}^{n} x_k\cdot y_k
\end{split}
\end{equation*}
\end{block}
\end{frame}

\begin{frame}{Dot Product}{Some Properties}
\begin{block}{Orthogonality}
Two vectors $\vect{x}$ and $\vect{y}$ in $\mathbb{R}^n$ are called \textbf{orthogonal} when:
\begin{equation*}
\vect{x}\pdot\vect{y}=0
\end{equation*}
\end{block}\pause
\begin{block}{Magnitude}
For any vector $\vect{v}$ in $\mathbb{R}^n$ the \textbf{length}, or \textbf{magnitude}, of $\vect{v}$ is a nonnegative scalar, denoted by $\vabs{\vect{v}}$ and defined to be
\begin{equation*}
\vabs{\vect{v}}=\sqrt{\vect{v}\pdot \vect{v}}
\end{equation*}
\end{block}\pause
\begin{block}{Unit Vectors}
Vectors of length one are called \textbf{unit vectors}.
\end{block}
\end{frame}

\begin{frame}{Matrix Multiplication}
\begin{block}{Matrix Product}
The \textbf{matrix product} of a $m\times r$ matrix $A$ and a $r\times n$ matrix $B$ is denoted
\begin{equation*}
C=A\cdot B=AB
\end{equation*}
where the $ij$th entry of $C$ is the dot product of the $i$th row vector of $A$ and the $j$th column vector of $B$:
\begin{equation*}
\begin{split}
c_{ij}&=\begin{bmatrix}a_{i1}&a_{2j}&\cdots&a_{ir}\end{bmatrix}\pdot
\begin{bmatrix}b_{1j}\\\vdots\\b_{rj}\end{bmatrix}\\
&=\sum_{k=1}^{r}a_{ik}b_{kj}
\end{split}
\end{equation*}
The matrix $C$ has order $m\times n$.
\end{block}
\end{frame}

\begin{frame}{Matrix Multiplication}{The Good Notation}
\begin{example}
Perform $AB$ where
\begin{equation*}
A=	\begin{bmatrix}
		1 & -1 & 3 \\
		0 & 4 & 2
	\end{bmatrix}
\quad\text{and}\quad
B=	\begin{bmatrix}
		3 & 1 \\
		2 & -4 \\
		-1 & 0 
	\end{bmatrix}
\end{equation*}
\visible<2->{
\begin{center}
\begin{tabular}{r|l}
&
\begin{minipage}{2.4cm}
%	\begin{equation*}
		$\left[
			\begin{array}{rr}
				\tikzmarkin<3,5>[ver=style cyan]{col-a}3  &\tikzmarkin<4,6>[ver=style cyan]{col-b} 1 \\
				2 &\ \ \ \ -4 \\
				-1\tikzmarkend{col-a} & 0\tikzmarkend{col-b} \\
			\end{array}
		\right]$
%	\end{equation*}
\end{minipage}\\\\
\hline\\
\begin{minipage}{2.3cm}
%	\begin{equation*}
		$\left[
			\begin{array}{rrr}
				\tikzmarkin<3-4>[hor=style orange]{row-a}1  & -1 & 3\tikzmarkend{row-a} \\
				\tikzmarkin<5-6>[hor=style orange]{row-b}0 & 4 & 2 \tikzmarkend{row-b}\\
			\end{array}
		\right]$
%	\end{equation*}
\end{minipage}
&
\begin{minipage}{2.4cm}
%	\begin{equation*}
		$\left[
			\begin{array}{rr}
				\tikzmarkin<3>[hor=style green]{end-a}\onslide<3->{-2} \tikzmarkend{end-a}
				& \tikzmarkin<4>[hor=style green]{end-b}\onslide<4->{5}\tikzmarkend{end-b}   \\
				 \tikzmarkin<5>[hor=style green]{end-c}\onslide<5->{6}\tikzmarkend{end-c} 
				 & \tikzmarkin<6>[hor=style green]{end-d}\onslide<6>{-16}\tikzmarkend{end-d}  \\
			\end{array}
		\right]$
%	\end{equation*}
\end{minipage}
\end{tabular}
\end{center}}
\vspace{0.05cm}
\end{example}
\end{frame}

\begin{frame}{Matrix Multiplication}{Properties}
\begin{block}{Properties of Matrix Multiplication}
\begin{itemize}
\item $(AB)C=A(BC)$\hfill (Associativity)
\item $A(B+C)=AB+AC$\hfill (Distributivity)
\item $(B+C)A=BA+CA$\hfill (Distributivity)\pause
\item $AB\neq BA$\hfill (Generally Noncommutative)
\end{itemize}
\end{block}\pause
\begin{block}{Properties of Identity Matrices}
For a $m\times n$ matrix $A$:
\begin{itemize}
\item $A\cdot I_n = A$\quad\text{and}\quad$I_m\cdot A = A$
\item $A\cdot\zeromat{n} = \zeromat{mn}$\quad\text{and}\quad$\zeromat{m}\cdot A=\zeromat{mn}$
\end{itemize}
\end{block}
\end{frame}

\begin{frame}{Matrix Transpose}
\begin{block}{Transpose}
For a matrix $A=[a_{ij}]$ the \textbf{transpose} of  the $m \by n$ matrix $A$ is defined as the $n \by m$ matrix:
\begin{equation*}
A^\intercal = 
\begin{bmatrix}
a_{11}&a_{12}&\cdots&a_{1n}\\
a_{21}&a_{22}&\cdots&a_{2n}\\
\vdots&\vdots&\ddots&\vdots\\
a_{m1}&a_{m2}&\cdots&a_{mn}
\end{bmatrix}^\intercal=
\begin{bmatrix}
a_{11}&a_{21}&\cdots&a_{m1}\\
a_{12}&a_{22}&\cdots&a_{m2}\\
\vdots&\vdots&\ddots&\vdots\\
a_{1n}&a_{2n}&\cdots&a_{mn}
\end{bmatrix}
\end{equation*}
\end{block}
\end{frame}
\begin{frame}{Matrix Transpose}{Examples}
\begin{example}
\centering
$\left[
	\begin{array}{rr}
		\tikzmarkin<2>[hor=style green]{tras-aa}1 & 3 \tikzmarkend{tras-aa}\\
		\tikzmarkin<3>[hor=style green]{tras-ba}2 & 4 \tikzmarkend{tras-ba}
	\end{array}
\right]^\intercal $
$= 
\left[
	\begin{array}{rr}
		\tikzmarkin<2>[ver2=style green]{tras-ab}\visible<2->{1} & \tikzmarkin<3>[ver2=style green]{tras-bb} \visible<3->{2} \\
		 \visible<2->{3} \tikzmarkend{tras-ab}& \visible<3->{4} \tikzmarkend{tras-bb}
	\end{array}
\right]$
\end{example}
\begin{example}<4->
\centering
$\left[
	\begin{array}{rrr}
		\tikzmarkin<5>[hor=style green]{tras-ca}1 & 2 & -1 \tikzmarkend{tras-ca}\\
		\tikzmarkin<6>[hor=style green]{tras-da}3 & 0 & 5 \tikzmarkend{tras-da}
	\end{array}
\right]^\intercal $
$= 
\left[
	\begin{array}{rr}
		\tikzmarkin<5>[ver=style green]{tras-cb}\visible<5->{1} & \tikzmarkin<6>[ver2=style green]{tras-db} \visible<6>{3} \\
		\visible<5->{2} & \visible<6>{0} \\
		\visible<5->{-1} \tikzmarkend{tras-cb}& \visible<6>{5} \tikzmarkend{tras-db}
	\end{array}
\right]$
\end{example}
\end{frame}

\begin{frame}{Matrices with Function Entries}
Matrices can have functions as entries, not just real numbers.
\begin{example}
\begin{equation*}
A(t)=\begin{bmatrix}
a_{11}(t)&a_{12}(t)&\cdots&a_{1n}(t)\\
a_{21}(t)&a_{22}(t)&\cdots&a_{2n}(t)\\
\vdots&\vdots&\ddots&\vdots\\
a_{m1}(t)&a_{m1}(t)&\cdots&a_{mn}(t)
\end{bmatrix}
\end{equation*}
\end{example}
\end{frame}

\begin{frame}{Derivative of a Matrix}
We say that a matrix is \textbf{continuous}, \textbf{piecewise continuous}, or \textbf{differentiable} if every element in the matrix has that property.\pause
\begin{block}{Derivative of a Matrix}
For a differentiable matrix $A$, the derivative of $A$ is defined as:
\begin{equation*}
A^\prime(t)=\dfrac{dA}{dt}=
\begin{bmatrix}
a_{11}^\prime(t)&a_{12}^\prime(t)&\cdots&a_{1n}^\prime(t)\\
a_{21}^\prime(t)&a_{22}^\prime(t)&\cdots&a_{2n}^\prime(t)\\
\vdots&\vdots&\ddots&\vdots\\
a_{m1}^\prime(t)&a_{m1}^\prime(t)&\cdots&a_{mn}^\prime(t)
\end{bmatrix}
\end{equation*}
\end{block}\pause
\begin{block}{Matrix Differentiation Rules}
For differentiable matrices $A(t)$ and $B(t)$ and scalar constant $c$.
\begin{itemize}
\item ${\left(A(t)+B(t)\right)}^\prime=A^\prime(t)+B^\prime(t)$
\item $\left(cA(t)\right)\prime=cA^\prime(t)$
\item ${\left(A(t)\cdot B(t)\right)}^\prime=A(t)\cdot B^\prime(t)+A^\prime(t)\cdot  B(t)$
\end{itemize}
\end{block}
\end{frame}

\begin{frame}{Derivative of a Matrix}{Examples}
\begin{example}
\centering
\begin{equation*}
g(t)=
\begin{bmatrix}
\ln t\\
-t^3\\
\cos 2t
\end{bmatrix}
\qquad g^\prime(t)=
\begin{bmatrix}
\visible<2->{\dfrac{1}{t}}\\
\visible<3->{-3t^2}\\
\visible<4->{-2\sin 2t}
\end{bmatrix}
\end{equation*}
\end{example}
\pause
\begin{example}<5->
\centering
\begin{equation*}
A(t)=
\begin{bmatrix}
e^t & t^2\\
\sin t & 2t
\end{bmatrix}
\qquad A^\prime(t)=
\begin{bmatrix}
\visible<6->{e^t} & \visible<7->{2t}\\
\visible<8->{\cos t} & \visible<9->{2}
\end{bmatrix}
\end{equation*}
\end{example}
\end{frame}
\end{document}
