\documentclass{beamer}

\usepackage[english]{babel}
\usepackage[utf8x]{inputenc}
\usepackage[inline]{asymptote}
\usepackage{slide_helper}
\usepackage{tabu}

\usepackage{tikz}
\usetikzlibrary{arrows.meta}

\begin{asydef}
import graph;
import slopefield;
import fontsize;
defaultpen(fontsize(9pt));
ngraph=1000;

int pen_pos=-1;
pen[] pens={blue, red, heavycyan, heavymagenta, lightolive};
pens.cyclic=true;
pen next_color() {return pens[++pen_pos];}

DefaultHead.size=new real(pen p=currentpen) {return 2.5mm;};
\end{asydef}

\title[MATH 2250 - Section 8.3]{The Step Function and the Delta Function}

\DeclareMathOperator{\step}{\textbf{step}}

\begin{document}

\begin{frame}
  \titlepage
\end{frame}

\begin{frame}[fragile]
\begin{block}{The Unit Step Function}
\begin{equation*}
\step(t)=
\begin{cases}
0 & \text{if}~t< 0 \\
1 & \text{if}~t\geq 0
\end{cases}
\end{equation*}
\begin{center}
\begin{asy}
size(245);

real min_x=-2, max_x=2;
real min_y=-1, max_y=2;

pair start=(min_x,min_y);
pair end=(max_x,max_y);

dot(start,invisible);
dot(end,invisible);

real a=0.0;

draw((min_x,0)--(a,0),blue+1.5bp,BeginArrow());
draw((a,1)--(max_x,1),blue+1.5bp,EndArrow());

unfill(circle(r=0.06,c=(a,0)));
draw(circle(r=0.06,c=(a,0)),blue+1bp);
fill(circle(r=0.06,c=(a,1)),blue+1bp);

//limits(start,end,Crop);

xaxis("$t$",YEquals(0),Ticks(NoZero));
yaxis("$y$",XEquals(0),Ticks(NoZero),autorotate=false);
\end{asy}
\end{center}
\end{block}
\end{frame}

\begin{frame}[fragile]
\begin{block}{The Translated Step Function}
\begin{equation*}
\step(t-a)=
\begin{cases}
0 & \text{if}~t< a \\
1 & \text{if}~t\geq a
\end{cases}
\end{equation*}
\begin{center}
\begin{asy}
size(245);

real min_x=-2, max_x=2;
real min_y=-1, max_y=2;

pair start=(min_x,min_y);
pair end=(max_x,max_y);

dot(start,invisible);
dot(end,invisible);

real a=0.7;

draw((min_x,0)--(a,0),blue+1.5bp,BeginArrow());
draw((a,1)--(max_x,1),blue+1.5bp,EndArrow());

draw((a,min_y)--(a,max_y),darkgray+dashed);
label("$a$",(a+0.1,0.5),darkgray);

unfill(circle(r=0.06,c=(a,0)));
draw(circle(r=0.06,c=(a,0)),blue+1bp);
fill(circle(r=0.06,c=(a,1)),blue+1bp);

//limits(start,end,Crop);

xaxis("$t$",YEquals(0),Ticks(NoZero));
yaxis("$y$",XEquals(0),Ticks(NoZero),autorotate=false);
\end{asy}
\end{center}
\end{block}
\end{frame}

\begin{frame}
\begin{block}{Laplace Transform of the Step Function}
\begin{equation*}
\laplace\{\step(t-a)\}=\dfrac{e^{-as}}{s}
\end{equation*}
\end{block}\pause
\begin{proof}
\begin{equation*}
\begin{aligned}
\laplace\{\step(t-a)\}
&= \int_0^\infty e^{-st}\step(t-a)~dt\\\pause
&= \int_0^a e^{-st}\cdot 0~dt + \int_a^\infty e^{-st}\cdot 1~dt\\\pause
&= \int_a^\infty e^{-st}~dt\\\pause
&= \lim_{b\rightarrow\infty}{\left[-\dfrac{e^{-st}}{s}\right]}^b_a\\\pause
&= \lim_{b\rightarrow\infty}-\dfrac{1}{s}\left[e^{-sb}-e^{-sa}\right]\\\pause
&= \dfrac{e^{as}}{s}
\end{aligned}
\end{equation*}
\end{proof}
\end{frame}

\begin{frame}[fragile]
\begin{example}
Consider
\begin{equation*}
\begin{aligned}
f(t)&=
\begin{cases}
\+2 & \text{if}~t<3 \\
 -4 & \text{if}~3\leq t< 4 \\
\+1 & \text{if}~t\geq 4
\end{cases}\pause
\qquad=2-6\step(t-3)+5\step(t-4)
\end{aligned}
\end{equation*}\pause
Which means that

\vspace{-5mm}
\begin{equation*}
\laplace\{f(t)\}=\dfrac{2-6e^{-3s}+5e^{-4s}}{s}
\end{equation*}\pause

\vspace{-8mm}
\begin{center}
\begin{asy}
picture pic_1, pic_2;
size(pic_1,130);
size(pic_2,130);

real min_x=0, max_x=7;
real min_y=-5, max_y=3;

pair start=(min_x,min_y);
pair end=(max_x,max_y);

real a=3;
real b=4;

draw(pic_1,(min_x,2)--(a,2),blue+1.5bp);
draw(pic_1,(a,-4)--(b,-4),blue+1.5bp);

draw(pic_1,(b,1)--(max_x,1),blue+1.5bp);

draw(pic_1,(a,min_y)--(a,max_y),darkgray+dashed);
draw(pic_1,(b,min_y)--(b,max_y),darkgray+dashed);

unfill(pic_1,circle(r=0.2,c=(a,2)));
draw(pic_1,circle(r=0.2,c=(a,2)),blue+1bp);
fill(pic_1,circle(r=0.2,c=(a,-4)),blue+1bp);

unfill(pic_1,circle(r=0.2,c=(b,-4)));
draw(pic_1,circle(r=0.2,c=(b,-4)),blue+1bp);
fill(pic_1,circle(r=0.2,c=(b,1)),blue+1bp);

limits(pic_1, start,end,Crop);

xaxis(pic_1,"$t$",YEquals(0),Ticks(NoZero));
yaxis(pic_1,"$f(t)$",XEquals(0),Ticks(NoZero),autorotate=false);

// ------

real min_x=0, max_x=5;
real min_y=0, max_y=5;

pair start=(min_x,min_y);
pair end=(max_x,max_y);

real F(real s) {return (2-6*exp(-3*s)+5*exp(-4*s))/s;}
draw(pic_2, graph(F,0.1, max_x), blue+1.5bp);

limits(pic_2, start,end,Crop);

xaxis(pic_2,"$s$",YEquals(0),Ticks(NoZero));
yaxis(pic_2,"$F(s)$",XEquals(0),Ticks(NoZero),autorotate=false);

// ------

// Fit pic to W of origin:
add(pic_1.fit(),(-7mm,0),W);

// Fit pic2 to E of (5mm,0):
add(pic_2.fit(),(7mm,0),E);

draw((-3mm,0)--(3mm,0),black+1bp,EndArrow());
label("$\laplace$",(0,3mm),black);
\end{asy}
\end{center}
\end{example}
\end{frame}

\begin{frame}[fragile]
\begin{example}
Consider 

\vspace{-7mm}
\begin{equation*}
g(t)=
\begin{cases}
0 & \text{if}~t<0 \\
t^2 & \text{if}~0\leq t \leq 1 \\
1 & \text{if}~t\geq 1 \\
\end{cases}\pause
\qquad =t^2\step(t)+(1-t^2)\step(t-1)
\end{equation*}

\vspace{-5mm}
\begin{overprint}
\onslide<1-2>
\onslide<3-7>
Which means
\begin{equation*}
\begin{aligned}
\laplace\{g(t)\} 
&=\int_0^\infty t^2~e^{-st}~\step(t)~dt + \int_0^\infty (1-t^2)~e^{-st}~\step(t-1)~dt\\
\visible<4->{&=\int_0^\infty t^2~e^{-st}~dt + \int_1^\infty e^{-st}~dt - \int_1^\infty t^2~e^{-st}~dt\\}
\visible<5->{&=\int_0^1 t^2~e^{-st}~dt + \int_1^\infty e^{-st}~dt \\}
\visible<6->{&=\dfrac{2}{s}-e^{-st}\left(\dfrac{1}{s}+\dfrac{2}{s^2}+\dfrac{2}{s^3}\right)+\dfrac{1}{s}e^{-s} \\}
\visible<7->{&=\dfrac{2}{s^2}-2e^{-s}\left(\dfrac{1}{s^2}+\dfrac{1}{s^3}\right) \\}
\end{aligned}
\end{equation*}
\onslide<8->
\begin{center}
\begin{asy}
picture pic_1, pic_2;
size(pic_1,150);
size(pic_2,150);

real min_x=-2, max_x=3;
real min_y=-2, max_y=2;

pair start=(min_x,min_y);
pair end=(max_x,max_y);

draw(pic_1,(min_x,0)--(0,0),blue+1.5bp);

real f(real t) {return t*t;}
draw(pic_1, graph(f,0,1),blue+1.5bp);

draw(pic_1,(1,1)--(max_x,1),blue+1.5bp);

limits(pic_1, start,end,Crop);

xaxis(pic_1,"$t$",YEquals(0),Ticks(NoZero));
yaxis(pic_1,"$f(t)$",XEquals(0),Ticks(NoZero),autorotate=false);

// ------

real min_x=0, max_x=5;
real min_y=0, max_y=5;

pair start=(min_x,min_y);
pair end=(max_x,max_y);

real F(real s) {return 2/(s*s*s)-2*exp(-s)*(1/(s*s)+1/(s*s*s));}
draw(pic_2, graph(F,0.1, max_x), blue+1.5bp);

limits(pic_2, start,end,Crop);

xaxis(pic_2,"$s$",YEquals(0),Ticks(NoZero));
yaxis(pic_2,"$F(s)$",XEquals(0),Ticks(NoZero),autorotate=false);

// ------

// Fit pic to W of origin:
add(pic_1.fit(),(-6mm,0),W);

// Fit pic2 to E of (5mm,0):
add(pic_2.fit(),(6mm,0),E);

draw((-3mm,0)--(3mm,0),black+1bp,EndArrow());
label("$\laplace$",(0,3mm),black);
\end{asy}
\end{center}
\end{overprint}
\vspace{-50mm}
\end{example}
\end{frame}

\begin{frame}
\begin{block}{Delayed Function}
For a given function $g(t)$, the \textbf{delayed function}
\begin{equation*}
f(t)=
\begin{cases}
0 & \text{if}~t<c \\
g(t-c) & \text{if}~t\geq c
\end{cases}
\end{equation*}
shifts $g(t)$ to the right $c$ units from the origin, and replaces it by zero to the left of $t=c$. Using the unit step function, the delayed function can also be written
\begin{equation*}
f(t)=g(t-c)\step(t-c)
\end{equation*}
\end{block}
\end{frame}

\begin{frame}[fragile]
\begin{example}
Consider the function $g(t)=\sqrt{t}$, which has the delayed function
\begin{equation*}
f(t)=
\begin{cases}
0 & \text{if}~t<3 \\
\sqrt{t-3} & \text{if}~t\geq 3
\end{cases}
\end{equation*}\pause
\begin{center}
\begin{asy}
picture pic_1, pic_2;
size(pic_1,150);
size(pic_2,150);

real min_x=-2, max_x=3;
real min_y=-2, max_y=2;

pair start=(min_x,min_y);
pair end=(max_x,max_y);

real g(real t) {return sqrt(t);}
draw(pic_1, graph(g,0,max_x),blue+1.5bp);

limits(pic_1, start,end,Crop);

xaxis(pic_1,"$t$",YEquals(0),Ticks(NoZero));
yaxis(pic_1,"$g(t)$",XEquals(0),Ticks(NoZero),autorotate=false);

// ------

real min_x=0, max_x=5;
real min_y=0, max_y=5;

pair start=(min_x,min_y);
pair end=(max_x,max_y);

draw(pic_2,(0,0)--(3,0), blue+2bp);
real f(real t) {return sqrt(t-3);}
draw(pic_2, graph(f,3, max_x), blue+1.5bp);

limits(pic_2, start,end,Crop);

xaxis(pic_2,"$s$",YEquals(0),Ticks(NoZero));
yaxis(pic_2,"$f(t)$",XEquals(0),Ticks(NoZero),autorotate=false);

// ------

// Fit pic to W of origin:
add(pic_1.fit(),(-6mm,0),W);

// Fit pic2 to E of (5mm,0):
add(pic_2.fit(),(6mm,0),E);

draw((-3mm,0)--(3mm,0),black+1bp,EndArrow());
label("$\laplace$",(0,3mm),black);
\end{asy}
\end{center}
\end{example}
\end{frame}

\begin{frame}
\small
\begin{block}{}
Let us now look at the Laplace transform of a function $f(t)$ that is delayed $c$ units.
\begin{equation*}
\begin{aligned}
\laplace\{f(t-c)\step(t-c)\}\pause
&=\int_0^\infty e^{-st}~f(t-c)\step(t-c)~dt\\\pause
&=\lim_{b\rightarrow\infty}\int_0^b e^{-st}~f(t-c)\step(t-c)~dt
\end{aligned}
\end{equation*}\pause
We may assume $b>c$, since $b\rightarrow\infty$.\pause 

\vspace{2mm}
Furthermore, $\step(t-c)=0$ for $t<c$ and $\step(t-c)=1$ for $t\geq c$.\pause
\begin{equation*}
\begin{aligned}
\lim_{b\rightarrow\infty}\int_0^b e^{-st}~f(t-c)\step(t-c)~dt
&= \lim_{b\rightarrow\infty}\int_c^b e^{-st}~f(t-c)~dt\\\pause
\text{let $w=t-c$}\qquad\qquad&= \lim_{b\rightarrow\infty}\int_0^{b-c} e^{-s(w+c)}~f(w)~dw\\\pause
&=\lim_{b\rightarrow\infty}e^{-cs} \int_0^{b-c} e^{-sw}~f(w)~dw\\\pause
&=e^{-cs}\int_0^\infty e^{-sw}~f(w)~dw\pause
=e^{-cs}~F(s)
\end{aligned}
\end{equation*}
\end{block}
\end{frame}

\begin{frame}
\begin{block}{Delay Theorem (or Shifting Theorem)}
\begin{equation*}
\laplace\{f(t-c)\step(t-c)\}=e^{-cs}~F(s)
\quad\text{where}~c>0
\end{equation*}
\end{block}\pause
\begin{block}{Alternate Form}
\begin{equation*}
\laplace\{g(t)\step(t-c)\}=e^{-cs}~\laplace\{g(t+c)\}
\end{equation*}
\end{block}\pause
\begin{example}
Consider
\begin{equation*}
h(t)=t^2\step(t-1)
\end{equation*}\pause
If we let $c=1$ and $g(t)=t^2$, then by the Delay theorem we have
\begin{equation*}
\begin{aligned}
\laplace\{h(t)\}=\laplace\{t^2\step(t-1)\}\pause
&= e^{-s}\laplace\{{(t+1)}^2\}\\\pause
&= e^{-s}\laplace\{t^2+2t+1\}\\\pause
&= e^{-s}\left(\dfrac{2}{s^2}+\dfrac{2}{s^2}+\dfrac{1}{s}\right)
\end{aligned}
\end{equation*}
\end{example}
\end{frame}

\begin{frame}
\begin{example}
Let us find the inverse Laplace transform of
\begin{equation*}
F(s)=\dfrac{1-e^{-3s}}{s^2}
=\dfrac{1}{s^2}-\dfrac{e^{-3s}}{s^2}
\end{equation*}\pause
We can treat $\dfrac{e^{-3s}}{s^2}$ as the transform of a delay function.\pause
\begin{equation*}
\laplace^{-1}\{F(s)\}=t-\underbrace{(t-3)\step(t-3)}_{\laplace^{-1}\left\{\tfrac{e^{-3s}}{s^2}\right\}}
\end{equation*}
\end{example}
\end{frame}

\begin{frame}[fragile]
\begin{block}{Chopper Function}
\begin{equation*}
\step(t-a)-\step(t-b)=
\begin{cases}
0 & \text{if}~t<a \\
1 & \text{if}~a\leq t< b \\
0 & \text{if}~t\geq b
\end{cases}
\end{equation*}
\end{block}\pause
\begin{example}
\begin{center}
\begin{asy}
picture pic_1, pic_2, pic_3;

real pic_size=90;
size(pic_1,pic_size);
size(pic_2,pic_size);
size(pic_3,pic_size);

real min_x=-1, max_x=3;
real min_y=-1, max_y=3;

pair start=(min_x,min_y);
pair end=(max_x,max_y);

real f(real t) {return -t*t*t+2t+1;}
real a=0.25;
real b=1.25;

pen chop_pen=black+1.5bp;
pen chop_fill_pen=black+1bp;
pen graph_pen=black+1.5bp;
pen graph_fill_pen=black+1bp;
pen final_pen=blue+1.5bp;
pen final_fill_pen=blue+1bp;
pen dashed_pen=gray+dashed;

real radius=0.1;

// ----- pic_1 chopper function -----
draw(pic_1, (min_x,0)--(a,0), chop_pen);
draw(pic_1, (a,1)--(b,1), chop_pen);
draw(pic_1, (b,0)--(max_x,0), chop_pen);

draw(pic_1, (a,0)--(a,1), dashed_pen);
draw(pic_1, (b,0)--(b,1), dashed_pen);

unfill(pic_1, circle(r=radius, c=(a,0)));
draw(pic_1, circle(r=radius, c=(a,0)),chop_fill_pen);

unfill(pic_1, circle(r=radius, c=(b,1)));
draw(pic_1, circle(r=radius, c=(b,1)),chop_fill_pen);

fill(pic_1, circle(r=radius, c=(a,1)),chop_fill_pen);
fill(pic_1, circle(r=radius, c=(b,0)),chop_fill_pen);

limits(pic_1, start, end, Crop);
xaxis(pic_1,"$t$",YEquals(0));
yaxis(pic_1,"$y$",XEquals(0),autorotate=false);
// ----------------------------------

// ----- pic_2 f(t) -----
draw(pic_2, graph(f,min_x,max_x), graph_pen);

limits(pic_2, start, end, Crop);
xaxis(pic_2,"$t$",YEquals(0));
yaxis(pic_2,"$y$",XEquals(0),autorotate=false);
// ----------------------

// ----- pic_3 chopped f(t) -----
draw(pic_3,(min_x,0)--(a,0), final_pen);
draw(pic_3, graph(f,a,b), final_pen);
draw(pic_3,(b,0)--(max_x,0), final_pen);

draw(pic_3,(a,0)--(a,f(a)), dashed_pen);
draw(pic_3,(b,0)--(b,f(b)), dashed_pen);

unfill(pic_3, circle(r=radius, c=(a,0)));
draw(pic_3, circle(r=radius, c=(a,0)),final_fill_pen);

unfill(pic_3, circle(r=radius, c=(b,f(b))));
draw(pic_3, circle(r=radius, c=(b,f(b))),final_fill_pen);

fill(pic_3, circle(r=radius, c=(a,f(a))),final_fill_pen);
fill(pic_3, circle(r=radius, c=(b,0)),final_fill_pen);

limits(pic_3, start, end, Crop);
xaxis(pic_3,"$t$",YEquals(0));
yaxis(pic_3,"$y$",XEquals(0),autorotate=false);
// ------------------------------

// ----- fit the pictures together -----

// Fit pic to W of origin:
add(pic_1.fit(),(-6mm,0),W);

// Fit pic2 to E of (5mm,0):
add(pic_2.fit(),(6mm,0),E);

// Fit pic3 to E of pic_2:
add(pic_3.fit(),(50mm,0),E);

label("times",(0,0));
label("gives",(46mm,0));
// -------------------------------------
\end{asy}
\end{center}
\end{example}\pause
\begin{block}{Laplace Transform of Chopper Function}

\vspace{-5mm}
\begin{equation*}
\laplace\left\{f(t)\cdot\left[\step(t-a)-\step(t-b)\right]\right\}=
e^{-as}\laplace\{f(t+a)\}-e^{-bs}\laplace\{f(t+b)\}
\end{equation*}
\end{block}
\end{frame}

\begin{frame}
\begin{example}
Let us find the Laplace transform of
\begin{equation*}
\begin{aligned}
f(t)&=
\begin{cases}
0 & \text{if}~t<1 \\
-\sin[\pi t] & \text{if}~1\leq t< 2 \\
0 & \text{if}~t\geq 2
\end{cases}\\\pause
&=-\sin[\pi t]\cdot\left[\step(t-1)-\step(t-2)\right]
\end{aligned}
\end{equation*}\pause
Thus,
\begin{equation*}
\begin{aligned}
\laplace\{f(t)\}&=-e^{-s}\laplace\{-\sin[\pi (t+1)]\}+e^{2s}\laplace\{\sin[\pi (t+2)]\}\\\pause
\laplace\{f(t)\}&=-e^{-s}\laplace\{-\sin[\pi t]\cos[\pi]-\cos[\pi t]\sin[\pi]\}\\
&\hspace{4.2mm}+e^{2s}\laplace\{\sin[\pi t]\cos[2\pi]+\cos[\pi t]\sin[2\pi]\}\\\pause
&=-e^{-s}\laplace\{\sin[\pi t]\}+e^{2s}\laplace\{\sin[\pi t]\}\\\pause
&=\laplace\{\sin[\pi t]\}(e^{-s}+e^{-2s})\\\pause
&=\dfrac{\pi}{s^2+\pi^2}(e^{-s}+e^{-2s})
\end{aligned}
\end{equation*}
\end{example}
\end{frame}

\begin{frame}[fragile]
\begin{example}
Consider the IVP
\begin{equation*}
x^{\prime\prime}+x=f(t)=
\begin{cases}
1 & \text{if}~0\leq t < \pi \\
0 & \text{if}~t\geq\pi
\end{cases}
\qquad\text{with}~
x(0)=0,~x^{\prime}(0)=0
\end{equation*}
\begin{overprint}
\onslide<1>
\onslide<2-4>
We can rewrite this DE using a step function
\begin{equation*}
x^{\prime\prime}+x=1-\step(t-\pi)
\qquad\text{with}~
x(0)=0,~x^{\prime}(0)=0
\end{equation*}
\visible<3->{%
Which has Laplace transformation
\begin{equation*}
s^2 X(s)+X(s)=\laplace\{1-\step(t-\pi)\}
\end{equation*}}
\visible<4->{%
We can then use the Delay Theorem on the RHS
\begin{equation*}
s^2 X(s)+X(s)=\dfrac{1}{s}+\dfrac{e^{-\pi s}}{s}
\end{equation*}}
\onslide<5-9>
We can now solve for $X(s)$. (As well as rearrange for the inverse.)
\begin{equation*}
\begin{aligned}
s^2 X(s)+X(s)&=\dfrac{1}{s}+\dfrac{e^{-\pi s}}{s}\\
\visible<6->{(s^2+1)X(s)&=\dfrac{1-e^{-\pi s}}{s}\\}
\visible<7->{X(s)&=\dfrac{1-e^{-\pi s}}{s(s^2+1)}\\}
\visible<8->{&=\dfrac{1}{s(s^2+1)}-e^{-\pi s}\dfrac{1}{s(s^2+1)}\\}
\visible<9->{&=\left(\dfrac{1}{s}-\dfrac{s}{s^2+1}\right)
-e^{-\pi s}\left(\dfrac{1}{s}-\dfrac{s}{s^2+1}\right)\\}
\end{aligned}
\end{equation*}
\onslide<10-12>
So, we can use the Delay Theorem again to find $x(t)$.
\begin{equation*}
x(t)=\laplace^{-1}\{X(s)\}=(1-\cos[t])-(1-\cos[t-\pi])\step(t-\pi)
\end{equation*}
\visible<11-12>{%
Which, when written as a piecewise function gives
\begin{equation*}
\begin{aligned}
x(t)&=
\begin{cases}
1-\cos[t] & \text{if}~0\leq t<\pi \\
1-\cos[t] -(1-\cos[t-\pi]) & \text{if}~ t\geq\pi \\
\end{cases}\\
\visible<12->{&=
\begin{cases}
1-\cos[t] & \text{if}~0\leq t<\pi \\
-2\cos[t] & \text{if}~ t\geq\pi \\
\end{cases}}
\end{aligned}
\end{equation*}}
\onslide<13->
\begin{center}
\begin{asy}
picture pic_1, pic_2;
size(pic_1,150);
size(pic_2,150);

real min_x=0, max_x=3*pi;
real min_y=-4, max_y=4;

pair start=(min_x,min_y);
pair end=(max_x,max_y);

pen dpen=blue+1.5bp;

// ----- pic_1 -----

draw(pic_1, (min_x,1)--(pi,1), dpen);
draw(pic_1, (pi,0)--(max_x,0), dpen);

unfill(pic_1, circle(r=0.1, c=(pi,1)));
draw(pic_1, circle(r=0.1, c=(pi,1)), dpen);
fill(pic_1, circle(r=0.1, c=(pi,0)), dpen);

draw(pic_1, (pi,0)--(pi,1), dashed+gray);

label(pic_1, "$\pi$", (pi,-0.5));

limits(pic_1, start, end, Crop);

xaxis(pic_1,"$t$",YEquals(0),Ticks(NoZero));
yaxis(pic_1,"$f(t)$",XEquals(0),Ticks(NoZero),autorotate=false);

// ----- pic_2 -----

real x(real t) 
{
	if (t<pi)
		return 1-cos(t);
	else
		return -2*cos(t);
}
draw(pic_2, graph(x,0.01, max_x), dpen);
draw(pic_2, (pi,0)--(pi,x(pi)), dashed+gray);

label(pic_2, "$\pi$", (pi,-0.5));

limits(pic_2, start, end, Crop);

xaxis(pic_2,"$t$",YEquals(0),Ticks(NoZero));
yaxis(pic_2,"$x(t)$",XEquals(0),Ticks(NoZero),autorotate=false);

// ------

// Fit pic to W of origin:
add(pic_1.fit(),(-1mm,0),W);

// Fit pic2 to E of (5mm,0):
add(pic_2.fit(),(1mm,0),E);

//draw((-3mm,0)--(3mm,0),black+1bp,EndArrow());
\end{asy}
\end{center}
\end{overprint}
\vspace{-35mm}
\end{example}
\end{frame}

\begin{frame}[fragile]
\begin{block}{}
\begin{columns}
\begin{column}{0.5\textwidth}
Physical systems often involved impulsive forces, which act over very short spans of time. To model these forces, the physicist Paul Dirac invented a \quotetext{function-like} object. 

\vspace{2mm}
Let us first look at a special function
\begin{equation*}
f_h(t)=
\begin{cases}
0 & \text{if}~t<0 \\
\tfrac{1}{h} & \text{if}~0\leq t<h \\
0 & \text{if}~y\geq h
\end{cases}
\end{equation*}
such that 
\begin{equation*}
\int_{-\infty}^{\infty} f_h(t)~dt=1
\end{equation*}
Dirac suggested that 

\vspace{-4mm}
\begin{equation*}
\delta(t)=\lim_{b\rightarrow 0} f_h(t)
\end{equation*}
\end{column}
\begin{column}{0.45\textwidth}
\begin{center}
\begin{asy}
size(175);

real min_x=-1, max_x=2;
real min_y=-1, max_y=3;

pair start=(min_x,min_y);
pair end=(max_x,max_y);

pen dpen=blue+2bp;
pen dotpen=blue+1bp;

real[] h_values={0.4,1.0,1.5};

for(real h : h_values)
{
	draw((0,1/h)--(h,1/h), dotpen);
	draw((h,0)--(max_x,0), dotpen);
	draw((h,0)--(h,1/h), dashed+grey);
	unfill(circle(r=0.05, c=(h,1/h)));
	draw(circle(r=0.05, c=(h,1/h)), dotpen);
	fill(circle(r=0.05, c=(0,1/h)), dotpen);
	fill(circle(r=0.05, c=(h,0)), dotpen);
}

label("$h_1$", (h_values[0],-0.25));
label("$\dfrac{1}{h_1}$", (-0.25,1/h_values[0]+0.1));

label("$h_2$", (h_values[1],-0.25));
label("$\dfrac{1}{h_2}$", (-0.25,1/h_values[1]+0.2));

label("$h_3$", (h_values[2],-0.25));
label("$\dfrac{1}{h_3}$", (-0.25,1/h_values[2]-0.1));

draw((min_x,0)--(0,0), dpen);

limits(start, end, Crop);

xaxis("$t$",YEquals(0),NoTicks());
yaxis("$y$",XEquals(0),NoTicks(),autorotate=false);

\end{asy}
\end{center}
\end{column}
\end{columns}
\end{block}
\end{frame}

\begin{frame}
\begin{block}{Dirac Delta Function}
The \textbf{Dirac Delta function} or \textbf{unit impulse function} $\delta(t)$ is defined by two condtitions:
\begin{enumerate}
\item \begin{equation*}
\delta(t)=
\begin{cases}
0 & \text{if}~ t\neq 0 \\
\lim_{h\rightarrow 0}\left(\dfrac{1}{h}\right) & \text{if}~t=0
\end{cases}
\end{equation*}
\item \begin{equation*}
\int_{-\infty}^{\infty} \delta(t)~dt=1
\end{equation*}
\end{enumerate}
\end{block}
\end{frame}

\begin{frame}
\begin{block}{}
To find that Laplace transform of $\delta(t)$, we will first calculate the transform of $f_h(t)$.
\begin{equation*}
\begin{aligned}
\laplace\{f_h(t)\}
&= \int_0^\infty e^{-st}f_h(t)~dt \\\pause
&= \int_0^h e^{-st}f_h(t)~dt \\\pause
&=\dfrac{1}{h} \int_0^h e^{-st}~dt \\\pause
&=\dfrac{1-e^{-hs}}{hs}
\end{aligned}
\end{equation*}\pause
We can then use l'H\^opital's rule to find
\begin{equation*}
\lim_{h\rightarrow 0}\laplace\{f_h(t)\}=1
\end{equation*}
\end{block}\pause
\begin{block}{Laplace Transform of the Delta Function}
\begin{equation*}
\laplace\{\delta(t)\}=1
\qquad\text{and}\qquad
\laplace\{\delta(t-a)\}=e^{-as}
\end{equation*}
\end{block}
\end{frame}
\end{document}