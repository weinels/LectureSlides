\documentclass{beamer}

\usepackage[english]{babel}
\usepackage[utf8x]{inputenc}
\usepackage{slide_helper}

\usepackage[customcolors, beamer]{hf-tikz}
\tikzset{style green/.style={
    set fill color=green!60!lime!50,
    set border color=green!100,
  },
  style cyan/.style={
    set fill color=cyan!50!blue!40,
    set border color=cyan!100,
  },
  style orange/.style={
    set fill color=orange!50!red!60,
    set border color=red!80,
  },
  hor/.style={
    above left offset={-0.15,0.35},
    below right offset={0.15,-0.125},
    #1
  },
  ver/.style={
    above left offset={-0.38,0.35},
    below right offset={0.15,-0.15},
    #1
  },
  ver2/.style={
    above left offset={-0.15,0.35},
    below right offset={0.15,-0.15},
    #1
  }
}

\title[MATH 2250 - Section 3.3]{The Inverse of a Matrix}

\begin{document}

\begin{frame}
  \titlepage
\end{frame}

\begin{frame}
\begin{block}{Inverse Matrix}
If there exists, for an $n \by n$ matrix $\mat{A}$, another matrix $\inverseof{\mat{A}}$ of the same order such that
\begin{equation*}
\inverseof{\mat{A}}\mat{A}=\mat{A}\inverseof{\mat{A}}=\identmat{n}
\end{equation*}
then $\inverseof{\mat{A}}$ is called the \textbf{inverse} of matrix $\mat{A}$, and $\mat{A}$ is called \textbf{invertible}.
\end{block}\pause
\begin{block}{Vocabulary}
\begin{itemize}
\item A square matrix that is not invertible is called \textbf{singular}.
\item A square matrix that is invertible is called \textbf{nonsingular}.
\end{itemize}
\end{block}
\end{frame}

\begin{frame}
\begin{block}{Invertible Matrix Properties}
\begin{itemize}
\item<+-> If $\mat{A}$ is invertible, then so is $\mat{A}^{-1}$ and 
\begin{equation*}
\inverseof{\left(\inverseof{\mat{A}}\right)}=\mat{A}
\end{equation*}
\item<+-> If $\mat{A}$ and $\mat{B}$ are invertible matrices of the same order, then their product $\mat{A}\mat{B}$ is invertible. In fact,
\begin{equation*}
\inverseof{\left(\mat{A}\mat{B}\right)}=\inverseof{\mat{B}}\inverseof{\mat{A}}
\end{equation*}
\item<+-> If $\mat{A}$ is invertible, then so is $\transposeof{\mat{A}}$, and
\begin{equation*}
\inverseof{\left(\transposeof{\mat{A}}\right)}=\transposeof{\left(\inverseof{\mat{A}}\right)}
\end{equation*}
\end{itemize}
\end{block}
\end{frame}

\begin{frame}
\begin{block}{Inverses by Reduced Row Echelon Form}
For an $n\by n$ matrix $\mat{A}$, the following process will calculate $\inverseof{\mat{A}}$, or show that $\mat{A}$ is not invertible.
\begin{description}
\item<+->[Step 1:] Form the $n\by 2n$ augmented matrix $\mat{M}=[\mat{A}|\identmat{n}]$.
\item<+->[Step 2:] Transform $\mat{M}$ into Reduced Row Echelon Form.
\item<+->[Step 3:] 
\begin{itemize}
\item<.-> If the left hand side of $\mat{M}$ is the identity matrix, then the right hand side is $\inverseof{\mat{A}}$.
\item<.-> Otherwise, $\mat{A}$ is a non-invertible matrix.
\end{itemize}
\end{description}
\end{block}

\onslide<+->
\begin{block}{Note}
This is far from the only method for calculating inverses, but it is the only one we will talk about. You are welcome to look up other methods on your own.
\end{block}
\end{frame}

\begin{frame}
\begin{example}\label{exmp1}
\begin{overprint}
\onslide<1-2>%
Consider the matrix:
\begin{equation*}
\mat{A}=
\begin{bmatrix}
\+1 & \+1 & \+1 \\
\+0 & \+2 & \+1 \\
\+1 & \+0 & \+1 \\
\end{bmatrix}
\end{equation*}
Find $\inverseof{\mat{A}}$

\visible<2>{%
Start by building the augmented matrix
\begin{equation*}
\mat{M_A}=
\begin{bmatrix}[rrr|rrr]
\+1 & \+1 & \+1 & \+1 & \+0 & \+0\\
\+0 & \+2 & \+1 & \+0 & \+1 & \+0\\
\+1 & \+0 & \+1 & \+0 & \+0 & \+1\\
\end{bmatrix}
\end{equation*}

Then transform $\mat{M_A}$ into Reduced Row Echelon Form.}
\onslide<3-5>
\Large
\begin{equation*}
	\begin{aligned}
		&	\begin{bmatrix}[rrr|rrr]
				\+1 & \+1 & \+1 & \+1 & \+0 & \+0\\
				\+0 & \+2 & \+1 & \+0 & \+1 & \+0\\
				\+1 & \+0 & \+1 & \+0 & \+0 & \+1\\
			\end{bmatrix}
			\visible<4-5>{\begin{aligned}
				& \phantom{R_1}\\
				& \phantom{R_2}\\
				& R_3=r_3-r_1
			\end{aligned}}\\
		\visible<5>{\Rightarrow
		&	\begin{bmatrix}[rrr|rrr]
				\+1 & \+1 & \+1 & \+1 & \+0 & \+0\\
				\+0 & \+2 & \+1 & \+0 & \+1 & \+0\\
				\+0 &  -1 & \+0 &  -1 & \+0 & \+1\\
			\end{bmatrix}}
	\end{aligned}
\end{equation*}
\onslide<6-8>
\Large
\begin{equation*}
	\begin{aligned}
		&	\begin{bmatrix}[rrr|rrr]
				\+1 & \+1 & \+1 & \+1 & \+0 & \+0\\
				\+0 & \+2 & \+1 & \+0 & \+1 & \+0\\
				\+0 &  -1 & \+0 &  -1 & \+0 & \+1\\
			\end{bmatrix}
			\visible<7-8>{\begin{aligned}
				& \phantom{R_1}\\
				& R_2=-r_3\\
				& R_3=r_2
			\end{aligned}}\\
		\visible<8>{\Rightarrow
		&	\begin{bmatrix}[rrr|rrr]
				\+1 & \+1 & \+1 & \+1 & \+0 & \+0\\
				\+0 & \+1 & \+0 & \+1 & \+0 &  -1\\
				\+0 & \+2 & \+1 & \+0 & \+1 & \+0\\
			\end{bmatrix}}
	\end{aligned}
\end{equation*}
\onslide<9-11>
\Large
\begin{equation*}
	\begin{aligned}
		&	\begin{bmatrix}[rrr|rrr]
				\+1 & \+1 & \+1 & \+1 & \+0 & \+0\\
				\+0 & \+1 & \+0 & \+1 & \+0 &  -1\\
				\+0 & \+2 & \+1 & \+0 & \+1 & \+0\\
			\end{bmatrix}
			\visible<10-11>{\begin{aligned}
				& \phantom{R_1}\\
				& \phantom{R_2}\\
				& R_3=r_3-2r_2
			\end{aligned}}\\
		\visible<11>{\Rightarrow
		&	\begin{bmatrix}[rrr|rrr]
				\+1 & \+1 & \+1 & \+1 & \+0 & \+0\\
				\+0 & \+1 & \+0 & \+1 & \+0 &  -1\\
				\+0 & \+0 & \+1 &  -2 & \+1 & \+2\\
			\end{bmatrix}}
	\end{aligned}
\end{equation*}
\onslide<12-14>
\Large
\begin{equation*}
	\begin{aligned}
		&	\begin{bmatrix}[rrr|rrr]
				\+1 & \+1 & \+1 & \+1 & \+0 & \+0\\
				\+0 & \+1 & \+0 & \+1 & \+0 &  -1\\
				\+0 & \+0 & \+1 &  -2 & \+1 & \+2\\
			\end{bmatrix}
			\visible<13-14>{\begin{aligned}
				& R_1=r_1-r_3\\
				& \phantom{R_2}\\
				& \phantom{R_3}
			\end{aligned}}\\
		\visible<14>{\Rightarrow
		&	\begin{bmatrix}[rrr|rrr]
				\+1 & \+1 & \+0 & \+3 &  -1 &  -2\\
				\+0 & \+1 & \+0 & \+1 & \+0 &  -1\\
				\+0 & \+0 & \+1 &  -2 & \+1 & \+2\\
			\end{bmatrix}}
	\end{aligned}
\end{equation*}
\onslide<15-17>
\Large
\begin{equation*}
	\begin{aligned}
		&	\begin{bmatrix}[rrr|rrr]
				\+1 & \+1 & \+0 & \+3 &  -1 &  -2\\
				\+0 & \+1 & \+0 & \+1 & \+0 &  -1\\
				\+0 & \+0 & \+1 &  -2 & \+1 & \+2\\
			\end{bmatrix}
			\visible<16-17>{\begin{aligned}
				& R_1=r_1-r_2\\
				& \phantom{R_2}\\
				& \phantom{R_3}
			\end{aligned}}\\
		\visible<17>{\Rightarrow
		&	\begin{bmatrix}[rrr|rrr]
				\+1 & \+0 & \+0 & \+2 &  -1 &  -1\\
				\+0 & \+1 & \+0 & \+1 & \+0 &  -1\\
				\+0 & \+0 & \+1 &  -2 & \+1 & \+2\\
			\end{bmatrix}}
	\end{aligned}
\end{equation*}
\onslide<18>
\begin{equation*}
\begin{bmatrix}[rrr|rrr]
	\+1 & \+0 & \+0 & \+2 &  -1 &  -1\\
	\+0 & \+1 & \+0 & \+1 & \+0 &  -1\\
	\+0 & \+0 & \+1 &  -2 & \+1 & \+2\\
\end{bmatrix}
\end{equation*}
Since the left hand side is $\identmat{3}$, we know the right hand side is the inverse:
\begin{equation*}
\inverseof{\mat{A}}=
\begin{bmatrix}[rrr]
\+2 &  -1 &  -1\\
\+1 & \+0 &  -1\\
 -2 & \+1 & \+2\\
\end{bmatrix}
\end{equation*}
\end{overprint}
\end{example}
\end{frame}

\begin{frame}
\begin{example}
\begin{overprint}
\onslide<1-2>%
Consider the matrix:
\begin{equation*}
\mat{B}=
\begin{bmatrix}
\+3 & \+0 & \+3 \\
 -1 & \+2 & \+1 \\
\+1 & \+1 & \+2 \\
\end{bmatrix}
\end{equation*}
Find $\inverseof{\mat{B}}$

\visible<2>{%
Start by building the augmented matrix
\begin{equation*}
\mat{M_B}=
\begin{bmatrix}[rrr|rrr]
\+3 & \+0 & \+3 & \+1 & \+0 & \+0\\
 -1 & \+2 & \+1 & \+0 & \+1 & \+0\\
\+1 & \+1 & \+2 & \+0 & \+0 & \+1\\
\end{bmatrix}
\end{equation*}

Then transform $\mat{M_B}$ into Reduced Row Echelon Form.}
\onslide<3-5>
\Large
\begin{equation*}
	\begin{aligned}
		&	\begin{bmatrix}[rrr|rrr]
				 \+3 & \+0 & \+3 & \+1 & \+0 & \+0\\
				  -1 & \+2 & \+1 & \+0 & \+1 & \+0\\
				 \+1 & \+1 & \+2 & \+0 & \+0 & \+1\\
			\end{bmatrix}
			\visible<4-5>{\begin{aligned}
				& R_1=r_3\\
				& \phantom{R_2}\\
				& R_3=r_1
			\end{aligned}}\\
		\visible<5>{\Rightarrow
		&	\begin{bmatrix}[rrr|rrr]
				 \+1 & \+1 & \+2 & \+0 & \+0 & \+1\\
				  -1 & \+2 & \+1 & \+0 & \+1 & \+0\\
				 \+3 & \+0 & \+3 & \+1 & \+0 & \+0\\
			\end{bmatrix}}
	\end{aligned}
\end{equation*}
\onslide<6-8>
\Large
\begin{equation*}
	\begin{aligned}
		&	\begin{bmatrix}[rrr|rrr]
				 \+1 & \+1 & \+2 & \+0 & \+0 & \+1\\
				  -1 & \+2 & \+1 & \+0 & \+1 & \+0\\
				 \+3 & \+0 & \+3 & \+1 & \+0 & \+0\\
			\end{bmatrix}
			\visible<7-8>{\begin{aligned}
				& \phantom{R_1}\\
				& R_2=r_2+r_1\\
				& R_3=r_2-3r_1
			\end{aligned}}\\
		\visible<8>{\Rightarrow
		&	\begin{bmatrix}[rrr|rrr]
				 \+1 & \+1 & \+2 & \+0 & \+0 & \+1\\
				 \+0 & \+3 & \+3 & \+0 & \+1 & \+1\\
				 \+0 &  -3 &  -3 & \+1 & \+0 &  -1\\
			\end{bmatrix}}
	\end{aligned}
\end{equation*}
\onslide<9-12>
\Large
\begin{equation*}
	\begin{aligned}
		&	\begin{bmatrix}[rrr|rrr]
				 \+1 & \+1 & \+2 & \+0 & \+0 & \+1\\
				 \+0 & \+3 & \+3 & \+0 & \+1 & \+1\\
				 \+0 &  -3 &  -3 & \+1 & \+0 &  -1\\
			\end{bmatrix}
			\visible<10-12>{\begin{aligned}
				& \phantom{R_1}\\
				& R_2=\tfrac{1}{3}r_2\\
				& R_3=r_3+r_2
			\end{aligned}}\\
		\visible<11-12>{\Rightarrow
		&	\begin{bmatrix}[rrr|rrr]
				 \+1 & \+1 & \+2 & \+0 & \+0 & \+1\\
				 \+0 & \+1 & \+1 & \+0 & \+\frac{1}{3} & \+\frac{1}{3}\\
				 \+0 & \+0 & \+0 & \+1 & \+1 &  0\\
			\end{bmatrix}}
	\end{aligned}
\end{equation*}
\visible<12>{%
\begin{center}
This means that $\mat{B}$ is a non-invertible matrix.
\end{center}}
\end{overprint}
\end{example}
\end{frame}

\begin{frame}
\begin{block}{Invertibility and Solutions}
Consider the matrix equation $\mat{A}\vect{x}=\vect{b}$.\\
Where $\mat{A}$ is an $n \by n$ matrix, and $\vect{x}$ and $\vect{b}$ are of length $n$.
\onslide<+->
\begin{itemize}
\item<+-> A unique solution exists if and only if $\mat{A}$ is invertible.
\item<+-> Otherwise there are either:
\begin{itemize}
\item No solutions.
\item Infinitely many solutions.
\end{itemize}
(Another method must be used to determine which.)
\end{itemize}
\end{block}
\end{frame}

\begin{frame}
\begin{example}
\usepercentframe{0.75}{
\only<1-2>{
Consider the system
\begin{center}
\begin{tabular}{rcrcrcr}
$x$ & $+$ & $y$  & $+$ & $z$ & $=$ & $2$\\
    &     & $2y$ & $+$ & $z$ & $=$ & $-1$\\
$x$ &     &      & $+$ & $z$ & $=$ & $3$
\end{tabular}
\end{center}\pause

We can can write this as the matrix equation:
\begin{equation*}
\underbrace{\begin{bmatrix}
\+1 & \+1 & \+1 \\
\+0 & \+2 & \+1 \\
\+1 & \+0 & \+1 \\
\end{bmatrix}}_{\mat{A}}
\underbrace{\begin{bmatrix}
x\\
y\\
z
\end{bmatrix}}_{\vect{x}}
=
\underbrace{\begin{bmatrix}
\+2\\
-1\\
\+3
\end{bmatrix}}_{\vect{b}}
\end{equation*}}
\only<3-7>{
We know from Example~\ref{exmp1} that $\mat{A}$ is invertible. 

\vspace{2mm}
This means we need solve the matrix equation for $\vect{x}$
\begin{equation*}
\begin{split}
\mat{A}\vect{x}&=\vect{b}\\
\visible<4->{\inverseof{\mat{A}}\mat{A}\vect{x}&=\inverseof{\mat{A}}\vect{b}}\\
\visible<5->{\identmat{3}\vect{x}&=\inverseof{\mat{A}}\vect{b}}\\
\visible<6->{\vect{x}&=\inverseof{\mat{A}}\vect{b}}
\end{split}
\end{equation*}
\visible<7->{So, if we can compute $\inverseof{\mat{A}}\vect{b}$ we will have solved the system.}}
\only<8->{
\begin{center}
\begin{tabular}{r|l}
&
\begin{minipage}{2.4cm}
%	\begin{equation*}
		$\left[
			\begin{array}{rr}
				\tikzmarkin<8-10>[ver2=style cyan]{col-a}\+2\\
				-1 \\
				\+0\tikzmarkend{col-a}\\
			\end{array}
		\right]$
%	\end{equation*}
\end{minipage}\\\\
\hline\\
\begin{minipage}{3.1cm}
%	\begin{equation*}
		$\left[
			\begin{array}{rrr}
				\tikzmarkin<8>[hor=style orange]{row-a}\+2  & -1 & -1\tikzmarkend{row-a} \\
				\tikzmarkin<9>[hor=style orange]{row-b}\+1 & \+0 & -1 \tikzmarkend{row-b}\\
				\tikzmarkin<10>[hor=style orange]{row-c}-2 & \+1 & \+2 \tikzmarkend{row-c}\\
			\end{array}
		\right]$
%	\end{equation*}
\end{minipage}
&
\begin{minipage}{2.4cm}
%	\begin{equation*}
		$\left[
			\begin{array}{rr}
				\tikzmarkin<8>[hor=style green]{end-a}\visible<8->{\+5} \tikzmarkend{end-a}\\
				\tikzmarkin<9>[hor=style green]{end-b}\visible<9->{\+2}\tikzmarkend{end-b}\\
				 \tikzmarkin<10>[hor=style green]{end-c}\visible<10->{-5}\tikzmarkend{end-c} 
			\end{array}
		\right]$
%	\end{equation*}
\end{minipage}
\end{tabular}
\end{center}
\visible<11->{So, we have 
\begin{equation*}
\begin{bmatrix}
x\\y\\z
\end{bmatrix}
=
\begin{bmatrix}
\+5\\\+2\\-5
\end{bmatrix}
\end{equation*}}}}
\end{example}
\end{frame}

\begin{frame}
\begin{block}{Invertible Matrix Characterization}
Let $\mat{A}$ be a $n \by n$ matrix. The following are equivalent:
\begin{itemize}
\item<+-> $\mat{A}$ is an invertible matrix.
\item<+-> $\transposeof{\mat{A}}$ is an invertible matrix.
\item<+-> $\mat{A}$ is row equivalent to $\identmat{n}$.\\ (This means when you put $\mat{A}$ in RREF, you get $\identmat{n}$)
\item<+-> The rank of $\mat{A}$ is $n$.
\item<+-> The equation $\mat{A}\vect{x}=\vect{0}$ has only the trivial solution $\vect{x}=\vect{0}$.
\item<+-> The equation $\mat{A}\vect{x}=\vect{b}$ has a unique solution for every $\vect{b}\in\R^n$.
\end{itemize}
\end{block}
\end{frame}

\begin{frame}
\begin{example}
\usepercentframe{0.77}{
\only<1-2>{An engineering consultant given the following IVP\@:
\begin{equation*}
y^{\prime\prime\prime} - 2y^{\prime\prime}-y^\prime+2y=0,\quad y(0)=b_1,\ y^\prime(0)=b_2,\ y^{\prime\prime}(0)=b_3
\end{equation*}
She must solve this IVP for many different sets of initial conditions, and expects to do the same tomorrow.\pause

The general solution is:
\begin{equation*}
y(t) = c_1 e^{2t}+c_2 e^{t} + c_3 e^{-t}
\end{equation*}
(We will talk about how to solve this type of DE in Chapter 4.)}
\only<3-4>{To determine $c_1$, $c_2$, and $c_3$, we must plug in each initial condition, giving the system:
\begin{equation*}
\begin{aligned}
y(0)                &=  \phantom{2}c_1 + c_2 + c_3 &= b_1\\
y^{\prime}(0)       &=           2 c_1 + c_2 - c_3 &= b_2\\
y^{\prime\prime}(0) &=           4 c_1 + c_2 + c_3 &= b_3
\end{aligned}
\end{equation*}
\visible<4>{
We can write this as the matrix equation:
\begin{equation*}
\underbrace{\begin{bmatrix}
\+1 & \+1 & \+1\\
\+2 & \+1 &  -1\\
\+4 & \+1 & \+1
\end{bmatrix}}_{\mat{A}}
\underbrace{\begin{bmatrix}
c_1\\
c_2\\
c_3
\end{bmatrix}}_{\vect{x}}
=
\underbrace{\begin{bmatrix}
b_1\\
b_2\\
b_3
\end{bmatrix}}_{\vect{b}}
\end{equation*}}}
\only<5->{If we can find the inverse of $\mat{A}$, then we can compute the constants for any set of initial conditions $\vect{b}$.

\visible<6->{
\begingroup
\renewcommand*{\arraystretch}{1.5}
\begin{equation*}
\inverseof{\mat{A}}=\begin{bmatrix}
 -\tfrac{1}{3} & \+0            & \+\tfrac{1}{3} \\
\+1            & \+\tfrac{1}{2} &  -\tfrac{1}{2} \\
\+\tfrac{1}{3} &  -\tfrac{1}{2} & \+\tfrac{1}{6}
\end{bmatrix}
\end{equation*}
\endgroup
\visible<7->{Thus, the solution for any $\vect{b}$ is:
\begingroup
\renewcommand*{\arraystretch}{1.5}
\begin{equation*}
\begin{bmatrix}
c_1\\
c_2\\
c_3
\end{bmatrix}
=
\begin{bmatrix}
 -\tfrac{1}{3} & \+0            & \+\tfrac{1}{3} \\
\+1            & \+\tfrac{1}{2} &  -\tfrac{1}{2} \\
\+\tfrac{1}{3} &  -\tfrac{1}{2} & \+\tfrac{1}{6}
\end{bmatrix}
\begin{bmatrix}
b_1\\
b_2\\
b_3
\end{bmatrix}
\end{equation*}
\endgroup}}}}
\end{example}
\end{frame}
\end{document}