\documentclass{beamer}

\usepackage[english]{babel}
\usepackage[utf8x]{inputenc}
\usepackage[inline]{asymptote}
\usepackage{slide_helper}
\usepackage{caption}
\usepackage{subcaption}

\begin{asydef}
import graph;
import fontsize;
defaultpen(fontsize(9pt));
ngraph=1000;
\end{asydef}

\title[MA245 - Section 4.6]{Forced Oscillations}

\makeatletter
\let\@@magyar@captionfix\relax
\makeatother

\begin{document}

\begin{frame}
  \titlepage
\end{frame}

\begin{frame}
\begin{example}
Let us look at the mass-spring system described by
\begin{equation*}
\ndot{x}{2}+x=2\cos[3t]
\end{equation*}\pause
The characteristic equation is $r^2+1=0$ and thus
\begin{equation*}
x_h=c_1\cos[t]+c_2\sin[t]
\end{equation*}\pause
Using the method of undetermined coefficients
\begin{equation*}
x_p=A\cos[3t]+B\sin[3t]
\end{equation*}\pause
Which leads to
\begin{equation*}
x_p=-\dfrac{1}{4}\cos[3t]
\end{equation*}\pause
and 
\begin{equation*}
x=c_1\cos[t]+c_2\sin[t]-\dfrac{1}{4}\cos[3t]
\end{equation*}
\end{example}
\end{frame}

\begin{frame}
\begin{block}{General Solution}
We can now look at the general solution for the undamped system
\begin{equation*}
m\ndot{x}{2}+kx=F_0\cos[\omega_f t]
\end{equation*}
Where $\omega_f$ is the \textbf{forcing frequency} and $F_0$ is the \textbf{forcing amplitute}.\pause

\vspace{2mm}
We know that
\begin{equation*}
x_h=c_1\cos[\omega_0 t]+c_2\sin[\omega_0 t]
\quad\text{where}\quad
\omega_0=\sqrt{\dfrac{k}{m}}
\end{equation*}\pause
This leaves two separate cases for $x_p$:
\begin{enumerate}
\item The frequencies $\omega_f$ and $\omega_0$ are different.
\item The frequencies $\omega_f$ and $\omega_0$ are the same.
\end{enumerate}
\end{block}
\end{frame}

\begin{frame}
\begin{block}{Unmatched Frequencies ($\omega_f\neq\omega_0$)}
This means we want to look for
\begin{equation*}
x_p=A\cos[\omega_f t]+B\sin[\omega_f t]
\end{equation*}\pause
and so, we find that
\begin{equation*}
A=\dfrac{F_0}{m\left(\omega_0^2-\omega_f^2\right)}
\quad\text{and}\quad
B=0
\end{equation*}\pause
So, where $c_1$ and $c_2$ are determined by initial conditions, we have
\begin{equation*}
\begin{aligned}
x(t)&=c_1\cos[\omega_0 t]+c_2\sin[\omega_0 t]+\dfrac{F_0}{m\left(\omega_0^2-\omega_f^2\right)}\cos[\omega_f t]\\\pause
&= C\cos[\omega_0 t-\delta]+\dfrac{F_0}{m\left(\omega_0^2-\omega_f^2\right)}\cos[\omega_f t]
\end{aligned}
\end{equation*}
where $C=\sqrt{c_1^2+c_2^2}$ and $\tan[\delta]=\dfrac{c_2}{c_1}$.
\end{block}
\end{frame}

\begin{frame}[fragile]
\begin{example}
An example of $\omega_f\neq\omega_0$ being different, but close, is
\begin{center}
\begin{asy}
size(300,180,IgnoreAspect);

real min_x=0, max_x=4;
real min_y=-0.2, max_y=0.2;

real m=1;
real F_0=-120;

real omega_o=23*pi;
real omega_f=20*pi;

real k=m*omega_o*omega_o;

real D=(2*F_0)/(m*(omega_o*omega_o-omega_f*omega_f));
real A=((omega_o-omega_f)/2);
real B=((omega_o+omega_f)/2);

real f(real t) 
{
	real part_1 = D;
	real part_2 = sin(A*t);
	real part_3 = sin(B*t);
	return part_1*part_2*part_3;
} 
real upper(real t) { return -D*sin(A*t);}
real lower(real t) { return D*sin(A*t);}

draw(graph(f,min_x, max_x), blue+0.5bp);
//draw(graph(upper, min_x, max_x), dashed+red+0.5bp);
//draw(graph(lower, min_x, max_x), dashed+red+0.5bp);

limits((min_x,min_y),(max_x,max_y),Crop);
xaxis("$t$",YEquals(0),min_x,max_x,Ticks(NoZero));
yaxis("$x$",XEquals(0),min_y,max_y,Ticks(NoZero));
\end{asy}
\end{center}\pause
The regular periodic patterns are called \textbf{beats}.
\end{example}
\end{frame}

\begin{frame}
\begin{block}{Resonance ($\omega_f=\omega_0$)}
This means we want to look at
\begin{equation*}
x_p=At\cos[\omega_0 t]+Bt\sin[\omega_0 t]
\end{equation*}\pause
and so, we find that
\begin{equation*}
A=0
\quad\text{and}\quad
B=\dfrac{F_0}{2m\omega_0}
\end{equation*}\pause
So, where $c_1$ and $c_2$ are determined by initial conditions, we have
\begin{equation*}
x(t)=c_1\cos[\omega_0 t]+c_2\sin[\omega_0 t]+\dfrac{F_0}{2m\omega_0}t\sin[\omega_0 t]
\end{equation*}
\end{block}
\end{frame}

\begin{frame}[fragile]
\begin{example}
We can see an example of resonance
\begin{equation*}
\ndot{x}{2}+x=2\cos[t]
,\quad x(0)=0
,\quad \ndot{x}{1}(0)=0
\end{equation*}\pause
Since $\dfrac{F_0}{2m\omega_0}=\dfrac{2}{2\cdot 1\cdot 1}=1$, we have $x_p=t\sin[t]$.\pause

\vspace{2mm}
So, given that $x(0)=0$ and $\ndot{x}{1}(0)=0$ we have $x=t\sin[t]$.\pause
\begin{center}
\begin{asy}
size(300,130,IgnoreAspect);

real min_x=0, max_x=100;
real min_y=-100, max_y=100;

real m=1;
real F_0=2;

real omega=1;

real A=F_0/(2*m*omega);
real c_1=0;
real c_2=0;

real f(real t) {return c_1*cos(t)+c_2*sin(t) + A*t*sin(t);}
real upper(real t) {return A*t;}
real lower(real t) {return -A*t;}

draw(graph(f,min_x, max_x), blue+0.5bp);
draw(graph(upper,min_x, max_x), dashed+red+0.5bp);
draw(graph(lower,min_x, max_x), dashed+red+0.5bp);

limits((min_x,min_y),(max_x,max_y),Crop);
xaxis("$t$",YEquals(0),min_x,max_x,Ticks(NoZero));
yaxis("$x$",XEquals(0),min_y,max_y,Ticks(NoZero));
\end{asy}
\end{center}
\end{example}
\end{frame}

\begin{frame}
\begin{block}{Beats}
\usepercentframe{0.87}{\only<1-4>{%
As we saw earlier when $\omega_f$ is close to $\omega_0$, though still not equal, there is clearly periodic behavior. 
\onslide<2->

\vspace{2mm}
This periodic pattern has a frequency, called the \textbf{beat frequency}, that is lower than both $\omega_f$ and $\omega_0$.
\onslide<3->

\vspace{2mm}
If the system is initially at rest ($x(0)=0$ and $\ndot{x}{1}(0)=0$) then the solution is
\begin{equation*}
x(t)=-\dfrac{F_0}{m\left(\omega_0^2-\omega_f^2\right)}\cos[\omega_0 t] + \dfrac{F_0}{m\left(\omega_0^2-\omega_f^2\right)}\cos[\omega_f t]
\end{equation*}
\onslide<4->
So, we can simplify using the trigonometric identity
\begin{equation*}
\cos[u]-\cos[v]=-2\sin[\dfrac{u-v}{2}]\sin[\dfrac{u+v}{2}]
\end{equation*}\vspace{1cm}}
\only<5->{%
\begin{equation*}
\begin{aligned}
x(t)&=-\dfrac{F_0}{m\left(\omega_0^2-\omega_f^2\right)}\cos[\omega_0 t] + \dfrac{F_0}{m\left(\omega_0^2-\omega_f^2\right)}\cos[\omega_f t]\\ 
\visible<6->{&=-\dfrac{F_0}{m\left(\omega_0^2-\omega_f^2\right)}\left(\cos[\omega_0 t] - \cos[\omega_f t]\right)\\}
\visible<7->{&=\dfrac{2F_0}{m\left(\omega_0^2-\omega_f^2\right)}\sin[\dfrac{\omega_0-\omega_f}{2}t]\sin[\dfrac{\omega_0+\omega_f}{2}t]}
\end{aligned}
\end{equation*}
\onslide<8->
When the difference between $\omega_f$ and $\omega_0$ is small, then $\omega_0-\omega_f$ is much smaller than $\omega_0+\omega_f$. 
\onslide<9->

\vspace{2mm}
Thus $\sin[\dfrac{\omega_0-\omega_f}{2}t]$ oscillates much slower than $\sin[\dfrac{\omega_0+\omega_f}{2}t]$.
\onslide<10->

\vspace{2mm}
The two curves
\begin{equation*}
\pm \dfrac{2F_0}{m\left(\omega_0^2-\omega_f^2\right)}\sin[\dfrac{\omega_0-\omega_f}{2}t]
\end{equation*}
form an envelope of the more rapid oscillation and is called the \textbf{sinusoidal amplitude}.}}
\end{block}
\end{frame}

\begin{frame}
\begin{block}{Solutions to the Undamped Forced Oscillator ($\omega_f\neq\omega_0$)}
For
\begin{equation*}
m\ndot{x}{2}+kx=F_0\cos[\omega_f t]
\end{equation*}\pause
The general solution is
\begin{equation*}
x(t)=c_1\cos[\omega_0 t]+c_2\cos[\omega_0 t] + \dfrac{F_0}{m\left(\omega_0^2-\omega_f^2\right)}\cos[\omega_f t]
\end{equation*}
where $c_1$ and $c_2$ are determined by initial conditions and $\omega_0=\sqrt{\tfrac{k}{m}}$.\pause

\vspace{2mm}
If the system starts from rest ($x(0)=0$ and $\ndot{x}{1}(0)=0$), the solution can be written as
\begin{equation*}
x(t)=\underbrace{\dfrac{2F_0}{m\left(\omega_0^2-\omega_f^2\right)}\sin[\dfrac{\omega_0-\omega_f}{2}t]}_{\text{sinusoidal amplitude}}
\underbrace{\sin[\dfrac{\omega_0+\omega_f}{2}t]}_{\substack{\text{rapid oscillation} \\ \text{within beats}}}
\end{equation*}
\end{block}
\end{frame}

\begin{frame}[fragile]
\begin{example}
\begin{overprint}
\onslide<1-9>
Given an oscillator with $\omega_0=22\pi$, $\omega_f=20\pi$, $m=1$, and $F_0=42\pi^2$.
\visible<2->{

\vspace{2mm}
We can then calculate $k$:
\visible<3->{%
\begin{equation*}
\omega_0=\sqrt{\dfrac{k}{m}}
\visible<4->{\quad\rightarrow\quad
\omega_0^2=\dfrac{k}{m}}
\visible<5->{\quad\rightarrow\quad
k=m\omega_0^2
\visible<6->{=484\pi^2}}
\end{equation*}}}

\visible<7->{\vspace{-5mm}
Since
\begin{equation*}
\dfrac{\omega_0-\omega_f}{2}=\pi
\qquad\text{and}\qquad
\dfrac{\omega_0+\omega_f}{2}=21\pi
\end{equation*}
\visible<8->{The solution is 
\begin{equation*}
x(t)=1\cdot\sin[\pi t]\sin[21\pi t]
\end{equation*}}}

\visible<9->{\vspace{-5mm}
Thus, envelope curves are
\begin{equation*}
y=\pm 1\cdot\sin[\pi t]
\end{equation*}}
\onslide<10->
\vspace{1mm}
\begin{center}
\begin{asy}
size(300,220,IgnoreAspect);

real min_x=0, max_x=3;
real min_y=-1.1, max_y=1.1;

real m=1;
real F_0=42*pi*pi;

real omega_o=22*pi;
real omega_f=20*pi;

real k=m*omega_o*omega_o;

real D=(2*F_0)/(m*(omega_o*omega_o-omega_f*omega_f));
real A=((omega_o-omega_f)/2);
real B=((omega_o+omega_f)/2);

real f(real t) 
{
	real part_1 = D;
	real part_2 = sin(A*t);
	real part_3 = sin(B*t);
	return part_1*part_2*part_3;
} 
real upper(real t) { return -D*sin(A*t);}
real lower(real t) { return D*sin(A*t);}

draw(graph(f,min_x, max_x), blue+0.5bp);
draw(graph(upper, min_x, max_x), dashed+red+0.5bp);
draw(graph(lower, min_x, max_x), dashed+red+0.5bp);

limits((min_x,min_y),(max_x,max_y),Crop);
xaxis("$t$",YEquals(0),min_x,max_x,Ticks(NoZero));
yaxis("$x$",XEquals(0),min_y,max_y,Ticks(NoZero));
\end{asy}
\end{center}
\vspace{-6mm}
\end{overprint}
\end{example}
\end{frame}

\begin{frame}[fragile]
\begin{example}
\begin{overprint}
\onslide<1-5>
Let us consider the damped forced oscillator
\begin{equation*}
\ndot{x}{2}+4\ndot{x}{1}+5x=10\cos[3t]
,\qquad x(0)=\ndot{x}{1}(0)=0
\end{equation*}
\visible<2->{First, we need to find the solution to the associated homogeneous equation.}

\visible<3->{\vspace{2mm}
The characteristic equation $r^2+4r+5=0$ has solutions $r=-2\pm i$, so we have
\begin{equation*}
x_h=e^{-2t}\left(c_1\cos[t]+c_2\sin[t]\right)
\end{equation*}}

\visible<4->{\vspace{-5mm}
Next we need to find a particular solution.}

\visible<5->{\vspace{2mm}
We can use the method of undetermined coefficients:
\begin{equation*}
\begin{matrix}[rrlrl]
x(t)        &=& \phantom{-3}A\cos[3t]   &+& \phantom{3}B\sin[3t]\\
\dot{x}(t)  &=& -3A\sin[3t] &+& 3B\cos[3t]\\
\ddot{x}(t) &=& -9A\cos[3t] &-& 9B\sin[3t]\\
\end{matrix}
\end{equation*}}

\onslide<6-9>
Substituting into the DE gives
\begin{equation*}
\begin{aligned}
\left(-9A\cos[3t]-9B\sin[3t]\right)\qquad\qquad\qquad&\\
+4\left(-3A\sin[3t]+3B\cos[3t]\right)\qquad&\\
+5\left(A\cos[3t]+B\sin[3t]\right)&=10\cos[3t]
\end{aligned}
\end{equation*}
\visible<7->{Which gives the system
\begin{equation*}
\begin{aligned}
-9A &+ 12B &+ 5A &=10\\
-9B &- 12A &+ 5B &=0
\end{aligned}
\end{equation*}}

\visible<8->{\vspace{-5mm}
Which has solution
\begin{equation*}
A=-\dfrac{1}{4}
\quad\text{and}\quad
B=\dfrac{3}{4}
\end{equation*}}

\visible<9->{\vspace{-5mm}
Thus,
\begin{equation*}
x_p=-\dfrac{1}{4}\cos[3t]+\dfrac{3}{4}\sin[3t]
\end{equation*}}

\onslide<10-14>
The general solution is
\begin{equation*}
x=e^{-2t}\left(c_1\cos[t]+c_2\sin[t]\right) -\dfrac{1}{4}\cos[3t]+\dfrac{3}{4}\sin[3t]
\end{equation*}
\visible<11->{To solve the IVP, we need to calculate
\begin{equation*}
\begin{aligned}
\dot{x}&=-2e^{-2t}\left(c_1\cos[t]+c_2\sin[t]\right) \\
&\qquad+ e^{-2t}\left(-c_1\sin[t]+c_2\cos[t]\right) \\
&\qquad\qquad+ \dfrac{1}{4}\sin[3t]-\dfrac{3}{4}\cos[3t]
\end{aligned}
\end{equation*}}
\visible<12->{%
Next, we need to plug in the initial conditions
\begin{equation*}
\begin{aligned}
x(0)=0
\visible<13->{&\quad\Rightarrow\quad c_1-\dfrac{1}{4}=0}
\visible<14->{&\quad\Rightarrow\quad c_1=\+\dfrac{1}{4}}\\
\dot{x}(0)=0
\visible<13->{&\quad\Rightarrow\quad c_2+\dfrac{7}{4}=0}
\visible<14->{&\quad\Rightarrow\quad c_2=-\dfrac{7}{4}}\\
\end{aligned}
\end{equation*}}

\onslide<15-17>
The solution to the IVP is
\begin{equation*}
\only<15>{x={e^{-2t}\left(\dfrac{1}{4}\cos[t]-\dfrac{7}{4}\sin[t]\right)}{-\dfrac{1}{4}\cos[3t]+\dfrac{3}{4}\sin[3t]}}
\only<16->{x=\underbrace{e^{-2t}\left(\dfrac{1}{4}\cos[t]-\dfrac{7}{4}\sin[t]\right)}_{\text{Transient}}\underbrace{-\dfrac{1}{4}\cos[3t]+\dfrac{3}{4}\sin[3t]}_{\text{Steady-State}}}
\end{equation*}
\visible<16->{We call $x_h$ \textbf{transient}, because for $b>0$ the solution tends towards zero.}

\visible<17->{\vspace{2mm}
The particular solution $x_p$ may either be constant or a periodic \textbf{steady-state} solution.}

\onslide<18->
\vspace{1cm}
\begin{figure}
\centering
\begin{subfigure}[b]{0.29\textwidth}
\begin{asy}
size(100,150,IgnoreAspect);

real min_x=-1, max_x=8;
real min_y=-1, max_y=1;

real f(real t) 
{
	real A=exp(-2*t);
	real B=(1/4)*cos(t);
	real C=(-7/4)*sin(t);
	return A*(B+C);
}

draw(graph(f ,0, max_x), blue+0.5bp);

limits((min_x,min_y),(max_x,max_y),Crop);
xaxis("$t$",YEquals(0),min_x,max_x,Ticks(NoZero));
yaxis("$y$",XEquals(0),min_y,max_y,Ticks(NoZero));
\end{asy}
\caption{Transient Solution}
\end{subfigure}
\begin{subfigure}[b]{0.29\textwidth}
\begin{asy}
size(100,150,IgnoreAspect);

real min_x=-1, max_x=8;
real min_y=-1, max_y=1;

real f(real t) 
{
	real D=(-1/4)*cos(3*t);
	real E=(3/4)*sin(3*t);
	return D+E;
}

draw(graph(f ,0, max_x), blue+0.5bp);

limits((min_x,min_y),(max_x,max_y),Crop);
xaxis("$t$",YEquals(0),min_x,max_x,Ticks(NoZero));
yaxis("$y$",XEquals(0),min_y,max_y,Ticks(NoZero));
\end{asy}
\caption{Steady-State}
\end{subfigure}
\begin{subfigure}[b]{0.29\textwidth}
\begin{asy}
size(100,150,IgnoreAspect);

real min_x=-1, max_x=8;
real min_y=-1, max_y=1;

real f(real t) 
{
	real A=exp(-2*t);
	real B=(1/4)*cos(t);
	real C=(-7/4)*sin(t);
	real D=(-1/4)*cos(3*t);
	real E=(3/4)*sin(3*t);
	return A*(B+C)+D+E ;
}

draw(graph(f ,0, max_x), blue+0.5bp);

limits((min_x,min_y),(max_x,max_y),Crop);
xaxis("$t$",YEquals(0),min_x,max_x,Ticks(NoZero));
yaxis("$y$",XEquals(0),min_y,max_y,Ticks(NoZero));
\end{asy}
\caption{IVP Solution}
\end{subfigure}
\end{figure}
\end{overprint}
\end{example}
\end{frame}

\begin{frame}
\begin{block}{Particular Solution $x_p$ of a Damped Mass-Spring System}
The damped mass-spring system
\begin{equation*}
m\ndot{x}{2} + b\ndot{x}{1} +kx=F_0\cos[\omega_f t]
\end{equation*}
has particular solution
\begin{equation*}
x_p=A\cos[\omega_f t] +B\sin[\omega_f t]
\end{equation*}
with
\begin{equation*}
A=\dfrac%
{m\left(\omega_0^2-\omega_f^2\right)F_0}% top
{m^2{\left(\omega_0^2-\omega_f^2\right)}^2+{\left(b\omega_f\right)}^2}% bottom
\quad\text{and}\quad
B=\dfrac%
{b\omega_f F_0}% top
{m^2{\left(\omega_0^2-\omega_f^2\right)}^2+{\left(b\omega_f\right)}^2}% bottom
\end{equation*}
with natural circular frequency $\omega_0=\sqrt{\tfrac{k}{m}}$.
\end{block}\pause
\begin{block}{Note}
You will verify this in the homework.
\end{block}
\end{frame}
\end{document}
