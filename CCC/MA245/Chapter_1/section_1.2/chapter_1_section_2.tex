\documentclass{beamer}
\usepackage[utf8]{inputenc}
\usepackage[english]{babel}
\usepackage[T1]{fontenc}
\usepackage[inline]{asymptote}
\usepackage{slide_helper}
\usepackage{asy_helper}

\begin{asydef}
import graph;
import slopefield;
import fontsize;
defaultpen(fontsize(9pt));
ngraph=1000;

int pen_pos=-1;
pen[] pens={blue, red, heavycyan, heavymagenta, lightolive};
pens.cyclic=true;
pen next_color() {return pens[++pen_pos];}
void reset_color() {pen_pos = -1;}

DefaultHead.size=new real(pen p=currentpen) {return 2.5mm;};
\end{asydef}

\title[MATH 2250 - Section 1.2]{Solution and Direction Fields: Qualitative Analysis}

\begin{document}
\begin{frame}
\titlepage
\end{frame}

\begin{frame}
\begin{block}{What is a Differential Equation?}
A \textbf{differential equation} (or DE) is an equation containing derivatives. The \textbf{order} of the equation refers to the highest-order derivative that occurs.\pause

\vspace{2mm}
In this chapter we will focus on DEs that can be written as:
\begin{equation*}
\dfrac{dy}{dt}=f(t,y)
\quad\text{or}\quad
y^{\prime}=f(t,y)
\end{equation*}
Where the dependent variable $y$ is an unknown function, the \textbf{solution}.
\end{block}\pause

\begin{block}{Note}
There may be more than one solution for a given differential equation.
\end{block}\pause

\begin{block}{Analytic Definition of a Solution}
Analytically, $y(t)$ is a \textbf{solution} of a differential equation if substituting $y(t)$ for $y$ reduced the equation to an identity:

\vspace{-4mm}
\begin{equation*}
y^{\prime}(t)=f(t, y(t))
\end{equation*}

\vspace{-2mm}
on an appropriate domain for $t$.
\end{block}
\end{frame}

\begin{frame}[fragile]
\begin{example}
Verify that $y(t)$ is a solution to the DE\@.
\begin{equation*}
y^{\prime}(t)=2y,\quad y(t)=e^{2t}
\end{equation*}
\begin{overprint}
\onslide<1>
\onslide<2-6>
Substituting into the DE gives:
\begin{equation*}
\begin{aligned}
y^{\prime}(t) &= \dfrac{d}{dt}e^{2t} \\
\visible<3->{&=2e^{2t} \\}
\visible<4->{&=2y(t) \\}
\visible<5->{&=f(t, y(t))}
\end{aligned}
\end{equation*}
\visible<6->{%
Similarly, we could show that
\begin{equation*}
y(t)=2e^{2t}
\quad\text{and}\quad
y(t)=\dfrac{-3}{2}e^{2t}
\end{equation*}
are also solutions. In fact, any constant multiple of $e^{2t}$ is a solution.}
\onslide<7->
\begin{center}
\begin{asy}
size(180, IgnoreAspect);

real min_x=-2, max_x=2;
real min_y=-15, max_y=15;

pair start=(min_x,min_y);
pair end=(max_x,max_y);

real y_1(real t) {return exp(2t);}
real y_2(real t) {return 2*exp(2t);}
real y_3(real t) {return (-3/2)*exp(2t);}

draw(graph(y_1, min_x, max_x), next_color() + 1bp);
draw(graph(y_2, min_x, max_x), next_color() + 1bp);
draw(graph(y_3, min_x, max_x), next_color() + 1bp);

reset_color();

label("$y=e^{2t}$", (1.3,5), next_color());
label("$y=2 e^{2t}$", (0.45,12), next_color());
label("$y=-\dfrac{3}{2} e^{2t}$", (1.5,-8), next_color());

limits(start,end,Crop);

xaxis("$t$",YEquals(0),Ticks(NoZero));
yaxis("$y$",XEquals(0),Ticks(NoZero),autorotate=false);
\end{asy}
\end{center}
\end{overprint}
\vspace{-58mm}
\end{example}
\end{frame}

\begin{frame}[fragile]
\begin{example}
Verify that $y(t)$ is a solution to the DE\@.
\begin{equation*}
y^{\prime}(t)=-\dfrac{t}{y},\quad y(t)=\sqrt{1-t^2}
\end{equation*}
\begin{overprint}
\onslide<1>
\onslide<2-6>
Substituting into the DE gives:
\begin{equation*}
\begin{aligned}
y^\prime(t) &= \dfrac{d}{dt}\left(\sqrt{1-t^2}\right) \\
\visible<3->{&= \dfrac{1}{2}{\left(1-t^2\right)}^{-\tfrac{1}{2}}\cdot(-2t)\\}
\visible<4->{&=\dfrac{-t}{\sqrt{1-t^2}}\\}
\visible<5->{&=-\dfrac{t}{y}}
\end{aligned}
\end{equation*}

\visible<6->{\vspace{-2mm}
Other solutions are of the form $y(t)=\sqrt{k-t^2}$.}
\onslide<7>
\begin{center}
\begin{asy}
size(250);

real min_x=-2, max_x=2;
real min_y=0, max_y=2;

pair start=(min_x,min_y);
pair end=(max_x,max_y);

real y_1(real t) {return sqrt(abs(1.0-t*t));}
real y_2(real t) {return sqrt(abs(0.5-t*t));}
real y_3(real t) {return sqrt(abs(3-t*t));}

draw(graph(y_1, -1, 1), next_color() + 1bp);
draw(graph(y_2, -sqrt(0.5), sqrt(0.5)), next_color() + 1bp);
draw(graph(y_3, -sqrt(3), sqrt(3)), next_color() + 1bp);

reset_color();

limits(start,end,Crop);

xaxis("$t$",YEquals(0),Ticks(NoZero));
yaxis("$y$",XEquals(0),Ticks(NoZero),autorotate=false);

label("$y=\sqrt{1-t^2}$",   (0.0,1.15), next_color(), UnFill());
label("$y=\sqrt{0.5-t^2}$", (0.0,0.25), next_color(), UnFill());
label("$y=\sqrt{3-t^2}$", (0.0,1.56), next_color(), UnFill());
\end{asy}
\end{center}
\end{overprint}
\vspace{-45mm}
\end{example}
\end{frame}

\begin{frame}
\begin{block}{Note}
It is no coincidence that the two previous examples had multiple solutions. Most differential equations have an infinite number of solutions.
\end{block}\pause
\begin{example}
Consider
\begin{equation*}
\dfrac{dy}{dt} = f(t)
\end{equation*}\pause
We can integrate both sides to get the solution to get

\vspace{-2mm}
\begin{equation*}
y=\int f(t)dt + c
\end{equation*}

\vspace{-2mm}
where $c$ is an arbitrary constant.
\end{example}\pause
\begin{block}{Family of Solutions}
In general, all solutions of a first-order DE form a \textbf{family} of solutions expressed with a single parameter $c$. Such a family is called the \textbf{general solution}. A member of the family that results from a specific value of $c$ is called a \textbf{particular solution}.
\end{block}
\end{frame}

\begin{frame}[fragile]
\begin{example}
The general solution of $y^\prime=3t^2$ is
\begin{equation*}
y=t^3+c
\end{equation*}
where $c$ may be any real value.

\begin{center}
\begin{asy}
size(150, IgnoreAspect);

real min_x=-2, max_x=2;
real min_y=-6, max_y=6;

pair start=(min_x,min_y);
pair end=(max_x,max_y);

real[] k_values = {1.0, -3.4, -1.5, 2.2, 4.0};

for (real k : k_values)
{
	real y(real t) {return t*t*t+k;}
	draw(graph(y, min_x, max_x), next_color() + 1bp, "c = " + string(k));
}

limits(start,end,Crop);

xaxis("$t$",YEquals(0),Ticks(NoZero));
yaxis("$y$",XEquals(0),Ticks(NoZero),autorotate=false);

attach(legend(), truepoint(E), 30E, UnFill());

real x_1 = 2.009;
real x_2 = 2.901;
unfill((x_1,0.3) -- (x_2,0.3) -- (x_2,-0.3) -- (x_1, -0.3) -- cycle);
\end{asy}
\end{center}
\end{example}
\end{frame}

\begin{frame}
\begin{block}{Initial-Value Problem}
The combination of a first-order differential equation and an \textbf{initial condition}
\begin{equation*}
\dfrac{dy}{dt}=f(t,y),\quad y(t_0)=y_0
\end{equation*}
is called an \textbf{initial-value problem}. It's solution will pass through the point $(t_0,y_0)$.
\end{block}\pause
\begin{block}{Note}
While a family of solutions for a DE contains multiple solutions, an IVP usually has only one solution. That is, the solution to an IVP is a particular solution to the DE\@.
\end{block}
\end{frame}

\begin{frame}[fragile]
\begin{example}
The function $y(t)=t^3+1$ is a solution to the IVP
\begin{equation*}
y^\prime = 3t^2,\quad y(0)=1
\end{equation*}\pause
Differentiating $y(t)$ confirms that 
\begin{equation*}
y^\prime(t)={(t^3+1)}^\prime=3t^2,\quad\text{and}\quad y(0)=0^3+1=1
\end{equation*}
\begin{center}
\begin{asy}
size(131, IgnoreAspect);

real min_x=-2, max_x=2;
real min_y=-6, max_y=6;

pair start=(min_x,min_y);
pair end=(max_x,max_y);

real[] k_values = {-3.4, -1.5, 2.2, 4.0};

for (real k : k_values)
{
	real y(real t) {return t*t*t+k;}
	draw(graph(y, min_x, max_x), gray + 1bp);
}

real y(real t) {return t*t*t+1.0;}
draw(graph(y, min_x, max_x), next_color() + 1bp);

limits(start,end,Crop);

xaxis("$t$",YEquals(0),Ticks(NoZero));
yaxis("$y$",XEquals(0),Ticks(NoZero),autorotate=false);
\end{asy}
\end{center}
\end{example}
\end{frame}

\begin{frame}
\begin{example}
Let us look again at the Malthusian population problem
\begin{equation*}
\dfrac{dy}{dt}=0.03y,\quad y(0)=0.9
\end{equation*}
Where we have the one-parameter family of solutions
\begin{equation*}
y(t)=ce^{0.03t}
\end{equation*}\pause
Substituting the initial conditions into the general solution gives
\begin{equation*}
\begin{aligned}
y(0) &= 0.9 \\ \pause
ce^{0.03\cdot 0} &= 0.9 \\ \pause
c\cdot 1 &= 0.9 \\ \pause
c &= 0.9
\end{aligned}
\end{equation*}\pause

\vspace{-8mm}
So, the solution to the IVP is 
\begin{equation*}
y(t)=0.9e^{0.03t}
\end{equation*}
\end{example}
\end{frame}

\begin{frame}
\begin{block}{Vocabulary}
The phrase \quotetext{solving a differential equation} can refer to:
\begin{itemize}[<+- | alert@+>]
\item Obtaining an \textbf{explicit} formula for $y(t)$. (Most common usage.)
\item Obtaining an \textbf{implicit} equation relating $y$ and $t$.
\item Obtaining a \textbf{power series representation} for $y(t)$.
\item Obtaining an appropriate \textbf{numerical approximation} to $y(t)$.
\item Informally, refer to a study of a \textbf{geometrical representation}.
\end{itemize}
\end{block}
\end{frame}

\begin{frame}
\begin{block}{Quantitative Analysis}
Historically, the study of differential equations was \textbf{quantitative}, to find explicit formulas or power series representations of solutions. This type of analysis dominated the thinking of the seventeenth and eighteenth centuries, and the work of Isaac Newton, Gottfried Leibniz, Leonhard Euler, and Joseph Lagrange.\pause

\vspace{2mm}
It is fairly rare to find an explicit solution because the family of \textbf{elementary functions} is simply too limited to express the solutions of most differential equations we care about.
\end{block}\pause

\begin{block}{Qualitative Analysis}
In the late nineteenth century, the French mathematician Henri Poincar\'{e}, while working on problems in celestial mechanics, started investigating the behavior of solutions. His new approach, now called \textbf{qualitative} theory, focuses on the properties of the solutions, instead of the search for an explicit formula. In this way, we are able to demonstrate the existence of constant or periodic solutions, as well as describe the long term behavior.
\end{block}
\end{frame}

\begin{frame}
\begin{block}{Graphical Definition of Solutions}
A \textbf{solution} to a first-order differential equation is a function whose slope at each point is specified by the derivative.
\end{block}
\onslide<2->

\begin{block}{Direction Fields}
We can see what solution curves look like by, on regular intervals, drawing short line segments with slope determined by the DE\@ for that point. The collection of these segments are called \textbf{direction field} (or a \textbf{slope field}).
\end{block}
\end{frame}

\begin{frame}[fragile]
\begin{example}
\vspace{-2mm}
\begin{equation*}
y^\prime=-2ty+t
\end{equation*}
\begin{overprint}
\onslide<1>
\begin{center}
\begin{asy}
size(220);

real min_x=-2, max_x=4;
real min_y=-3, max_y=3;

pair start=(min_x,min_y);
pair end=(max_x,max_y);

for(real gx=min_x+1; gx<=max_x-1; ++gx)
	draw((gx,min_y)--(gx,max_y),dotted+darkgray);
    
for(real gy=min_y+1; gy<=max_y-1; ++gy)
	draw((min_x,gy)--(max_x,gy),dotted+darkgray); 
	
real yp (real t, real y) { return -2*t*y+t; }

add(slopefield(yp, start, end, 20));

limits(start,end,Crop);

xaxis("$t$",YEquals(min_y),min_x,max_x,LeftTicks());
xaxis(YEquals(max_y),min_x,max_x);
yaxis("$y$",XEquals(min_x),min_y,max_y,LeftTicks());
yaxis(XEquals(max_x),min_y,max_y);
\end{asy}
\end{center}
\onslide<2>
\begin{center}
\begin{asy}
size(220);

real min_x=-2, max_x=4;
real min_y=-3, max_y=3;

pair start=(min_x,min_y);
pair end=(max_x,max_y);

for(real gx=min_x+1; gx<=max_x-1; ++gx)
	draw((gx,min_y)--(gx,max_y),dotted+darkgray);
    
for(real gy=min_y+1; gy<=max_y-1; ++gy)
	draw((min_x,gy)--(max_x,gy),dotted+darkgray); 
	
real yp (real t, real y) { return -2*t*y+t; }

add(slopefield(yp, start, end, 20));

pair[] pts = {(0,-2), (2.5,1), (2.1,-2), (0,0.5), (0,1.75)};

for (pair p : pts)
{
	real c = (p.y - 0.5)/exp(-p.x*p.x);
	real f (real t) { return c*exp(-1*t*t) + 0.5; }
	
	pen clr = next_color();
	draw(graph(f, min_x, max_x), clr+1bp);
	dot(p, clr);
}

limits(start,end,Crop);

xaxis("$t$",YEquals(min_y),min_x,max_x,LeftTicks());
xaxis(YEquals(max_y),min_x,max_x);
yaxis("$y$",XEquals(min_x),min_y,max_y,LeftTicks());
yaxis(XEquals(max_x),min_y,max_y);
\end{asy}
\end{center}
\end{overprint}
\end{example}
\end{frame}

\begin{frame}[fragile]
\begin{example}
\vspace{-2mm}
\begin{equation*}
y^\prime=y+t^2
\end{equation*}
\begin{overprint}
\onslide<1>
\begin{center}
\begin{asy}
size(220);

real min_x=-4, max_x=4;
real min_y=-4, max_y=4;

pair start=(min_x,min_y);
pair end=(max_x,max_y);

for(real gx=min_x+1; gx<=max_x-1; ++gx)
	draw((gx,min_y)--(gx,max_y),dotted+darkgray);
    
for(real gy=min_y+1; gy<=max_y-1; ++gy)
	draw((min_x,gy)--(max_x,gy),dotted+darkgray); 
	
real yp (real t, real y) { return y+t*t; }

add(slopefield(yp, start, end, 20));

limits(start,end,Crop);

xaxis("$t$",YEquals(min_y),min_x,max_x,LeftTicks());
xaxis(YEquals(max_y),min_x,max_x);
yaxis("$y$",XEquals(min_x),min_y,max_y,LeftTicks());
yaxis(XEquals(max_x),min_y,max_y);
\end{asy}
\end{center}
\onslide<2>
\begin{center}
\begin{asy}
size(220);

real min_x=-4, max_x=4;
real min_y=-4, max_y=4;

pair start=(min_x,min_y);
pair end=(max_x,max_y);

for(real gx=min_x+1; gx<=max_x-1; ++gx)
	draw((gx,min_y)--(gx,max_y),dotted+darkgray);
    
for(real gy=min_y+1; gy<=max_y-1; ++gy)
	draw((min_x,gy)--(max_x,gy),dotted+darkgray); 
	
real yp (real t, real y) { return y+t*t; }

add(slopefield(yp, start, end, 20));

pair[] pts = {(0.0,-2.1), (-2,1), (1.9,1), (0,2), (2.5,-0.5)};

for (pair p : pts)
{
	real c = (p.y + p.x * p.x + 2 * p.x + 2)/exp(p.x);
	real f (real t) { return c*exp(t)-t*t-2*t-2; }
	
	pen clr = next_color();
	draw(graph(f, min_x, max_x), clr+1bp);
	dot(p, clr);
}

limits(start,end,Crop);

xaxis("$t$",YEquals(min_y),min_x,max_x,LeftTicks());
xaxis(YEquals(max_y),min_x,max_x);
yaxis("$y$",XEquals(min_x),min_y,max_y,LeftTicks());
yaxis(XEquals(max_x),min_y,max_y);
\end{asy}
\end{center}
\end{overprint}
\end{example}
\end{frame}

\begin{frame}[fragile]
\begin{example}
\vspace{-2mm}
\begin{equation*}
y^\prime=t-y
\only<3>{\quad\text{and}\quad y^{\prime\prime}=1-y^\prime=1-t+y}
\end{equation*}
\begin{overprint}
\onslide<1>
\begin{center}
\begin{asy}
size(220);

real min_x=-1, max_x=3;
real min_y=-1, max_y=3;

pair start=(min_x,min_y);
pair end=(max_x,max_y);

for(real gx=min_x+1; gx<=max_x-1; ++gx)
	draw((gx,min_y)--(gx,max_y),dotted+darkgray);
    
for(real gy=min_y+1; gy<=max_y-1; ++gy)
	draw((min_x,gy)--(max_x,gy),dotted+darkgray); 
	
real yp (real t, real y) { return t-y; }

add(slopefield(yp, start, end, 20));

limits(start,end,Crop);

xaxis("$t$",YEquals(min_y),min_x,max_x,LeftTicks());
xaxis(YEquals(max_y),min_x,max_x);
yaxis("$y$",XEquals(min_x),min_y,max_y,LeftTicks());
yaxis(XEquals(max_x),min_y,max_y);
\end{asy}
\end{center}
\onslide<2>
\begin{center}
\begin{asy}
size(220);

real min_x=-1, max_x=3;
real min_y=-1, max_y=3;

pair start=(min_x,min_y);
pair end=(max_x,max_y);

for(real gx=min_x+1; gx<=max_x-1; ++gx)
	draw((gx,min_y)--(gx,max_y),dotted+darkgray);
    
for(real gy=min_y+1; gy<=max_y-1; ++gy)
	draw((min_x,gy)--(max_x,gy),dotted+darkgray); 
	
real yp (real t, real y) { return t-y; }

add(slopefield(yp, start, end, 20));

pair[] pts = {(1,2), (2.1,1), (1.9,1), (0,2), (2.5,-0.5)};

for (pair p : pts)
{
	real c = exp(p.x)*(p.y-p.x+1);
	real f (real t) { return (exp(t)*t-exp(t)+c)/exp(t); }
	
	pen clr = next_color();
	draw(graph(f, min_x, max_x), clr+1bp);
	dot(p, clr);
}

limits(start,end,Crop);

xaxis("$t$",YEquals(min_y),min_x,max_x,LeftTicks());
xaxis(YEquals(max_y),min_x,max_x);
yaxis("$y$",XEquals(min_x),min_y,max_y,LeftTicks());
yaxis(XEquals(max_x),min_y,max_y);
\end{asy}
\end{center}
\onslide<3>
\begin{center}
\begin{asy}
size(220);

real min_x=-1, max_x=3;
real min_y=-1, max_y=3;

pair start=(min_x,min_y);
pair end=(max_x,max_y);

for(real gx=min_x+1; gx<=max_x-1; ++gx)
	draw((gx,min_y)--(gx,max_y),dotted+darkgray);
    
for(real gy=min_y+1; gy<=max_y-1; ++gy)
	draw((min_x,gy)--(max_x,gy),dotted+darkgray); 
	
real yp (real t, real y) { return t-y; }

add(slopefield(yp, start, end, 20));

pair[] pts = {(1,2), (2.1,1), (1.9,1), (0,2), (2.5,-0.5)};

for (pair p : pts)
{
	real c = exp(p.x)*(p.y-p.x+1);
	real f (real t) { return (exp(t)*t-exp(t)+c)/exp(t); }
	
	pen clr = next_color();
	draw(graph(f, min_x, max_x), clr+1bp);
	dot(p, clr);
}

real zed(real t) { return t-1; }
draw(graph(zed, min_x, max_x), gray+dashed+1bp);

label("Concave up", (0,0.5), black, UnFill());
label("$y^{\prime\prime}>0$", (0, 0.3), black, UnFill());
label("Concave down", (1.97, 0), black, UnFill());
label("$y^{\prime\prime}<0$", (1.97, -0.21), black, UnFill());

limits(start,end,Crop);

xaxis("$t$",YEquals(min_y),min_x,max_x,LeftTicks());
xaxis(YEquals(max_y),min_x,max_x);
yaxis("$y$",XEquals(min_x),min_y,max_y,LeftTicks());
yaxis(XEquals(max_x),min_y,max_y);
\end{asy}
\end{center}
\end{overprint}
\end{example}
\end{frame}

\begin{frame}[fragile]
\begin{block}{Equilibria}
For a differential equation, a solution that does not change over time is called an \textbf{equilibrium solution}.\pause

\vspace{2mm}
For a first-order DE $y^{\prime}=f(t,y)$, an equilibrium solution is always a horizontal line $y(t)=C$, which can be obtained by setting $y^{\prime}=0$.
\end{block}\pause

\begin{block}{Stability}
For a differential equation $y^\prime=f(t,y)$, an equilibrium solution $y(t)=C$ is called
\begin{itemize}[<+- | alert@+>]
\item \textbf{stable} if solutions \textquote{near} it tend toward it as $t\rightarrow\infty$.
\item \textbf{unstable} if solutions \textquote{near} it tend away from it as $t\rightarrow\infty$.
\end{itemize}
\end{block}

\onslide<+->
\begin{block}{Note}
A equilibrium solution is often called \textbf{semistable} if it is stable on one side and unstable on the other.
\end{block}
\end{frame}

\begin{frame}[fragile]
\begin{example}
The DE $y^\prime=-2ty+t$ has the constant solution $y(t)=\dfrac{1}{2}$.
\begin{center}
\begin{asy}
size(205);

real min_x=-2, max_x=4;
real min_y=-3, max_y=3;

pair start=(min_x,min_y);
pair end=(max_x,max_y);

for(real gx=min_x+1; gx<=max_x-1; ++gx)
	draw((gx,min_y)--(gx,max_y),dotted+darkgray);
    
for(real gy=min_y+1; gy<=max_y-1; ++gy)
	draw((min_x,gy)--(max_x,gy),dotted+darkgray); 
	
real yp (real t, real y) { return -2*t*y+t; }

add(slopefield(yp, start, end, 20));

pair[] pts = {(0,-2), (2.5,1), (2.1,-2), (0,1.75)};

for (pair p : pts)
{
	real c = (p.y - 0.5)/exp(-p.x*p.x);
	real f (real t) { return c*exp(-1*t*t) + 0.5; }
	
	pen clr = grey;
	draw(graph(f, min_x, max_x), clr+1bp);
	//dot(p, clr);
}

real f (real t) { return 0.5; }
draw(graph(f, min_x, max_x),heavymagenta+1bp); 
//dot((0, 0.5), heavymagenta);

limits(start,end,Crop);

xaxis("$t$",YEquals(min_y),min_x,max_x,LeftTicks());
xaxis(YEquals(max_y),min_x,max_x);
yaxis("$y$",XEquals(min_x),min_y,max_y,LeftTicks());
yaxis(XEquals(max_x),min_y,max_y);
\end{asy}
\end{center}
\end{example}
\end{frame}

\begin{frame}[fragile]
\begin{example}
\begin{equation*}
y^\prime=y^2-4
\end{equation*}
\begin{overprint}
\onslide<1>
\begin{center}
\begin{asy}
size(215);

real min_x=-4, max_x=4;
real min_y=-4, max_y=4;

pair start=(min_x,min_y);
pair end=(max_x,max_y);

for(real gx=min_x+1; gx<=max_x-1; ++gx)
	draw((gx,min_y)--(gx,max_y),dotted+darkgray);
    
for(real gy=min_y+1; gy<=max_y-1; ++gy)
	draw((min_x,gy)--(max_x,gy),dotted+darkgray); 
	
real yp (real t, real y) { return y*y-4; }

add(slopefield(yp, start, end, 20));

limits(start,end,Crop);

xaxis("$t$",YEquals(min_y),min_x,max_x,LeftTicks());
xaxis(YEquals(max_y),min_x,max_x);
yaxis("$y$",XEquals(min_x),min_y,max_y,LeftTicks());
yaxis(XEquals(max_x),min_y,max_y);
\end{asy}
\end{center}
\onslide<2>
\begin{center}
\begin{asy}
size(215);

real min_x=-4, max_x=4;
real min_y=-4, max_y=4;

pair start=(min_x,min_y);
pair end=(max_x,max_y);

for(real gx=min_x+1; gx<=max_x-1; ++gx)
	draw((gx,min_y)--(gx,max_y),dotted+darkgray);
    
for(real gy=min_y+1; gy<=max_y-1; ++gy)
	draw((min_x,gy)--(max_x,gy),dotted+darkgray); 
	
real yp (real t, real y) { return y*y-4; }

add(slopefield(yp, start, end, 20));

// draw middle solutions
pen eqclr = next_color();
pen clr = next_color();

for (real x=min_x-1; x <= max_x+1; x += 1)
{
	draw(curve((x,0), yp, start, end), clr+1bp);
}

// draw top solutions
pen clr = next_color();

for (real x=min_x-1; x <= max_x+1; x += 1)
{
	draw(curve((x,3.5), yp, start, end), clr+1bp);
}

// draw bottom solutions
pen clr = next_color();

for (real x=min_x-1; x <= max_x+1; x += 1)
{
	draw(curve((x,-3.5), yp, start, end), clr+1bp);
}

// draw equilibiria
pen clr = blue;

pair[] pts = {(0,-2), (0,2)};

for (pair p : pts)
{
	draw(curve(p, yp, start, end), eqclr+1.5bp);
}

limits(start,end,Crop);

xaxis("$t$",YEquals(min_y),min_x,max_x,LeftTicks());
xaxis(YEquals(max_y),min_x,max_x);
yaxis("$y$",XEquals(min_x),min_y,max_y,LeftTicks());
yaxis(XEquals(max_x),min_y,max_y);
\end{asy}
\end{center}
\end{overprint}
\end{example}
\end{frame}

\begin{frame}
\begin{block}{Isoclines}
An \textbf{isocline} of a differential equation $y^\prime=f(t,y)$ is a curve in the $ty$-plane along which the slope is constant.\pause

\vspace{2mm}
In other words, it is the set of all points $(t,y)$ where the slope has the value $m$, and is therefore the graph of $f(t,y)=m$.
\end{block}\pause

\begin{block}{Method of Isoclines}
Drawing multiple isoclines forms a handy guide to the slopes of solutions. Though, they rarely coincide with solutions. 

\vspace{2mm}
Draw the isoclines with dashed lines, so you don't confuse the two.
\end{block}
\end{frame}

\begin{frame}[fragile]
\begin{example}
The differential equation

\vspace{-4mm}
\begin{equation*}
y^\prime = y^2-4
\end{equation*}
has isoclines

\vspace{-4mm}
\begin{equation*}
y^2-4=m
\quad\Rightarrow\quad
y=\pm\sqrt{m+4}
\end{equation*}
\begin{overprint}
\onslide<1>
\begin{center}
\begin{asy}
size(165);

real min_x=-4, max_x=4;
real min_y=-4, max_y=4;

pair start=(min_x,min_y);
pair end=(max_x,max_y);

for(real gx=min_x+1; gx<=max_x-1; ++gx)
	draw((gx,min_y)--(gx,max_y),dotted+darkgray);
    
for(real gy=min_y+1; gy<=max_y-1; ++gy)
	draw((min_x,gy)--(max_x,gy),dotted+darkgray); 
	
real yp (real t, real y) { return y*y-4; }

//add(slopefield(yp, start, end, 20));

real[] vals = {0, 5, -3, -4};

for (real m : vals)
{
    real fp (real t) { return sqrt(m+4); }
    real fn (real t) { return -1*sqrt(m+4); }
    
    draw(graph(fp,min_x, max_x), grey+0.75bp);
    draw(graph(fn,min_x, max_x), grey+0.75bp);
    
    draw(isocline(fp, yp, min_x, max_x), blue+1bp);
    draw(isocline(fn, yp, min_x, max_x), blue+1bp);
    
    label("$m="+string(m)+"$", (2,sqrt(m+4)+.2), Align, black, UnFill());
    label("$m="+string(m)+"$", (2,-sqrt(m+4)+.2), Align, black, UnFill());
}

limits(start,end,Crop);

xaxis("$t$",YEquals(min_y),min_x,max_x,LeftTicks());
xaxis(YEquals(max_y),min_x,max_x);
yaxis("$y$",XEquals(min_x),min_y,max_y,LeftTicks());
yaxis(XEquals(max_x),min_y,max_y);
\end{asy}
\end{center}
\onslide<2>
\begin{center}
\begin{asy}
size(165);

real min_x=-4, max_x=4;
real min_y=-4, max_y=4;

pair start=(min_x,min_y);
pair end=(max_x,max_y);

for(real gx=min_x+1; gx<=max_x-1; ++gx)
	draw((gx,min_y)--(gx,max_y),dotted+darkgray);
    
for(real gy=min_y+1; gy<=max_y-1; ++gy)
	draw((min_x,gy)--(max_x,gy),dotted+darkgray); 
	
real yp (real t, real y) { return y*y-4; }

//add(slopefield(yp, start, end, 20));

// draw middle solutions
pen eqclr = next_color();
pen clr = next_color();

for (real x=min_x-1; x <= max_x+1; x += 1)
{
	draw(curve((x,0), yp, start, end), clr+1bp);
}

// draw top solutions
pen clr = next_color();

for (real x=min_x-1; x <= max_x+1; x += 1)
{
	draw(curve((x,3.5), yp, start, end), clr+1bp);
}

// draw bottom solutions
pen clr = next_color();

for (real x=min_x-1; x <= max_x+1; x += 1)
{
	draw(curve((x,-3.5), yp, start, end), clr+1bp);
}

// draw equilibiria
pen clr = blue;

pair[] pts = {(0,-2), (0,2)};

for (pair p : pts)
{
	draw(curve(p, yp, start, end), eqclr+1.5bp);
}

real[] vals = {0, 5, -3, -4};

for (real m : vals)
{
    real fp (real t) { return sqrt(m+4); }
    real fn (real t) { return -1*sqrt(m+4); }
    
    draw(graph(fp,min_x, max_x), grey+0.75bp);
    draw(graph(fn,min_x, max_x), grey+0.75bp);
    
    draw(isocline(fp, yp, min_x, max_x), blue+1bp);
    draw(isocline(fn, yp, min_x, max_x), blue+1bp);
    
    label("$m="+string(m)+"$", (2,sqrt(m+4)+.2), Align, black, UnFill());
    label("$m="+string(m)+"$", (2,-sqrt(m+4)+.2), Align, black, UnFill());
}

limits(start,end,Crop);

xaxis("$t$",YEquals(min_y),min_x,max_x,LeftTicks());
xaxis(YEquals(max_y),min_x,max_x);
yaxis("$y$",XEquals(min_x),min_y,max_y,LeftTicks());
yaxis(XEquals(max_x),min_y,max_y);
\end{asy}
\end{center}
\end{overprint}
\end{example}
\end{frame}

\begin{frame}[fragile]
\begin{example}
The differential equation

\vspace{-4mm}
\begin{equation*}
y^\prime=y^2-t
\end{equation*}
has isoclines

\vspace{-4mm}
\begin{equation*}
y^2-t=m
\quad\rightarrow\quad
t=y^2-m
\end{equation*}
\begin{overprint}
\onslide<1>
\begin{center}
\begin{asy}
size(1.25*160, 160, IgnoreAspect);

real min_x=-3, max_x=3;
real min_y=-3, max_y=3;

pair start=(min_x,min_y);
pair end=(max_x,max_y);

for(real gx=min_x+1; gx<=max_x-1; ++gx)
	draw((gx,min_y)--(gx,max_y),dotted+darkgray);
    
for(real gy=min_y+1; gy<=max_y-1; ++gy)
	draw((min_x,gy)--(max_x,gy),dotted+darkgray); 
	
real yp (real t, real y) { return y*y-t; }

//add(slopefield(yp, start, end, 20));

real[] vals = {3,2,1,0,-1,-2};
int i = 0;
for (real m : vals)
{
    real fp (real t) { return sqrt(m+t); }
    real fn (real t) { return -1*sqrt(m+t); }
    
    draw(graph(fp,-1*m, max_x), grey+0.75bp);
    draw(graph(fn,-1*m, max_x), grey+0.75bp);
    
    draw(isocline(fp, yp, -1*m, max_x, n=10-i, width=0.2), blue+1bp);
    draw(isocline(fn, yp, -1*m, max_x, n=10-i, width=0.2), blue+1bp);
    
    i = i + 1;
    
    //label("$m="+string(m)+"$", (2,sqrt(m+4)+.2), Align, black, UnFill());
    //label("$m="+string(m)+"$", (2,-sqrt(m+4)+.2), Align, black, UnFill());
}

limits(start,end,Crop);

xaxis("$t$",YEquals(min_y),min_x,max_x,LeftTicks());
xaxis(YEquals(max_y),min_x,max_x);
yaxis("$y$",XEquals(min_x),min_y,max_y,LeftTicks());
yaxis(XEquals(max_x),min_y,max_y);

int i = -1;
for (real m : vals)
{
	if (i < 0)
	    label("$m="+string(m)+"$", (3.1,sqrt(m+3)-.1), Align, black, UnFill());
	else
	    label("$m="+string(m)+"$", (3.1,-sqrt(m+3)-.1), Align, black, UnFill());
	    
	i = -1*i;
}
\end{asy}
\end{center}
\onslide<2>
\begin{center}
\begin{asy}
size(1.25*160, 160, IgnoreAspect);

real min_x=-3, max_x=3;
real min_y=-3, max_y=3;

pair start=(min_x,min_y);
pair end=(max_x,max_y);

for(real gx=min_x+1; gx<=max_x-1; ++gx)
	draw((gx,min_y)--(gx,max_y),dotted+darkgray);
    
for(real gy=min_y+1; gy<=max_y-1; ++gy)
	draw((min_x,gy)--(max_x,gy),dotted+darkgray); 
	
real yp (real t, real y) { return y*y-t; }

//add(slopefield(yp, start, end, 20));

//pair[] pts = {(-2,0), (-1,0), (1,0), (2,0), (0,-2), (0,-0.5), (-1.5, 0)};

//for (pair p : pts)
real inc=1.1;
for(real x=min_x; x<max_x; x+=inc)
	for(real y=min_y; y<max_y; y+=inc) 
		draw(curve((x,y), yp, start, end), red+0.5bp);

real[] vals = {3,2,1,0,-1,-2};
int i = 0;
for (real m : vals)
{
    real fp (real t) { return sqrt(m+t); }
    real fn (real t) { return -1*sqrt(m+t); }
    
    draw(graph(fp,-1*m, max_x), grey+0.75bp);
    draw(graph(fn,-1*m, max_x), grey+0.75bp);
    
    draw(isocline(fp, yp, -1*m, max_x, n=10-i, width=0.2), blue+1bp);
    draw(isocline(fn, yp, -1*m, max_x, n=10-i, width=0.2), blue+1bp);
    
    i = i + 1;
    
    //label("$m="+string(m)+"$", (2,sqrt(m+4)+.2), Align, black, UnFill());
    //label("$m="+string(m)+"$", (2,-sqrt(m+4)+.2), Align, black, UnFill());
}

limits(start,end,Crop);

xaxis("$t$",YEquals(min_y),min_x,max_x,LeftTicks());
xaxis(YEquals(max_y),min_x,max_x);
yaxis("$y$",XEquals(min_x),min_y,max_y,LeftTicks());
yaxis(XEquals(max_x),min_y,max_y);

int i = -1;
for (real m : vals)
{
	if (i < 0)
	    label("$m="+string(m)+"$", (3.1,sqrt(m+3)-.1), Align, black, UnFill());
	else
	    label("$m="+string(m)+"$", (3.1,-sqrt(m+3)-.1), Align, black, UnFill());
	    
	i = -1*i;
}
\end{asy}
\end{center}
\end{overprint}
\end{example}
\end{frame}

\begin{frame}
\begin{block}{Direction Field Checklist}
We can usually answer the following questions for DE $y^\prime=f(t,y)$.
\begin{enumerate}[<+- | alert@+>]
\item Is the field well defined? That is, are there any points $(t,y)$ such that $f(t,y)$ does not exist?
\item Does there appear to be a unique solution curve passing through each point of the plane?
\item Are there equilibrium solutions? Are they stable, unstable, or semistable?
\item What is the concavity of solutions?
\item Do any solutions appear to \quotetext{blow up}? That is, do there appear to be any vertical asymptotes?
\item What is the pattern of the isoclines?
\item Are there any periodic solutions?
\item What is the long term behavior or solutions as $t\rightarrow\infty$ and $t\rightarrow-\infty$?
\item Does the field have any symmetries?
\end{enumerate}
\end{block}
\end{frame}
\end{document}
