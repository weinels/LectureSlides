\documentclass{beamer}
\usepackage[utf8]{inputenc}
\usepackage[english]{babel}
\usepackage[T1]{fontenc}
\usepackage[inline]{asymptote}
\usepackage{slide_helper}

\title[MATH 2250 - Section 1.1]{Dynamical Systems: Modeling}

\begin{document}
\begin{frame}
\titlepage
\end{frame}

\begin{frame}
\begin{block}{}
Models, a hallmark of the scientific method, are the way we understand the world around us. A model is not intended to be the \quotetext{real thing}, but instead a representation that selects features or aspects of the real thing.
\end{block}\pause

\begin{block}{Types of Models}
The most common type of model is a \textbf{continuous-time} system, which are modeled by \textbf{differential equations}.\pause

\vspace{2mm}
It can often be useful to think of changes to a system as happening in separate jumps, such as daily, weekly, etc\ldots Such systems are called \textbf{discrete-time} or \textbf{sampled-data} systems.
\end{block}\pause

\begin{block}{}
We use \textbf{scalar} models when a system is described by a single measurement and \textbf{vector} models for systems with several varying components. \pause

\vspace{2mm}
The study of multidimensional systems will be aided by the study of \textbf{linear algebra} in chapters 3 and 5.
\end{block}
\end{frame}

\begin{frame}
\begin{block}{}
In this class, we will study mathematical models applied to \textbf{dynamical systems}, which are systems that change over time.\pause

\vspace{1mm}
Dynamical systems are used to model many physical systems, such as earthquakes, turbulence around a wing, electrical circuits, and so many more.
\end{block}\pause

\begin{block}{}
The phenomena we study are found in different \textbf{states}, characterized by a set of measurements and which evolve with the passage of time.
\end{block}\pause

\begin{example}
A cup of coffee sitting on a desk seems like a simple physical system. To understand completely the coffee's interactions with the air, the cup, the table, or your digestive, circulatory, and nervous system would involve all fields of science.\pause

\vspace{1mm}
But, if all we care about is the temperature of the coffee, we can use a limited model called Newton's Law of Cooling, which incorporates the surrounding temperature to give an accurate description of the coffee's temperature.
\end{example}
\end{frame}

\begin{frame}
\begin{block}{Differential Equations}
A \textbf{differential equation (DE)} is an equation that contains \emph{derivatives} of one or more dependent variables with respect to time.
\onslide<+->
\begin{itemize}[<+- | alert@+>]
\item An \textbf{ordinary differential equation (ODE)} contains only ordinary derivatives.
\item A \textbf{partial differential equation (PDE)} contains partial derivaties. 
\end{itemize}
\onslide<+->
The \textbf{order} of a differential equation refers to the highest-order derivatives that appears in the equation. 
\end{block}
\onslide<+->
\begin{block}{Note}
We will only be studying ODE's in this class.
\end{block}
\end{frame}

\begin{frame}
\begin{example}
\begin{itemize}[<+- | alert@+>]
\item $\dfrac{dy}{dt}=f(t,y)$ is a first-order ODE with independent variable $t$ and dependent variable $y$.
\item $\dfrac{d^2 y}{d t^2}=f(t,y,y^\prime)$ is a second-order ODE with independent variable $t$ and dependent variable $y$.
\item $2\dfrac{d^2 y}{d t^2}+y\dfrac{dy}{dt}+ty^2=0$ is a second-order ODE with independent variable $t$ and dependent variable $y$.
\item $\dfrac{d^5 y}{d t^5}-\dfrac{dy}{dt}=4yt$ is a fifth-order ODE with independent variable $t$ and dependent variable $y$.
\item $\dfrac{\partial^2 y}{\partial x^2}+\dfrac{\partial^2 z}{\partial t^2}=xyz$ is a second-order PDE with independent variables $x$ and $t$ and dependent variables $y$ and $z$.
\end{itemize}
\end{example}
\end{frame}

\begin{frame}
\begin{block}{Constants of Proportionality}
Let $y$ be an unknown differentiable function of time. We can express each of the following statements as an equation, using $k$ as a constant of proportionality.
\onslide<+->
\begin{itemize}[<+- | alert@+>]
\item The rate of change of $y$ is \textbf{(directly) proportional} to $y$:

\vspace{-3mm}
\begin{equation*}
\dfrac{dy}{dt}=ky
\end{equation*}

\vspace{-2mm}
\item The rate of change of $y$ is \textbf{proportional} to the product of $y^2$ and $t$:

\vspace{-3mm}
\begin{equation*}
\dfrac{dy}{dt}=ky^2 t
\end{equation*}

\vspace{-2mm}
\item The rate of change of $y$ is \textbf{inversely proportional} to $y$:

\vspace{-3mm}
\begin{equation*}
\dfrac{dy}{dt}=\dfrac{k}{y}
\end{equation*}

\vspace{-2mm}
\item The rate of change of $y$ is \textbf{directly proportional} to $y^2$ and \textbf{inversely proportional} to $\sqrt{t}$:

\vspace{-3mm}
\begin{equation*}
\dfrac{dy}{dt}=k\dfrac{y^2}{\sqrt{t}}
\end{equation*}

\vspace{-2mm}
\end{itemize}
\end{block}
\end{frame}

\begin{frame}
\begin{example}
\textbf{Exponential Growth} The population $P$ is growing at a rate proportional to the population at any time $t$:
\begin{equation*}
\dfrac{dP}{dt}=kP,\quad k>0
\end{equation*}
\end{example}\pause
\begin{example}
\textbf{Exponential Decay} Let $A$ be the amount of radioactive material in a sample at any time $t$. The amount $A$ is decreasing at a rate proportional to the amount at any time $t$:
\begin{equation*}
\dfrac{dA}{dt}=kA,\quad k<0
\end{equation*}
\end{example}
\end{frame}

\begin{frame}
\begin{example}
\textbf{Newton's Law of Cooling or Heating} The rate of change of temperature $T$ of an object is proportional to the difference between the temperature $M$ of the surroundings and the temperature of the object:
\begin{equation*}
\dfrac{dT}{dt}=k(M-T),\quad k>0
\end{equation*}
\end{example}\pause

\begin{example}
\textbf{Logistic Growth} The rate a which a disease is spread (i.e\@. the rate of increase of the number $N$ of people infected) in a fixed population $L$ is proportional to the product of the number of people infected and the number of people not yet infected:
\begin{equation*}
\dfrac{dN}{dt}=kN(L-N),\quad k>0
\end{equation*}
\end{example}
\end{frame}

\begin{frame}
\begin{example}
\textbf{Voltage Across an Inductor} The voltage drop $V$ is proportional to the rate of current $I$ in the inductor:
\begin{equation*}
V=L\dfrac{dI}{dt}
\end{equation*}
(The proportionality constant is this instance is written as $L$ (instead of $k$) and is called the \textbf{inductance}.)
\end{example}
\end{frame}

\begin{frame}
\begin{block}{The Malthus Model for Population Growth}
In 1798 an English clergyman Thomas Malthus argued that the worlds population was growing geometrically, while the world's food supply was growing arithmetically. From this, he concluded that the end result would be mass starvation.\pause

\vspace{1mm}
Using these assumptions, he constructed one of the first mathematical models for population growth. (Sparking class, social, and religious controversy along the way.)\pause

\vspace{1mm}
Malthus assumed that the rate of increase of the worlds population, $y(t)$, was proportional to it's size. We can state this as a DE\@:

\vspace{-2mm}
\begin{equation*}
\dfrac{dy}{dt}=ky
\end{equation*}

\vspace{-1mm}
where the positive number $k$ is called the \textbf{growth} or \textbf{rate} constant.\pause

\vspace{1mm}
In 1798, the population was about 0.9 million people. Malthus assumed the growth rate was a 3\% annual increase. Giving the DE\@:

\vspace{-2mm}
\begin{equation*}
\dfrac{dy}{dt}=0.03y,\quad y(0)=0.9
\end{equation*}
\end{block}
\end{frame}

\begin{frame}
\begin{block}{Accuracy of the Malthus Model}
\begin{center}
\begin{tabular}{cccc|cccc}
\textbf{Year} & \textbf{t} & \textbf{Malthus} & \textbf{Actual} &
\textbf{Year} & \textbf{t} & \textbf{Malthus} & \textbf{Actual} \\\hline
1800 & 0   & \phantom{1}0.90 & 0.9 & 1910 & 110 & \phantom{1}24.42 & 1.8 \\  
1810 & 10  & \phantom{1}1.21 & 0.9 & 1920 & 120 & \phantom{1}32.98 & 1.9 \\
1820 & 20  & \phantom{1}1.64 & 1.0 & 1930 & 130 & \phantom{1}44.52 & 2.1 \\
1830 & 30  & \phantom{1}2.21 & 1.0 & 1940 & 140 & \phantom{1}60.10 & 2.3 \\
1840 & 40  & \phantom{1}2.99 & 1.1 & 1950 & 150 & \phantom{1}81.13 & 2.7 \\
1850 & 50  & \phantom{1}4.03 & 1.2 & 1960 & 160 &           109.53 & 3.0 \\
1860 & 60  & \phantom{1}5.45 & 1.3 & 1970 & 170 &           147.87 & 3.5 \\
1870 & 70  & \phantom{1}7.35 & 1.4 & 1980 & 180 &           199.62 & 4.2 \\
1880 & 80  & \phantom{1}9.93 & 1.5 & 1990 & 190 &           269.49 & 5.1 \\
1890 & 90  &           13.40 & 1.6 & 2000 & 200 &           363.81 & 6.0 \\
1900 & 100 &           18.09 & 1.7 &      &     &                  &
\end{tabular}
\end{center}
\end{block}
\end{frame}

\begin{frame}
\begin{example}
\textbf{Hooke's Law} The restoring force on a spring is proportional to the displacement $x$ but opposite in direction:
\begin{equation*}
F_{res}=-kx,\quad k>0
\end{equation*}
If friction is negligible, we can assume Newton's First Law of Motion:
\begin{equation*}
m\dfrac{d^2 x}{dt^2} = -kx
\end{equation*}
\end{example}\pause

\begin{example}
\textbf{Hooke's Law as a System} If we substitute $dx/dt=y$ into the previous example, we can convert it to an equivalent system of first-order equations:
\begin{equation*}
\begin{aligned}
\dfrac{dx}{dt} & =y \\
\dfrac{dy}{dt} & =-\dfrac{k}{m}x
\end{aligned}
\end{equation*}
\end{example}
\end{frame}
\end{document}
