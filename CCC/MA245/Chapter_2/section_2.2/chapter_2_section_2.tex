\documentclass{beamer}
\usepackage[utf8]{inputenc}
\usepackage[english]{babel}
\usepackage[T1]{fontenc}
\usepackage[inline]{asymptote}
\usepackage{slide_helper}

\title[MA245 - Section 2.2]{Solving the First-Order Linear Differential Equation}

\begin{document}
\begin{frame}
\titlepage
\end{frame}

\begin{frame}
\begin{block}{Euler-Lagrange Two-Stage Method}
We saw last section that the general solution for the linear first-order differential equation
\begin{equation*}
y^\prime+p(t) y = f(t)
\end{equation*}
has the form $y=y_h+y_p$.\pause

\vspace{2mm}
We solved the corresponding homogenous equation using separation of variables, getting a one-parameter family
\begin{equation*}
y_h=c e^{-\int p(t) dt}
\end{equation*}
where $c\in\R$.\pause

\vspace{2mm}
The second step is to find a particular solution, which we will accomplish using \textbf{variation of parameters}, which was developed by French mathematician Joseph Louis Lagrange.
\end{block}
\end{frame}

\begin{frame}
\begin{block}{Variation of Parameters}
The idea of variation of parameters is to start with
\begin{equation*}
y_h(t)=c e^{-\int p(t) dt}
\end{equation*}
\onslide<2->
and change the constant $c$ to a function $v(t)$ and try a solution of the form
\begin{equation*}
y_p(t)=v(t) e^{-\int p(t) dt}
\end{equation*}
where the unknown function $v(t)$ is called the \textbf{varying parameter}.

\onslide<3->
\vspace{2mm}
Our goal is to find $v(t)$, to do so we need to substitute $y_p$ into the DE\@.

\vspace{-4mm}
\begin{overprint}
\onslide<3>
\begin{equation*}
\underbrace{\left(v^\prime(t) e^{-\int p(t)dt}-p(t)v(t) e^{-\int p(t)dt}\right)}_{y^\prime_p}
+
\underbrace{p(t)v(t)e^{-\int p(t)dt}}_{p(t)y_p} = f(t)
\end{equation*}
\onslide<4>
\vspace{4mm}
\begin{equation*}
v^\prime(t) e^{-\int p(t)dt} = f(t)
\end{equation*}
\onslide<5>
\vspace{4mm}
\begin{equation*}
v^\prime(t) = f(t) e^{\int p(t)dt}
\end{equation*}
\onslide<6->
\vspace{3mm}
\begin{equation*}
v(t) = \int f(t) e^{\int p(t)dt} dt
\end{equation*}
\end{overprint}
\onslide<7->
\vspace{-4mm}
Now that we have $v(t)$, we have determined a particular solution.
\begin{equation*}
y_p(t) = v(t) e^{-\int p(t) dt} = e^{-\int p(t) dt} \int f(t) e^{\int p(t)dt} dt 
\end{equation*}
\end{block}
\end{frame}

\begin{frame}
\begin{example}
Consider the IVP
\begin{equation*}
y^\prime + \left(\dfrac{1}{t+1}\right)y = 2,\quad y(0)=0,\quad t\geq 0
\end{equation*}
\begin{overprint}
\onslide<2-8>
\textbf{Step 1.} We start by solving the associated homogeneous equation.
\begin{equation*}
y^\prime + \left(\dfrac{1}{t+1}\right)y = 0
\end{equation*}
\visible<3->{Let us assume for the moment that $y\neq0$ and use separation of variables.
\begin{center}
\begin{tabular}{rcl}
\only<3>{$\dfrac{dy}{y}$ &$=$& $-\dfrac{dt}{t+1}$ \\}
\only<4>{$\ln \abs{y}$ &$=$& $-\ln[t+1] + c$ \\}
\only<5-6>{$\abs{y}$ & $=$ &}\only<5>{$e^{{\ln[t+1]}^{-1} + c}$}\only<6>{$e^c {\left(t+1\right)}^{-1}$}\only<5-6>{\\}
\only<7->{$y_h$ &$=$&} \only<7>{$\pm\dfrac{e^c}{t+1}$}\only<8->{$\dfrac{k}{t+1}$}
\end{tabular}
\end{center}}
\visible<8->{where $k=\pm e^c$}
\onslide<9-14>
\textbf{Step 2.} Next, using variation of parameters, we try
\begin{equation*}
y_p = \dfrac{v(t)}{t+1}
\end{equation*}
\visible<10->{which gives:
\begin{equation*}
\vphantom{\dfrac{v^\prime(t)}{t+1} = 2}
\only<10>{v^\prime(t) e^{-\int p(t) dt} = f(t)}
\only<11>{\dfrac{v^\prime(t)}{t+1} = 2}
\only<12>{v^\prime(t) = 2t+2}
\only<13->{v(t) = t^2+2t+c}
\end{equation*}}

\vspace{-5mm}
\visible<14->{But, we only need a single $v(t)$, so we can let $c=0$, giving
\begin{equation*}
y_p=\dfrac{t^2+2t}{t+1}
\end{equation*}}
\onslide<15->
\textbf{Step 3.} Thus, the general solution is:
\begin{equation*}
y(t) = y_h + y_p = \dfrac{k}{t+1}+\dfrac{t^2+2t}{t+1}
\end{equation*}
\visible<16->{\textbf{Step 4.} Substituting the initial condition into the general solution gives:
\begin{equation*}
0 = y(0) = \dfrac{k}{(0)+1}+\dfrac{{(0)}^2+2(0)}{(0)+1}
\visible<17->{\Rightarrow k=0}
\end{equation*}}
\visible<18->{Which means that the solution to the IVP is
\begin{equation*}
y(t)=\dfrac{t^2+2t}{t+1}
\end{equation*}}
\end{overprint}
\vspace{-5mm}
\end{example}
\end{frame}

\begin{frame}
\begin{block}{Euler Lagrange Method for Solving Linear First-Order DEs}
To solve a linear differential equation
\begin{equation*}
y^\prime + p(t) y = f(t)
\end{equation*}
where $p$ and $f$ are continuous on a domain $I$, use the following steps.
\begin{description}
\item[Step 1.] Solve the corresponding homogenous equation $y^\prime+p(t)y=0$ to obtain the one-parameter family.

\vspace{-3mm}
\begin{equation*}
y_h=c e^{-\int p(t) dt}
\end{equation*}\vspace{-8mm}
\item[Step 2.] Solve
\begin{equation*}
v^\prime(t) e^{-\int p(t)dt} = f(t)
\end{equation*}
for $v(t)$ to obtain a particular solution $y_p = v(t) e^{-\int p(t)dt}$.
\item[Step 3.] Combine the results of Step 1 and Step 2 to form the general solution
\begin{equation*}
y(t)=y_h + y_p
\end{equation*}
\item[Step 4.] If you are solving an IVP, only after Step 3 can you plug in the initial condition.
\end{description}
\end{block}
\end{frame}

\begin{frame}
\begin{block}{Note}
Variation of Parameters is a very powerful method, and we will see it again in our study of higher order differential equations later in the course. But, for first-order (and \emph{only} first-order) equations we have a second method, called the \textbf{Integrating Factor Method} which may also be used. \pause

\vspace{2mm}
For the differential equation
\begin{equation*}
y^\prime + p(t) y = f(t)
\end{equation*}
we will break this new method down into two cases:
\begin{itemize}
\item $p(t)$ is constant.
\item $p(t)$ is variable.
\end{itemize}
\end{block}
\end{frame}

\begin{frame}
\begin{block}{Integrating Factor Method (Constant Coefficient)}
Let us look at the first-order linear differential equation
\begin{equation*}
y^\prime+ ay = f(t),\quad a\in\R
\end{equation*}
\onslide<2->
This method uses a simple observation made by Euler:
\begin{equation*}
e^{at}\left(y^\prime+ay\right) = \dfrac{d}{dt}\left(e^{at} y\right)
\end{equation*}
\begin{overprint}
\onslide<3>
Let us start with the differential equation.
\begin{equation*}
y^\prime+ ay = f(t)
\end{equation*}
\onslide<4>
We first multiply both sides of the equation by $e^{at}$.
\begin{equation*}
e^{at}\left(y^\prime+ay\right) = e^{at} f(t) 
\end{equation*}
\onslide<5>
We then apply Euler's observation to the left-hand side.
\begin{equation*}
\dfrac{d}{dt}\left(e^{at} y\right) = e^{at} f(t) 
\end{equation*}
\onslide<6>
Next we integrate both sides.
\begin{equation*}
e^{at} y = \int e^{at} f(t)  dt + c
\end{equation*}
\onslide<7->
Solving for $y$ gives:
\begin{equation*}
y(t) = e^{-at} \int e^{at} f(t)  dt + c e^{-at}
\end{equation*}
\end{overprint}
\end{block}
\onslide<8->
\begin{block}{Note}
This is the same answer we got from Variation of Parameters, though achieved through a different route. We have obtained both $y_h$ and $y_p$ at the same time.
\end{block}
\end{frame}

\begin{frame}
\begin{block}{Integrating Factor Method (Variable Coefficient)}
Now let us look at the more general first-order differential equation
\begin{equation*}
y^\prime + p(t) y = f(t)
\end{equation*}
\begin{overprint}
\onslide<2>
We seek a function $\mu(t)$ that satisfies Euler's observation, i.e.
\begin{equation*}
\mu(t)\cdot\left(y^\prime + p(t) y\right) = \dfrac{d}{dt}\left(\mu(t)\cdot y\right)
\end{equation*}
\onslide<3>
Let us carry out the differentiation on the right-hand side
\begin{equation*}
\mu(t) y^\prime + p(t)\mu(t) y = \mu^\prime (t) y + \mu(t) y^\prime
\end{equation*}
\onslide<4>
If we assume $y(t)\neq 0$, this simplifies to
\begin{equation*}
\mu^\prime (t) = p(t)\mu(t)
\end{equation*}
\onslide<5>
We can find a solution $\mu(t)>0$ by Separation of Variables.
\begin{equation*}
\dfrac{\mu^\prime (t)}{\mu(t)} = p(t)
\end{equation*}
\onslide<6>
We can find a solution $\mu(t)>0$ by Separation of Variables.
\begin{equation*}
\ln\abs{\mu(t)} = \int p(t) dt
\end{equation*}
\onslide<7->
We can find a solution $\mu(t)>0$ by Separation of Variables.
\begin{equation*}
\mu(t) = e^{\int p(t) dt}
\end{equation*}
\end{overprint}
\begin{overprint}
\onslide<8>
We now know the integrating factor, and perform the same steps as before.
\begin{equation*}
y^\prime + p(t) y = f(t)
\end{equation*}
\onslide<9>
Multiply both sides by the integrating factor.
\begin{equation*}
\mu(t)\cdot\left(y^\prime + p(t) y\right) = \mu(t)\cdot f(t)
\end{equation*}
\onslide<10>
Apply the property $\mu(t)\cdot\left(y^\prime + p(t) y\right) = {\left(\mu(t)\cdot y\right)}^\prime$ to the left-hand side.
\begin{equation*}
{\left(\mu(t) y\right)}^\prime = \mu(t) f(t)
\end{equation*}
\onslide<11>
Integrate both sides.
\begin{equation*}
\mu(t) y(t) = \int \mu(t) f(t) dt + c
\end{equation*}
\onslide<12->
Assuming $\mu(t)\neq 0$, we can solve for $y$.
\begin{equation*}
y(t) = \dfrac{1}{\mu(t)} \int \mu(t) f(t) dt + \dfrac{c}{\mu(t)}
\end{equation*}
\end{overprint}
\end{block}
\onslide<13->
\begin{block}{Note}
We have again found $y_h$ and $y_p$ at the same time.
\end{block}
\end{frame}

\begin{frame}
\begin{block}{Integrating Factor Method for First-Order Linear DEs}
\small
To solve the linear first-order DE, where $p$ and $f$ are continuous on a domain $I$.
\begin{equation*}
y^\prime + p(t) y = f(t)
\end{equation*}
\begin{description}
\vspace{-8mm}
\item[Step 1.] Find the integrating factor $\mu(t) = e^{\int p(t) dt}$, where $\int p(t) dt$ represents \emph{any} anti-derivative of $p(t)$.
\item[Step 2.] Multiply both sides of the DE by $mu(t)$, which always simplifies to:

\vspace{-2mm}
\begin{equation*}
{\left(e^{\int p(t) dt} y(t)\right)}^\prime = e^{\int p(t) dt} f(t)
\end{equation*}
\item[Step 3.] Find the anti-derivative to get:

\vspace{-2mm}
\begin{equation*}
e^{\int p(t) dt} y(t) = \int e^{\int p(t) dt} f(t) dt +c
\end{equation*}
\item[Step 4.] Solve algebraically for $y$.

\vspace{-2mm}
\begin{equation*}
y = e^{-\int p(t) dt} \int e^{\int p(t) dt} f(t) dt + c e^{-\int p(t) dt}
\end{equation*}
\item[Step 5.] For IVPs, substitute the initial conditions in to find $c$.
\end{description}
\end{block}
\end{frame}

\begin{frame}
\begin{example}
Consider the IVP
\begin{equation*}
y^\prime - y = t,\quad y(0) = 1
\end{equation*}
Let us solve this DE using the Integrating Factor method.

\vspace{2mm}
\begin{overprint}
\onslide<2-6>
\textbf{Step 1.} Find the integrating factor:
\begin{equation*}
\mu(t) = \alt<3->{e^{\int (-1) dt}}{e^{\int p(t) dt}} \visible<4->{=e^{-t}}
\end{equation*}
\visible<5->{\textbf{Step 2.} Multiply both sides of the DE by $\mu(t)$:
\begin{equation*}
e^{-t}\left(y^\prime - y\right) = te^{-t}
\end{equation*}}
\visible<6->{Which reduces to:
\begin{equation*}
{\left(e^{-t} y\right)}^\prime = t e^{-t}
\end{equation*}}
\onslide<7->
\textbf{Step 3.} Find the antiderivative:
\begin{equation*}
e^{-t} y = \int t e^{-t} dt \visible<8->{= e^{-t}(-t-1)+c}
\end{equation*}
\visible<9->{\textbf{Step 4.} Solve for $y$:
\begin{equation*}
y(t) = e^t \left(e^{-t}\right)\left(-t-1\right) + c e^{t} \visible<10->{=-t-1+ce^{t}}
\end{equation*}}
\visible<11->{\textbf{Step 5.} Plug in the initial conditions to find the solution to the IVP\@:
\begin{equation*}
1 = y(0) = -0-1+c e^{0} \visible<12->{\Rightarrow c = 2}
\end{equation*}}
\visible<13->{Thus, the solution to the IVP is $y(t) = -t -1 + 2 e^t$}
\end{overprint}
\end{example}
\end{frame}
\end{document}
