\documentclass{beamer}
\usepackage[utf8]{inputenc}
\usepackage[english]{babel}
\usepackage[T1]{fontenc}
\usepackage{slide_helper}
\usepackage[super]{nth}
\usepackage{array}
\usepackage{wasysym}

\DeclareSymbolFont{extraup}{U}{zavm}{m}{n}
\DeclareMathSymbol{\varheart}{\mathalpha}{extraup}{86}
\DeclareMathSymbol{\vardiamond}{\mathalpha}{extraup}{87}
\DeclareMathSymbol{\varclub}{\mathalpha}{extraup}{84} 
\DeclareMathSymbol{\varspade}{\mathalpha}{extraup}{85}

\newcommand{\suitheart}[1][]{{\color{red}\text{#1}\varheart}}
\newcommand{\suitspade}[1][]{{\color{black}\text{#1}\spadesuit}}
\newcommand{\suitdiamond}[1][]{{\color{red}\text{#1}\vardiamond}}
\newcommand{\suitclub}[1][]{{\color{black}\text{#1}\varclub}}

\newcommand{\prob}[1]{P\left(#1\right)}
\newcommand{\condprob}[2]{\prob{#1~\middle|~#2}}
\newcommand{\comb}[2]{_{#1}C_{#2}}

\title[MA205 - Section 4.2]{Addition Rule and Multiplication Rule}

\begin{document}
\begin{frame}
\titlepage
\end{frame}

\begin{frame}
\begin{definition}
A \textbf{compound event} is any event combining two or more simple events.
\end{definition}\pause

\begin{definition}
Events $A$ and $B$ are \textbf{disjoint} (or \textbf{mutually exclusive}) if they cannot occur at the same time.
\end{definition}\pause

\begin{definition}
Events $A$ and $B$ are \textbf{independent events} if the probability of $B$ occurring is the same, whether or not $A$ occurs.
\end{definition}\pause

\begin{example}
Are these events independent?
\begin{enumerate}
\item A fair coin is tossed two times. The two events are (1) first toss is a head and (2) second toss is a head.\pause \quad\textbf{Independent}\pause
\item You draw a card from a deck, then without replacing the first, draw a second card?\pause \quad\textbf{Dependent}
\end{enumerate}
\end{example}
\end{frame}

\begin{frame}
\begin{block}{Multiplication Rule}
If events $A$ and $B$ are independent, then the probability of both $A$ and $B$ occurring is
\begin{equation*}
\prob{A~\text{and}~B}=\prob{A} \cdot \prob{B}
\end{equation*}
\end{block}\pause

\begin{example}
Suppose we flip a fair coin and roll a fair die. 

\vspace{2mm}
The sample space is $\set{\text{H1}, \text{H2}, \text{H3}, \text{H4}, \text{H5}, \text{H6}, \text{T1}, \text{T2}, \text{T3}, \text{T4}, \text{T5}, \text{T6}}$.\pause

\vspace{2mm}
The probability of getting tails on the coin and three on the die is 
\begin{equation*}
\prob{\text{T3}}=\dfrac{1}{12}
\end{equation*}\pause

We could have also calculated
\begin{equation*}
\prob{\text{H and 3}} = \prob{\text{H}}\cdot\prob{3} = \dfrac{1}{2}\cdot\dfrac{1}{6} = \dfrac{1}{12}
\end{equation*}
\end{example}
\end{frame}

\begin{frame}
\begin{example}
Modern aircraft are now highly reliable, and one design feature contributing to that reliability is the use of \textbf{redundancy}, whereby critical components are duplicated so that if one fails, the other will still work.\pause

\vspace{1mm}
The Airbus 310 airliner has three independent hydraulic systems, so if one system fails, full flight control is maintained. Let us assume the probability of a hydraulic system failing is 0.002.\pause

\begin{itemize}
\item If the Airbus 310 had only a single hydraulic system, what is the probability that the flight control system would work for an entire flight?\pause \quad $\prob{\text{safe flight}}=1-\prob{\text{system fail}}=1-0.002=0.998$\pause
\item Given the Airbus 310 has three independent hydraulic systems, what is the probability that the flight control system would work for an entire flight?\pause
\vspace{-4mm}
\begin{equation*}
\begin{aligned}
\prob{\text{safe flight}}
&=1-\prob{\text{system}_1\text{ fail}\text{ and }\text{system}_2\text{ fail}\text{ and }\text{system}_3\text{ fail}}\\\pause
&=1-\prob{\text{system}_1\text{ fail}}\cdot\prob{\text{system}_2\text{ fail}}\cdot\prob{\text{system}_3\text{ fail}}\\\pause
&=1-0.002\cdot0.002\cdot0.002\\\pause
&=1-0.000000008=.999999992
\end{aligned}
\end{equation*}
\end{itemize}
\vspace{-5mm}
\end{example}
\end{frame}

\begin{frame}
\begin{block}{Addition Rule}
The probability of either $A$ or $B$ occurring (or both) is
\begin{equation*}
\prob{A~\text{or}~B}=\prob{A}+\prob{B}-\prob{A~\text{and}~B}
\end{equation*}
\end{block}\pause

\begin{example}
Suppose we flip a fair coin and roll a fair die. 

\vspace{2mm}
The sample space is $\set{\text{H1}, \text{H2}, \text{H3}, \text{H4}, \text{H5}, \text{H6}, \text{T1}, \text{T2}, \text{T3}, \text{T4}, \text{T5}, \text{T6}}$.

\vspace{2mm}
We want to calculate the probability of getting a head or a six.\pause

\vspace{2mm} 
The outcomes are: H1, H2, H3, H4, H5, H6, T6. Giving $\prob{\text{H or 6}} = \tfrac{7}{12}$.\pause

\vspace{2mm}
Notice that $\tfrac{6}{12}=\tfrac{1}{2}$ of the outcomes have heads and $\tfrac{2}{12}=\tfrac{1}{6}$ have a six. \pause

\vspace{2mm}
But, $\tfrac{1}{2}+\tfrac{1}{6}=\frac{8}{12}$, is wrong because we have double counted H6. Thus, we need to subtract $\prob{\text{H6}}=\tfrac{1}{12}$.
\begin{equation*}
\prob{\text{H or 6}} = \prob{\text{H}}+\prob{6}-\prob{\text{H and 6}} = \tfrac{1}{2}+\tfrac{1}{6}-\tfrac{1}{12}=\tfrac{7}{12}
\end{equation*}
\end{example}
\end{frame}

\begin{frame}
\begin{example}
Suppose we draw one card from a standard deck. Let us find the probability that we get a Queen or a King.\pause

\vspace{2mm}
There are 4 Queens and 4 Kings in the deck, hence eight outcomes out of 52 possible outcomes. So, the probability is
\begin{equation*}
\prob{\text{Q or K}} = \dfrac{8}{52}
\end{equation*}\pause

Since there are no cards that are both Kings and Queens, we have
\begin{equation*}
\prob{\text{Q or K}} = \prob{\text{Q}} + \prob{K} - \prob{\text{Q and K}} = \dfrac{4}{52}+\dfrac{4}{52}-\dfrac{0}{52} = \dfrac{8}{52}
\end{equation*}
\end{example}\pause

\begin{note}
If two events are \textbf{disjoint}, then $\prob{A~\text{or}~B} = \prob{A}+\prob{B}$.
\end{note}
\end{frame}

\begin{frame}
\begin{example}
Suppose we draw one card from a standard deck. Let us calculate the probability that we get a red card or a King.\pause

\vspace{2mm}
Half the cards are red, so $\prob{\text{Red}}=\dfrac{26}{52}$.\pause

\vspace{2mm}
Four cards are Kings, so $\prob{\text{K}}=\dfrac{4}{52}$.\pause

\vspace{2mm}
Two cards are red kings, so $\prob{\text{Red and K}}=\dfrac{2}{52}$.\pause

\vspace{2mm}
Thus,
\begin{equation*}
\prob{\text{Red or K}} 
= \prob{\text{Red}} + \prob{\text{K}} - \prob{\text{Red and K}}
= \dfrac{26}{52} + \dfrac{4}{52} - \dfrac{2}{52}
= \dfrac{7}{13}
\end{equation*}
\end{example}
\end{frame}

\begin{frame}
\begin{example}
What is the probability that two cards drawn at random from a deck will both be aces?\pause

\vspace{2mm}
You might guess that since there are four aces in the deck of 52 cards, the probability would be $\dfrac{4}{52}\cdot\dfrac{4}{52}=\dfrac{1}{169}$.\pause

\vspace{2mm}
The problem here is that the events are not independent. Once one card is drawn, there are only 51 cards remaining. Once an ace has been drawn, there are only three remaining.\pause

\vspace{2mm}
This means that the the probability of drawing two aces is $\dfrac{4}{52}\cdot\dfrac{3}{51}=\dfrac{1}{221}$.
\end{example}\pause

\begin{definition}
The probability the event $B$ occurs, given that event $A$ has happened, is represented as $\condprob{B}{A}$. This is called a \textbf{conditional probability}.

\vspace{2mm}
Read as \textquote{the probability of $B$ given $A$.}
\end{definition}
\end{frame}

\begin{frame}
\begin{example}
\begin{center}
\begin{tabular}{|l|c|c|c|}
\hline
Car color & Speeding ticket & No speeding ticket & Total \\ \hline
Red & 15 & 135 & 150 \\ \hline
Not red & 45 & 470 & 515 \\ \hline
Total & 60 & 605 & 665 \\ \hline
\end{tabular}
\end{center}

Find the probability someone has gotten a speeding ticket \emph{given} they drive a red car.\pause
\begin{equation*}
\condprob{\text{ticket}}{\text{red}} = \dfrac{15}{150} = \dfrac{1}{10} = 10\%
\end{equation*}\pause

Find the probability someone drives a red car \emph{given} they have gotten a speeding ticker.\pause
\begin{equation*}
\condprob{\text{red}}{\text{ticket}} = \dfrac{15}{60} = \dfrac{1}{4} = 25\%
\end{equation*}
\end{example}\pause

\begin{note}
In general $\condprob{B}{A}\neq\condprob{A}{B}$.
\end{note}
\end{frame}

\begin{frame}
\begin{block}{Multiplication Rule}
If $A$ and $B$ are not independent, then $\prob{A~\text{and}~B}=\prob{A}\cdot\condprob{B}{A}$.
\end{block}\pause

\begin{example}
If you pull two cards out of a deck, find the probability that both are hearts.\pause

\vspace{2mm}
The probability that the first card is a heart is $\prob{\text{\nth{1} $\suitheart$}}=\dfrac{13}{52}$.\pause

\vspace{2mm}
The probability that the second card is a heart, given that the first card was a heart, is $\condprob{\text{\nth{2} $\suitheart$}}{\text{\nth{1} $\suitheart$}}=\dfrac{12}{51}$.\pause

\vspace{2mm}
So, the probability that both are hearts is
\begin{equation*}
\prob{\text{both $\suitheart$}}=\prob{\text{\nth{1} $\suitheart$}}\cdot\condprob{\text{\nth{2} $\suitheart$}}{\text{\nth{1} $\suitheart$}} =  \dfrac{13}{52} \cdot \dfrac{12}{51} = \dfrac{156}{2652} \approx 5.9\%
\end{equation*}
\end{example}
\end{frame}

\begin{frame}
\begin{example}
If you draw two cards from a deck, find the probability that you will get the Ace of Diamonds and a black card.\pause

\vspace{1mm}
The order you draw the two cards doesn't matter, so there are two events:\pause
\begin{description}
\item[Event $A$] Drawing the Ace of Diamonds then drawing a black card.
\begin{equation*}
\begin{aligned}
\prob{\suitdiamond[A]~\text{and Black}} &= \prob{\suitdiamond[A]}\cdot\condprob{\text{Black}}{\suitdiamond[A]}\\
&= \dfrac{1}{52}\cdot\dfrac{26}{51} = \dfrac{1}{102}
\end{aligned}
\end{equation*}\pause

\vspace{-4mm}
\item[Event $B$] Drawing a black card then drawing the Ace of Diamonds.
\begin{equation*}
\begin{aligned}
\prob{\text{Black and}~\suitdiamond[A]} &= \prob{\text{Black}}\cdot\condprob{\suitdiamond[A]}{\text{Black}} \\
&= \dfrac{26}{52}\cdot\dfrac{1}{51}=\dfrac{1}{102}
\end{aligned}
\end{equation*}\pause
\end{description}

\vspace{-4mm}
These events are independent and mutually exclusive, so 

\vspace{-2.5mm}
\begin{equation*}
\prob{A~\text{or}~B} = \prob{A}+\prob{B} = \dfrac{1}{102}+\dfrac{1}{102} = \dfrac{2}{102} \approx 1.96\%
\end{equation*}
\end{example}
\end{frame}

\begin{frame}
\begin{block}{Sampling}
Sampling methods are critically important, and the following relationships hold:
\begin{itemize}
\item Sampling \emph{with replacement}: Selections are \emph{independent} events.
\item Sampling \emph{without replacement}: Selections are \emph{dependent} events.
\end{itemize}
\end{block}\pause

\begin{block}{Note}
Calculations involving conditional probabilities can often be very cumbersome.
\end{block}\pause

\begin{block}{Treating Dependent Events as Independent}
When sampling without replacement and the sample size is no more than 5\% of the size of the population, treat the selections as being independent (even though they are actually dependent).
\end{block}
\end{frame}

\begin{frame}
\begin{example}
Assume that three adults are randomly selected without replacement from the $247,436,830$ adults in the United States. If we assume that 10\% of adults use drugs, what is the probability that the three selected adults all use drugs?\pause

Because the three adults are randomly selected without replacement, the three events are dependent. This means the exact probability would be rather cumbersome. 
\vspace{-2mm}
\begin{equation*}
\begin{aligned}
\prob{\text{\small all use drugs}} &= \prob{\text{\small first use drugs and second use drugs and third use drugs}}\\
&=\left(\dfrac{24,743,683}{247,436,830}\right)\cdot\left(\dfrac{24,743,682}{247,436,829}\right)\cdot\left(\dfrac{24,743,681}{247,436,828}\right)\\
&=0.0009999998909\pause\quad\text{(Imagine selecting 10,000 adults!)}
\end{aligned}
\end{equation*}\pause
Since 5 adults is less that 5\% of the total population, we can simplify the calculations considerably.
\vspace{-2mm}
\begin{equation*}
\begin{aligned}
\prob{\text{\small all use drugs}} &= \prob{\text{\small first use drugs and second use drugs and third use drugs}}\\
&=0.1\cdot0.1\cdot0.1=0.00100
\end{aligned}
\end{equation*}
\end{example}
\end{frame}

\begin{frame}
\begin{example}
When two different people are randomly selected from those at your school, we can assume that birthdays occur on the days of the week with equal frequencies.

\vspace{2mm}
Find the probability that two people are born on the same day of the week:\pause 

\vspace{1mm}
Because no particular day of the week is specified, the first person can be born on any day of the week. This means the probability that the second person is born on the same day is 1/7.\pause

\vspace{2mm}
Find the probability that both people are born on a Monday:\pause

\vspace{1mm}
The probability that the any person is born on a Monday is 1/7. Since birth dates are independent, the probability that both are born on Monday is 
\vspace{-2mm}
\begin{equation*}
\dfrac{1}{7}\cdot\dfrac{1}{7}=\dfrac{1}{49}
\end{equation*}
\end{example}\pause

\begin{block}{Caution}
In any probability calculation, it is very important to carefully identify the event being considered.
\end{block}
\end{frame}
\end{document}
