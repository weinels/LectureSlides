\documentclass{beamer}
\usepackage[utf8]{inputenc}
\usepackage[english]{babel}
\usepackage[T1]{fontenc}
\usepackage{slide_helper}
\usepackage[super]{nth}
\usepackage{array}
\usepackage{wasysym}

\DeclareSymbolFont{extraup}{U}{zavm}{m}{n}
\DeclareMathSymbol{\varheart}{\mathalpha}{extraup}{86}
\DeclareMathSymbol{\vardiamond}{\mathalpha}{extraup}{87}
\DeclareMathSymbol{\varclub}{\mathalpha}{extraup}{84} 
\DeclareMathSymbol{\varspade}{\mathalpha}{extraup}{85}

\newcommand{\suitheart}[1][]{{\color{red}\text{#1}\varheart}}
\newcommand{\suitspade}[1][]{{\color{black}\text{#1}\spadesuit}}
\newcommand{\suitdiamond}[1][]{{\color{red}\text{#1}\vardiamond}}
\newcommand{\suitclub}[1][]{{\color{black}\text{#1}\varclub}}

\newcommand{\prob}[1]{P\left(#1\right)}
\newcommand{\condprob}[2]{\prob{#1~\middle|~#2}}
\newcommand{\comb}[2]{_{#1}C_{#2}}
\newcommand{\perm}[2]{_{#1}P_{#2}}

\title[MA205 - Section 4.4]{Counting}

\begin{document}
\begin{frame}
\titlepage
\end{frame}

\begin{frame}
\begin{block}{Multiplication Counting Rule}
For a sequence of events in which the first event can occur $n_1$ ways, the second event can occur $n_2$ ways, the third event can occur $n_3$ ways, and so on. The total number of outcomes $n_1\cdot n_2\cdot n_3\cdot\cdots$.
\end{block}\pause

\begin{example}
When making random guesses for an unknown four-digit passcode, each digit can be $0,1,\ldots,9$. What is the total number of different possible passcodes?\pause

\vspace{1mm}
There are 10 possible choices for each digit, so the total number of passcodes is $10\cdot 10\cdot 10\cdot 10 = 10,000$.
\end{example}
\end{frame}

\begin{frame}
\begin{definition}
The \textbf{factorial symbol (!)} denotes the product of decreasing positive whole numbers. By special definition, $0!=1$.
\end{definition}\pause

\begin{example}
\vspace{-3mm}
\begin{equation*}
4!=4\cdot3\cdot2\cdot1=24
\end{equation*}
\end{example}\pause

\begin{block}{Factorial Rule}
The number of different arrangements (where order matters) of $n$ different items when all $n$ items are selected is $n!$. %chktex 40
\end{block}\pause

\begin{example}
A researcher must visit the presidents of the Gallup, Neilsen, Harris, Pew, and Zogby polling companies.
\begin{itemize}
\item How many different travel iterations are there?\pause\ $5!=5\cdot4\cdot3\cdot2\cdot1=120$.\pause
\item What is the probability that the presidents are visited in order from younger to oldest?\pause~There is only one: $1/120=.0083$.
\end{itemize}
\end{example}
\end{frame}

\begin{frame}
\begin{definition}
When $n$ different items are available and $r$ of them are selected without replacement, the number of different permutations (where order counts) is given by
\begin{equation*}
\perm{n}{r}=\dfrac{n!}{(n-r)!}
\end{equation*}
\end{definition}\pause

\begin{example}
In a state lottery, 48 balls numbered 1 to 48 are placed in a machine. The balls are drawn without replacement, where the order determines the winning number. 

\vspace{1mm}
How many possible lottery number are there?\pause

\vspace{1mm}
There are $n=48$ balls and $r=6$ are chosen, so the total number of permutations are:
\begin{equation*}
\perm{48}{r} =\dfrac{48!}{(48-6)!}=\pause 8,835,488,640
\end{equation*}
\end{example}
\end{frame}

\begin{frame}
\begin{example}
In a horse race, a trifecta bet is won by correctly selecting, in the correct order, the horses that finish first, second, and third. The 104th running Kentucky Derby has a field of 19 horses.

\vspace{1mm}
How many different trifecta bets are possible?\pause

\vspace{1mm}
There are $n=19$ horses available, and we must select $r=3$ of them without replacement. The number of permutations is

\vspace{-2mm}
\begin{equation*}
\perm{19}{3}=\dfrac{19!}{(19-3)!}=5814
\end{equation*}\pause

\vspace{-4mm}
If a bettor randomly selects three horses for a trifecta bet, what is the probability of winning?\pause

\vspace{1mm}
There are 5814 arrangements of three horses, but only one of them will win the trifecta bet. So the probability is

\vspace{-3mm}
\begin{equation*}
\prob{\text{trifecta win}}=\dfrac{1}{5814}=.00017
\end{equation*}\pause

\vspace{-4mm}
We are assuming that all horses are equally likely to win the Kentucky Derby. In practice, this is not true. Some horses are faster than others.
\end{example}
\end{frame}

\begin{frame}
\begin{definition}
When $n$ different items are available, but only $r$ of them are selected without replacement, the number of different combinations (order does not matter) if found as follows

\vspace{-4mm}
\begin{equation*}
\comb{n}{k}=\binom{n}{k} = \dfrac{n!}{k!\left(n-k\right)!}
\end{equation*}
\end{definition}\pause

\begin{example}
How many five-card poker hands are there?\pause

\begin{equation*}
\comb{52}{5} = \dfrac{52!}{5!\left(52-5\right)!} = 2,598,960
\end{equation*}
\end{example}\pause

\begin{example}\label{ex:poker_straight}
What is the probability that a poker hand will contain exactly two jacks?\pause

\begin{equation*}
\dfrac{\left(\comb{4}{2}\right)\left(\comb{48}{3}\right)}{\comb{52}{5}} = \dfrac{6\cdot17,296}{2,598,960} = \dfrac{103,776}{2,598,960} \approx 4\%
\end{equation*}
\end{example}
\end{frame}

\begin{frame}
\begin{example}
What is the probability that a poker hand will contain a straight?\pause

\vspace{2mm}
The lowest ranked straight is A,2,3,4,5 and the highest ranked straight is 10,J,Q,K,A. Thus, for the any of the ten ranks A through 10, we can build a straight. Each card can be from any of the four suits.\\ This means the probability is:
\begin{equation*}
\prob{\text{straight}} = \dfrac{10\left(\comb{4}{1}\right)\left(\comb{4}{1}\right)\left(\comb{4}{1}\right)\left(\comb{4}{1}\right)\left(\comb{4}{1}\right)}{\comb{52}{5}} = \dfrac{10,240}{2,598,960} \approx 3.94\%
\end{equation*}
\end{example}\pause

\begin{example}
What is the probability that a poker hand will contain a straight, but not a straight flush?\pause

We calculated in Example~\ref{ex:poker_straight} the number of straights possible. Now, just need to subtract out the number of straight flushes.
\begin{equation*}
\prob{\text{straight}} = \dfrac{10,240 - 40}{\comb{52}{5}} = \dfrac{10,200}{2,598,960} \approx 3.92\%
\end{equation*}

\end{example}
\end{frame}
\end{document}
