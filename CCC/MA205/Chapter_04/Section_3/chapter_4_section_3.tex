\documentclass{beamer}
\usepackage[utf8]{inputenc}
\usepackage[english]{babel}
\usepackage[T1]{fontenc}
\usepackage{slide_helper}
\usepackage[super]{nth}
\usepackage{array}
\usepackage{wasysym}

\DeclareSymbolFont{extraup}{U}{zavm}{m}{n}
\DeclareMathSymbol{\varheart}{\mathalpha}{extraup}{86}
\DeclareMathSymbol{\vardiamond}{\mathalpha}{extraup}{87}
\DeclareMathSymbol{\varclub}{\mathalpha}{extraup}{84} 
\DeclareMathSymbol{\varspade}{\mathalpha}{extraup}{85}

\newcommand{\suitheart}[1][]{{\color{red}\text{#1}\varheart}}
\newcommand{\suitspade}[1][]{{\color{black}\text{#1}\spadesuit}}
\newcommand{\suitdiamond}[1][]{{\color{red}\text{#1}\vardiamond}}
\newcommand{\suitclub}[1][]{{\color{black}\text{#1}\varclub}}

\newcommand{\prob}[1]{P\left(#1\right)}
\newcommand{\condprob}[2]{\prob{#1~\middle|~#2}}
\newcommand{\comb}[2]{_{#1}C_{#2}}

\title[MA205 - Section 4.3]{Complements,\\ Conditional Probability,\\ and Bayes's Theorem}

\begin{document}
\begin{frame}
\titlepage
\end{frame}

\begin{frame}
\begin{block}{At Least One}
When finding the probability of some event occurring \textquote{at lease once,} we should keep the following in mind:
\begin{itemize}\pause
\item \textquote{At least one} has the same meaning as \textquote{one or more.}\pause
\item The \emph{complement} of getting \textquote{at least one} particular event is that you get \emph{no} occurrences of that event.
\end{itemize}
\end{block}\pause

\begin{example}
The following are the same event:
\begin{itemize}
\item \textquote{Not getting at least 1 girl in 10 births.}
\item \textquote{Getting no girls in 10 births.}
\item \textquote{Getting 10 boys in 10 births.}
\end{itemize}
\end{example}
\end{frame}

\begin{frame}
\begin{block}{Finding the Probability of \textquote{At Least One}}
\begin{enumerate}
\item Let $A$ = \textquote{getting at least one event.}\pause
\item Then $\bar{A}$ = \textquote{getting none of the event being considered.}\pause
\item Find $\prob{\bar{A}}$ = \textquote{probability that event $A$ does not occur.}\pause
\item Thus $\prob{A}=1-\prob{\bar{A}}$.
\end{enumerate}
\end{block}\pause

\begin{example}
A study by SquareTrade found that 6\% of damaged iPads were damaged by \textquote{bags/backpacks.} If 20 damaged iPads are randomly selected, find the probability of getting \emph{at least one} that was damaged in a bag/backpack.\pause
\begin{enumerate}
\item Let $A$ = \textquote{at least 1 iPad was damaged in a bag/backpack.}\pause
\item Then $\bar{A}$ = \textquote{none of the iPads were damaged in a bag/backpack.}\pause
\item Find $\prob{\bar{A}}=0.94\cdot0.94\cdot\cdots\cdot0.94={0.94}^{20}=0.290$\pause
\item Thus $\prob{A}=1-\prob{\bar{A}}=1-0.290=0.710$.
\end{enumerate}
\end{example}
\end{frame}

\begin{frame}
\begin{definition}
A \textbf{conditional probability} of an event is a probability obtained with the additional information that some other event has already occurred.
\end{definition}\pause

\begin{block}{Intuitive Approach}
The conditional probability of $B$ occurring given that $A$ has occurred can be found by assuming that event $A$ has occurred and then calculating the probability that event $B$ will occur. 
\end{block}\pause

\begin{block}{Formal Approach}
\begin{equation*}
\condprob{A}{B}=\dfrac{\prob{A~\text{and}~B}}{\prob{A}}
\end{equation*}
\end{block}
\end{frame}

\begin{frame}
\begin{example}
\begin{center}
\begin{tabular}{|l|c|c|}\hline
&\textbf{Positive Test Result} & \textbf{Negative Test Result} \\\hline
\textbf{Uses Drugs} & 45 & 5 \\
&(True Positive)&(False Negative)\\\hline
\textbf{Doesn't Use Drugs} & 25 & 480\\
&(False Positive) & (True Negative)\\\hline
\end{tabular}
\end{center}
Find the following probabilities:
\begin{itemize}
\item $\condprob{\text{positive test result}}{\text{subject uses drugs}}=\pause\dfrac{45}{50}=0.900$\pause
\item $\condprob{\text{subject uses drugs}}{\text{positive test result}}=\pause\dfrac{45}{70}=0.643$
\end{itemize}
\end{example}\pause

\begin{note}
In general $\condprob{B}{A}\neq\condprob{A}{B}$.
\end{note}
\end{frame}

\begin{frame}
\begin{example}
\only<1-8>{%
\onslide<+->
Let us assume that cancer has a 1\% prevalence, that is 1\% of the population has cancer. Let us also assume that $C$ is the event that a randomly selected subject has cancer.

\onslide<+->
\vspace{1mm}
A cancer test has the following performance characteristics:
\begin{itemize}
\item $\prob{C}=0.01$
\item The false positive rate is 10\%. That is, $\condprob{\text{positive test result}}{\bar{C}}$
\item The true positive rate is 80\%. That is, $\condprob{\text{positive test result}}{C}$
\end{itemize}
\onslide<+->
Suppose we want to compute $\condprob{C}{\text{positive test result}}$.

\onslide<+->
\vspace{1mm}
We can make the following observations.
\begin{itemize}[<+->]
\item<.-> Assume we have 100 subjects. Then, 10 subjects are expected to have cancer. The other 990 do not have cancer.
\item Among the 990 subjects without cancer, 99 will get a positive result.
\item Among the 990 subjects without cancer, 891 will get a negative result.
\item Among the 10 subject with cancer, 8 will get a positive result.
\item Among the 10 subjects with cancer, 2 will get a negative result.
\end{itemize}
}
\only<9->{%
We can summarize the results in the following table.
\begin{center}
\begin{tabular}{|l|c|c|c|}\hline
&\textbf{Positive Test Result} & \textbf{Negative Test Result} & \textbf{Total} \\\hline
\textbf{Has Cancer} & 8 & 2 & 10 \\
&(True Positive)&(False Negative)&\\\hline
\textbf{No Cancer} & 99 & 891 & 990\\
&(False Positive) & (True Negative)&\\\hline
\textbf{Total} & 107 & 893 &\\\hline
\end{tabular}
\end{center}\pause[10]

So, $\condprob{C}{\text{positive test result}}=8/107=0.0748$.\pause

\vspace{1mm}
We can also use the following formula:
\begin{equation*}
\begin{aligned}
\condprob{C}{\text{positive}}
&=\dfrac{\prob{C}\cdot\condprob{\text{positive}}{C}}{\left(\prob{C}\cdot\condprob{\text{positive}}{C}\right)+\left(\prob{\bar{C}}\cdot\condprob{\text{positive}}{\bar{C}}\right)}\\\pause
&=\dfrac{0.1\cdot(8/10)}{\left(0.01\cdot0.80\right)+\left(0.99\cdot0.10\right)}\\\pause
&=0.0748
\end{aligned}
\end{equation*}
\vspace{-2.2mm}
}
\end{example}
\end{frame}

\begin{frame}
\begin{block}{Bayes' Theorem}
\begin{equation*}
\condprob{A}{B}=\dfrac{\prob{A}\condprob{B}{A}}{\prob{A}} = \dfrac{\prob{A}\cdot\condprob{B}{A}}{\left(\prob{A}\cdot\condprob{B}{A}\right)+\left(\prob{\bar{A}}\cdot\condprob{B}{\bar{A}}\right)}
\end{equation*}
\end{block}\pause

\begin{block}{Note}
Bayes' Theorem is useful for sequential events, where new information is obtained for a subsequent event, and the new information is used to revise the probability of the original event.
\end{block}\pause

\begin{definition}
A \textbf{prior probability} is an initial probability value originally obtained before any additional information is obtained.
\end{definition}\pause

\begin{definition}
A \textbf{posterior probability} is a probability value that has been revised by using additional information that is later obtained.
\end{definition}
\end{frame}
\end{document}
