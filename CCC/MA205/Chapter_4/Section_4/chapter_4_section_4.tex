\documentclass{beamer}
\usepackage[utf8]{inputenc}
\usepackage[english]{babel}
\usepackage[T1]{fontenc}
\usepackage{slide_helper}
\usepackage[super]{nth}
\usepackage{array}
\usepackage{wasysym}

\DeclareSymbolFont{extraup}{U}{zavm}{m}{n}
\DeclareMathSymbol{\varheart}{\mathalpha}{extraup}{86}
\DeclareMathSymbol{\vardiamond}{\mathalpha}{extraup}{87}
\DeclareMathSymbol{\varclub}{\mathalpha}{extraup}{84} 
\DeclareMathSymbol{\varspade}{\mathalpha}{extraup}{85}

\newcommand{\suitheart}[1][]{{\color{red}\text{#1}\varheart}}
\newcommand{\suitspade}[1][]{{\color{black}\text{#1}\spadesuit}}
\newcommand{\suitdiamond}[1][]{{\color{red}\text{#1}\vardiamond}}
\newcommand{\suitclub}[1][]{{\color{black}\text{#1}\varclub}}

\newcommand{\prob}[1]{P\left(#1\right)}
\newcommand{\condprob}[2]{\prob{#1~\middle|~#2}}
\newcommand{\comb}[2]{_{#1}C_{#2}}

\title[MA205 - Section 4.3]{Counting}

\begin{document}
\begin{frame}
\titlepage
\end{frame}

\begin{frame}
\begin{definition}
The number of combinations when choosing $k$ objects out of a pool of $n$ objects is:
\begin{equation*}
\comb{n}{k}=\binom{n}{k} = \dfrac{n!}{k!\left(n-k\right)!}
\end{equation*}
\end{definition}\pause

\begin{example}
How many five-card poker hands are there?\pause

\begin{equation*}
\comb{52}{5} = \dfrac{52!}{5!\left(52-5\right)!} = 2,598,960
\end{equation*}
\end{example}\pause

\begin{example}\label{ex:poker_straight}
What is the probability that a poker hand will contain exactly two jacks?\pause

\begin{equation*}
\dfrac{\left(\comb{4}{2}\right)\left(\comb{48}{3}\right)}{\comb{52}{5}} = \dfrac{6\cdot17,296}{2,598,960} = \dfrac{103,776}{2,598,960} \approx 4\%
\end{equation*}
\end{example}
\end{frame}

\begin{frame}
\begin{example}
What is the probability that a poker hand will contain a straight?\pause

\vspace{2mm}
The lowest ranked straight is A,2,3,4,5 and the highest ranked straight is 10,J,Q,K,A. Thus, for the any of the ten ranks A through 10, we can build a straight. Each card can be from any of the four suits.\\ This means the probability is:
\begin{equation*}
\prob{\text{straight}} = \dfrac{10\left(\comb{4}{1}\right)\left(\comb{4}{1}\right)\left(\comb{4}{1}\right)\left(\comb{4}{1}\right)\left(\comb{4}{1}\right)}{\comb{52}{5}} = \dfrac{10,240}{2,598,960} \approx 3.94\%
\end{equation*}
\end{example}\pause

\begin{example}
What is the probability that a poker hand will contain a straight, but not a straight flush?\pause

We calculated in Example~\ref{ex:poker_straight} the number of straights possible. Now, just need to subtract out the number of straight flushes.
\begin{equation*}
\prob{\text{straight}} = \dfrac{10,240 - 40}{\comb{52}{5}} = \dfrac{10,200}{2,598,960} \approx 3.92\%
\end{equation*}

\end{example}
\end{frame}

\begin{frame}
\begin{example}
In the casino game Roulette, a wheel with 38 spaces (18 red, 18 black, and 2 green) is spun. In one possible bet, the players bet \$1 on a single number. If that number is spun on the wheel, then they receive \$36. Otherwise, they lose their \$1.

\vspace{2mm}
On average, how much money should a player expect to win or lose if they play this game repeatedly?\pause

\vspace{2mm}
Any number you bet on will have the following probabilities:
\begin{center}
\begin{tabular}{|c|c|} \hline
Outcome & Probability \\ \hline
\$35 (win) & 1/38 \\\hline
-\$1 (lose) & 37/38 \\ \hline
\end{tabular}
\end{center}\pause

So, \textbf{on average}, we will have a net change of
\begin{equation*}
\$35\cdot\dfrac{1}{38} + -\$1\cdot\dfrac{37}{38} = \$0.9211 - \$0.9737 \approx -\$0.053
\end{equation*}

That is, \textbf{on average}, we will lose 5.3 cents per space we bet on.
\end{example}
\end{frame}
\end{document}
