\documentclass{beamer}
\usepackage[utf8]{inputenc}
\usepackage[english]{babel}
\usepackage{helvet}
\usepackage[T1]{fontenc}
\usepackage[inline]{asymptote}
\usepackage{asy_helper}
\usepackage{slide_helper}
\usepackage{multirow}
\usepackage{cancel}
\usepackage{tikz}
\usetikzlibrary{shapes.geometric, arrows}
\usepackage{pgfplots}
\pgfplotsset{compat=1.5} 
\usepgfplotslibrary{statistics}
\usetikzlibrary{external}
\tikzexternalize%

\begin{asydef}
  real nd_func(real mu, real sigma, real x)
  {
    return 1/sqrt(2*pi*sigma*sigma)*exp((-1*(x-mu)*(x-mu))/(2*sigma*sigma));
  }
  
  guide normal_dist(real mu, real sigma, real xmin, real xmax)
  {
    real nd(real x) { return nd_func(mu, sigma, x); }
    return graph(nd, xmin, xmax);
  }

  void shade_below(real mu, real sigma, real b, real xmin, real xmax, pen p=royalblue)
  {
    real nd_func(real x) { return 1/sqrt(2*pi*sigma*sigma)*exp((-1*(x-mu)*(x-mu))/(2*sigma*sigma)); }
    
    guide g = graph(nd_func, xmin, b);
    
    filldraw(g -- (b,0) -- cycle, p, black);
    
    draw((xmin,0)--(b,0));
  }

  void shade_above(real mu, real sigma, real b, real xmin, real xmax, pen p=royalblue)
  {
    real nd_func(real x) { return 1/sqrt(2*pi*sigma*sigma)*exp((-1*(x-mu)*(x-mu))/(2*sigma*sigma)); }
    
    guide g = graph(nd_func, b, xmax);
    
    filldraw(g -- (b,0) -- cycle, p, black);
    
    draw((xmax,0)--(b,0));
  }

  void shade_between(real mu, real sigma, real a, real b, pen p=royalblue)
  {
    real nd_func(real x) { return 1/sqrt(2*pi*sigma*sigma)*exp((-1*(x-mu)*(x-mu))/(2*sigma*sigma)); }
    
    guide g = graph(nd_func, a, b);
    
    filldraw((a,0) -- g -- (b,0) -- cycle, p, black);
  }

  void multiple_nd_curves_example(real std_dev)
  {
    size(300, 190, IgnoreAspect);
    
    draw(normal_dist(0, std_dev, -6,6));
    shade_between(0,std_dev,-std_dev,std_dev);
    draw((0,0)--(0,0.45));

    label("$\sigma="+format("%#.2f", std_dev)+"$", (-4.2,0.45), Fill(paleyellow));

    xaxis(Bottom(), RightTicks(new real[] {-6,-4,-2,0,2,4,6}));
    yaxis(Left(), LeftTicks(size=nan),ymin = 0, ymax = 0.5);
  }
\end{asydef}

\newcommand{\satisfied}[0]{{\color{green!70!black}\checkmark}}

\title[MA205 - Section 6.2]{Difference of Two Proportions}

\newcommand{\prob}[1]{P\left({#1}\right)}
\newcommand{\jointprob}[3]{\prob{{#1}~\text{#2}~{#3}}}
\newcommand{\condprob}[2]{\prob{{#1}~|~{#2}}}
\newcommand{\comb}[2]{_{#1}C_{#2}}

\begin{document}
\begin{frame}
  \titlepage
\end{frame}

\begin{frame}
  \begin{example}\label{CPR}
    \vspace{-2mm}
    Consider an experiment for patients who underwent cardiopulmonary resucitation (CPR) for a hear attack and we subsequently admitted to a hospital.\pause

    \vspace{1mm}
    These patients were randomly divided into a treatment group, where they received a blood thinner or the control group where they did not receive a blood thinner.\pause

    \vspace{1mm}
    The variable of interest was whether they survived for at least 24 hours.\pause

    \vspace{1mm}
    We really have two samples, and hence two sample proportions:
    \begin{itemize}
    \item The proportion of the treatment group that survived: $\hat{p}_{t}$
    \item The proportion of the control group that survived: $\hat{p}_{c}$
    \end{itemize}\pause

    \question{How would we determine if blood thinners actually make a difference with these patients?}\pause
    \answer{We can look at $\hat{p}_{t} - \hat{p}_{c}$.}\pause

    \vspace{1mm}
    But, what we really want to know is, if blood thinners have an effect of heart attack survival rates in the general population?
  \end{example}
\end{frame}

\begin{frame}
  \begin{note}
    The best point estimate for $p_1-p_2$ is\pause~$\hat{p}_1-\hat{p}_2$.
  \end{note}\pause

  \begin{block}{\normalsize Conditions For The Sampling Distribution Of $\hat{p}_1-\hat{p}_2$ To Be Normal}
    $\hat{p}_1-\hat{p}_2$ can be modeled using a normal distribution when:\pause
    \begin{description}
    \item[Independence:] The data are independent within and between the two groups. Generally this is satisfied if the data come from a randomized experiment.\pause
    \item[Success-Failure:] The success-failure condition holds for both groups, where we check successes and failures in each group separately.\pause
    \end{description}

    When these conditions are satisfied, the standard error of $\hat{p}_1-\hat{p}_2$ is

    \vspace{-2mm}
    \begin{equation*}
      \begin{aligned}
        SE = \sqrt{\dfrac{p_1(1-p_1)}{n_1}+\dfrac{p_2(1-p_2)}{n_2}}
      \end{aligned}
    \end{equation*}

    \vspace{-1mm}
    where $p_1$ and $p_2$ represent the population proportions, and $n_1$ and $n_2$ represent the sample sizes.
  \end{block}
\end{frame}

\begin{frame}
  \begin{block}{Confidence Intervals for $\hat{p}_1-\hat{p}_2$}
    When the independence and success-failure conditions are met, we can build confidence interval in the same general manner and before:
    \begin{equation*}
      \begin{array}{rcl}
        \text{point estimate} & \pm & z^*\cdot SE \\
        &\Downarrow\\
        \hat{p}_1-\hat{p}_2 & \pm & z^*\sqrt{\dfrac{p_1(1-p_1)}{n_1}+\dfrac{p_2(1-p_2)}{n_2}}
      \end{array}
    \end{equation*}
  \end{block}
\end{frame}

\begin{frame}
  \begin{example}
    We can summarize the results from the experiment in Example~\ref{CPR}:
    \begin{center}
      \begin{tabular}{lccc}\hline
        & Survived & Died & Total \\\hline
        Control & 11 & 29 & 50 \\
        Treatment & 14 & 26 & 40 \\\hline
        Total & 25 & 65 & 90\\\hline
      \end{tabular}
    \end{center}\pause

    \question{Is independence satisfied?}\pause
    \answer{This is a randomized experiment, so yes.}\pause

    \vspace{1mm}
    \question{Are the success-failure conditions satisfied?}\pause
    \answer{The treatment group had 11 survivals and 29 deaths, and the control group had 14 survivals and 26 deaths. All are more than 10, so yes.}
  \end{example}
\end{frame}

\begin{frame}
  \begin{examplecont}
    Let us create a 90\% confidence interval.\pause

    \vspace{1mm}
    We first need to calculate the point estimate:

    \vspace{-3mm}
    \begin{equation*}
      \begin{aligned}
        \hat{p}_{t} - \hat{p}_c\pause
        &= \dfrac{14}{40} - \dfrac{11}{50}\pause
        = 0.35 - 0.22\pause
        = 0.13
      \end{aligned}
    \end{equation*}\pause

    \vspace{-4mm}
    Next, the standard error:

    \vspace{-2mm}
    \begin{equation*}
      \begin{aligned}
        SE \approx \pause \sqrt{\dfrac{0.35(1-0.35)}{40}+\dfrac{0.22(1-0.22)}{50}}\pause
        =0.095
      \end{aligned}
    \end{equation*}

    Recall that the critical value for 90\% confidence is 1.65.\pause

    \vspace{1mm}
    The confidence interval is:

    \vspace{-4mm}
    \begin{equation*}
      \begin{aligned}
        \hat{p}_{t} - \hat{p}_c \pm z^*\cdot SE \pause
        \rightarrow
        0.13 \pm 1.65\cdot0.095 \pause
        \rightarrow
        \interval{\open{-0.027}}{\open{0.287}} \pause
      \end{aligned}
    \end{equation*}

    \vspace{-1mm}
    We are 90\% confident that blood thinners have a difference of -2.7\% to 28.7\% percentage point impact on survival rate for patients.\pause

    \vspace{1mm}
    \question{What can be conclude about whether blood thinners help or harm?}\pause
    \answer{Since 0\% is in the confidence interval, we don't have enough evidence to say if blood thinners had any impact.}
  \end{examplecont}
\end{frame}

\begin{frame}
  \begin{example}
    A 5-year experiment was conducted to evaluate the effectiveness of fish oils on reducing cardiovascular events, where each subject was randomized into one of two groups.\pause

    \vspace{1mm}
    We'll consider heart attack outcomes in these patients:
    \begin{center}
      \begin{tabular}{lccc}\hline
        & heart attack & no event & Total \\\hline
        fish oil & 145 & 12788 & 12933 \\
        placebo & 200 & 12738 & 12938 \\\hline
      \end{tabular}
    \end{center}
  \end{example}
\end{frame}

\begin{frame}
  \begin{example}

  \end{example}
  \addtocounter{theorem}{-1}
  \begin{example}[Continued]

  \end{example}
  \begin{example}

  \end{example}
  \begin{examplecont}

  \end{examplecont}
\end{frame}
\end{document}
