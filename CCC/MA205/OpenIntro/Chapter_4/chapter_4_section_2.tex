\documentclass{beamer}
\usepackage[utf8]{inputenc}
\usepackage[english]{babel}
\usepackage{helvet}
\usepackage[T1]{fontenc}
\usepackage[inline]{asymptote}
\usepackage{asy_helper}
\usepackage{slide_helper}

\title[MA205 - Section 4.2]{Geometric Distribution}

\newcommand{\prob}[1]{P\left({#1}\right)}
\newcommand{\jointprob}[3]{\prob{{#1}~\text{#2}~{#3}}}
\newcommand{\condprob}[2]{\prob{{#1}~|~{#2}}}
\newcommand{\comb}[2]{_{#1}C_{#2}}

\begin{document}
\begin{frame}
\titlepage
\end{frame}

\begin{frame}
  \begin{definition}
    When an individual trial only has two possible outcomes, often labeled \texttt{success} or \texttt{failure}, it is called a \textbf{Bernoulli random variable}.
  \end{definition}\pause

  \begin{note}
    It does not matter which outcome is labeled as \texttt{success} or \texttt{failure}, just that there are only two outcomes.
  \end{note}\pause

  \begin{note}
    Bernoulli random variables are often denoted with
    \begin{itemize}
    \item 1 for \texttt{success}
    \item 0 for \texttt{failure}
    \end{itemize}
  \end{note}\pause

  \begin{note}
    The events \texttt{success} and \texttt{failure} are complements.
  \end{note}
\end{frame}

\begin{frame}
  \begin{example}\label{universal donor}
    \vspace{-2mm}
    Subjects are randomly selected for the National Health and Nutrition Examination Survey conducted by the National Center for Health Statistics, Centers for Disease Control and Prevention.\pause
    
    \vspace{1mm}
    A person is a universal donor if they have group O and type Rh blood.\pause
    
    \vspace{1mm}
    If we think of each subject as a trial then:\pause
    \begin{itemize}
    \item If a person is a universal donor, we label them a \texttt{success}.\pause
    \item If a person is not a universal donor, we label them a \texttt{failure}.
    \end{itemize}\pause

    If there is a 6\% chance that a person is a universal donor, then:\pause
    \begin{itemize}
    \item The probability of a success is $p=0.06$\pause
    \item The probability of a failure is $q=1-p=\pause 0.94$
    \end{itemize}
  \end{example}\pause

  \begin{note}
    \texttt{success} and \texttt{failure} are not moral descriptions. We could have just as easily labeled the universal donors as \texttt{failure}.
  \end{note}
\end{frame}

\begin{frame}
  \begin{definition}
    The \textbf{sample proportion}, $\hat{p}$, is the sample mean:
    \begin{equation*}
      \begin{aligned}
        \hat{p} = \dfrac{\text{\# of successes}}{\text{\# of trials}}
      \end{aligned}
    \end{equation*}
  \end{definition}\pause

  \begin{example}
    Suppose we observe the ten trials of a Bernoulli random variable:
    \begin{equation*}
      \begin{aligned}
        1~1~1~0~1~0~0~1~1~0
      \end{aligned}
    \end{equation*}\pause
    The sample proportion of these observations would be:
    \begin{equation*}
      \begin{aligned}
        \hat{p} = \dfrac{1+1+1+0+1+0+0+1+1+0}{10}\pause = 0.6
      \end{aligned}
    \end{equation*}
  \end{example}
\end{frame}

\begin{frame}
  \begin{block}{Bernoulli Random Variable}
    If $X$ is a random variable that takes value 1 with probability $p$ and 0 with probability $q=1-p$, then $X$ is a Bernoulli random variable with mean and standard deviation:

    \vspace{-4mm}
    \begin{equation*}
      \begin{aligned}
        \mu=p
        \qquad
        \sigma=\sqrt{p(1-p)}
      \end{aligned}
    \end{equation*}
  \end{block}\pause

  \begin{example}
    In Example~\ref{universal donor}, $X$ describes the chances a subject is a universal donor, with probability of success $p=0.06$.\pause

    \vspace{1mm}
    The mean of $X$ is:
    \begin{equation*}
      \begin{aligned}
        \mu=p\pause
        =0.06\pause
      \end{aligned}
    \end{equation*}
    The standard deviation of $X$ is:
    \begin{equation*}
      \begin{aligned}
        \sigma = \sqrt{p(1-p)}\pause
        =\sqrt{0.06(1-0.06)}\pause
        =\sqrt{0.0564}\pause
        =0.237486842
      \end{aligned}
    \end{equation*}
  \end{example}
\end{frame}

\begin{frame}
  \begin{definition}
    The \textbf{geometric distribution} is used to describe how many trails it takes to observe a success.
  \end{definition}
\end{frame}
\end{document}
