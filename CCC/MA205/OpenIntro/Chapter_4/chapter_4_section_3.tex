\documentclass{beamer}
\usepackage[utf8]{inputenc}
\usepackage[english]{babel}
\usepackage{helvet}
\usepackage[T1]{fontenc}
\usepackage[inline]{asymptote}
\usepackage{asy_helper}
\usepackage{slide_helper}
\usepackage{cancel}
\usepackage{tikz}
\usetikzlibrary{shapes.geometric, arrows}
\usepackage{pgfplots}
\pgfplotsset{compat=1.5} 
\usepgfplotslibrary{statistics}
\usetikzlibrary{external}
\tikzexternalize%

\title[MA205 - Section 4.3]{Binomial Distribution}

\newcommand{\prob}[1]{P\left({#1}\right)}
\newcommand{\jointprob}[3]{\prob{{#1}~\text{#2}~{#3}}}
\newcommand{\condprob}[2]{\prob{{#1}~|~{#2}}}
\newcommand{\comb}[2]{_{#1}C_{#2}}

\newcommand<>{\success}[1]{{\color#2{green!70!black}#1}}
\newcommand<>{\failure}[1]{{\color#2{red!80}#1}}

\begin{document}
\begin{frame}
\titlepage
\end{frame}

\begin{frame}
  \begin{example}\label{twitter}
    \vspace{-2mm} % beamer bug
    Let us assume there is a 0.85 probability that a randomly chosen adult has heard of Twitter.\pause

    \vspace{1mm}
    Four people are chosen at random:

    \vspace{-2.5mm}
    \begin{center}
    Ariana ($A$), \quad
    Brittany ($B$), \quad
    Carlton ($C$), \quad
    Damian ($D$)
    \end{center}

    \vspace{-2mm}
    We want the probability exactly one of them will have heard of Twitter.\pause

    \vspace{1mm}
    There are four combinations possible:

    \vspace{-7mm}
    \begin{equation*}
      \begin{array}{l}
        \prob{A = \success<3-5>{\text{yes}}~\text{and}~
              B = \failure<3-5>{\text{no}}~\text{and}~
              C = \failure<3-5>{\text{no}}~\text{and}~
              D = \failure<3-5>{\text{no}}}\\
        \onslide<4->
        \qquad = \success<4-5>{0.85} \cdot \failure<4-5>{0.15}\cdot \failure<4-5>{0.15}\cdot \failure<4-5>{0.15}
        \onslide<5->
        = {(\success<5>{0.85})}^1{(\failure<5>{0.15})}^3 = 0.002869 \\
        \onslide<6->
        \prob{A = \failure<6-8>{\text{no}}~\text{and}~
              B = \success<6-8>{\text{yes}}~\text{and}~
              C = \failure<6-8>{\text{no}}~\text{and}~
              D = \failure<6-8>{\text{no}}}\\
        \onslide<7->
        \qquad = \failure<7-8>{0.15} \cdot \success<7-8>{0.85}\cdot \failure<7-8>{0.15}\cdot \failure<7-8>{0.15}
        \onslide<8->
        = {(\success<8>{0.85})}^1{(\failure<8>{0.15})}^3 = 0.002869 \\
        \onslide<9->
        \prob{A = \failure<9-11>{\text{no}}~\text{and}~
              B = \failure<9-11>{\text{no}}~\text{and}~
              C = \success<9-11>{\text{yes}}~\text{and}~
              D = \failure<9-11>{\text{no}}}\\
        \onslide<10->
        \qquad = \failure<10-11>{0.15} \cdot \failure<10-11>{0.15}\cdot \success<10-11>{0.85}\cdot \failure<10-11>{0.15}
        \onslide<11->
        = {(\success<11>{0.85})}^1{(\failure<11>{0.15})}^3 = 0.002869 \\
        \onslide<12->
        \prob{A = \failure<12-14>{\text{no}}~\text{and}~
              B = \failure<12-14>{\text{no}}~\text{and}~
              C = \failure<12-14>{\text{no}}~\text{and}~
              D = \success<12-14>{\text{yes}}}\\
        \onslide<13->
        \qquad = \failure<13-14>{0.15} \cdot \failure<13-14>{0.15}\cdot \failure<13-14>{0.15}\cdot \success<13-14>{0.85}
        \onslide<14->
        = {(\success<14>{0.85})}^1{(\failure<14>{0.15})}^3 = 0.002869 \\
      \end{array}
    \end{equation*}

    \vspace{-3mm}
    \onslide<15->
    So, the probability exactly one has heard of Twitter is
    \vspace{-4mm}
    \begin{equation*}
      0.002869 + 0.002869 + 0.002869 + 0.002869 = 0.11475 = 11.475\%
    \end{equation*}
  \end{example}
\end{frame}

\begin{frame}
  \begin{definition}
    The \textbf{binomial distribution} is used to describe the number of successes in a fixed number or trials.
  \end{definition}\pause

  \begin{block}{Notation}
    \begin{center}
      \begin{tabular}{ll}
        $p$ & The probability of a success. \\
        $q = 1-p$ & The probability of a failure. \\
        $n$ & The fixed number of trials. \\
        $k$ & The number of successes.
      \end{tabular}
    \end{center}
  \end{block}\pause

  \begin{note}
    Example~\ref{twitter} is how to find a binomial distribution the hard way.
  \end{note}\pause

  \begin{note}
    If all the scenarios are independent of each other, then we can calculate the final probability as:

    \vspace{-4mm}
    \begin{equation*}
      \begin{aligned}
        \left[\text{\# of scenarios}\right] \cdot \prob{\text{single scenario}}
      \end{aligned}
    \end{equation*}
  \end{note}
\end{frame}

\begin{frame}
  \begin{definition}
    The \textbf{factorial}, for any positive integer $n$, is

    \vspace{-3mm}
    \begin{equation*}
      \begin{aligned}
        0! &= 1 \\
        1! &= 1 \\
        2! &= 2\cdot 1 = 2 \\
        3! &= 3\cdot 2\cdot 1 = 6 \\
        4! &= 4\cdot 3\cdot 2\cdot 1 = 24 \\
        &\vdots \\
        n! &= n\cdot(n-1) \cdot(n-2)\cdot \cdots \cdot 4\cdot 3\cdot 2\cdot 1
      \end{aligned}
    \end{equation*}
  \end{definition}\pause

  \begin{note}
    Factorials can be calculated iteratively.\ i.e.\@
    \begin{equation*}
      \begin{aligned}
        (n+1)! &= n!\cdot (n+1)
      \end{aligned}
    \end{equation*}
  \end{note}
\end{frame}

\begin{frame}
  \begin{definition}
    The \textbf{binomial coefficients} gives the number of ways to choose $k$ successes in $n$ trials.:
    \begin{equation*}
      \begin{aligned}
        \binom{n}{k} = \dfrac{n!}{k!(n-k)!}
      \end{aligned}
    \end{equation*}
    Read \textquote{$n$ choose $k$.}
  \end{definition}\pause

  \begin{example}
    The number of ways to choose $k=3$ successes in $n=4$ trials:
    \begin{equation*}
      \begin{aligned}
        \binom{4}{3} \pause
        = \dfrac{4!}{3!(4-3)!} \pause
        = \dfrac{4!}{3!1!} \pause
        = \dfrac{4\cdot 3\cdot 2\cdot 1}{3\cdot 2\cdot 1\cdot 1} \pause
        = \dfrac{4\cdot \cancel{3}\cdot \cancel{2}\cdot 1}{\cancel{3}\cdot \cancel{2}\cdot 1\cdot 1} \pause
        = 4
      \end{aligned}
    \end{equation*}
  \end{example}
\end{frame}

\begin{frame}
  \begin{block}{Binomial Distribution}
    Suppose the probability of a single trial being a success is $p$. Then the probability of observing exactly $k$ successes in $n$ independent trials is given by
    \begin{equation*}
      \begin{aligned}
        \binom{n}{k}p^k{(1-p)}^{n-k} = \dfrac{n!}{k!(n-k)!} p^k {(1-p)}^{n-k}
      \end{aligned}
    \end{equation*}
    The mean, variance, and standard deviation of the number of observed successes are
    \begin{equation*}
      \mu = np, \qquad
      \sigma^2 = np(1-p), \qquad
      \sigma = \sqrt{np(1-p)}
    \end{equation*}
  \end{block}\pause
  
  \begin{block}{Is It Binomial?}
    Every binomial distribution has to satisfy the following:
    \begin{itemize}
    \item The trials are independent.\pause
    \item The number of trials, $n$, is fixed.\pause
    \item Each trial outcome can be classified as either a \emph{success} or \emph{failure}.\pause
    \item The probability of a success, $p$, is the same for each trial.
    \end{itemize}
  \end{block}
\end{frame}

\begin{frame}
  \begin{example}
    From Example~\ref{twitter} we have $p=0.85$, $n=4$, and $k=1$.\pause

    \vspace{-6mm}
    \begin{equation*}
      \begin{aligned}
        \prob{\text{exactly one has heard of twitter}}
        &= \binom{n}{k}p^k{(1-p)}^{n-k} \\\pause
        &= \binom{4}{1}{(0.85)}^{1}{(1-0.85)}^{4-1} \\\pause
        &= \dfrac{4!}{1!(4-1)!} {(0.85)}^{1} {(0.15)}^{3} \\\pause
        &= \dfrac{4 \cdot 3\cdot 2\cdot 1}{1\cdot 3\cdot 2\cdot 1} {(0.85)}^{1} {(0.15)}^{3} \\\pause
        &= \dfrac{4 \cdot \cancel{3}\cdot \cancel{2}\cdot 1}{1\cdot \cancel{3}\cdot \cancel{2}\cdot 1} {(0.85)}^{1} {(0.15)}^{3} \\\pause
        &= 4\cdot {(0.85)}^{1} {(0.15)}^{3} \\\pause
        &= 0.11475
      \end{aligned}
    \end{equation*}
  \end{example}
\end{frame}

\begin{frame}
  \begin{example}
    Assume that 70\% of customers won't exceed their car insurance deductible. Let's the probability that 5 of 8 randomly selected customers won't exceed their premium.\pause

    \vspace{1mm}
    Start by identifying

    \vspace{-3.5mm}
    \begin{equation*}
      \begin{aligned}
        p=0.7, \qquad
        q=1-p=0.3, \qquad
        n=8, \qquad
        k=5
      \end{aligned}
    \end{equation*}\pause

    \vspace{-11mm}
    \begin{equation*}
      \begin{aligned}
        \binom{n}{k}p^k{(1-p)}^{n-k}
        &= \binom{8}{5}{(0.7)}^{5}{(0.3)}^{8-5} \\\pause
        &= \dfrac{8!}{5!(8-5)!} {(0.7)}^{5}{(0.3)}^{3}\pause
        = \dfrac{8!}{5!(3)!} {(0.7)}^{5}{(0.3)}^{3} \\\pause
        &= \dfrac{8\cdot 7\cdot 6\cdot 5\cdot 4\cdot 3\cdot 2\cdot 1}{5\cdot 4\cdot 3\cdot 2\cdot 1\cdot 3\cdot 2\cdot 1} {(0.7)}^{5}{(0.3)}^{3} \\\pause
        &= \dfrac{8\cdot 7\cdot 6\cdot \cancel{5}\cdot \cancel{4}\cdot \cancel{3}\cdot \cancel{2}\cdot 1}{\cancel{5}\cdot \cancel{4}\cdot \cancel{3}\cdot \cancel{2}\cdot 1\cdot 3\cdot 2\cdot 1} {(0.7)}^{5}{(0.3)}^{3} \\\pause
        &= 56\cdot {(0.7)}^{5}{(0.3)}^{3} \\\pause
        &= 0.254122
      \end{aligned}
    \end{equation*}
  \end{example}
\end{frame}

\begin{frame}
  \begin{example}
    Assume the probability that a smoker will develop a severe lung condition in their life time is 0.3.

    \vspace{1mm}
    \question{If you have four friends who smoke, are the conditions for the binomial model satisfied?}\pause
    \answer{It is likely that independence is not satisfied, since they probably all know each other.}
  \end{example}\pause

  \begin{example}
    Suppose instead four people are randomly selected.

    \vspace{1mm}
    \question{Is the binomial model appropriate to find the probability that none of them will develop a severe lung condition?}
    \answer{We are assuming that the four are randomly selected, yes.

      \vspace{-5mm}
      \begin{equation*}
        \begin{aligned}
          \binom{n}{k}p^k{(1-p)}^{n-k}
          &= \binom{4}{0}{(0.3)}^0{(1-0.3)}^{4-0}
          = \dfrac{4!}{0!(4-0)!} {(0.3)}^0 {(0.7)}^{4} \\
          &= 1\cdot 1\cdot {(0.7)}^{4}
          = 0.2401
        \end{aligned}
      \end{equation*}
    }
    \vspace{-4mm}
  \end{example}
\end{frame}

\begin{frame}
  \begin{example}\label{smokers}
    \vspace{-2mm} % beamer bug
    Let consider finding the probability than no more than one of the four people will develop a severe lung condition.\pause

    \vspace{1mm}
    The events that \textquote{none of them develops a severe lung condition} and \textquote{exactly one develops a severe lung condition} are mutually exclusive.

      \vspace{-5mm}
      \begin{equation*}
        \begin{aligned}
          \prob{\text{none}} &+ \prob{\text{exactly one}} \\\pause
          & = \binom{4}{0}{(0.3)}^0{(1-0.3)}^{4-0} + \binom{4}{1}{(0.3)}^1{(1-0.3)}^{4-1} \\\pause
          &= \dfrac{4!}{0!(4-0)!} {(0.3)}^0 {(0.7)}^{4} + \dfrac{4!}{1!(4-1)!} {(0.3)}^1 {(0.7)}^{3} \\\pause
          &= \dfrac{4\cdot 3\cdot 2\cdot 1}{1\cdot 4\cdot 3\cdot 2\cdot 1} {(0.3)}^0 {(0.7)}^{4} + \dfrac{4\cdot 3\cdot 2\cdot 1}{1\cdot 3\cdot 2\cdot 1}  {(0.3)}^1 {(0.7)}^{3} \\\pause
          &= \dfrac{\cancel{4}\cdot \cancel{3}\cdot \cancel{2}\cdot 1}{1\cdot \cancel{4}\cdot \cancel{3}\cdot \cancel{2}\cdot 1} {(0.3)}^0 {(0.7)}^{4} + \dfrac{4\cdot \cancel{3}\cdot \cancel{2}\cdot 1}{1\cdot \cancel{3}\cdot \cancel{2}\cdot 1}  {(0.3)}^1 {(0.7)}^{3} \\\pause
          &= 0.2401 + 0.4116 \\\pause
          &= 0.6517 = 65.17\%
        \end{aligned}
      \end{equation*}
  \end{example}
\end{frame}

\begin{frame}
  \begin{example}
    Lets consider finding the probability that at least two of the four people will develop a severe lung condition.\pause

    \vspace{1mm}
    The complement of \textquote{at least two will develop a severe lung condition} is \textquote{no more than one will develop a severe lung condition.}\pause

    \vspace{1mm}
    We know from Example~\ref{smokers} that $\prob{\text{no more than one}} = 0.6517$.\pause

    \vspace{2mm}
    So,

    \vspace{-6mm}
    \begin{equation*}
      \begin{aligned}
        \prob{\text{at least two}}
        &= 1 - \prob{\text{no more than one}} \\\pause
        &= 1 - 0.6517 \pause
        = 0.3483 = 34.83\%
      \end{aligned}
    \end{equation*}
  \end{example}\pause

  \begin{example}
    \question{Out of seven randomly selected smokers, how many would we expect to develop a severe lung condition?}\pause
    \answer{The mean of the binomial model is

      \vspace{-4.3mm}
      \begin{equation*}
        \begin{aligned}
          \mu &= np \pause
          = 7\cdot 0.3 \pause
          = 2.1
        \end{aligned}
      \end{equation*}

      \vspace{-2mm}
      On average, we would expect 2.1 of 7 randomly chosen smokers to develop a severe lung condition.
      }
  \end{example}
\end{frame}
\end{document}
