\documentclass{beamer}
\usepackage[utf8]{inputenc}
\usepackage[english]{babel}
\usepackage{helvet}
\usepackage[T1]{fontenc}
\usepackage{textcomp}
\usepackage[inline]{asymptote}
\usepackage{asy_helper}
\usepackage{slide_helper}
\usepackage{multirow}
\usepackage{tikz}
\usepackage{subcaption}
\usepackage{fp}
\usepackage{cancel}
\usetikzlibrary{shapes.geometric, arrows}
\usepackage{pgfplots}
\pgfplotsset{compat=1.5} 
\usepgfplotslibrary{statistics}
\usetikzlibrary{external}
\tikzexternalize%

\begin{asydef}
guide normal_dist(real mu, real sigma, real xmin, real xmax)
{
	real nd_func(real x) { return 1/sqrt(2*pi*sigma*sigma)*exp((-1*(x-mu)*(x-mu))/(2*sigma*sigma)); }
	return graph(nd_func, xmin, xmax);
}

void shade_below(real mu, real sigma, real b, real xmin, real xmax, pen p=royalblue)
{
	real nd_func(real x) { return 1/sqrt(2*pi*sigma*sigma)*exp((-1*(x-mu)*(x-mu))/(2*sigma*sigma)); }
		
	guide g = graph(nd_func, xmin, b);
	
	filldraw(g -- (b,0) -- cycle, p, black);
	
	draw((xmin,0)--(b,0));
}

void shade_above(real mu, real sigma, real b, real xmin, real xmax, pen p=royalblue)
{
	real nd_func(real x) { return 1/sqrt(2*pi*sigma*sigma)*exp((-1*(x-mu)*(x-mu))/(2*sigma*sigma)); }
		
	guide g = graph(nd_func, b, xmax);
	
	filldraw(g -- (b,0) -- cycle, p, black);
	
	draw((xmax,0)--(b,0));
}

void shade_between(real mu, real sigma, real a, real b, pen p=royalblue)
{
	real nd_func(real x) { return 1/sqrt(2*pi*sigma*sigma)*exp((-1*(x-mu)*(x-mu))/(2*sigma*sigma)); }
		
	guide g = graph(nd_func, a, b);
	
	filldraw((a,0) -- g -- (b,0) -- cycle, p, black);
}

void multiple_nd_curves_example(real std_dev)
{
	size(300, 190, IgnoreAspect);
	
	draw(normal_dist(0, std_dev, -6,6));
	shade_between(0,std_dev,-std_dev,std_dev);
	draw((0,0)--(0,0.45));

	label("$\sigma="+format("%#.2f", std_dev)+"$", (-4.2,0.45), Fill(paleyellow));

	xaxis(Bottom(), RightTicks(new real[] {-6,-4,-2,0,2,4,6}));
	yaxis(Left(), LeftTicks(size=nan),ymin = 0, ymax = 0.5);
}

string axis_label(real x) 
{ 
	if (x == 0) 
		return "$\mu$"; 
	if (x == 1)
		return "$\sigma$";
	if (x == -1)
		return "$-\sigma$";
	return "$" + string(x) + "\sigma$";
}

real small_normal_size = 69;
\end{asydef}

\title[MA205 - Section 4.1]{Normal Distribution}

\newcommand{\prob}[1]{P\left({#1}\right)}
\newcommand{\jointprob}[3]{\prob{{#1}~\text{#2}~{#3}}}
\newcommand{\condprob}[2]{\prob{{#1}~|~{#2}}}
\newcommand{\comb}[2]{_{#1}C_{#2}}

\begin{document}
\begin{frame}
\titlepage
\end{frame}

%\begin{frame}
%\begin{example}
%\begin{columns}
%\begin{column}{0.4\linewidth}
%If a satellite crashes at a random point on the earth, what is the probability that it will crash on land?
%
%\vspace{2mm}
%\onslide<2->{%
%The surface of the earth is 
%
%\vspace{-6mm}
%\begin{center}
%\begin{tabular}{rr}
%Land: & 54,255,000 sq mi\\
%Water: & 142,715,000 sq mi
%\end{tabular}
%\end{center}
%}
%\onslide<3->{%
%So the probability of crashing on land is
%\begin{equation*}
%\dfrac{54,255,000}{196,970,000}=0.275
%\end{equation*}
%}
%\end{column}
%\begin{column}{0.58\linewidth}
%\begin{center}
%\onslide<1->{%
%\includegraphics[width=\linewidth]{earth.eps}
%}
%\end{center}
%\end{column}
%\end{columns}
%\end{example}
%\end{frame}

\begin{frame}
\begin{example}
\begin{columns}
\begin{column}{0.45\linewidth}
\includegraphics[width=\linewidth]{popcorn.eps}
\end{column}
\begin{column}{0.5\linewidth}
Consider making popcorn.\pause

\vspace{1mm}
You put some oil and corn kernels in a pan and start heating.\pause

\vspace{1mm}
For the first few minutes nothing happens, then a few kernels start to pop.\pause

\vspace{1mm}
A little while later more and more start to pop.\pause

\vspace{1mm}
This goes one for a minute or so, and the popping gradually tappers off.\pause

\vspace{1mm}
Most of the popping happens in that brief, noisy moment.\pause

\vspace{1mm}
This demonstrates a typical pattern that is part of many phenomena.
\end{column}
\end{columns}
\end{example}
\end{frame}

\begin{frame}[fragile]
\begin{definition}
A \textbf{normal distribution} is a perfectly symmetric, bell-shaped distribution. It is also referred to as a \textbf{normal curve} or a \textbf{bell curve}.
\begin{center}
\begin{asy}
size(320);

shade_below(0, 0.3, 0, -1.5, 1.5);
shade_above(0, 0.3, 0, -1.5, 1.5, yellow);
draw(normal_dist(0, 0.3, -1.5, 1.5));
draw((0,0)--(0,1.45));

xaxis();
//yaxis();

label(minipage("Mean Median Mode", 30),(.8,1.1));
draw((.8,1.25){up}..{left}(0.01,1.4),Arrow);

label("Lower 50\%", (-0.3,0.15));
label("Upper 50\%", (0.3,0.15));

draw((-1,-0.1)--(1,-0.1), Arrows);
label("Curves continues forever in both directions", (0, -0.2));

label(minipage("Curve never touches the $x$-axis", 45), (-1.2,0.25));
\end{asy}
\end{center}
\end{definition}
\end{frame}

\begin{frame}[fragile]
\begin{note}
The normal distribution with mean $\mu$ and standard distribution $\sigma$ is denoted $N(\mu, \sigma)$, where $\mu$ and $\sigma$ are called the parameters.
\end{note}\pause

\begin{example}
Both are normal distributions, but with different center and spread.
\begin{figure}
\centering
\begin{subfigure}[b]{0.45\linewidth}
\centering
\begin{asy}
size(150, 100, IgnoreAspect);

draw(normal_dist(0, 1, -4, 4));

xaxis(Bottom(), RightTicks(new real [] {-3,-2,-1,0,1,2,3}, size=nan));
\end{asy}
\caption{$N(\mu=0, \sigma=1)$}
\end{subfigure}
\begin{subfigure}[b]{0.45\linewidth}
\centering
\begin{asy}
size(150, 100, IgnoreAspect);

draw(normal_dist(19, 4, 3, 35));

xaxis(Bottom(), RightTicks(new real [] {7,11,15,19,23,27,31}, size=nan));
\end{asy}
\caption{$N(\mu=19, \sigma=4)$}
\end{subfigure}
\end{figure}
\end{example}
\end{frame}

\begin{frame}[fragile]
\begin{definition}
The graph of any continuous probability distribution is called a \textbf{density curve} if the total area under the curve is exactly 1.
\end{definition}\pause

\begin{note}
This means there is a correspondence between the area under a density curve and probabilities.
\end{note}\pause

\begin{definition}
The special case $N(\mu=0, \sigma=1)$ is called the \textbf{standard normal distribution}. The total area under the curve is exactly equal to 1.
\begin{center}
\begin{asy}
size(300, 75, IgnoreAspect);

draw((1,0)--(1,0.55), red);
draw((-1,0)--(-1,0.55), red);

draw(normal_dist(0, 1, -4, 4));
draw((0,0)--(0,0.5));

xaxis(Bottom(), RightTicks(size=nan));
yaxis(Left(), LeftTicks(size=nan),ymin = 0, ymax = 0.5);

label("$\mu+\sigma$", (1, 0.45), UnFill);
label("$\mu-\sigma$", (-1, 0.45), UnFill);
\end{asy}
\end{center}
\end{definition}
\end{frame}

\begin{frame}[fragile]
\begin{definition}
The \textbf{empirical rule}, or \textbf{68-95-99.7 rule}, states approximately how much of the area is contained when stepping one, two, or three standard deviations from the mean.

\vspace{0.5mm}
One standard deviation from the mean.
\begin{center}
\begin{asy}
size(300, 170, IgnoreAspect);

shade_between(0,1,-1,1);
draw(normal_dist(0, 1, -4, 4));
draw((0,0)--(0,0.45));

label("\Large\textbf{68\%}", (0,0.1), white);

xaxis(Bottom(), RightTicks(new real [] {-4,-3,-2,-1,0,1,2,3,4}, size=nan, ticklabel = axis_label));
\end{asy}
\end{center}
\end{definition}
\end{frame}

\begin{frame}[fragile]
\begin{definition}
The \textbf{empirical rule}, or \textbf{68-95-99.7 rule}, states approximately how much of the area is contained when stepping one, two, or three standard deviations from the mean.

\vspace{0.5mm}
Two standard deviations from the mean.
\begin{center}
\begin{asy}
size(300, 170, IgnoreAspect);

shade_between(0,1,-2,2);
draw(normal_dist(0, 1, -4, 4));
draw((0,0)--(0,0.45));

label("\Large\textbf{95\%}", (0,0.1), white);

xaxis(Bottom(), RightTicks(new real [] {-4,-3,-2,-1,0,1,2,3,4}, size=nan, ticklabel = axis_label));
\end{asy}
\end{center}
\end{definition}
\end{frame}

\begin{frame}[fragile]
\begin{definition}
The \textbf{empirical rule}, or \textbf{68-95-99.7 rule}, states approximately how much of the area is contained when stepping one, two, or three standard deviations from the mean.

\vspace{0.5mm}
Three standard deviations from the mean.
\begin{center}
\begin{asy}
size(300, 170, IgnoreAspect);

shade_between(0,1,-3,3);
draw(normal_dist(0, 1, -4, 4));
draw((0,0)--(0,0.45));

label("\Large\textbf{99.7\%}", (0,0.1), white);

xaxis(Bottom(), RightTicks(new real [] {-4,-3,-2,-1,0,1,2,3,4}, size=nan, ticklabel = axis_label));
\end{asy}
\end{center}
\end{definition}
\end{frame}

\begin{frame}
\begin{definition}
A \textbf{$\boldsymbol{z}$-score} is a measure of the number of standard deviations a particular data point is away from the mean.

\vspace{-2mm}
\begin{equation*}
z=\dfrac{\left(\text{data point}\right)-\left(\text{mean}\right)}{\text{standard deviation}}=\dfrac{x-\mu}{\sigma}
\end{equation*}
\end{definition}\pause

\begin{example}
On a college entrance exam, the mean was 70, and the standard deviation was 8. Rose scored a 85, what is her $z$-score?\pause

\vspace{-2mm}
\begin{equation*}
z=\dfrac{x-\mu}{\sigma}\pause=\dfrac{85-70}{8}\pause\approx 1.875
\end{equation*}

\vspace{-4.3mm}
\end{example}\pause

\vspace{-2mm}
\begin{example}
On the same exam, George has a $z$-score of $-1.3$. What was his score?\pause
\begin{equation*}
z=\dfrac{x-\mu}{\sigma}\pause
~~\Rightarrow~~
z\sigma=x-\mu\pause
~~\Rightarrow~~
x=z\sigma+\mu\pause = \left(-1.3\right)\left(8\right)+70=59.6
\end{equation*}

\vspace{-4.5mm}
\end{example}
\end{frame}

\begin{frame}
\begin{example}
The mean on a exam was 82, with a standard deviation of 7 points. An \textquote{A} on the exam is a 93, what is the $z$-score?\pause
\begin{equation*}
z=\dfrac{x-\mu}{\sigma}=\dfrac{93-82}{7}\approx 1.57
\end{equation*}
\end{example}\pause

\begin{note}
We know from the empirical rule that roughly 68\% of the scores fall within one standard deviation of the mean.\pause 

\vspace{1mm}
This means that 68\% of the students scored between 75 and 89.\pause

\vspace{1mm}
Moreover, we know that roughly 95\% of the scores fall within two standard deviations of the mean. \pause

\vspace{1mm}
Which means that $95\%-68\%=27\%$ of the scores are more than one standard deviation from the mean, but less than two. \pause

\vspace{1mm}
Since the curve is symmetric, we know that 13.5\% of the students scored between 89 and 96, as well as 13.5\% between 68 and 75
\end{note}
\end{frame}

\begin{frame}[fragile]
\begin{definition}
The \textbf{empirical rule}, or \textbf{68-95-99.7 rule}, states approximately how much of the area is contained when stepping one, two, or three standard deviations from the mean.

\vspace{0.5mm}
For each standard deviation.
\begin{center}
\begin{asy}
size(300, 170, IgnoreAspect);


shade_between(0, 1,-4, -3, white);
shade_between(0, 1,-3, -2, green);
shade_between(0, 1,-2, -1, orange);
shade_between(0, 1,-1,  0, royalblue);
shade_between(0, 1, 0,  1, yellow);
shade_between(0, 1, 1,  2, purple);
shade_between(0, 1, 2,  3, mediumcyan);
shade_between(0, 1, 3,  4, white);

draw((0,0)--(0,0.45));

real nd_func(real x, real mu, real sigma) { return 1/sqrt(2*pi*sigma*sigma)*exp((-1*(x-mu)*(x-mu))/(2*sigma*sigma)); }

void sndlabel(string s, pair pt, real offset=0.01)
{
	pair c = (pt.x, nd_func(pt.x, 0, 1) + offset);
	
	draw(pt -- c, EndArrow);
	label(s , pt, UnFill());
}

sndlabel("34\%",   (-0.5, 0.43));
sndlabel("34\%",   ( 0.5, 0.43));
sndlabel("13.5\%",   (-1.5, 0.25));
sndlabel("13.5\%",   ( 1.5, 0.25));
sndlabel("2.35\%", (-2.5, 0.15));
sndlabel("2.35\%", ( 2.5, 0.15));
sndlabel("0.15\%",  (-3.5, 0.05));
sndlabel("0.15\%",  ( 3.5, 0.05));

xaxis(Bottom(), RightTicks(new real [] {-4,-3,-2,-1,0,1,2,3,4}, size=nan, ticklabel = axis_label));
\end{asy}
\end{center}
\end{definition}
\end{frame}

\begin{frame}[fragile]
\begin{definition}
An \textbf{inflection point} is where a curve changes from being concave up to concave down, or vice versa
\end{definition}\pause

\begin{note}
A normal density curve always has two inflection points, each one standard deviation from the mean.
\begin{center}
\begin{asy}
size(300, 140, IgnoreAspect);

real nd_func(real x, real mu, real sigma) { return 1/sqrt(2*pi*sigma*sigma)*exp((-1*(x-mu)*(x-mu))/(2*sigma*sigma)); }

draw(normal_dist(0, 1, -5, 5));
draw((0,0)--(0,nd_func(0,0,1)));

draw((1,0) -- (1,0.45), red);
draw((0,nd_func(1,0,1)) -- (1,nd_func(1,0,1)), red);
dot((1,nd_func(1,0,1)), red);

label("Inflection point", (1,nd_func(1,0,1)), E);

draw((-1,0) -- (-1,0.45), red);
draw((0,nd_func(-1,0,1)) -- (-1,nd_func(-1,0,1)), red);
dot((-1,nd_func(-1,0,1)), red);

label("Inflection point", (-1,nd_func(-1,0,1)), W);

label("Concave up", (2,nd_func(2,0,1)), E);
label("Concave up", (-2,nd_func(-2,0,1)), W);

label("Concave down", (0,nd_func(0,0,1)), N);

xaxis(Bottom(), RightTicks(new real [] {-4,-3,-2,-1,0,1,2,3,4}, size=nan, ticklabel = axis_label));
\end{asy}
\end{center}
\end{note}
\end{frame}

\begin{frame}[fragile]
\begin{example}
\begin{center}
\begin{asy}
size(300, 50, IgnoreAspect);

draw(normal_dist(0, 1, -4, 4));
shade_between(0,1,-4,0.8);
//draw((0,0)--(0,0.5));

xaxis(Bottom(), RightTicks(new real[] {0,0.8}, ticklabel=new string(real x) { if (x==0) return "0"; else return "$x$";}, size=nan));
//yaxis(Left(), LeftTicks(size=nan),ymin = 0, ymax = 0.5);
\end{asy}
\end{center}
The area of the shaded region is the probability that a $z$ score is less than or equal to $x$, $\prob{z\leq x}$.
\end{example}\pause

\begin{note}
Most statistical software, programming languages, spreadsheets programs, and calculators are able to calculate the area for you.
\end{note}
\end{frame}

\begin{frame}[fragile]
\begin{example}
\begin{center}
\begin{asy}
size(300, 50, IgnoreAspect);

draw(normal_dist(0, 1, -4, 4));
shade_between(0,1,-0.8,4);
//draw((0,0)--(0,0.5));

xaxis(Bottom(), RightTicks(new real[] {0,-0.8}, ticklabel=new string(real x) { if (x==0) return "0"; else return "$x$";}, size=nan));
//yaxis(Left(), LeftTicks(size=nan),ymin = 0, ymax = 0.5);
\end{asy}
\end{center}
The area of the shaded region is the probability that a $z$ score is greater than or equal to $x$, $\prob{z\geq x}$.
\end{example}\pause

\begin{note}
\begin{center}
\setlength\tabcolsep{1pt}
\begin{tabular}{ccccccc}
\begin{asy}
size(small_normal_size);

real mu = 0;
real sigma = 0.5;
real x = mu - 0.8*sigma;
draw(normal_dist(mu, sigma, -1.8, 1.8));
shade_above(mu,sigma,x, -1.8, 1.8);

xaxis(Bottom(), NoTicks);
\end{asy}
&$=$&
\begin{asy}
size(small_normal_size);

real mu = 0;
real sigma = 0.5;
draw(normal_dist(mu, sigma, -1.8, 1.8));
shade_between(mu,sigma,-1.8,1.8);

xaxis(Bottom(), NoTicks);
\end{asy}
&$-$&
\begin{asy}
size(small_normal_size);

real mu = 0;
real sigma = 0.5;
real x = mu - 0.8*sigma;
draw(normal_dist(mu, sigma, -1.8, 1.8));
shade_below(mu,sigma,x,-1.8,1.8);

xaxis(Bottom(), NoTicks);
\end{asy}
\\
$\prob{z\geq x}$ & & $1$ & & $\prob{z\leq x}$
\end{tabular}
\end{center}
\end{note}
\end{frame}

\begin{frame}[fragile]
\begin{example}
\begin{center}
\begin{asy}
size(300, 50, IgnoreAspect);

draw(normal_dist(0, 1, -4, 4));
shade_between(0,1,-0.7,1.2);
//draw((0,0)--(0,0.5));

xaxis(Bottom(), RightTicks(new real[] {0,-0.7, 1.2}, ticklabel=new string(real x) { if (x==0) return "0"; else if (x < 0) return "$x_1$"; else return "$x_2$";}, size=nan));
//yaxis(Left(), LeftTicks(size=nan),ymin = 0, ymax = 0.5);
\end{asy}
\end{center}
The area of the shaded region is the probability that a $z$ score lies between $x_1$ and $x_2$, $\prob{x_1\leq z\leq x_2}$. 
\end{example}\pause

\begin{note}
\begin{center}
\setlength\tabcolsep{1pt}
\begin{tabular}{ccccc}
\begin{asy}
size(small_normal_size);

real mu = 0;
real sigma = 0.5;
real x1 = mu - 0.7*sigma;
real x2 = mu + 1.2*sigma;
draw(normal_dist(mu, sigma, -1.8, 1.8));
shade_between(mu,sigma,x1,x2);

xaxis(Bottom(), NoTicks);
\end{asy}
&$=$&
\begin{asy}
size(small_normal_size);

real mu = 0;
real sigma = 0.5;
real x1 = mu - 0.7*sigma;
real x2 = mu + 1.2*sigma;
draw(normal_dist(mu, sigma, -1.8, 1.8));
shade_below(mu,sigma,x2,-1.8,1.8);

xaxis(Bottom(), NoTicks);
\end{asy}
&$-$&
\begin{asy}
size(small_normal_size);

real mu = 0;
real sigma = 0.5;
real x1 = mu - 0.7*sigma;
real x2 = mu + 1.2*sigma;
draw(normal_dist(mu, sigma, -1.8, 1.8));
shade_below(mu,sigma,x1,-1.8,1.8);

xaxis(Bottom(), NoTicks);
\end{asy}
\\
$\prob{x_1\leq z\leq x_2}$ & & $\prob{z\leq x_2}$ & & $\prob{z\leq x_1}$
\end{tabular}
\end{center}
\end{note}\pause

\begin{note}
\begin{center}
\setlength\tabcolsep{1pt}
\begin{tabular}{ccccccc}
\begin{asy}
size(small_normal_size);

real mu = 0;
real sigma = 0.5;
real x1 = mu - 0.7*sigma;
real x2 = mu + 1.2*sigma;
draw(normal_dist(mu, sigma, -1.8, 1.8));
shade_between(mu,sigma,x1,x2);

xaxis(Bottom(), NoTicks);
\end{asy}
&$=$&
\begin{asy}
size(small_normal_size);

real mu = 0;
real sigma = 0.5;
real x1 = mu - 0.7*sigma;
real x2 = mu + 1.2*sigma;
draw(normal_dist(mu, sigma, -1.8, 1.8));
shade_between(mu,sigma,-1.8,1.8);

xaxis(Bottom(), NoTicks);
\end{asy}
&$-$&
\begin{asy}
size(small_normal_size);

real mu = 0;
real sigma = 0.5;
real x1 = mu - 0.7*sigma;
real x2 = mu + 1.2*sigma;
draw(normal_dist(mu, sigma, -1.8, 1.8));
shade_below(mu,sigma,x1,-1.8,1.8);

xaxis(Bottom(), NoTicks);
\end{asy}
&$-$&
\begin{asy}
size(small_normal_size);

real mu = 0;
real sigma = 0.5;
real x1 = mu - 0.7*sigma;
real x2 = mu + 1.2*sigma;
draw(normal_dist(mu, sigma, -1.8, 1.8));
shade_above(mu,sigma,x2,-1.8,1.8);

xaxis(Bottom(), NoTicks);
\end{asy}
\\
$\prob{x_1\leq z\leq x_2}$ & & $1$ & & $\prob{z\leq x_1}$ & & $\prob{z\geq x_2}$
\end{tabular}
\end{center}
\end{note}
\end{frame}

\begin{frame}[fragile]
\begin{block}{Procedure for Finding Areas with a Nonstandard Normal Distribution}
\begin{enumerate}
\item Sketch a normal curve, label the mean and any specific $x$ values, and then shade the region representing the desired probability.
\item \phantom{For each relevant $x$ value that is a boundary for the shaded region, convert that value to the equivalent $z$ score.}
\item \phantom{Use technology to find the area of the shaded region.}
\end{enumerate}
\begin{center}
\begin{asy}
size(300, 130, IgnoreAspect);

shade_between(0,1,-0.7,1.3);
draw(normal_dist(0, 1, -4.5, 4.5));

label("$x_1$",(-0.7,0),NW);
dot((-0.7,0));
label("$x_2$",(0.4,0),NE);
dot((0.4,0));
label("$x_3$",(1.3,0),NE);
dot((1.3,0));
label("$x_4$",(2.3,0),NW);
dot((2.3,0));

xaxis("$x$", Bottom(), RightTicks(new real [] {-4,-3,-2,-1,0,1,2,3,4}, size=nan, ticklabel = axis_label));
\end{asy}
\end{center}
\end{block}
\end{frame}

\begin{frame}[fragile]
\begin{block}{Procedure for Finding Areas with a Nonstandard Normal Distribution}
\begin{enumerate}
\item Sketch a normal curve, label the mean and any specific $x$ values, and then shade the region representing the desired probability.
\item For each relevant $x$ value that is a boundary for the shaded region, convert that value to the equivalent $z$ score.
\item \phantom{Use technology to find the area of the shaded region.}
\end{enumerate}
\begin{center}
\begin{asy}
size(300, 130, IgnoreAspect);

shade_between(0,1,-0.7,1.3);
draw(normal_dist(0, 1, -4.5, 4.5));

label("$z_1$",(-0.7,0),NW,blue);
dot((-0.7,0));
label("$z_3$",(1.3,0),NE,blue);
dot((1.3,0));


xaxis("$z$",Bottom(), RightTicks(new real [] {-4,-3,-2,-1,0,1,2,3,4}, size=nan));
\end{asy}
\end{center}
\end{block}
\end{frame}

\begin{frame}[fragile]
\begin{block}{Procedure for Finding Areas with a Nonstandard Normal Distribution}
\begin{enumerate}
\item Sketch a normal curve, label the mean and any specific $x$ values, and then shade the region representing the desired probability.
\item For each relevant $x$ value that is a boundary for the shaded region, convert that value to the equivalent $z$ score.
\item Use technology to find the area of the shaded region.
\end{enumerate}
\begin{center}
\begin{asy}
size(300, 130, IgnoreAspect);

shade_between(0,1,-0.7,1.3);
draw(normal_dist(0, 1, -4.5, 4.5));

label("$z_1$",(-0.7,0),NW,blue);
dot((-0.7,0));
label("$z_3$",(1.3,0),NE,blue);
dot((1.3,0));

real nd_func(real x, real mu, real sigma) { return 1/sqrt(2*pi*sigma*sigma)*exp((-1*(x-mu)*(x-mu))/(2*sigma*sigma)); }

label("$P(z_1\leq z\leq z_2)$",(0.3,0.4*nd_func(0,0,1)));

xaxis("$z$",Bottom(), RightTicks(new real [] {-4,-3,-2,-1,0,1,2,3,4}, size=nan));
\end{asy}
\end{center}
\end{block}
\end{frame}

\begin{frame}[fragile]
\begin{example}
The heights of men are normally distributed with a mean of 68.6 in\@. and a standard deviation of 2.8 in.\pause

\vspace{1mm}
Let's find the percentage of men who are taller than a shower head installed at a height of 72 in\@.\pause

\vspace{1mm}
We start by finding the $z$-value of the shower head.

\vspace{-2mm}
\begin{equation*}
z=\dfrac{x-\mu}{\sigma}\pause = \dfrac{72-68.6}{2.8}\pause = 1.21
\end{equation*}\pause

\vspace{-4mm}
Next, we sketch a picture and shade the area we wish to find:

\vspace{-2mm}
\begin{center}
\begin{asy}
size(225, 60, IgnoreAspect);

real xmin=-4; real xmax=4;

shade_between(0,1,1.21,xmax);
draw(normal_dist(0, 1, xmin, xmax));

label("$1.21$",(1.21,0),NW);
dot((1.21,0));

xaxis("$z$",Bottom(), RightTicks(new real [] {-3,-2,-1,0,1,2,3}, size=nan));
\end{asy}
\end{center}\pause

\vspace{-3.5mm}
We can then use technology to compute:

\vspace{-3mm}
\begin{equation*}
\prob{z\geq 1.21} \approx 0.1123\quad\text{(rounded)}
\end{equation*}\pause

\vspace{-6mm}
So, about 11.23\% of men are taller than the shower head.
\end{example}
\end{frame}

\begin{frame}[fragile]
\begin{example}
Cumulative SAT scores are approximately normal with a mean of 1100 and standard distribution of 200. Edward earned a 1030 on his SAT\@.\pause

\vspace{1mm}
The $z$-value of his score is:

\vspace{-2mm}
\begin{equation*}
z=\dfrac{x-\mu}{\sigma}\pause = \dfrac{1030-1100}{200}\pause = -0.35
\end{equation*}\pause

\vspace{-4mm}
Recall that the percentile of a data value is the percentage of data less than the data value.\pause 

\vspace{-2mm}
\begin{center}
\begin{asy}
size(225, 60, IgnoreAspect);

real xmin=-4; real xmax=4;

shade_below(0,1,-0.35,xmin,xmax);
draw(normal_dist(0, 1, xmin, xmax));

label("$-0.35$",(-0.35,0),NE);
dot((-0.35,0));

xaxis("$z$",Bottom(), RightTicks(new real [] {-3,-2,-1,0,1,2,3}, size=nan));
\end{asy}
\end{center}

\vspace{-3.5mm}
This means $\prob{z\leq -0.35}$ is the percentile of Edwards SAT score.\pause

\vspace{1mm}
We can then use technology to compute:

\vspace{-3mm}
\begin{equation*}
\prob{z\leq -0.35} \approx 0.3632\quad\text{(rounded)}
\end{equation*}\pause

\vspace{-6mm}
So, Edward is in the ${36}^{\text{th}}$ percentile.
\end{example}
\end{frame}

\begin{frame}[fragile]
\begin{example}\label{airforce_mens_example} 
\vspace{-2mm}
The U.S.\@ Air Force requires that pilots have heights between 64 and 77 in.\@ The heights of men are normally distributed with a mean of 68.6 in.\@ and a standard deviation of 2.8 in.\@

\vspace{1mm}
Let's find the percentage of men meet that requirement.\pause

\vspace{1mm}
We start by finding the $z$-values of the height requirements.

\vspace{-5mm}
\begin{equation*}
z_1=\dfrac{x-\mu}{\sigma} = \dfrac{64-68.6}{2.8} = -1.64\pause
~~\text{and}~~
z_2=\dfrac{x-\mu}{\sigma} = \dfrac{77-68.6}{2.8} = 3.00
\end{equation*}\pause

\vspace{-5mm}
Next, we sketch a picture and shade the area we wish to find:

\vspace{-2mm}

\begin{center}
\begin{asy}
size(225, 60, IgnoreAspect);

real xmin=-3.34483; real xmax=5.27586;

shade_between(0,1,-1.64,3);
draw(normal_dist(0, 1, xmin, xmax));

label("$-1.64$",(-1.64,0),NE);
dot((-1.64,0));

label("$3.00$",(3,0),N);
dot((3,0));

xaxis("$z$",Bottom(), RightTicks(new real [] {-3,-2,-1,0,1,2,3,4,5}, size=nan));
\end{asy}
\end{center}\pause

\vspace{-5mm}
We can then use technology to compute:

\vspace{-3mm}
\begin{equation*}
\prob{-1.64\leq z\leq 3.00} \approx 0.9484\quad\text{(rounded)}
\end{equation*}\pause

\vspace{-6mm}
So, we see that about 94.84\% of men meet the requirements.
\end{example}
\end{frame}

\begin{frame}[fragile]
\begin{example}\label{airforce_womens_example} 
\vspace{-2mm}
The U.S.\@ Air Force requires that pilots have heights between 64 and 77 in.\@ The heights of women are normally distributed with a mean of 63.7 in.\@ and a standard deviation of 2.9 in.\@

\vspace{1mm}
Let's find the percentage of women meet that requirement.\pause

\vspace{1mm}
We start by finding the $z$-values of the height requirements.

\vspace{-5mm}
\begin{equation*}
z_1=\dfrac{x-\mu}{\sigma} = \dfrac{64-63.7}{2.9} = 0.10\pause
~~\text{and}~~
z_2=\dfrac{x-\mu}{\sigma} = \dfrac{77-63.7}{2.9} = 4.59
\end{equation*}\pause

\vspace{-5mm}
Next, we sketch a picture and shade the area we wish to find:

\vspace{-2mm}

\begin{center}
\begin{asy}
size(225, 60, IgnoreAspect);

real xmin=-3.34483; real xmax=5.27586;

shade_between(0,1,0.1,4.59);
draw(normal_dist(0, 1, xmin, xmax));

label("$0.10$",(0.1,0),NW);
dot((0.1,0));

label("$4.59$",(4.59,0),NW);
dot((4.59,0));

xaxis("$z$",Bottom(), RightTicks(new real [] {-3,-2,-1,0,1,2,3,4,5}, size=nan));
\end{asy}
\end{center}\pause

\vspace{-5mm}
We can then use technology to compute:

\vspace{-3mm}
\begin{equation*}
\prob{0.10\leq z\leq 4.59} \approx 0.4601\quad\text{(rounded)}
\end{equation*}\pause

\vspace{-6mm}
So, we see that only about 46\% of women meet the requirements.
\end{example}
\end{frame}

\begin{frame}
\begin{block}{When Finding Values from Known Areas}
\begin{itemize}[<+- | alert@+>]
\item Draw a sketch of the graph.
\item Don't confuse $z$ scores and areas.
\item Choose the correct side of the graph.
\item A $z$ score must be negative whenever it is located in the left half of the normal distribution.
\item Areas are always between 0 and 1, and are never negative.
\end{itemize}
\end{block}

\onslide<+->
\begin{block}{Procedure}
\begin{enumerate}[<+- | alert@+>]
\item<.-> Sketch the normal distribution curve, write the given probability or percentage in the appropriate region of the graph, and identify the $x$ values being sought.
\item Either use technology or a table to identify the $z$ scores corresponding to that area.
\item Convert to $x$ values: $x=\mu + z\cdot\sigma$
\item Use your sketch to verify that the solution makes sense.
\end{enumerate}
\end{block}
\end{frame}

\begin{frame}[fragile]
\begin{example}\label{airforce_womens_lower_example}
\vspace{-2mm}
When designing equipment, one common criterion is to use a design that accommodates 95\% of the population. In Example~\ref{airforce_womens_example} we saw that only 46\% of women satisfy the U.S.\@ Air Force pilot height requirement. 

\onslide<2->
\vspace{1mm}
What would be the maximum acceptable height of a woman if the requirements were changed to allow the shortest 95\% of women?

\vspace{1mm}
\begin{overprint}
\onslide<3 | handout:0>
\begin{center}
\begin{asy}
size(300, 80, IgnoreAspect);

real xmin=-3.5; real xmax=3.5;

real nd_func(real x, real sigma, real mu) { return 1/sqrt(2*pi*sigma*sigma)*exp((-1*(x-mu)*(x-mu))/(2*sigma*sigma)); }
shade_between(0,1,xmin,1.645);
draw(normal_dist(0, 1, xmin, xmax));

label("$95\%$", (0,0.5*nd_func(0,1,0)));

label("$z=??$",(1.645,0),NW);
dot((1.645,0));

xaxis("$z$",Bottom(), RightTicks(new real [] {-3,-2,-1,0,1,2,3}, size=nan));
\end{asy}
\end{center}
\onslide<4->
\begin{center}
\begin{asy}
size(300, 80, IgnoreAspect);

real xmin=-3.5; real xmax=3.5;

real nd_func(real x, real sigma, real mu) { return 1/sqrt(2*pi*sigma*sigma)*exp((-1*(x-mu)*(x-mu))/(2*sigma*sigma)); }

shade_between(0,1,xmin,1.645);
draw(normal_dist(0, 1, xmin, xmax));

label("$95\%$", (0,0.5*nd_func(0,1,0)));

label("$z=1.645$",(1.645,0),NW);
dot((1.645,0));

xaxis("$z$",Bottom(), RightTicks(new real [] {-3,-2,-1,0,1,2,3}, size=nan));
\end{asy}
\end{center}
\end{overprint}
\onslide<4->
\vspace{0mm}
Using either technology or a table, we find that $z=1.645$.

\onslide<5->
\vspace{1mm}
We then need to convert to the $x$ value.
\vspace{-2.75mm}
\begin{equation*}
x = \mu + z\cdot\sigma
\onslide<6->
= 63.7 + 1.645\cdot 2.9 
\onslide<7->
= 68.4705
\end{equation*}

\onslide<8->
\vspace{-7.5mm}
A requirement of a height less than or equal to 68.5 in.\@ would allow 95\% of women to be eligible.
\end{example}
\end{frame}
\end{document}
