\documentclass{beamer}
\usepackage[utf8]{inputenc}
\usepackage[english]{babel}
\usepackage{helvet}
\usepackage[T1]{fontenc}
\usepackage{textcomp}
\usepackage[inline]{asymptote}
\usepackage{slide_helper}
\usepackage{tikz}
\usetikzlibrary{shapes.geometric, arrows}
\usepackage{pgfplots}
\pgfplotsset{compat=1.5} 
\usepgfplotslibrary{statistics}
\usetikzlibrary{external}
\tikzexternalize%

\title[MA205 - Section 1.4]{Experiments}

\begin{document}
\begin{frame}
\titlepage
\end{frame}

\begin{frame}
\begin{definition}
In an \textbf{experiment}, we apply some treatment and then proceed to observe its effects on the individuals.
\end{definition}\pause

\begin{definition}
The individuals in an experiment are called \textbf{subjects}.
\end{definition}\pause

\begin{definition}
A \textbf{Placebo} is a treatment that has no medicinal effect. (Such as a sugar pill or saline injection.)
\end{definition}\pause

\begin{definition}
The group that receives a placebo is called the \textbf{control group}.
\end{definition}\pause

\begin{definition}
The group that receives a treatment is called the \textbf{treatment group}.
\end{definition}
\end{frame}

\begin{frame}
\begin{definition}
As researchers assign treatments to cases, they do their best to \textbf{control} any differences in the groups.
\end{definition}\pause

\begin{example}
Consider an experiment that wants to test a new drug in pill form.\pause

\vspace{2mm}
Some patients may take their pills with only a sip of water and some may take their pills with an entire glass.\pause

\vspace{2mm}
To control for the effect of water consumption, the researchers ask all subjects take the treatment pill with a 12oz glass of water.\pause

\vspace{2mm}
The treatment drug may have a different reaction to full versus empty stomachs.\pause

\vspace{2mm}
To control for food consumption, researchers ask all subjects to take the treatment pill immediately after a meal.
\end{example}
\end{frame}

\begin{frame}
\begin{definition}
A \textbf{negative control} is a control group that receives a treatment that is expected to not have a noticeable effect.
\end{definition}\pause

\begin{note}
Placebos like sugar pills and saline injections are common examples.
\end{note}\pause

\begin{example}\label{caffeine}
Researchers want to study the effects of caffeine on blood pressure.

\vspace{2mm}
The researchers split their sample into four groups:\pause
\begin{description}
\item[Treatment Group:] This group drinks two cups of coffee.\pause
\item[Negative Control:] This group drinks two cups of decaf coffee.\pause
\item[Negative Control:] This group drinks two cups of water.\pause
\item[Negative Control:] This group drinks nothing.\pause
\end{description}

\vspace{1mm}
Afterwards the blood pressure of each subject is measured.
\end{example}
\end{frame}

\begin{frame}
\begin{definition}
A \textbf{positive control} is a sample group that receives a known treatment and is expected to change the subjects in a predictable way.
\end{definition}\pause

\begin{example}
It's unlikely, but the blood pressure machine used in Example~\ref{caffeine} may be faulty or miscalibrated.\pause

\vspace{2mm}
The researchers could add a fifth group who drinks nothing, but their blood pressure is measured using a completely independent second machine.
\end{example}\pause

\begin{example}
A researcher swabs an existing colony of bacteria and wipes it on a growth plate.
\end{example}
\end{frame}

\begin{frame}
\begin{example}
An experiment of a new acne treatment randomly assigns 300 patients into the following groups:
\begin{description}
\item[Treatment Group:] Receives the treatment being tested.
\item[Negative Control:] Receives a placebo treatment.
\item[Positive Control:] Receives a commercially available medication.
\end{description}
\end{example}\pause

\begin{note}
The negative control is used to show that any positive effects of the new treatment aren't caused by some confounding variable.

\vspace{2mm}
The positive control is used to detect any problems with the new treatment or how it is administered.
\end{note}\pause

\begin{note}
A positive control can also be used to benchmark the results of the new treatment against existing treatments.
\end{note}
\end{frame}

\begin{frame}
\begin{note}
Some times researchers will suspect that variables, other than the treatment, will influence the result.
\end{note}\pause

\begin{definition}
Researchers can first group individuals based on a suspected confounding variable into \textbf{blocks} and then randomize the cases within each block to the treatment groups. This is called \textbf{blocking}.
\end{definition}\pause

\begin{example}
If researchers are looking into the effect of a drug on heart attack patients, they might split all the patients into high-risk and low-risk blocks. Then half of each block is assigned to the treatment group and half to the control group.
\end{example}
\end{frame}

\newcommand{\low}[1]{{\color{blue}#1}}
\newcommand{\high}[1]{{\color{red}#1}}

\newcommand{\unsorted}[0]{
\begin{tabular}{|cccccccccccccc|}\hline
\low{1} & \low{2} & \low{3} & \high{4} & \low{5} & \high{6} & \low{7} & \high{8} & \high{9} & \high{10} & \low{11} & \low{12} & \low{13} & \low{14} \\
\low{15} & \low{16} & \low{17} & \low{18} & \high{19} & \low{20} & \high{21} & \high{22} & \high{23} & \low{24} & \low{25} & \low{26} & \low{27} & \high{28} \\
\high{29} & \high{30} & \low{31} & \high{32} & \high{33} & \low{34} & \high{35} & \low{36} & \high{37} & \high{38} & \low{39} & \high{40} & \low{41} & \low{42} \\\hline
\end{tabular}
} %chktex 31
\newcommand{\highrisk}[0]{
\begin{tabular}{|ccccc|}
\multicolumn{5}{c}{High-risk patients}\\\hline
\high{4} & \high{6} & \high{8} & \high{9} & \high{10} \\
\high{19} & \high{21} & \high{22} & \high{23} & \high{28} \\
\high{29} & \high{30} & \high{32} & \high{33} & \high{35} \\
\high{37} & \high{38} & \high{40} & & \\\hline
\end{tabular}
} %chktex 31
\newcommand{\lowrisk}[0]{
\begin{tabular}{|ccccccc|}
\multicolumn{7}{c}{Low-risk patients}\\\hline
\low{1} & \low{2} & \low{3} & \low{5} & \low{7} & \low{11} & \low{12} \\
\low{13} & \low{14} & \low{15} & \low{16} & \low{17} & \low{18} & \low{20} \\
\low{24} & \low{25} & \low{26} & \low{27} & \low{31} & \low{34} & \low{36} \\
\low{39} & \low{41} & \low{42} & & & & \\\hline
\end{tabular}
} %chktex 31
\newcommand{\randhighcontrol}[0]{
\begin{tabular}{|cccccc|}\hline
\high{4} & \high{6} & \high{19} & \high{28} & \high{30} & \high{32} \\
\high{35} & \high{38} & \high{40} & & & \\\hline
\end{tabular}
} %chktex 31
\newcommand{\randlowcontrol}[0]{
\begin{tabular}{|cccccc|}\hline
\low{2} & \low{5} & \low{7} & \low{12} & \low{13} & \low{17} \\
\low{18} & \low{20} & \low{25} & \low{36} & \low{39} & \low{42} \\\hline
\end{tabular}
} %chktex 31
\newcommand{\randhightreatment}[0]{
\begin{tabular}{|cccccc|}\hline
\high{8} & \high{9} & \high{10} & \high{21} & \high{22} & \high{23} \\
\high{29} & \high{33} & \high{37} & & & \\\hline
\end{tabular}
} %chktex 31
\newcommand{\randlowtreatment}[0]{
\begin{tabular}{|cccccc|}\hline
\low{1} & \low{3} & \low{11} & \low{14} & \low{15} & \low{16} \\
\low{24} & \low{26} & \low{27} & \low{31} & \low{34} & \low{41} \\\hline
\end{tabular}
} %chktex 31


\newcommand{\control}[0]{\begin{tabular}{cc}
\multicolumn{2}{l}{Control group} \\\hline
\vspace{-3mm}\\
\randlowcontrol &
\randhighcontrol
\end{tabular}}% chktex 31

\newcommand{\treatment}[0]{\begin{tabular}{cc}
\multicolumn{2}{l}{Treatment group} \\\hline
\vspace{-3mm}\\
\randlowtreatment &
\randhightreatment
\end{tabular}}% chktex 31

\newcommand{\slidea}[0]{\unsorted} %chktex 31
\newcommand{\slideb}[0]{$\swarrow$ \hspace{10mm}Split into blocks\hspace{10mm} $\searrow$ \\
\begin{tabular}{cc}
\lowrisk&\highrisk
\end{tabular}} %chktex 31

\newcommand{\slidec}[0]{$\downarrow$\hspace{5mm}Randomly split each block in half\hspace{5mm}$\downarrow$\\
\control%

\treatment} %chktex 31

\begin{frame}
\begin{example}
\begin{center}
\slidea\pause

\slideb\pause

\slidec%
\end{center}
\end{example}
\end{frame}

\begin{frame}
\begin{block}{Experimental Design}
A good experiment is built on four principles.

\begin{tabular}{rl}
\textbf{Controlling} & Researchers do their best to control for\\
& differences in the treatment and control groups. \\
\textbf{Randomization} & Sampling and assignment into treatment \\
& and/or control groups are done randomly. \\
\textbf{Replication} & A sufficiently large sample is used. \\
\textbf{Blocking} & Researchers suspect that variables other than \\
& the treatment may influence the response.
\end{tabular}
\end{block}\pause

\begin{note}
While blocking is a slightly more advanced topic, the statistical methods we discuss in this course can be extended to analyze such experiments.
\end{note}
\end{frame}

\begin{frame}
\begin{note}
Ethics in human experimentation is a very complicated topic, and there are multiple viewpoints on the use of placebos.
\end{note}\pause

\begin{example}
Suppose researchers want to test the effectiveness of a new treatment for cervical cancer. They decided to use a control group that receives no treatment.

\vspace{1mm}
\question{Is this ethical?}\pause
\answer{No, there are existing, effective treatments for cervical cancer. It is unethical to withhold all treatment from a patient.}
\end{example}\pause

\begin{note}
If there is no known effective treatment, then having a control group that receives no treatment may be ethical.
\end{note}\pause
\end{frame}

\begin{frame}
\begin{example}
If researchers want to study the effectiveness of a new surgical procedure, they may need to perform a sham surgery on the control group patients.\pause

\vspace{1mm}
\question{Is it ethical to use a sham surgery?}\pause
\answer{It's complicated. Surgery always carries the risk of infection and complication. But at the same time, you don't want to promote a new, costly surgery if it doesn't have proven benefit for the patient.}
\end{example}\pause

\begin{note}
In practice, research groups are responsible to review boards which must weigh the ethical concerns of an experiment before any patients are treated. 
\end{note}
\end{frame}
\end{document}
