\documentclass{beamer}
\usepackage[utf8]{inputenc}
\usepackage[english]{babel}
\usepackage{helvet}
\usepackage[T1]{fontenc}
\usepackage{textcomp}
\usepackage[inline]{asymptote}
\usepackage{slide_helper}
\usepackage{tikz}
\usetikzlibrary{shapes.geometric, arrows}
\usepackage{pgfplots}
\pgfplotsset{compat=1.5} 
\usepgfplotslibrary{statistics}
\usetikzlibrary{external}
\tikzexternalize%

\title[MA205 - Section 1.2]{Data Basics}

\begin{document}
\begin{frame}
\titlepage
\end{frame}

\begin{frame}
\begin{definition}
A \textbf{data matrix} is common way to organize data, especially if collecting data in a spreadsheet.
\end{definition}\pause

\begin{example}\label{loans}
\begin{center}
\usebeamerfont{datamatrix}
\begin{tabular}{cccccccc}\hline
& loan\textunderscore amount & interest\textunderscore rate & term & grade & state & total\textunderscore income & homeownership \\\hline
1 & 7500 & 7.34 & 36 & A & MD & 70000 & rent \\
2 & 25000 & 9.43 & 60 & B & OH & 254000 & mortgage \\
3 & 14500 & 6.08 & 36 & A & MO & 80000 & mortgage \\
\vdots & \vdots & \vdots & \vdots & \vdots & \vdots & \vdots & \vdots \\
50 & 3000 & 7.96 & 36 & A & CA & 34000 & rent 
\end{tabular}
\end{center}
\end{example}\pause

\begin{definition}
Each row is called a \textbf{case} or \textbf{observational unit}.
\end{definition}\pause

\begin{definition}
Each column is called a \textbf{variable}.
\end{definition}
\end{frame}

\begin{frame}
\begin{note}
It is very important to provide descriptions of all the variables in a data matrix. Be sure to include the units of measurement.
\end{note}\pause

\begin{example}
\begin{center}
\usebeamerfont{datamatrix}
\begin{tabular}{cccccccc}\hline
& loan\textunderscore amount & interest\textunderscore rate & term & grade & state & total\textunderscore income & homeownership \\\hline
1 & 7500 & 7.34 & 36 & A & MD & 70000 & rent \\
2 & 25000 & 9.43 & 60 & B & OH & 254000 & mortgage \\
3 & 14500 & 6.08 & 36 & A & MO & 80000 & mortgage \\
\vdots & \vdots & \vdots & \vdots & \vdots & \vdots & \vdots & \vdots \\
50 & 3000 & 7.96 & 36 & A & CA & 34000 & rent 
\end{tabular}

\vspace{2mm}
\begin{tabular}{ll}\hline
\textbf{variable} & \textbf{description} \\\hline
 loan\textunderscore amount & Amount of the load received, in US dollars. \\
 interest\textunderscore rate & Interest rate on the loan, in an annual percentage. \\
 term & The length of the loan, which is always a whole number of months. \\
 grade & Loan grade, which takes values A through G and represents the quality\\ 
 & of the loan and its likelihood of being repaid. \\
 state & US state where the borrower resides. \\
 total\textunderscore income & Borrower's total income, including any second income, in US dollars. \\
 homeownership & Indicates whether the person owns, owns but has a mortgage, or rents.
\end{tabular}
\end{center}
\end{example}
\end{frame}

\begin{frame}
\begin{definition}
\textbf{Numerical} data consisting of numbers representing counts or measurements. Sometimes referred to as \textbf{Quantitative} data.
\end{definition}\pause

\begin{note}
Most of the data we will be considering in this course will be numerical.
\end{note}\pause

\begin{definition}
\textbf{Categorical} data consisting of names or labels (not numbers). Sometimes referred to as \textbf{Qualitative} data.
\end{definition}\pause

\begin{note}
The names and labels in categorical data are sometimes coded with numbers. When a number is used as a name it is \textbf{not} numerical data.
\end{note}
\end{frame}

\begin{frame}
\begin{example}
\question{The ages (in years) of subjects enrolled in a clinical trail are?}\pause
\answer{Numerical data.}
\end{example}\pause

\begin{example}
\question{The genders (male / female / non-binary) of subjects enrolled in a clinical trial are?}\pause
\answer{Categorical data.}
\end{example}\pause

\begin{example}\label{idnums}
\question{Identification numbers $1, 2, 3, \ldots, 25$ are assigned randomly to 25 subjects in a clinical trail. The identification numbers are?}\pause
\answer{Categorical Data.}
\end{example}\pause

\begin{note}
The numbers in Example~\ref{idnums} don't actually measure or count anything.
\end{note}
\end{frame}

\begin{frame}
\begin{definition}
\textbf{Discrete data} result when the data values are numerical and the number of values is finite, or countable.
\end{definition}\pause

\begin{example}
Each of several physicians plans to count the number of physical examinations given during the next full week.
\end{example}\pause

\begin{example}
Casino employees plan to roll a fair die until the number 5 is rolled, and they count the number of rolls required to get a 5. (It is possible, but unlikely, that a 5 could never be rolled.)
\end{example}
\end{frame}

\begin{frame}
\begin{definition}
\textbf{Continuous data} result from infinitely many possible numerical values, where the collection of values is not countable.
\end{definition}\pause

\begin{example}
When a typical patient has blood drawn as part of a routine examination, the volume of blood drawn is between 0 mL and 50 mL.
\end{example}\pause

\begin{example}
The unemployment rate for Los Angeles County is 4.69\%.
\end{example}\pause

\begin{example}
A grade school teacher measures the heights of his students.
\end{example}
\end{frame}

\begin{frame}
\begin{definition}
The \textbf{nominal level of measurement} is characterized by data that consist of names, labels, or categories only. The data cannot be arranged in some meaningful order, such as low to high.
\end{definition}\pause

\begin{example}
A survey has responses of \textquote{yes}, \textquote{no}, and \textquote{undecided}
\end{example}\pause

\begin{example}
For an item on a survey, respondents are given a choice of possible answers, and they are coded as follows:
\begin{itemize}
\item[] 1 is coded as \textquote{I agree}
\item[] 2 is coded as \textquote{I disagree}
\item[] 3 is coded as \textquote{I don't care}
\item[] 4 is coded as \textquote{I refuse to answer}
\end{itemize}
The numbers 1,2,3, and 4 don't count or measure anything.
\end{example}
\end{frame}

\begin{frame}
\begin{definition}
Data are at the \textbf{ordinal level of measurement} if they can be arranged in some order, but differences between data values either cannot be determined or are meaningless.
\end{definition}\pause

\begin{example}
A college professor assigns grades of A, B, C, D, or F. These grades can be arranged in a meaningful order, but grades are very individualized so the difference between two students grades cannot be calculated.
\end{example}\pause

\begin{example}
A survey asks people what they felt their blood pressure was. The possible answers are \textquote{Low}, \textquote{Normal}, \textquote{High.} These responses can be arranged in a meaningful order, but the differences between \textquote{Low} and \textquote{High} doesn't make sense.
\end{example}
\end{frame}

\begin{frame}
\begin{definition}
Data are at the \textbf{interval level of measurement} if they can be arranged in order, and differences between data values can be found and are meaningful. \emph{Data at this level do not have a natural zero starting point at which none of the quantity is present.}
\end{definition}\pause

\begin{example}
Body temperatures of $\degrees{98.2}$F and $\degrees{98.9}$F are examples of data at this interval level of measurement. The values are ordered, and we can calculate their difference. There is no natural starting point. (The value $\degrees{0}$F is an arbitrary choice and doesn't represent the complete absence of heat.)
\end{example}\pause

\begin{example}
The years 1492 and 1776 can be arranged in order, and the difference of 284 years is meaningful. But, time doesn't not have a natural starting point that represents \textquote{no time.}
\end{example}
\end{frame}

\begin{frame}
\begin{definition}
Data are at the \textbf{ratio level of measurement} if they can be arranged in order, differences can be found and are meaningful, and there is a natural starting point which indicates that none of the quantity is present. 
\end{definition}\pause

\begin{example}
Heights of 180cm and 90cm for a high school student and a preschool student are at the ratio level of measurement. (The starting point is 0cm and 180cm is twice as tall as 90cm.)
\end{example}\pause

\begin{example}
The times of 50 minutes and 100 minutes for a math class are at the ratio level of measurement. (The starting point is 0 minutes and 100 minutes is twice as long as 50 minutes.)
\end{example}
\end{frame}

\begin{frame}
\begin{block}{Classification of Variables}
\begin{center}
\definecolor{flow_clr}{RGB}{238, 214, 234}
\tikzstyle{box} = [rectangle, rounded corners, minimum width=2cm, minimum height=0.7cm, outer sep=.1cm, text centered, draw=black, fill=flow_clr]
\tikzstyle{arrow} = [thick,->,>=to]
\begin{tikzpicture}[node distance=1.4cm]
\node (variables) [box] {all variables};
\node (numerical) [box, below of=variables, left of=variables, xshift=-1.5cm] {numerical};
\node (categorical) [box, below of=variables, right of=variables, xshift=1.5cm] {categorical};
\node (continuous) [box, below of=numerical, left of=numerical] {continuous};
\node (discrete) [box, below of=numerical, right of=numerical] {discrete};
\node(nominal) [box, below of=categorical, left of=categorical] {nominal};
\node(ordinal) [box, below of=categorical, right of=categorical] {ordinal};
\draw [arrow] (variables) -- (numerical);
\draw [arrow] (variables) -- (categorical);
\draw [arrow] (numerical) -- (continuous);
\draw [arrow] (numerical) -- (discrete);
\draw [arrow] (categorical) -- (nominal);
\draw [arrow] (categorical) -- (ordinal);
\end{tikzpicture}
\end{center}
\end{block}\pause

\begin{note}
For simplicities sake, we will often treat ordinal data as nominal data.
\end{note}\pause 
\begin{note} 
If your data consists of real numbers, then it is usually continuous.
\end{note}\pause

\begin{note}
If your data consists of only integers, then it is usually discrete.
\end{note}
\end{frame}

\begin{frame}
\begin{block}{Relationships Between Variables}
Often, researchers will want to study the relationship between variables.\pause

\begin{itemize}
\item If homeownership is lower than the national average in one county, will the percent of multi-unit structures in that county tend to be above or below the national average?\pause
\item Does a higher than average increase in county population tend to correspond to counties with higher or lower median household incomes?\pause
\item How useful a predictor is median education level for the median household income for US counties?
\end{itemize}
\end{block}
\end{frame}

\begin{frame}
\begin{definition}
A \textbf{scatterplot} is a plot of paired $(x,y)$ numerical data with a horizontal x-axis and a vertical y-axis. The horizontal axis is used for the first variable ($x$), and the vertical axis for the second variable ($y$).
\end{definition}\pause
\begin{example}
\begin{center}
\begin{tikzpicture}
\tikzstyle{every pin}=[
fill=white,
draw=black,
font=\tiny,
]
\pgfkeys{/pgf/number format/.cd,
fixed,
precision=999,
set thousands separator={},
1000 sep in fractionals=false,
}
\begin{axis}[
clip mode=individual,
clip marker paths=true,
width=11cm,
height=5.0cm,
xlabel={Percent of Units in Multi-Unit Structures},
ylabel={Homeownership Rate},
%ymajorgrids=true,
%xmajorgrids=true,
%enlarge x limits=false,
%enlarge y limits=false,
%xticklabel style={/pgf/number format/.cd,fixed,precision=0},
xticklabel=\pgfmathprintnumber{\tick}\%,
yticklabel=\pgfmathprintnumber{\tick}\%,
xtick={100, 80, 60, 40, 20, 0},
ytick={100, 80, 60, 40, 20, 0},
ymin=-5,
ymax=100,
xmin=-5,
xmax=100,
scatter/use mapped color={
%draw=mapped color,
fill=black,
},
]
\addplot [scatter, only marks, blue!50!black, scatter src=y, mark size=0.3pt] 
table [y=homeownership, x=multi_unit, col sep=comma] {county.csv};
\node [coordinate,pin={[pin distance=0.65cm]75:{Washington D.C. $(43.5\%, 61.7\%)$}}] at (axis cs:43.5, 61.7) {};
\node [coordinate,pin={[pin distance=0.75cm]235:{Charlottesville, VA $(43.5\%, 41.3\%)$}}] at (axis cs:43.2, 41.3) {};
\node [coordinate,pin={[pin distance=0.5cm]270:{San Fransisco, CA $(66.6\%, 37.5\%)$}}] at (axis cs:66.6, 37.5) {};
\node [coordinate,pin={[pin distance=0.9cm]30:{St. Louis, MO $(53.5\%, 47.2\%)$}}] at (axis cs:53.5, 47.2) {};
\node [coordinate,pin={[pin distance=0.95cm]87:{Thomas County, KS. $(18.3\%, 65.1\%)$}}] at (axis cs:18.3, 65.1) {};
\end{axis}
\end{tikzpicture}
\end{center}
\end{example}
\end{frame}

\begin{frame}
\begin{definition}
When two variables show some connection with one another, they are called \textbf{associated} variables. Sometimes referred to as \textbf{dependent}.
\end{definition}\pause

\begin{definition}
If there is a downward trend in the scatter plot, the variables are said to be \textbf{negatively associated}.
\end{definition}\pause

\begin{definition}
If there is an upward trend in the scatter plot, the variables are said to be \textbf{positive associated}.
\end{definition}\pause

\begin{definition}
If variables are not associated, then they are said to be \textbf{independent}.
\end{definition}\pause

\begin{note}
It is impossible to be both associated and independent.
\end{note}
\end{frame}

\begin{frame}
\begin{example}
\begin{center}
\begin{tikzpicture}
\tikzstyle{every pin}=[
fill=white,
draw=black,
font=\tiny,
]
\pgfkeys{/pgf/number format/.cd,
fixed,
precision=999,
set thousands separator={},
1000 sep in fractionals=false,
}
\begin{axis}[
clip mode=individual,
clip marker paths=true,
width=11cm,
height=5.9cm,
xlabel={Percent of Units in Multi-Unit Structures},
ylabel={Homeownership Rate},
%ymajorgrids=true,
%xmajorgrids=true,
%enlarge x limits=false,
%enlarge y limits=false,
%xticklabel style={/pgf/number format/.cd,fixed,precision=0},
xticklabel=\pgfmathprintnumber{\tick}\%,
yticklabel=\pgfmathprintnumber{\tick}\%,
xtick={100, 80, 60, 40, 20, 0},
ytick={100, 80, 60, 40, 20, 0},
ymin=-5,
ymax=100,
xmin=-5,
xmax=100,
scatter/use mapped color={
%draw=mapped color,
fill=black,
},
]
\addplot [scatter, only marks, blue!50!black, scatter src=y, mark size=0.3pt] 
table [y=homeownership, x=multi_unit, col sep=comma] {county.csv};
\end{axis}
\end{tikzpicture}
\end{center}
\question{Are the Multi-Unit Structure Rate and the Homeownership Rate associated?}\pause
\answer{Yes, they are negatively associated.}
\end{example}
\end{frame}

\begin{frame}
\begin{example}
\begin{center}
\begin{tikzpicture}
\tikzstyle{every pin}=[
fill=white,
draw=black,
font=\tiny,
]
\pgfkeys{/pgf/number format/.cd,
fixed,
precision=999,
set thousands separator={},
1000 sep in fractionals=false,
}
\begin{axis}[
clip mode=individual,
clip marker paths=true,
width=11cm,
height=5.9cm,
xlabel={Median Household Income},
ylabel style={align=center},
ylabel={Population Change\\over 7 years},
%ymajorgrids=true,
%xmajorgrids=true,
%enlarge x limits=false,
%enlarge y limits=false,
%xticklabel style={/pgf/number format/.cd,fixed,precision=0},
xticklabel=\$\pgfmathprintnumber{\tick}k,
yticklabel=\pgfmathprintnumber{\tick}\%,
xtick={20000, 40000, 60000, 80000, 100000, 120000},
ytick={-10, 0, 10, 20},
scaled x ticks=base 10:-3
xtick scale label code/.code={}
ymin=-15,
ymax=30,
xmin=0,
xmax=130000,
scatter/use mapped color={
%draw=mapped color,
fill=black,
},
]
\addplot [scatter, only marks, blue!50!black, scatter src=y, mark size=0.3pt,restrict y to domain=-15:30] 
table [y=pop_change, x=median_hh_income, col sep=comma] {county.csv};
\end{axis}
\end{tikzpicture}
\end{center}
\question{Are the Median Household Income and the Population Change associated?}\pause
\answer{Yes, they are positively associated.}
\end{example}
\end{frame}

\begin{frame}
\begin{example}
Consider the question:
\blockquote{If there is an increase in the median household income in a county, does this drive an increase in its population?}\pause

A higher median income is likely one of the causes of an increased population.
\end{example}\pause

\begin{definition}
When we suspect one variable might causally affect another, we label the first variable the \textbf{explanatory variable} and the second the \textbf{response variable}.
\vspace{-4mm}
\begin{center}
\definecolor{box_clr}{RGB}{205, 228, 236}
\tikzstyle{box} = [rectangle, rounded corners, minimum width=2cm, minimum height=0.5cm, outer sep=.1cm, text centered, draw=black, fill=box_clr]
\tikzstyle{arrow} = [thick,->,>=to]
\begin{tikzpicture}[node distance=1.4cm]
\node (left) [box] {explanatory variable};
\node (right) [box, right of=left, xshift=6cm] {response variable};
\draw [arrow] (left) --  node[midway, above] {might affect} (right);
\end{tikzpicture}
\end{center}
\end{definition}\pause

\begin{note}
Labeling a variable like this does \textbf{nothing} to guarantee that a causal relationship exists!
\end{note}
\end{frame}

\begin{frame}
\begin{definition}
In an \textbf{experiment}, we apply some treatment and then proceed to observe its effects on the individuals.
\end{definition}\pause

\begin{definition}
The individuals in an experiment are called \textbf{subjects}.
\end{definition}\pause

\begin{definition}
A \textbf{Placebo} is a treatment that has no medicinal effect. (Such as a sugar pill or saline injection.)
\end{definition}\pause

\begin{definition}
The group that receives a placebo is called the \textbf{control group}.
\end{definition}\pause

\begin{definition}
The group that receives a treatment is called the \textbf{treatment group}.
\end{definition}
\end{frame}

\begin{frame}
\begin{definition}
\textbf{Replication} is the repetition of an experiment on more than one individual. This means larger sample sizes are often needed.
\end{definition}\pause

\begin{definition}
\textbf{Blinding} is used when the subject doesn't know if they are receiving a placebo or a real treatment.
\end{definition}\pause

\begin{definition} 
\textbf{Double blinding} is when both the patients and the researchers are unaware of who is getting the placebo and who is getting the treatment.
\end{definition}\pause

\begin{definition}
\textbf{Randomization} is used when individuals are assigned to different groups through a process of random selection.
\end{definition}
\end{frame}

\begin{frame}
\begin{example}
In 1954, an experiment was designed to test the effectiveness of the vaccine developed by Jonas Salk in preventing polio, which had killed or paralyzed thousands of children.\pause

\vspace{2mm}
By random selection $401,974$ children were assigned to two groups:
\begin{itemize}
\item $200,745$ children were given injections of the Salk vaccine.
\item $201,229$ children were given placebo injections that contained no drug.
\end{itemize}\pause

\vspace{2mm}
Among the children given the Salk vaccine, 33 later developed paralytic polio, and among the children given a placebo, 115 later developed paralytic polio.
\end{example}\pause

\begin{note}
In general, causation can only be inferred from a randomized experiment.
\end{note}
\end{frame}

\begin{frame}
\begin{definition}
In an \textbf{observational study}, we observe and measure specific characteristics, but we don't attempt to \emph{modify} the subjects.
\end{definition}\pause

\begin{note}
Experiments are preferable to observational studies. But there may be cost, time, or ethical concerns that prohibit an experiment.
\end{note}\pause

\begin{example}
Suppose we want to determine if listening to music while driving increases the chance of being in an collision.\pause
\begin{itemize}
\item \textbf{Observational study:} We gather police reports about collisions and use them to determine if the person was listening to music or not.\pause
\item \textbf{Experiment:} We randomly assign subjects to either listen to music while driving or listen to nothing. We then count how many collisions each subject is involved in.
\end{itemize}
\end{example}
\end{frame}
\end{document}
