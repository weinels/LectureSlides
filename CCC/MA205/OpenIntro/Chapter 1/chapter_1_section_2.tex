\documentclass{beamer}
\usepackage[utf8]{inputenc}
\usepackage[english]{babel}
\usepackage{helvet}
\usepackage[T1]{fontenc}
\usepackage{textcomp}
\usepackage[inline]{asymptote}
\usepackage{slide_helper}

\setbeamerfont{datamatrix}{family=\ttfamily,size*={7pt}{10pt}}
\setbeamerfont{quote}{size=\small, shape=\itshape} %chktex 6
\setbeamerfont{footnote}{size*={6pt}{2pt}}

\title[MA205 - Section 1.2]{Data Basics}

\begin{document}
\begin{frame}
\titlepage
\end{frame}

\begin{frame}
\begin{definition}
A \textbf{data matrix} is common way to organize data, especially if collecting data in a spreadsheet.
\end{definition}\pause

\begin{example}
\begin{center}
\usebeamerfont{datamatrix}
\begin{tabular}{cccccccc}\hline
& loan\textunderscore amount & interest\_rate & term & grade & state & total\_income & homeownership \\\hline
1 & 7500 & 7.34 & 36 & A & MD & 70000 & rent \\
2 & 25000 & 9.43 & 60 & B & OH & 254000 & mortgage \\
3 & 14500 & 6.08 & 36 & A & MO & 80000 & mortgage \\
\vdots & \vdots & \vdots & \vdots & \vdots & \vdots & \vdots & \vdots \\
50 & 3000 & 7.96 & 36 & A & CA & 34000 & rent 
\end{tabular}
\end{center}
\end{example}
\end{frame}

\end{document}
