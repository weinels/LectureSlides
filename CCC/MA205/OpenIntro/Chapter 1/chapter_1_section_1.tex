\documentclass{beamer}
\usepackage[utf8]{inputenc}
\usepackage[english]{babel}
\usepackage{helvet}
\usepackage[T1]{fontenc}
\usepackage{textcomp}
\usepackage[inline]{asymptote}
\usepackage{slide_helper}
\usepackage{blindtext}
\newcommand\blfootnote[1]{%
\begingroup
\renewcommand\thefootnote{}\footnote{#1}%
\addtocounter{footnote}{-1}%
\endgroup
}

\setbeamerfont{datamatrix}{family=\ttfamily,size*={7pt}{10pt}}
\setbeamerfont{quote}{size=\small, shape=\itshape} %chktex 6
\setbeamerfont{footnote}{size*={6pt}{2pt}}

\title[MA205 - Section 1.1]{Case Study: Using Stents to Prevent Strokes}

\begin{document}
\begin{frame}
\titlepage
\end{frame}

\begin{frame}
\begin{block}{Efficacy of a Medical Treatment}
A classic challenge in statistics is to determine how effective a medical treatment truly is.
\end{block}\pause

\begin{note}
The terms introduced in this first chapter will be revisited later. 
\end{note}\pause

\begin{definition}
A \textbf{stent} is a device put inside blood vessels that assist in patient recovery after cardiac events and reduce the risk of an additional heart attack or death.
\end{definition}\pause

\begin{block}{Case Study}
Many doctors have hoped that stents would have similar benefits for patients at risk of strokes.\pause

The question researchers need to answer is: 
\vspace{-1.5mm}\begin{center}\textbf{Does the use of stents reduce the risk of stroke?}\end{center}\vspace{0.1mm}
\end{block}
\end{frame}

\begin{frame}
\begin{block}{Experiment}
The researchers conducted an experiment with 451 at-risk patients.

Each volunteer patient was randomly assigned into one of two groups.
\end{block}\pause

\begin{block}{Treatment group (224 patients)}
These patients received a stent and medical management. 
\end{block}\pause

\begin{note}
The medical management included medications, management of risk factors, and help in lifestyle modification.
\end{note}\pause

\begin{block}{Control group (227 patients)}
These patients received the same medical management, but did not receive a stent.
\end{block}
\end{frame}

\begin{frame}
\begin{block}{Data Gathering}
The researchers studied the effect of stents at two time points:\pause
\begin{itemize}
\item 30 days after enrollment\pause
\item 365 days after enrollment\pause
\end{itemize}
\end{block}

\begin{block}{Data}
\begin{center}
\usebeamerfont{datamatrix}
\begin{tabular}{lccc}\hline
Patient & group & 0-30 days & 0-365 days \\\hline
1 & treatment & no event & no event \\
2 & treatment & stroke & stroke \\
3 & treatment & no event & no event \\
4 & treatment & no event & stroke \\
\vdots & \vdots & \vdots & \vdots \\
451 & control & no event & no event
\end{tabular}
\end{center}
\end{block}\pause

\begin{note}
Listing each patient line-by-line is very cumbersome.
\end{note}
\end{frame}

\begin{frame}
\begin{block}{Descriptive Statistics}
\begin{center}
\usebeamerfont{datamatrix}
\begin{tabular}{lccccc}
&\multicolumn{2}{c}{0-30 days}&&\multicolumn{2}{c}{0-365 days} \\
&stroke & no event & & stroke& no event \\\hline
treatment & 33 & 191 & & 45 & 179 \\
control & 13 & 214 & & 28 & 199 \\\hline
total & 46 & 405 & & 73 & 378\\
\end{tabular}
\end{center}
\end{block}\pause

\begin{block}{Question}
\question{What percentage of the treatment group had a stroke in the first year?}\pause
\vspace{-2mm}
\begin{equation*}
\dfrac{\text{number of treatment group that had a stroke}}{\text{total size of treatment group}}\pause
=\dfrac{45}{224}\pause
 = 0.20 = 20\%
\end{equation*}
\vspace{-3mm}
\end{block}\pause
\begin{block}{Question}
\question{What percentage of the control group had a stroke in the first year?}\pause
\vspace{-2mm}
\begin{equation*}
\dfrac{\text{number of control group that had a stroke}}{\text{total size of control group}}\pause
= \dfrac{28}{224}\pause
 = 0.12 = 12\%
\end{equation*}
\vspace{-3mm}
\end{block}
\end{frame}

\begin{frame}
\begin{note}
This means an additional 8\% of patients with a stent had a stroke!
\end{note}\pause

\begin{definition}
A \textbf{summary statistic} is a single number summarizing a large amount of data.
\end{definition}\pause

\begin{block}{Why is this important?}
\begin{enumerate}
\item Many doctors expected stents to reduce the chance of a stroke.\pause
\item Does the data show a \textquote{real} difference between the groups?
\end{enumerate}
\end{block}\pause

\begin{note}
The second question is a real subtle one and most of the statistical tools we discuss will be used to address this question.
\end{note}
\end{frame}

\begin{frame}
\begin{block}{Signifigance}
What is the chance of getting a head when flipping a quarter?\pause 

Theoretically it is 50\%. But if you flip a large number of coins, you rarely get exactly half heads and half tails.\pause

\begin{center}
\usebeamerfont{datamatrix}
\begin{tabular}{ccccc}\hline
\multicolumn{2}{c}{heads} & \multicolumn{2}{c}{tails} & total \\\hline
5,045 & 50.4\% & 4,955 & 49.5\% & 10,000\\
4,969 & 49.7\% & 5,031 & 50.3\% & 10,000\\
5,064 & 50.6\% & 4,936 & 49.4\% & 10,000\\
5,091 & 50.9\% & 4,909 & 49.1\% & 10,000\\
4,972 & 49.7\% & 5,028 & 50.3\% & 10,000\\
5,021 & 50.2\% & 4,979 & 49.8\% & 10,000\\
5,007 & 50.1\% & 4,993 & 49.9\% & 10,000\\
5,031 & 50.3\% & 4,969 & 49.7\% & 10,000\\
5,056 & 50.6\% & 4,944 & 49.4\% & 10,000\\
5,006 & 50.1\% & 4,994 & 49.9\% & 10,000\\
\end{tabular}
\end{center}
\end{block}
\end{frame}

\begin{frame}
\begin{note}
The published results of the study were:
\vspace{-1.5mm}\begin{center}\usebeamerfont{quote}There was compelling evidence of harm by stents in this study of stroke patients.\end{center}\vspace{-4mm}
\rule{0.5\linewidth}{0.4pt}

{\usebeamerfont{footnote} Chimowitz MI, Lynn MJ, Derdeyn CP, et al. 2011. Stenting versus Aggressive Medical Therapy for Intracranial Arterial Stenosis. New England Journal of Medicine 365:993-1003. \url{http://nejm.org/doi/full/10.1056/NEJMoa1105335}}
\end{note}\pause

\begin{block}{Be careful}
Do not generalize the results of this study to all patients and all stents.\pause

\begin{itemize}
\item This study considered patients with very specific characteristics who volunteered to be a part of the study and may not be representative of all stroke patients.\pause
\item There are many types of stents and this study only considered the self-expanding Wingspan stent.
\end{itemize}
\end{block}
\end{frame}

\begin{frame}
\begin{block}{Percentages Review}
\begin{itemize}
\item \textbf{Percentage of:} To find a percentage of an amount, replace the \% symbol with division by 100 and multiply by the amount.
\begin{description}
\item[Example:] 6\% of 1200 responses is $\frac{6}{100}\cdot 1200=72$
\end{description}
\item \textbf{Decimal to Percentage:} To convert from a decimal to a percentage, multiply by 100\%.
\begin{description}
\item[Example:] $0.25\rightarrow 0.25\cdot100\%=25\%$
\end{description}
\item \textbf{Fraction to Percentage:} To convert from a fraction to a percentage, divide the denominator into the numerator and multiply by 100\%.
\begin{description}
\item[Example:] $\frac{3}{4}=0.75\rightarrow 0.75\cdot 100\%=75\%$
\end{description}
\item \textbf{Percentage to Decimal:} To convert from a percentage to a decimal number, replace the \% by division by 100. 
\begin{description}
\item[Example:] $85\%\rightarrow \frac{85}{100}=0.85$
\end{description}
\end{itemize}
\end{block}
\end{frame}
\end{document}
