\documentclass{beamer}
\usepackage[utf8]{inputenc}
\usepackage[english]{babel}
\usepackage{helvet}
\usepackage[T1]{fontenc}
\usepackage{textcomp}
\usepackage[inline]{asymptote}
\usepackage{slide_helper}
\usepackage{multirow}
\usepackage{tikz}
\usepackage{subfigure}
\usepackage{cancel}
\usetikzlibrary{shapes.geometric, arrows}
\usepackage{pgfplots}
\pgfplotsset{compat=1.5} 
\usepgfplotslibrary{statistics}
\usetikzlibrary{external}
\tikzexternalize%

\title[MA205 - Section 3.3]{Sampling From A Small Population}

\DeclareSymbolFont{extraup}{U}{zavm}{m}{n}
\DeclareMathSymbol{\varheart}{\mathalpha}{extraup}{86}
\DeclareMathSymbol{\vardiamond}{\mathalpha}{extraup}{87}
\DeclareMathSymbol{\varclub}{\mathalpha}{extraup}{84} 
\DeclareMathSymbol{\varspade}{\mathalpha}{extraup}{85}

\newcommand{\suitheart}[1][]{{\color{red}\text{#1}\varheart}}
\newcommand{\suitspade}[1][]{{\color{black}\text{#1}\spadesuit}}
\newcommand{\suitdiamond}[1][]{{\color{red}\text{#1}\vardiamond}}
\newcommand{\suitclub}[1][]{{\color{black}\text{#1}\varclub}}
\newcommand{\card}[2]{{#1{\color{black}\text{#2}}}}

\newcommand{\prob}[1]{P\left({#1}\right)}
\newcommand{\jointprob}[3]{\prob{{#1}~\text{#2}~{#3}}}
\newcommand{\condprob}[2]{\prob{{#1}~|~{#2}}}
\newcommand{\comb}[2]{_{#1}C_{#2}}

\newcommand\encircle[1]{%
  \tikz[baseline=(X.base)]  % chktex 36 chktex 1
    \node (X) [draw, shape=circle, inner sep=0, fill=yellow!10] {\strut #1};} % chktex 36 chktex 1

\begin{document}
\begin{frame}
\titlepage
\end{frame}

\begin{frame}
\begin{note}
While we usually sample from a much larger population, there are times where our sample size is large enough or the population small enough that we sample more that 10\% of a population without replacement.
\end{note}\pause

\begin{example}
Teachers sometimes select a student at random to answer a question. We assume each student has an equal chance of being selected and there are 15 students in the class. 

\question{What is the chance you will be picked for the next question?}\pause
\answer{Probability is $\dfrac{1}{15}\approx 0.067$.}
\end{example}
\end{frame}

\begin{frame}
\begin{example}
\question{If the teacher asks 3 questions, what is the probability that you will not be selected? (Assume that she only picks a student once.)}\pause
\answer{For any single question, if you are not picked, then she picked one of the other students.\pause

\vspace{2mm}
Using the General Multiplication rule we get:

\vspace{-6mm}
\begin{equation*}
\begin{array}{l}
\prob{\text{\small not picked in 3 questions}} \\
\quad = \prob{\text{\small\variable{Q1} is \outcome{not picked} and \variable{Q2} is \outcome{not picked} and \variable{Q3} is \outcome{not picked}}}\\\pause
\quad = \condprob{\text{\small\variable{Q3} is \outcome{not picked}}}{\text{\small\variable{Q1} is \outcome{not picked} and \variable{Q2} is \outcome{not picked}}}\\
\qquad\quad\times\jointprob{\text{\small\variable{Q1} is \outcome{not picked}}}{and}{\text{\small\variable{Q2} is \outcome{not picked}}}\\\pause
\quad = \condprob{\text{\small\variable{Q3} is \outcome{not picked}}}{\text{\small\variable{Q1} is \outcome{not picked} and \variable{Q2} is \outcome{not picked}}}\\
\qquad\quad\times\condprob{\text{\small\variable{Q2} is \outcome{not picked}}}{\text{\small{\variable{Q1} is \outcome{not picked}}}}\times\prob{\text{\small\variable{Q1} is \outcome{not picked}}} \\\pause
\quad = \dfrac{12}{13}\cdot\dfrac{13}{14}\cdot\dfrac{14}{15} \pause = 0.80
\end{array}
\end{equation*}\pause
So, there is a 80\% chance you won't be picked.}
\end{example}
\end{frame}

\begin{frame}
\begin{example}
\question{If the teacher asks 3 questions without regard to who she has already selected, what is the probability that you will not be selected?}\pause
\answer{If she is willing to selected the same student twice, then the questions become independent events and the calculations become easier.\pause

\vspace{2mm}
Using the Multiplication Rule for Independent Events we get:

\vspace{-6mm}
\begin{equation*}
\begin{array}{l}
\prob{\text{\small not picked in 3 questions}} \\
\quad = \prob{\text{\small\variable{Q1} is \outcome{not picked} and \variable{Q2} is \outcome{not picked} and \variable{Q3} is \outcome{not picked}}}\\\pause
\quad = \prob{\text{\small\variable{Q1} is \outcome{not picked}}}\cdot\prob{\text{\small\variable{Q2} is \outcome{not picked}}}\cdot\prob{\text{\small\variable{Q3} is \outcome{not picked}}} \\\pause
\quad = \dfrac{14}{15}\cdot\dfrac{14}{15}\cdot\dfrac{14}{15}\pause = 0.813
\end{array}
\end{equation*}\pause
So, there is a 81.3\% chance you won't be picked.}
\end{example}\pause

\begin{note}
Notice that this is different than the 80\% chance of not being picked when she was picking without replacement.
\end{note}
\end{frame}

\begin{frame}
\begin{example}
Your department is holding a raffle. They sell 30 tickets and offer seven prizes. They place the tickets in a hat and draw one for each prize, without replacing the winning tickets. 

\vspace{2mm}
\question{What is your chance of winning a prize if you buy one ticket?}\pause
\answer{This sampling is without replacement, so the events are not independent and we have to use the General Multiplication Rule.

\vspace{-3mm}
\begin{equation*}
\begin{aligned}
\prob{\text{\small win at least one prize}}
&= 1 - \prob{\text{\small win no prizes}} \\\pause
&= 1 - \dfrac{29}{30}\cdot\dfrac{28}{29}\cdot\dfrac{27}{28}\cdot\dfrac{26}{27}\cdot\dfrac{25}{26}\cdot\dfrac{24}{25}\cdot\dfrac{23}{24} \\\pause
&= 1- \dfrac{29\cdot28\cdot27\cdot26\cdot25\cdot24\cdot23}{30\cdot29\cdot28\cdot27\cdot26\cdot25\cdot24}\\\pause
&= 1- \dfrac{\cancel{29}\cdot\cancel{28}\cdot\cancel{27}\cdot\cancel{26}\cdot\cancel{25}\cdot\cancel{24}\cdot23}{30\cdot\cancel{29}\cdot\cancel{28}\cdot\cancel{27}\cdot\cancel{26}\cdot\cancel{25}\cdot\cancel{24}}\\\pause
& = 1 - \dfrac{23}{30} \pause
= \dfrac{7}{30} \approx 0.233
\end{aligned}
\end{equation*}
\vspace{-2mm}
}
\end{example}
\end{frame}

\begin{frame}
\begin{example}
Your department is holding a raffle. They sell 30 tickets and offer seven prizes. They place the tickets in a hat and draw one for each prize, with replacing the winning tickets. 

\vspace{2mm}
\question{What is your chance of winning a prize if you buy one ticket?}\pause
\answer{This sampling is with replacement, so the events are independent and we have to use the Multiplication Rule for Independent Events.

\vspace{-3mm}
\begin{equation*}
\begin{aligned}
\prob{\text{\small win at least one prize}}
&= 1 - \prob{\text{\small win no prizes}} \\\pause
&= 1 - \dfrac{29}{30}\cdot\dfrac{29}{30}\cdot\dfrac{29}{30}\cdot\dfrac{29}{30}\cdot\dfrac{29}{30}\cdot\dfrac{29}{30}\cdot\dfrac{29}{30} \\\pause
&= 1- {\left(\dfrac{29}{30}\right)}^7 \pause \approx 0.211
\end{aligned}
\end{equation*}
\vspace{-2mm}
}
\end{example}\pause

\begin{note}
The chances of winning a prize when sampling without replacement almost 10\% larger than when sampling with replacement.
\end{note}
\end{frame}
\end{document}
