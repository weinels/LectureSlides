\documentclass{beamer}
\usepackage[utf8]{inputenc}
\usepackage[english]{babel}
\usepackage{helvet}
\usepackage[T1]{fontenc}
\usepackage{textcomp}
\usepackage[inline]{asymptote}
\usepackage{slide_helper}
\usepackage{multirow}
\usepackage{tikz}
\usepackage{subfigure}
\usetikzlibrary{shapes.geometric, arrows}
\usepackage{pgfplots}
\pgfplotsset{compat=1.5} 
\usepgfplotslibrary{statistics}
\usetikzlibrary{external}
\tikzexternalize%

\title[MA205 - Section 3.1]{Defining Probability}

\DeclareSymbolFont{extraup}{U}{zavm}{m}{n}
\DeclareMathSymbol{\varheart}{\mathalpha}{extraup}{86}
\DeclareMathSymbol{\vardiamond}{\mathalpha}{extraup}{87}
\DeclareMathSymbol{\varclub}{\mathalpha}{extraup}{84} 
\DeclareMathSymbol{\varspade}{\mathalpha}{extraup}{85}

\newcommand{\suitheart}[1][]{{\color{red}\text{#1}\varheart}}
\newcommand{\suitspade}[1][]{{\color{black}\text{#1}\spadesuit}}
\newcommand{\suitdiamond}[1][]{{\color{red}\text{#1}\vardiamond}}
\newcommand{\suitclub}[1][]{{\color{black}\text{#1}\varclub}}

\newcommand{\prob}[1]{P\left(#1\right)}
\newcommand{\condprob}[2]{\prob{#1~\middle|~#2}}
\newcommand{\comb}[2]{_{#1}C_{#2}}

\begin{document}
\begin{frame}
\titlepage
\end{frame}

\begin{frame}
\begin{definition}
The result of \textbf{random process} is called an \textbf{outcome}.
\end{definition}\pause

\begin{note}
A \textbf{die}, the singular of \textbf{dice}, is a cube with six sides numbered 1 to 6.
\end{note}\pause

\begin{example}
Assume we roll a die and get a 1.

\vspace{1mm}
\question{What is the random process?}\pause
\answer{Rolling the die.}\pause

\vspace{1mm}
\question{What is the outcome?}\pause
\answer{The 1 that was rolled.}\pause

\vspace{1mm}
\question{What is the chance of rolling a 1 on this die?}\pause
\answer{If the dice is fair, each side has the same chance of being rolled. So a 1 has a one-in-six chance, equivalently $\dfrac{1}{6}$.}
\end{example}
\end{frame}

\begin{frame}
\begin{definition}
A \textbf{probability} of an outcome is the proportion of times the outcome would occur if we observed the random process an infinite number of times.
\end{definition}\pause

\begin{note}
\begin{equation*}
\prob{X}=\dfrac{\text{Number of outcomes corresponding to $X$}}{\text{Total number of outcomes}}
\end{equation*}
\end{note}\pause

\begin{example}
\question{What is the probability of rolling a 1 or 2 on a die?}\pause
\answer{There are two outcomes, a 1 or a 2, and six faces on a die.
\begin{equation*}
\prob{\text{roll}~1~\text{or}~2} = \dfrac{2}{6} = \dfrac{1}{3}
\end{equation*}}
\end{example}
\end{frame}

\begin{frame}
\begin{definition}
A standard deck of 52 playing cards consists of four \textbf{suits} in two colors: 
Hearts $\suitheart$, Spades $\suitspade$, Diamonds $\suitdiamond$, and Clubs $\suitclub$

\vspace{2mm}
Each suit contains 13 cards, each of a different \textbf{rank}: 

2 through 10, Jack, Queen, King, and Ace

\vspace{-6mm}
\begin{equation*}
\begin{array}{ccccccccccccc}
\suitclub 2 & \suitclub 3 & \suitclub 4 & \suitclub 5 & \suitclub 6 & \suitclub 7 & \suitclub 8 & \suitclub 9 & \suitclub 10 & \suitclub \text{J} & \suitclub \text{Q} & \suitclub \text{K} &\suitclub \text{A} \\
\suitdiamond 2 & \suitdiamond 3 & \suitdiamond 4 & \suitdiamond 5 & \suitdiamond 6 & \suitdiamond 7 & \suitdiamond 8 & \suitdiamond 9 & \suitdiamond 10 & \suitdiamond \text{J} & \suitdiamond \text{Q} & \suitdiamond \text{K} &\suitdiamond \text{A} \\
\suitheart 2 & \suitheart 3 & \suitheart 4 & \suitheart 5 & \suitheart 6 & \suitheart 7 & \suitheart 8 & \suitheart 9 & \suitheart 10 & \suitheart \text{J} & \suitheart \text{Q} & \suitheart \text{K} &\suitheart \text{A} \\
\suitspade 2 & \suitspade 3 & \suitspade 4 & \suitspade 5 & \suitspade 6 & \suitspade 7 & \suitspade 8 & \suitspade 9 & \suitspade 10 & \suitspade \text{J} & \suitspade \text{Q} & \suitspade \text{K} &\suitspade \text{A}
\end{array}
\end{equation*}
\end{definition}\pause

\begin{example}
\question{What is the probability of drawing a single card from a deck and getting an Ace?}\pause
\answer{There are four aces in a deck of 52 cards. Which gives the probability
\begin{equation*}
\prob{\text{Ace}}=\dfrac{4}{52}=\dfrac{1}{13}= 0.0769=7.67\%
\end{equation*}}
\vspace{-5mm}
\end{example}
\end{frame}

\begin{frame}
\begin{example}
\question{What is the probability of rolling a 1, 2, 3, 4, 5, or 6 on a die?}\pause
\answer{Every side of the die is listed, so
\begin{equation*}
\prob{\text{roll}~1~\text{or}~2~\text{or}~3~\text{or}~4~\text{or}~5~\text{or}~6} = \dfrac{6}{6} = 1 = 100\%
\end{equation*}}
\end{example}\pause

\begin{definition}
An outcome with a probability of 1 is called \textbf{certain}.
\end{definition}
\end{frame}

\begin{frame}
\begin{example}
\question{If a year is selected at random, what is the probability that Thanksgiving Day (in the United States) will be on a Wednesday?}\pause
\answer{In the United States, Thanksgiving Day always falls on the fourth Thursday in November. \pause

\vspace{1mm}
This means it is impossible for Thanksgiving Day to fall on a Wednesday.
\begin{equation*}
\begin{aligned}
\prob{\text{Thanksgiving on a Wednesday}}&=0=0\%\\
\prob{\text{Thanksgiving on a Thursday}}&=1=100\%
\end{aligned}
\end{equation*}}
\end{example}\pause

\begin{definition}
An outcome with a probability of 0 is called \textbf{impossible}.
\end{definition}\pause

\begin{note}
Probabilities are always between 0 and 1.
\end{note}
\end{frame}

\begin{frame}
\begin{example}
The probability of rolling a 1 on a die is $p=1/6\approx0.167$, but if we roll six dice, we may get no 1's or multiple 1's.\pause

\vspace{1mm}
Let $\hat{p}_n$ be the proportion of number of 1's rolled after $n$ rolls.

\begin{center}
\begin{tikzpicture}
%\pgfkeys{/pgf/number format/.cd,
%fixed,
%precision=999,
%set thousands separator={,},
%1000 sep in fractionals=true,
%}
\begin{semilogxaxis}[
clip mode=individual,
clip marker paths=true,
width=11.5cm,
height=5cm,
xlabel={$n$ (number of rolls)},
ylabel={$\hat{p}_n$},
ylabel style={rotate=-90},
grid=major,
minor tick style={draw=none},
xtick={1, 10, 100, 1000, 10000, 100000},
xticklabels={1,10,100,{1,000},{10,000},{100,000}},
]
\draw[red!90, dashed] (-1.1, 166) -- (12.6,166); % super hack-y, but I couldn't figure how how to get a wide enough line in axis cs
\addplot [blue!85] table [y=pn, x=n, col sep=comma] {diceroll.csv};
\end{semilogxaxis}
\end{tikzpicture}
\end{center}\pause
\end{example}
\begin{note}
It is not a coincidence that $\hat{p}_n$ get closer to $p$ as $n$ increases.
\end{note}
\end{frame}

\begin{frame}
\begin{block}{Law of Large Numbers}
As more observations are collected, the proportion $\hat{p}_n$ of occurrences with a particular outcome converges to the probability $p$ of that outcome.
\end{block}\pause

\begin{block}{Cautions}
\begin{itemize}
\item The law of large numbers applies to behavior over a large number of trails, and it does not apply to any one individual outcome.\pause
\begin{itemize}
\item Gamblers sometimes foolishly loose large sums of money by incorrectly thinking that a string of losses increases the chances of a win on the next bet, or that a string of wins is likely to continue.
\end{itemize}\pause
\item If we know nothing about the likelihood of different possible outcomes, we should not assume that they are equally likely.\pause
\begin{itemize}
\item You should not think that the probability of passing the next exam is $\tfrac{1}{2}$, or 0.5. The actual probability depends on factors such as the amount of preparation and the difficulty of the exam.
\end{itemize}
\end{itemize}
\end{block}
\end{frame}

\begin{frame}
\begin{definition}
Outcomes $A$ and $B$ are \textbf{disjoint} (or \textbf{mutually exclusive}) if they cannot occur at the same time.
\end{definition}\pause

\end{document}
