\documentclass{beamer}
\usepackage[utf8]{inputenc}
\usepackage[english]{babel}
\usepackage{helvet}
\usepackage[T1]{fontenc}
\usepackage[inline]{asymptote}
\usepackage{asy_helper}
\usepackage{slide_helper}
\usepackage{cancel}
\usepackage{tikz}
\usetikzlibrary{shapes.geometric, arrows}
\usepackage{pgfplots}
\pgfplotsset{compat=1.5} 
\usepgfplotslibrary{statistics}
\usetikzlibrary{external}
\tikzexternalize%

\begin{asydef}
  real nd_func(real mu, real sigma, real x)
  {
    return 1/sqrt(2*pi*sigma*sigma)*exp((-1*(x-mu)*(x-mu))/(2*sigma*sigma));
  }
  
  guide normal_dist(real mu, real sigma, real xmin, real xmax)
  {
    real nd(real x) { return nd_func(mu, sigma, x); }
    return graph(nd, xmin, xmax);
  }

  void shade_below(real mu, real sigma, real b, real xmin, real xmax, pen p=royalblue)
  {
    real nd_func(real x) { return 1/sqrt(2*pi*sigma*sigma)*exp((-1*(x-mu)*(x-mu))/(2*sigma*sigma)); }
    
    guide g = graph(nd_func, xmin, b);
    
    filldraw(g -- (b,0) -- cycle, p, black);
    
    draw((xmin,0)--(b,0));
  }

  void shade_above(real mu, real sigma, real b, real xmin, real xmax, pen p=royalblue)
  {
    real nd_func(real x) { return 1/sqrt(2*pi*sigma*sigma)*exp((-1*(x-mu)*(x-mu))/(2*sigma*sigma)); }
    
    guide g = graph(nd_func, b, xmax);
    
    filldraw(g -- (b,0) -- cycle, p, black);
    
    draw((xmax,0)--(b,0));
  }

  void shade_between(real mu, real sigma, real a, real b, pen p=royalblue)
  {
    real nd_func(real x) { return 1/sqrt(2*pi*sigma*sigma)*exp((-1*(x-mu)*(x-mu))/(2*sigma*sigma)); }
    
    guide g = graph(nd_func, a, b);
    
    filldraw((a,0) -- g -- (b,0) -- cycle, p, black);
  }

  void multiple_nd_curves_example(real std_dev)
  {
    size(300, 190, IgnoreAspect);
    
    draw(normal_dist(0, std_dev, -6,6));
    shade_between(0,std_dev,-std_dev,std_dev);
    draw((0,0)--(0,0.45));

    label("$\sigma="+format("%#.2f", std_dev)+"$", (-4.2,0.45), Fill(paleyellow));

    xaxis(Bottom(), RightTicks(new real[] {-6,-4,-2,0,2,4,6}));
    yaxis(Left(), LeftTicks(size=nan),ymin = 0, ymax = 0.5);
  }
\end{asydef}

\begin{asydef}
  pair ex2 = (250, 70);
  pair blkpq = (90, 90);
\end{asydef}

\title[MA205 - Section 5.2]{Confidence Intervals for a Proportion}

\newcommand{\prob}[1]{P\left({#1}\right)}
\newcommand{\jointprob}[3]{\prob{{#1}~\text{#2}~{#3}}}
\newcommand{\condprob}[2]{\prob{{#1}~|~{#2}}}
\newcommand{\comb}[2]{_{#1}C_{#2}}

\begin{document}
\begin{frame}
  \titlepage
\end{frame}

\begin{frame}
  \begin{example}
    In a Gallup poll of 1487 adults, 43\% of them said that they have a Facebook page.

    \vspace{1mm}
    \question{Based on this result, what is the best point estimate of the proportion of \emph{all} adults who have a Facebook page.}\pause
    \answer{The sample proportion, 0.43, is the best point estimate of the population proportion.}
  \end{example}\pause

  \begin{note}
    We have no indication of how \emph{good} of an estimate 0.43 is, just that it is the best of the available options.
  \end{note}
\end{frame}

\begin{frame}
  \begin{definition}
    A \textbf{confidence interval} is a range of values around the point estimate used to estimate the true value of a population parameter.

    \vspace{-2mm}
    \begin{equation*}
      \begin{aligned}
        \interval{\closed{\text{point estimate} - \text{some value}}}{\closed{\text{point estimate}+\text{some value}}}
      \end{aligned}
    \end{equation*}

    \vspace{1mm}
    A confidence interval is sometimes abbreviated as CI\@.
  \end{definition}\pause

  \begin{definition}
    The \textbf{confidence level} is the probability that the confidence interval actually does contain the population parameter, assuming that the estimation process is repeated a large number of times.
  \end{definition}\pause

  \begin{note}
    Round the confidence interval limits to three significant digits.
  \end{note}
\end{frame}

\begin{frame}[fragile]
  \begin{block}{The Process Success Rate}
    A confidence level of 95\% tells us that the process we use should, given enough iterations, result in a confidence interval that contains the true population proportion 95\% of the time.\pause

    \vspace{2mm}
    If the true population proportion is $p=0.5$, then we expect around 19 of 20 confidence intervals to contain the true value of $p$.
    \begin{center}
      \begin{asy}
        size(300,150,IgnoreAspect);

        real xmin = 0;		real xmax = 10;
        real ymin = 0.43;	real ymax = 0.57;

        draw((0,0.5)--(21,0.5));
        draw((0,ymin)--(0,ymax));

        label("$0.55$",(0,0.55),W);
        draw((-0.1,0.55)--(0.1,0.55));
        label("$0.45$",(0,0.45),W);
        draw((-0.1,0.45)--(0.1,0.45));
        label("$p=0.50$",(0,0.5),W);

        int bad = randint(1,21);

        for (int i=1; i<21; ++i)
        {
	  if (i == bad)
	  {
	    real a = randreal(0.48,0.49);
	    real b = a-0.05;
	    draw((i,a)--(i,b), heavyred+1.5bp);
	    dot((i,a), heavyred+4bp);
	    dot((i,b), heavyred+4bp);
	  }
	  else
	  {
	    real a = randreal(0.52,0.54);
	    real b = a-0.05;
	    draw((i,a)--(i,b), heavygreen+1.5bp);
	    dot((i,a), heavygreen+4bp);
	    dot((i,b), heavygreen+4bp);
	  }
        }

      \end{asy}
    \end{center}
  \end{block}
\end{frame}

\begin{frame}[fragile]
  \begin{block}{A Few Observations}
    \begin{overprint}
      \onslide<1 | handout: 0>
      \begin{center}
        \begin{asy}
          size(300, 55, IgnoreAspect);

          real mu = 0;
          real sigma = 1;

          real left = -1.1;
          real right = 1.1;

          real xmin=mu-2.5*sigma; real xmax=mu+2.5*sigma;

          dot((right,0), invisible);
          
          shade_between(mu,sigma,xmin,xmax);
          draw(normal_dist(mu, sigma, xmin, xmax));
        \end{asy} 
      \end{center}
      \onslide<2 | handout:0>
      \begin{center}
        \begin{asy}
          size(300, 55, IgnoreAspect);

          real mu = 0;
          real sigma = 1;

          real left = -1.1;
          real right = 1.1;

          real xmin=mu-2.5*sigma; real xmax=mu+2.5*sigma;

          dot((right,0), invisible);
          
          shade_between(mu,sigma,left,right, mediumyellow);
          shade_between(mu,sigma,xmin,left);
          shade_between(mu,sigma,right,xmax, mediumred);
          draw(normal_dist(mu, sigma, xmin, xmax));

          label("Confidence Level", (mu,0.5*nd_func(mu,sigma,mu)));
          label("$1-\alpha$", (mu,0.25*nd_func(mu,sigma,mu)));
          label("$\alpha/2$", (left,0),NW);
          label("$\alpha/2$", (right,0),NE);

        \end{asy}
      \end{center}
      \onslide<3- | handout:1>
      \begin{center}
        \begin{asy}
          size(300, 55, IgnoreAspect);

          real mu = 0;
          real sigma = 1;

          real left = -1.1;
          real right = 1.1;

          real xmin=mu-2.5*sigma; real xmax=mu+2.5*sigma;

          shade_between(mu,sigma,left,right, mediumyellow);
          shade_between(mu,sigma,xmin,left);
          shade_between(mu,sigma,right,xmax);
          draw(normal_dist(mu, sigma, xmin, xmax));

          label("Confidence Level", (mu,0.5*nd_func(mu,sigma,mu)));
          label("$1-\alpha$", (mu,0.25*nd_func(mu,sigma,mu)));
          label("$\alpha/2$", (left,0),NW);
          label("$\alpha/2$", (right,0),NE);

          //label("$-z^*$",(left,0),NE);
          //dot((right,0));
          label("$z^*$",(right,0),NW);
          dot((right,0));
        \end{asy}
      \end{center}
    \end{overprint}
    \vspace{-1mm}
    \begin{itemize}[<+- | alert@+>]
    \item When the requirements of the Central Limit Theorem are met, the sampling distribution of sample proportions can be approximated by a normal distribution.
    \item A $z$ score associated with a sample proportion has a probability $\alpha/2$ of falling in the {\color<2 | handout:0>{red}right tail portion}.
    \item The $z$ score at the boundary of the right-tail region is commonly denoted by $z^*$.
    \end{itemize}
  \end{block}

  \onslide<+->
  \begin{definition}
    The value $z^*$ is called a \textbf{critical value}.
  \end{definition}
\end{frame}

\begin{frame}[fragile]
  \begin{example}
    Let us find the critical value corresponding to a 95\% confidence level.

    \vspace{1mm}
    \visible<2->{A 95\% confidence interval gives $\alpha=0.05$ and $\alpha/2=0.025$.}

    \vspace{1mm}
    \visible<3->{To find the $z$ value using the inverse normal distribution, we need to know the area to the left of the right tail, $0.025+0.95=0.9750$.}

    \vspace{1mm}
    \visible<4->{Using technology we get $z^* = 1.96$.}
    
    \vspace{1mm}
    \begin{overprint}
      \onslide<1 | handout:0>
      \begin{center}
        \begin{asy}
          size(ex2.x, ex2.y, IgnoreAspect);

          real mu = 0;
          real sigma = 1;

          real left = -1.96;
          real right = 1.96;

          real xmin=mu-3*sigma; real xmax=mu+3*sigma;

          shade_between(mu,sigma,left,right, mediumyellow);
          shade_between(mu,sigma,xmin,left);
          shade_between(mu,sigma,right,xmax);
          draw(normal_dist(mu, sigma, xmin, xmax));

          draw((left,0)--(left,0.53));
          draw((right,0)--(right,0.53));
          draw((left,0.5)--(right,0.5), Arrows);
          label("Confidence Level 95\%",(0,0.5), UnFill);

          label("$1-\alpha=0.95$", (mu,0.4*nd_func(mu,sigma,mu)), invisible);
          label("$\alpha/2=0.025$", (left,nd_func(mu,sigma, left)),NW, invisible);
          label("$\alpha/2=0.025$", (right,nd_func(mu,sigma, right)),NE, invisible);

          //label("$-z^*$",(left,0),S);
          //dot((left,0));

          label("$z^*$",(right,0),S);
          dot((right,0));

          xaxis(Bottom(), RightTicks(new real [] {mu}, size=nan));
        \end{asy}
      \end{center}
      \onslide<2 | handout:1>
      \begin{center}
        \begin{asy}
          size(ex2.x, ex2.y, IgnoreAspect);

          real mu = 0;
          real sigma = 1;

          real left = -1.96;
          real right = 1.96;

          real xmin=mu-3*sigma; real xmax=mu+3*sigma;

          shade_between(mu,sigma,left,right, mediumyellow);
          shade_between(mu,sigma,xmin,left);
          shade_between(mu,sigma,right,xmax);
          draw(normal_dist(mu, sigma, xmin, xmax));

          label("$1-\alpha=0.95$", (mu,0.4*nd_func(mu,sigma,mu)));
          label("$\alpha/2=0.025$", (left,nd_func(mu,sigma, left)),NW);
          label("$\alpha/2=0.025$", (right,nd_func(mu,sigma, right)),NE);

          label("$z^*$",(right,0),S);
          dot((right,0));

          draw((left,0)--(left,0.53));
          draw((right,0)--(right,0.53));
          draw((left,0.5)--(right,0.5), Arrows);
          label("Confidence Level 95\%",(0,0.5), UnFill);

          xaxis(Bottom(), RightTicks(new real [] {mu}, size=nan));
        \end{asy}
      \end{center}
      \onslide<3 | handout:0>
      \begin{center}
        \begin{asy}
          size(ex2.x, ex2.y, IgnoreAspect);

          real mu = 0;
          real sigma = 1;

          real left = -1.96;
          real right = 1.96;

          real xmin=mu-3*sigma; real xmax=mu+3*sigma;

          shade_between(mu,sigma,xmin,right, pink);
          shade_between(mu,sigma,right,xmax);
          draw(normal_dist(mu, sigma, xmin, xmax));

          label("$0.9750$", (mu,0.4*nd_func(mu,sigma,mu)));
          //label("$\alpha/2=0.025$", (right,nd_func(mu,sigma, right)),NE);

          draw((left,0)--(left,0.53), invisible);
          draw((right,0)--(right,0.53), invisible);
          draw((left,0.5)--(right,0.5), invisible, Arrows);
          label("Confidence Level 95\%",(0,0.5), invisible,  UnFill);
          label("$1-\alpha=0.95$", (mu,0.4*nd_func(mu,sigma,mu)), invisible);
          label("$\alpha/2=0.025$", (left,nd_func(mu,sigma, left)),NW, invisible);
          label("$\alpha/2=0.025$", (right,nd_func(mu,sigma, right)),NE, invisible);

          label("$z^*$",(right,0),S);
          dot((right,0));

          xaxis(Bottom(), RightTicks(new real [] {mu}, size=nan));
        \end{asy}
      \end{center}
      \onslide<4- | handout:0>
      \begin{center}
        \begin{asy}
          size(ex2.x, ex2.y, IgnoreAspect);

          real mu = 0;
          real sigma = 1;

          real left = -1.96;
          real right = 1.96;

          real xmin=mu-3*sigma; real xmax=mu+3*sigma;

          shade_between(mu,sigma,xmin,right, pink);
          shade_between(mu,sigma,right,xmax);
          draw(normal_dist(mu, sigma, xmin, xmax));

          label("$0.9750$", (mu,0.4*nd_func(mu,sigma,mu)));
          //label("$\alpha/2=0.025$", (right,nd_func(mu,sigma, right)),NE);

          draw((left,0)--(left,0.53), invisible);
          draw((right,0)--(right,0.53), invisible);
          draw((left,0.5)--(right,0.5), invisible, Arrows);
          label("Confidence Level 95\%",(0,0.5), invisible,  UnFill);
          label("$1-\alpha=0.95$", (mu,0.4*nd_func(mu,sigma,mu)), invisible);
          label("$\alpha/2=0.025$", (left,nd_func(mu,sigma, left)),NW, invisible);
          label("$\alpha/2=0.025$", (right,nd_func(mu,sigma, right)),NE, invisible);
          
          label("$z^*=1.96$",(right,0),S);
          dot((right,0));

          xaxis(Bottom(), RightTicks(new real [] {mu}, size=nan));
        \end{asy}
      \end{center}
    \end{overprint}
  \end{example}

  \onslide<5->
  \begin{block}{Common Confidence Levels}\small
    \begin{center}
      \begin{tabular}{|c|c|c|}\hline
        \textbf{Confidence Level} & $\boldsymbol{\alpha}$ & \textbf{Critical Value}\\\hline
        90\% & 0.10 & 1.645 \\\hline
        95\% & 0.05 & 1.960 \\\hline
        99\% & 0.01 & 2.575 \\\hline
      \end{tabular}
    \end{center}
  \end{block}
\end{frame}

\begin{frame}
  \begin{definition}
    Anytime a point estimate is used to estimate a parameter, there will be some amount of error.\pause

    \vspace{1mm}
    The maximum likely amount of error is called the \textbf{margin of error}\pause

    \vspace{1mm}
    When the point estimate closely follows a normal model, the margin of error is found by multiplying the critical value and the standard error:

    \vspace{-3mm}
    \begin{equation*}
      \begin{aligned}
        z^{*} \cdot SE
      \end{aligned}
    \end{equation*}
  \end{definition}\pause

  \begin{note}
    The margin of error for $\hat{p}$ is:
    \begin{equation*}
      z^{*}\cdot {SE}_{\hat{p}} = z^{*}\sqrt{\dfrac{\hat{p}(1-\hat{p})}{n}}
    \end{equation*}
  \end{note}
\end{frame}

\begin{frame}
  \begin{block}{Procedure for Constructing a Confidence Interval for $p$}
    \begin{enumerate}[<+- | alert@+>]
    \item Verify that $\hat{p}$ is nearly normal:
      \begin{itemize}
      \item The observations are independent.
      \item $n\hat{p}\geq 10$ and $n(1-\hat{p})\geq 10$.
      \end{itemize}
    \item Find the critical value $z^{*}$.
    \item Evaluate the margin of error: $E=z^{*} \sqrt{\dfrac{\hat{p}(1-\hat{p})}{n}}$
    \item Construct the interval: $\interval{\open{\hat{p} - E}}{\open{\hat{p} + E}}$
    \item Interpret the confidence interval in the context of the problem.
    \end{enumerate}
  \end{block}
  \onslide<+->
  \begin{note}
    Statistics software and graphing calculators, can calculate the confidence interval for you.
  \end{note}
  \onslide<+->
  \begin{note}
    Round the confidence interval limits to four digits.
  \end{note}
\end{frame}

\begin{frame}
  \begin{example}
    A 2018 Pew Research poll found that 88.7\% of a random sample of 1000 American adults supported expanding solar power. Let's compute a 90\% confidence interval.\pause

    \vspace{1mm}
    \begin{enumerate}
    \item $n\hat{p}=\pause 1000\cdot 0.887\pause =887\geq 10$\\\pause $n(1-\hat{p})=\pause1000\cdot(1-0.887)\pause=113\geq 10$\pause
    \item The critical value for 90\% confidence interval is $z^{*}$ is 1.645.\pause
    \item The margin of error is

      \vspace{-6mm}
      \begin{equation*}
        E=z^{*}\sqrt{\dfrac{\hat{p}(1-\hat{p})}{n}}\pause
        =1.645\sqrt{\dfrac{0.887(1-0.887)}{1000}}\pause
        = 0.016469\pause
      \end{equation*}
      \vspace{-5mm}
    \item The confidence interval is

      \vspace{-3mm}
      \begin{equation*}
        \begin{matrix}
          ( & \hat{p} - E &,& \hat{p} + E & )\\\pause
          ( & 0.887 - 0.016469 &,& 0.887 + 0.016469 & ) \\\pause
          ( & 0.8705 &,& 0.9035 & ) \\
        \end{matrix}
      \end{equation*}

      \vspace{-2mm}\pause
      \item We are 90\% confident that between 87.1\% and 90.4\% of American adults supported the expansion of solar power in 2018.
    \end{enumerate}
  \end{example}
\end{frame}

\begin{frame}
  \begin{example}
    A 2014 NBC 4 New York/The Wallstreet Journal/Marist Poll found 82\% of the 1,042 NYC adults polled favored a \textquote{mandatory 21-day quarantine for anyone who has come in contact with an Ebola patient}. Lets compute a 95\% confidence interval.\pause

    \vspace{1mm}
    \begin{enumerate}
    \item $n\hat{p}=\pause 1042\cdot 0.82\pause = 854.44\geq 10$\\\pause $n(1-\hat{p})=\pause1042\cdot(1-0.82)\pause=187.56\geq 10$\pause
    \item The critical value for 95\% confidence interval is $z^{*}$ is 1.96.\pause
    \item The margin of error is

      \vspace{-3mm}
      \begin{equation*}
        E=z^{*}\sqrt{\dfrac{\hat{p}(1-\hat{p})}{n}}\pause
        =1.96\sqrt{\dfrac{0.82(1-0.82)}{1042}}\pause
        = 0.023327\pause
      \end{equation*}
      \vspace{-5mm}
    \item The confidence interval is

      \vspace{-3mm}
      \begin{equation*}
        \begin{matrix}
          ( & 0.82 - 0.023327 &,& 0.82 + 0.023327 & ) \\\pause
          ( & 0.7967 &,& 0.8433 & ) \\
        \end{matrix}
      \end{equation*}

      \vspace{-2mm}\pause
      \item We are 95\% confident that between 79.7\% and 84.3\% of NYC adults supported the mandatory Ebola quarantine in 2014.
    \end{enumerate}
  \end{example}
\end{frame}

\begin{frame}
  \begin{block}{Interpreting a Confidence Interval}
    For the confidence interval $\interval{\open{0.8705}}{\open{0.9035}}$ there is one correct interpretation and many creatively incorrect interpretations.
    \begin{description}
    \item[\textbf{Correct:}]<2-| alert@2> \textquote{We are 90\% confident that the interval from 0.8705 to 0.9035 actually does contain the true value of the population proportion $p$.}
    \item[\textbf{Reason:}]<2- | alert@2> The confidence level 95\% refers to the success rate of the process used to estimate the population proportion.
    \item[\textbf{Incorrect:}]<3- | alert@3> \textquote{There is a 90\% chance that the true value of $p$ will fall between 0.8705 and 0.9035.}
    \item[\textbf{Reason:}]<3- | alert@3> The population proportion $p$ is a fixed value.
    \item[\textbf{Incorrect:}]<4- | alert@4> \textquote{90\% of sample proportions will fall between 0.8705 and 0.9035.}
    \item[\textbf{Reason:}]<4- | alert@4> The values 0.8705 and 0.9035 result from one sample, they are not parameters describing the behavior of all samples.
    \end{description}
  \end{block}
\end{frame}

\begin{frame}
  \begin{block}{Analyzing Polls}
    When analyzing results from polls, consider the following:
    \begin{itemize}[<+- | alert@+>]
    \item The sample should be a simple random sample.
    \item The confidence interval should be provided.
    \item The sample size should be provided.
    \item Except for relatively rare cases, the quality of the poll results depends on the sampling method and the size of the sample, but the size of the population is usually not a factor.
    \end{itemize}
  \end{block}
  \onslide<+->
  \begin{block}{Caution}
    Never think that poll results are unreliable if the sample size if a small percentage of the population size.
  \end{block}
\end{frame}

\begin{frame}
  \begin{block}{Finding $\hat{p}$ from a Confidence Interval}
    If you know the confidence interval, $\interval{\open{\text{upper limit}}}{\open{\text{lower limit}}}$, we can calculate the point estimate:
    \begin{equation*}
      \hat{p} = \dfrac{(\text{upper limit})+(\text{lower limit})}{2}
    \end{equation*}
  \end{block}\pause

  \begin{block}{Finding $E$ from a Confidence Interval}
    If you know the confidence interval, $\interval{\open{\text{upper limit}}}{\open{\text{lower limit}}}$, we can calculate the margin of error:
    \begin{equation*}
      E = \dfrac{(\text{upper limit})-(\text{lower limit})}{2}
    \end{equation*}
  \end{block}
\end{frame}

\begin{frame}
  \begin{example}
    The article \textquote{High-Dose Nicotine Patch Therapy}, by Dale, Hurt, et al. (\emph{Journal of the American Medical Association}, Vol 274, No. 17) includes the statement:
    \vspace{-1mm}
    \begin{center}
      Of the 71 subjects, 70\% were abstinent from smoking at 8 weeks\\ (95\% confidence interval [CI], 58\% to 81\%).
    \end{center}\pause

    \vspace{-1mm}
    The point estimate $\hat{p}$ is
    \begin{equation*}
      \begin{aligned}
        \hat{p} &= \dfrac{(\text{upper limit})+(\text{lower limit})}{2} \pause
        = \dfrac{0.81+0.58}{2}\pause
        = 0.695\pause
      \end{aligned}
    \end{equation*}

    The margin of error $E$ is
    \begin{equation*}
      \begin{aligned}
        E &= \dfrac{(\text{upper limit})-(\text{lower limit})}{2} \pause
        = \dfrac{0.81-0.58}{2}\pause
        = 0.115
      \end{aligned}
    \end{equation*}
  \end{example}
\end{frame}

\begin{frame}
  \begin{example}\label{ex:online}
    \vspace{-2mm} % beamer bug
    Let us assume a researcher can find no information about the percentage of adults who make online purchases. This information is extremely important to online stores, so the researcher decides to conduct a survey.\pause

    \vspace{1mm}
    How many adults must be surveyed in order to be 95\% confident that the sample percentage is in error by no more than three percentage points?\pause

    \vspace{1mm}
    We can solve the margin of error equation for $n$:
    \begin{equation*}
      E=z^{*}\sqrt{\dfrac{\hat{p}(1-\hat{p})}{n}} \pause
      \quad\Rightarrow\quad
      n=\dfrac{ {(z^{*})}^2 \hat{p} (1-\hat{p}) }{E^2}
    \end{equation*}\pause
    We have $z^{*}=1.96$ and $E=0.03$, but what about $\hat{p}$?
  \end{example}
\end{frame}

\begin{frame}
  \begin{block}{Sample Size Required to Estimate a Population Proportion}
    The sample must be a simple random sample of independent sample units.\pause

    \vspace{2mm}
    If a reasonable estimate of $\hat{p}$ can be made by using previous samples, a pilot study, or someone's expert knowledge:
    \begin{equation*}
      n=\dfrac{ {(z^{*})}^2 \hat{p} (1-\hat{p}) }{E^2}
    \end{equation*}\pause

    \vspace{-3mm}
    If nothing is known about the value $\hat{p}$:
    \begin{equation*}
      n=\dfrac{ {(z^{*})}^2 0.25 }{E^2}
    \end{equation*}
  \end{block}\pause

  \begin{block}{Rounding}
    If the computed sample size $n$ is not a whole number, round the value of $n$ up to the next larger whole number.
  \end{block}
\end{frame}

\begin{frame}[fragile]
  \begin{block}{Why 0.25?}
    If $\hat{p}=0.5$, then $\hat{p} (1-\hat{p})=0.25$ is the largest possible product.

    \vspace{-2mm}
    \begin{overprint}
      \onslide<2 | handout:0>
      \vspace{1mm}
      To see why, consider the rectangle:
      \begin{center}
        \begin{asy}
          size(blkpq.x, blkpq.y);

          draw((1,1)--(1,0)--(-1,0)--(-1,1)--cycle);

          label("$W$", (0,0),S);
          label("$L$", (1,0.5),E);
        \end{asy}
      \end{center}
      \onslide<3 | handout:0>
      \vspace{1mm}
      To see why, consider the rectangle:
      \begin{center}
        \begin{asy}
          size(blkpq.x, blkpq.y);

          draw(( 1,1)--( 1,0),red);
          draw(( 1,0)--(-1,0),green*0.92);
          draw((-1,0)--(-1,1),red);
          draw((-1,1)--( 1,1),green*0.92);

          label("$W$", (0,0),S,green*0.92);
          label("$L$", (1,0.5),E,mediumred);
        \end{asy}
      \end{center}
      Let us assume that the perimeter of this rectangle is 2, which means:

      \vspace{-2mm}
      \begin{equation*}
        {\color{red!80}2L}+{\color{green!85!black}2W}=2 
        \phantom{\qquad\Rightarrow\qquad
          L+W=1}
      \end{equation*}
      \onslide<4 | handout:0>
      \vspace{1mm}
      To see why, consider the rectangle:
      \begin{center}
        \begin{asy}
          size(blkpq.x, blkpq.y);

          draw(( 1,1)--( 1,0),red);
          draw(( 1,0)--(-1,0),green*0.92);
          draw((-1,0)--(-1,1));
          draw((-1,1)--( 1,1));

          label("$W$", (0,0),S,green*0.92);
          label("$L$", (1,0.5),E,red);
        \end{asy}
      \end{center}
      Let us assume that the perimeter of this rectangle is 2, which means:

      \vspace{-2mm}
      \begin{equation*}
        2L+2W=2 
        \qquad\Rightarrow\qquad
            {\color{red!80}L}+{\color{green!85!black}W}=1
      \end{equation*}
      \onslide<5- | hanout:1>
      \vspace{1mm}
      To see why, consider the rectangle:
      \begin{center}
        \begin{asy}
          size(blkpq.x, blkpq.y);

          filldraw((1,1)--(1,0)--(-1,0)--(-1,1)--cycle, pink,black);
          label("Area",(0,0.5));

          label("$W$", (0,0),S);
          label("$L$", (1,0.5),E);
        \end{asy}
      \end{center}
      Let us assume that the perimeter of this rectangle is 2, which means

      \vspace{-2mm}
      \begin{equation*}
        2L+2W=2 
        \qquad\Rightarrow\qquad
        L+W=1
      \end{equation*}

      We can then write the area of this rectangle as

      \vspace{-2mm}
      \begin{equation*}
        \text{Area} = LW = L(1-L) = -L^2+L
      \end{equation*}

      \visible<6->{Since this is a parabola that opens down, we know that the\\ vertex, $(0.5,0.5)$, is the maximum value.}
    \end{overprint}
  \end{block}
\end{frame}

\begin{frame}
  \begin{example}
    Let us return to Example~\ref{ex:online}, finding the sample size for a survey.\pause

    \vspace{1mm}
    We have $z^{*}=1.96$ and $E=0.03$, but we know nothing about $\hat{p}$.\pause
    \begin{equation*}
      n=\dfrac{ {(z^{*})}^2 0.25 }{E^2} \pause
      =\dfrac{ {(1.96)}^2 0.25 }{{(0.03)}^2}\pause
      =1067.11 \pause
    \end{equation*}
    So, we need at least 1068 adults in our survey.\pause
  \end{example}
  \begin{block}{Caution}
    \begin{enumerate}
    \item Don't make the mistake of using $E=3$ as the margin of error corresponding to \textquote{three percentage points.}\pause
    \item Be sure to substitute correct the critical $z$ score for $z^{*}$ for the confidence level.\pause
    \item Be sure to \emph{round up to the next highest integer}, do not round using the usual rounding rules.
    \end{enumerate}
  \end{block}
\end{frame}
\end{document}
