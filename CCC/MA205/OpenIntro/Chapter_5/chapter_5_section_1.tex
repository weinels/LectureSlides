\documentclass{beamer}
\usepackage[utf8]{inputenc}
\usepackage[english]{babel}
\usepackage{helvet}
\usepackage[T1]{fontenc}
\usepackage[inline]{asymptote}
\usepackage{asy_helper}
\usepackage{slide_helper}
\usepackage{cancel}
\usepackage{tikz}
\usetikzlibrary{shapes.geometric, arrows}
\usepackage{pgfplots}
\pgfplotsset{compat=1.5} 
\usepgfplotslibrary{statistics}
\usetikzlibrary{external}
\tikzexternalize%

\begin{asydef}
guide normal_dist(real mu, real sigma, real xmin, real xmax)
{
	real nd_func(real x) { return 1/sqrt(2*pi*sigma*sigma)*exp((-1*(x-mu)*(x-mu))/(2*sigma*sigma)); }
	return graph(nd_func, xmin, xmax);
}

void shade_below(real mu, real sigma, real b, real xmin, real xmax, pen p=royalblue)
{
	real nd_func(real x) { return 1/sqrt(2*pi*sigma*sigma)*exp((-1*(x-mu)*(x-mu))/(2*sigma*sigma)); }
		
	guide g = graph(nd_func, xmin, b);
	
	filldraw(g -- (b,0) -- cycle, p, black);
	
	draw((xmin,0)--(b,0));
}

void shade_above(real mu, real sigma, real b, real xmin, real xmax, pen p=royalblue)
{
	real nd_func(real x) { return 1/sqrt(2*pi*sigma*sigma)*exp((-1*(x-mu)*(x-mu))/(2*sigma*sigma)); }
		
	guide g = graph(nd_func, b, xmax);
	
	filldraw(g -- (b,0) -- cycle, p, black);
	
	draw((xmax,0)--(b,0));
}

void shade_between(real mu, real sigma, real a, real b, pen p=royalblue)
{
	real nd_func(real x) { return 1/sqrt(2*pi*sigma*sigma)*exp((-1*(x-mu)*(x-mu))/(2*sigma*sigma)); }
		
	guide g = graph(nd_func, a, b);
	
	filldraw((a,0) -- g -- (b,0) -- cycle, p, black);
}

void multiple_nd_curves_example(real std_dev)
{
	size(300, 190, IgnoreAspect);
	
	draw(normal_dist(0, std_dev, -6,6));
	shade_between(0,std_dev,-std_dev,std_dev);
	draw((0,0)--(0,0.45));

	label("$\sigma="+format("%#.2f", std_dev)+"$", (-4.2,0.45), Fill(paleyellow));

	xaxis(Bottom(), RightTicks(new real[] {-6,-4,-2,0,2,4,6}));
	yaxis(Left(), LeftTicks(size=nan),ymin = 0, ymax = 0.5);
}
\end{asydef}

\title[MA205 - Section 5.1]{Point Estimates and Sampling Variability}

\newcommand{\prob}[1]{P\left({#1}\right)}
\newcommand{\jointprob}[3]{\prob{{#1}~\text{#2}~{#3}}}
\newcommand{\condprob}[2]{\prob{{#1}~|~{#2}}}
\newcommand{\comb}[2]{_{#1}C_{#2}}

% Gauss function, parameters mu and sigma
\newcommand\gauss[2]{1/(#2*sqrt(2*pi))*exp(-((x-#1)^2)/(2*#2^2))} % chktex 36


\begin{document}
\begin{frame}
\titlepage
\end{frame}

\begin{frame}
  \begin{definition}
    A \textbf{point estimate} is a single value used to estimate a parameter.
  \end{definition}\pause

  \begin{note}
    The sample proportion, $\hat{p}$, the best point estimate of the population proportion $p$. But, is it a \emph{good} estimate?
  \end{note}\pause

  \begin{definition}
    The difference between a point estimate and the true parameter is called the \textbf{error} in the estimate.
  \end{definition}\pause

  \begin{note}
    In general, there are two sources of error: sampling error and bias.
  \end{note}
\end{frame}

\begin{frame}
  \begin{definition}
    \textbf{Sampling error} describes how much an estimate will tend to vary from one sample to the next.
  \end{definition}\pause

  \begin{example}
    One sample may have $\hat{p}=1\%$ and another sample may have $\hat{p}=3\%$.
  \end{example}\pause

  \begin{note}
    Much of statistics is focused on understanding and quantifying sampling error.
  \end{note}
\end{frame}

\begin{frame}
  \begin{definition}
    \textbf{Bias} describes a systematic tendency to over-estimate or under-estimate the true population value.
  \end{definition}\pause

  \begin{example}
    If a university took a student poll asking about support for a new stadium, they'd get a biased response if they asked:

    \vspace{-2.5mm}
    \begin{center}\small
      \textquote{Do you support your school by supporting funding for the new stadium?}
    \end{center}
  \end{example}\pause

  \begin{note}
    We try to minimize bias by using thoughtful data collection procedures.
  \end{note}
\end{frame}

\begin{frame}
  \begin{example}\label{solar energy}
    \vspace{-2mm} %beamer bug
    Suppose the proportion of American adults who support the expansion of solar energy is $p=0.88$.\pause

    \vspace{1mm}
    If we take a poll of 1000 American adults on this topic, the estimate would not be perfect. But, how close can we expect $\hat{p}$ to be to $p$?\pause

    \vspace{1mm}
    We can simulate such a sample:
    \begin{enumerate}
    \item As of 2021, there are about 258 million adults in America. Let us get 258 million slips of paper and write \textquote{support} on 88\% of them and \textquote{not} on the remaining 12\%.\pause
    \item Mix up the slips and pull out 1000, to represent our sample.\pause
    \item Compute the fraction of the sample that says \textquote{support}.\pause
    \end{enumerate}
  \end{example}

  \begin{note}
    While this method seems silly, a compute can do these steps in a short amount of time.
  \end{note}
\end{frame}

\begin{frame}
  \begin{example}
    I wrote a short program to run this simulation, but one simulation isn't enough to get a sense of the distribution of the point estimates.\pause

    \vspace{1mm}
    So, I ran nine simulations:
    \begin{center}
      \begin{tabular}{cr|cr|cr}
        $\hat{p}$ & Error & $\hat{p}$ & Error & $\hat{p}$ & Error \\\hline
        0.867 & -0.013 & 0.876 & -0.004 & 0.883 & 0.003 \\
        0.889 & 0.009 & 0.887 & 0.007 & 0.874 & -0.006 \\
        0.896 & 0.016 & 0.898 & 0.018 & 0.874 & -0.006 \\
      \end{tabular}
    \end{center}

    Notice that they are all kinda close to $p=0.88$, but there is variation.\pause

    \vspace{1mm}
    The mean of all these $\hat{p}$ values is $0.8827$, which is pretty close to $p$.
  \end{example}\pause

  \begin{definition}
    The \textbf{sampling distribution} is the distribution of sample proportions.
  \end{definition}
\end{frame}

\begin{frame}
  \begin{example}
    Running the simulation 10,000 times gives the histogram:

    \vspace{-3mm}
    \begin{center}
      \begin{tikzpicture}
        \begin{axis}[
            small,
            height=5.0cm,
            width=11.8cm,
            enlarge x limits=false,
            enlarge y limits=false,
            %ybar,
            ymajorgrids=true,
            minor y tick num=1,
            yminorgrids=true,
            ylabel={Frequency},
            xlabel={Sample Proportions},
            xtick={0.84,0.85,0.86,...,1},
            xmin=0.84,
            xmax=0.92,
            ymin=0,
            ymax=900
          ]
          \addplot [ybar, fill=blue!30!white, hist={bins=40}]
          table [y=data] {solar_sampling_dist_n_1000.dat};
          \only<5->{\addplot[red, thick, domain={0.84:0.92},yscale=19.94,samples=150]{\gauss{0.881}{0.0102761}};}
        \end{axis}
      \end{tikzpicture}
    \end{center}\pause

    \vspace{-5mm}
    \begin{description}
    \item[Center]: The center of this distribution is $\bar{x}_{\hat{p}}=0.8799$, which is very close to $p=0.88$.\pause
    \item[Spread]: The standard deviation of this distribution is $s_{\hat{p}}= 0.0102$.\pause\ This is often called the \textbf{standard error}.\pause
    \item[Shape]: This distribution is approximately normal.
    \end{description}
  \end{example}
\end{frame}

\begin{frame}
  \begin{example}
    What if we used a much smaller sample size of $n=50$?\pause
    
    \vspace{0mm}
    \begin{description}
    \item[Center]: The center of this distribution is $\bar{x}_{\hat{p}}=0.8791$, which is still very close to $p=0.88$.\pause
    \item[Spread]: The standard deviation of this distribution is $s_{\hat{p}}= 0.0462$, which is much bigger.\pause
    \end{description}
  \end{example}

  \begin{note}
    This highlights an important property: a bigger sample tends to provide a more precise point estimate than a smaller sample.
  \end{note}\pause

  \begin{note}
    In real-world applications, we never actually observe the sampling distribution, yet it is useful to always think of a point estimate as coming from a hypothetical distribution.
  \end{note}
\end{frame}

\begin{frame}
  \begin{block}{Central Limit Theorem}
    When observations are independent and the sample size, $n$, is sufficiently large, the sample proportions $\hat{p}$ will tend to follow a normal distribution with the following mean and standard deviation:

    \vspace{-1mm}
    \begin{equation*}
      \begin{aligned}
        \mu_{\hat{p}}=p
        \qquad\text{and}\qquad
        \sigma_{\hat{p}}={SE}_{\hat{p}}=\sqrt{\dfrac{p(1-p)}{n}}
      \end{aligned}
    \end{equation*}\pause

    \vspace{-1mm}
    In order for the Central Limit Theorem to hold, the sample size is typically considered sufficiently large when
    \vspace{-2mm}
    \begin{equation*}
      \begin{aligned}
        np\geq 10
        \qquad\text{and}\qquad
        n(1-p)\geq 10
      \end{aligned}
    \end{equation*}

    \vspace{-2mm}
    which is called the \textbf{success-failure conditions}.
  \end{block}\pause

  \begin{note}
    The Central Limit Theorem is incredibly important, and provides a foundation for much of statistics.
  \end{note}
\end{frame}

\begin{frame}
  \begin{example}\label{solar CLT}
    \vspace{-2mm}% beamer bug
    In Example~\ref{solar energy}, we had a sample size of $n=1000$ and $p=0.88$.\pause

    \vspace{1mm}
    Before we can apply the Central Limit Theorem, we need to check the success-failure conditions:
    \begin{equation*}
      \begin{aligned}
        np &= \pause
        1000\cdot 0.88\pause
        = 880\pause
        \geq 10~{\color{green!70!black}\checkmark} \\\pause
        n(1-p) &= \pause
        1000(1-0.88)\pause
        = 1000\cdot 0.12\pause
        = 120\pause
        \geq~{\color{green!70!black}\checkmark}
      \end{aligned}
    \end{equation*}\pause

    Applying the Central Limit Theorem gives:
    \begin{equation*}
      \begin{aligned}
        \mu_{\hat{p}} &=\pause
        p = 0.88 \\\pause
        {SE}_{\hat{p}} &= \pause
        \sqrt{\dfrac{p(1-p)}{n}}\pause
        = \sqrt{\dfrac{0.88(1-0.12)}{1000}}\pause
        = 0.0103
      \end{aligned}
    \end{equation*}\pause
    This is very close to the observed standard error, $0.0102$.
  \end{example}
\end{frame}

\begin{frame}
  \begin{block}{How to Verify Sample Observations are Independent}
    \begin{itemize}
    \item Subjects in an experiment are considered independent if they undergo random assignment to the treatment groups.\pause
    \item If the observations are from a simple random sample, then they are independent.\pause
    \item If a sample is from a seemingly random process, e.g.\  an occasional error on an assembly line, checking independence is more difficult. In this case, use your best judgment.
    \end{itemize}
  \end{block}\pause

  \begin{note}
    If a sample is larger than 10\% of the population, the methods we discuss tend to overestimate the sampling error slightly. In these cases more advanced methods are needed.
  \end{note}
\end{frame}

\begin{frame}[fragile]
  \begin{example}
    Using $n=1000$ and $p=0.88$ from Example~\ref{solar energy}, let us find out how often $\hat{p}$ is within 0.02 (2\%) of the population value $p=0.88$.\pause

    \vspace{1mm}
    In Example~\ref{solar CLT}, we applied the Central Limit Theorem, getting $\mu_{\hat{p}} = 0.88$ and ${SE}_{\hat{p}}=0.0103$.\pause

    \vspace{1mm}
    We start by calculating the $z$-values:

    \vspace{-5mm}
    \begin{equation*}
      \begin{aligned}
        z_{0.86} &= \pause
        \dfrac{0.86-0.88}{0.0103}\pause
        = -1.942 \pause
        \quad\text{and}\quad
        z_{0.90} = \pause
        \dfrac{0.90-0.88}{0.0103}\pause
        = 1.942
      \end{aligned}
    \end{equation*}\pause

    \vspace{-6mm}
    \begin{center}
      \begin{asy}
        size(300, 70, IgnoreAspect);

        shade_between(0,1,-1.94175,1.94175);
        draw(normal_dist(0, 1, -4.5, 4.5));

        label("$z_{0.86}$",(-1.94175,0),SE,blue);
        dot((-1.94175,0));
        label("$z_{0.90}$",(1.94175,0),SW,blue);
        dot((1.94175,0));

        xaxis("$z$",Bottom(), RightTicks(new real [] {-4,-3,-2,-1,0,1,2,3,4}, size=nan));
      \end{asy}
    \end{center}\pause

    \vspace{-4mm}
    Using technology gives:

    \vspace{-4mm}
    \begin{equation*}
      \begin{aligned}
        \prob{-1.94175 \leq z \leq 1.94175} = 0.947833
      \end{aligned}
    \end{equation*}\pause

    \vspace{-6mm}
    We expect $\hat{p}$ to be within 0.02 of 0.88 about 94.78\% of the time.
  \end{example}
\end{frame}

\begin{frame}
  \begin{example}
    We do not actually know the population proportion unless we conduct a full census of the entire population.\pause

    \vspace{1mm}
    The value $p=0.88$ was based on a Pew Research poll of 1000 adults that found $\hat{p}=0.887$ of them favored expanding solar energy.\pause

    \vspace{1mm}
    A question the researchers might have asked is:

    \vspace{-3mm}
    \begin{center}\small
      \textquote{Does the sample proportion from the poll approximately\\ follow a normal distribution?}
    \end{center}\pause

    \vspace{-4mm}
    %We can check the requirements of the Central Limit Theorem:
    \begin{description}
    \item[Independence] Pew Research is a well known non-profit think tank, so we can believe that the poll is a simple random sample, and hence the observations are independent.\pause
    \item[Success-Failure Conditions] Since we don't actually know $p$, the next best thing we have is $\hat{p}$.

      \vspace{-6.5mm}
      \begin{equation*}
        \begin{aligned}
          n\hat{p} &= \pause
          1000\cdot 0.887\pause
          =887\pause
          \geq 10~{\color{green!70!black}\checkmark} \\\pause
          n(1-\hat{p}) &=\pause
          1000(1-0.887)\pause
          = 1000\cdot 0.113 \pause
          = 113\pause
          \geq 10~{\color{green!70!black}\checkmark}
        \end{aligned}
      \end{equation*}\pause

      \vspace{-4.5mm}
      Because $n\hat{p}$ and $n(1-\hat{p})$ are both well above 10, we can conclude that $\hat{p}$ is a reasonable substitute for $p$.
    \end{description}
  \end{example}
\end{frame}

\begin{frame}
  \begin{block}{Substitution Approximation}
    When $np$ and $n(1-p)$ are much larger than 10, we can use $\hat{p}$ in place of $p$ and the Centeral Limit Theorem becomes:
    \begin{equation*}
      \begin{aligned}
        \mu_{\hat{p}}&=p\approx \hat{p} \\
        {SE}_{\hat{p}}&=\sqrt{\dfrac{p(1-p)}{n}}\approx \sqrt{\dfrac{\hat{p}(1-\hat{p})}{n}}
      \end{aligned}
    \end{equation*}
  \end{block}\pause

  \begin{example}
    For $n=1000$, $p-0.88$, $\hat{p}=0.887$:

    \vspace{-2mm}
    \begin{equation*}
      \begin{aligned}
        {SE}_{\hat{p}}&=\pause
        \sqrt{\dfrac{p(1-p)}{n}} \pause
        = \sqrt{\dfrac{0.88(1-0.88)}{1000}}\pause
        = 0.010276 \\\pause
        {SE}_{\hat{p}}&\approx \pause
        \sqrt{\dfrac{\hat{p}(1-\hat{p})}{n}}\pause
        = \sqrt{\dfrac{0.887(1-0.887)}{1000}}\pause
        = 0.010012
      \end{aligned}
    \end{equation*}\pause
    These values are the same to three decimal places, so using the substitution approximation won't make a major difference.
  \end{example}
\end{frame}

\begin{frame}
  \begin{block}{Trends in the Sampling Distribution}
    \begin{itemize}
    \item The smaller either $np$ or $n(1-p)$ is, the more discrete the sampling distribution is.\pause
    \item When $np$ or $n(1-p)$ is smaller than 10, the skew in the distribution is more pronounced.\pause
      \begin{itemize}
      \item If $p$ is close to 0, the distribution will be more right skewed.\pause
      \item If $p$ is close to 1, the distribution will be more left skewed.\pause
      \item if $p$ is close to 0.5, the distribution will be more symmetric.\pause
      \end{itemize}
    \item The larger both $np$ and $n(1-p)$ are, the more normal the sampling distribution is.\pause
      \begin{itemize}
      \item This may be harder to see for larger sample sizes, as the variability also becomes smaller.\pause
      \end{itemize}
    \item When $np$ and $n(1-p)$ are both very large, the distributions discreteness is hardly evident, and the distribution looks much more like a normal distribution.
    \end{itemize}
  \end{block}
\end{frame}
\end{document}
