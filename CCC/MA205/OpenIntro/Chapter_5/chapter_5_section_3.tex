\documentclass{beamer}
\usepackage[utf8]{inputenc}
\usepackage[english]{babel}
\usepackage{helvet}
\usepackage[T1]{fontenc}
\usepackage[inline]{asymptote}
\usepackage{asy_helper}
\usepackage{slide_helper}
\usepackage{cancel}
\usepackage{tikz}
\usetikzlibrary{shapes.geometric, arrows}
\usepackage{pgfplots}
\pgfplotsset{compat=1.5} 
\usepgfplotslibrary{statistics}
\usetikzlibrary{external}
\tikzexternalize%

\begin{asydef}
  real nd_func(real mu, real sigma, real x)
  {
    return 1/sqrt(2*pi*sigma*sigma)*exp((-1*(x-mu)*(x-mu))/(2*sigma*sigma));
  }
  
  guide normal_dist(real mu, real sigma, real xmin, real xmax)
  {
    real nd(real x) { return nd_func(mu, sigma, x); }
    return graph(nd, xmin, xmax);
  }

  void shade_below(real mu, real sigma, real b, real xmin, real xmax, pen p=royalblue)
  {
    real nd_func(real x) { return 1/sqrt(2*pi*sigma*sigma)*exp((-1*(x-mu)*(x-mu))/(2*sigma*sigma)); }
    
    guide g = graph(nd_func, xmin, b);
    
    filldraw(g -- (b,0) -- cycle, p, black);
    
    draw((xmin,0)--(b,0));
  }

  void shade_above(real mu, real sigma, real b, real xmin, real xmax, pen p=royalblue)
  {
    real nd_func(real x) { return 1/sqrt(2*pi*sigma*sigma)*exp((-1*(x-mu)*(x-mu))/(2*sigma*sigma)); }
    
    guide g = graph(nd_func, b, xmax);
    
    filldraw(g -- (b,0) -- cycle, p, black);
    
    draw((xmax,0)--(b,0));
  }

  void shade_between(real mu, real sigma, real a, real b, pen p=royalblue)
  {
    real nd_func(real x) { return 1/sqrt(2*pi*sigma*sigma)*exp((-1*(x-mu)*(x-mu))/(2*sigma*sigma)); }
    
    guide g = graph(nd_func, a, b);
    
    filldraw((a,0) -- g -- (b,0) -- cycle, p, black);
  }

  void multiple_nd_curves_example(real std_dev)
  {
    size(300, 190, IgnoreAspect);
    
    draw(normal_dist(0, std_dev, -6,6));
    shade_between(0,std_dev,-std_dev,std_dev);
    draw((0,0)--(0,0.45));

    label("$\sigma="+format("%#.2f", std_dev)+"$", (-4.2,0.45), Fill(paleyellow));

    xaxis(Bottom(), RightTicks(new real[] {-6,-4,-2,0,2,4,6}));
    yaxis(Left(), LeftTicks(size=nan),ymin = 0, ymax = 0.5);
  }
\end{asydef}

\begin{asydef}
  pair ex2 = (250, 70);
  pair blkpq = (90, 90);
\end{asydef}

\newcommand{\satisfied}[0]{{\color{green!70!black}\checkmark}}

\title[MA205 - Section 5.3]{Hypothesis Testing For A Proportion}

\newcommand{\prob}[1]{P\left({#1}\right)}
\newcommand{\jointprob}[3]{\prob{{#1}~\text{#2}~{#3}}}
\newcommand{\condprob}[2]{\prob{{#1}~|~{#2}}}
\newcommand{\comb}[2]{_{#1}C_{#2}}

\begin{document}
\begin{frame}
  \titlepage
\end{frame}

\begin{frame}
  \begin{example}\label{infant vaccination}
    \vspace{-2mm} %beamer spacing is bugged
    The following question comes from a book written by Hans Rosling, Anna Rosling R\"{o}nnlund, and Ola Rosling called \emph{Factfulness}.

    \vspace{1mm}
    \emph{How many of the world's 1 year old children today have been vaccinated against some disease?}

    \vspace{-2.5mm}
    \begin{center}
      (a) 20\%
      \qquad\quad
      (b) 50\%
      \qquad\quad
      (c) 80\%
    \end{center}

    \vspace{-2.5mm}
    \question{What is your answer (or guess)?}\pause
    \answer{The correct answer is 80\%.}\pause
  \end{example}
  
  \begin{note}
    If we take a multiple choice test, then we might like to distinguish between the two possibilities:
    \begin{itemize}
    \item People never learn these particular topics and their responses are simply equivalent to random guessing.
    \item People have knowledge that helps them do better than random guessing, or perhaps, they have false knowledge that leads them to actually do worse than random guessing.
    \end{itemize}
  \end{note}
\end{frame}

\begin{frame}
\begin{definition}
In statistics, a \textbf{hypothesis} is a claim or statement about some property of a population.
\end{definition}\pause

\begin{definition}
A \textbf{hypothesis test} (or \textbf{test of significance}) is a procedure for testing a claim about some property of a population.
\end{definition}\pause

\begin{definition}
The \textbf{null hypothesis} ($H_0$) represents a skeptical perspective or a claim to be tested.
\end{definition}\pause

\begin{definition}
The \textbf{alternative hypotheses} ($H_A$) represents an alternative claim under consideration and is often represented by a range of possible parameter values.
\end{definition}
\end{frame}

\begin{frame}
  \begin{example}
    A US court considers two possible claims about a defendant: they are either innocent or guilty.

    \vspace{1mm}
    \question{Which would be $H_0$ and which $H_A$?}\pause
    \answer{The jury considers whether the evidence is so convincing that there is no reasonable doubt regarding the person's guilt.

      \vspace{-3mm}
      \begin{equation*}
        \begin{aligned}
          H_0 &: \text{Innocence} \\
          H_A &: \text{Guilty}
        \end{aligned}
      \end{equation*}
      \vspace{-3mm}}
  \end{example}\pause

  \begin{note}
    Just because the jurors leave unconvinced of guilt beyond a reasonable doubt, does not mean they believe the defendant is innocent.\pause

    \vspace{1mm}
    We may reject or fail to reject the alternative hypothesis, but we typically never accept the null hypothesis as true.
  \end{note}
\end{frame}

\begin{frame}
  \begin{example}
    In Example~\ref{infant vaccination} we has three choices for the question:
    
    \vspace{1mm}
    \emph{How many of the world's 1 year old children today have been vaccinated against some disease?}

    \vspace{-2.5mm}
    \begin{center}
      (a) 20\%
      \qquad\quad
      (b) 50\%
      \qquad\quad
      (c) 80\%
    \end{center}\pause

    \vspace{-2mm}
    \question{If someone has pick an answer completely at random, what would the probability of selecting the correct answer be?}\pause
    \answer{$p = \dfrac{1}{3} = 0.333$}\pause

    \vspace{1mm}
    So, if the null hypothesis is that someone was guessing at random, the alternative would be they were not guessing.\pause

    \vspace{-3mm}
    \begin{equation*}
      \begin{aligned}
        H_0 &: p = 0.333 \\
        H_A &: p \neq 0.333
      \end{aligned}
    \end{equation*}
  \end{example}\pause

  \begin{definition}
    The value we are comparing the parameter to is called the \textbf{null value}.
  \end{definition}
\end{frame}

\begin{frame}
  \begin{example}
    It may seem incredibly unlikely that the proportion of people who get the correct answer is \emph{exactly} 33.3\%.

    \vspace{1mm}
    \question{If we don't believe the null hypothesis, should we simply reject it?}\pause
    \answer{No. While we may not believe the null hypotheses, we need strong evidence before we reject the null hypothesis and conclude something more interesting.\pause

      \vspace{1mm}
      Even if we don't believe the proportion is \emph{exactly} 33.3\%, that doesn't tell us anything useful about if people do better or worse than random guessing.}
  \end{example}
\end{frame}

\begin{frame}
  \begin{note}
    We will be using the \dataset{rosling\_responses} data set to evaluate the hypothesis test evaluating whether college-educated adults who get the question about infant vaccination correct is different from 33.3\%.\pause

    \vspace{1mm}
    This data set summarizes the answers of 50 college-educated adults. Of these 50 adults, 24\% of respondents got the question correct.
  \end{note}
\end{frame}

\begin{frame}
  \begin{example}
    For this data set, we have $n=50$ and $\hat{p}=0.24$. Lets see if it's reasonable to construct a confidence interval.\pause

    \vspace{-3mm}
    \begin{equation*}
      \begin{aligned}
        np &=\pause 50\cdot 0.24\pause = 12\pause \geq 10~\satisfied \\\pause
        n(1-p) &=\pause 50(1-0.24)\pause = 50\cdot 0.76\pause = 38\pause \geq 10~\satisfied
      \end{aligned}
    \end{equation*}\pause

    \vspace{-3mm}
    The conditions are met, so let's construct a 95\% confidence interval.

    \vspace{-6mm}
    \begin{equation*}
      \begin{aligned}
        \hat{p} \pm z^* \sqrt{\dfrac{\hat{p}(1-\hat{p})}{n}}\pause
        = 0.24 \pm 1.96\sqrt{\dfrac{0.24(1-0.24)}{5}}\pause
        = 0.24 \pm 0.118381\pause
      \end{aligned}
    \end{equation*}

    \vspace{-2mm}
    The confidence interval is $\interval{\open{0.122}}{\open{0.358}}$, which means we are 95\% confident that the proportion of college-educated adults to correctly answer the question is between 12.2\% and 35.8\%.\pause

    \vspace{1mm}
    Since the null hypothesis of $p=0.333$ falls in this range, we cannot say the null hypotheses is implausible. We fail to reject the null hypothesis.

    \vspace{1mm}
    Just because we conclude that it's plausible that $p=0.333$ does not mean we actually accept the null hypothesis.
  \end{example}
\end{frame}

\begin{frame}
  \begin{example}
    \emph{There are 2 billion children in the world today aged 0-15, how many children will there be in 2100 according to the United Nations?}

    \vspace{-2mm}
    \begin{center}
      (a) 4 billion
      \qquad
      (b) 3 billion
      \qquad
      (c) 2 billion
    \end{center}\pause

    \vspace{-2mm}
    \question{What are the null and alternative hypotheses?}
    \answer{%
      
      \vspace{-5mm}
      \begin{equation*}
        \begin{aligned}
          H_0 : p=0.333
          \qquad
          H_A : p\neq 0.333
        \end{aligned}
      \end{equation*}
    }\pause

    \vspace{-6mm}
    If we take a larger sample of 228 college-educated adults, 34 (14.9\%) selected the correct answer: (c) 2 billion\pause

    \vspace{1mm}
    Let's construct a 95\% confidence interval.\pause

    \vspace{-6mm}
    \begin{equation*}
      \begin{aligned}
        \hat{p} \pm z^* \sqrt{\dfrac{\hat{p}(1-\hat{p})}{n}}\pause
        = 0.149 \pm 1.96 \sqrt{\dfrac{0.149(0.851)}{228}}\pause
        = 0.149 \pm 0.04622
      \end{aligned}
    \end{equation*}\pause

    \vspace{-4mm}
    The confidence interval is $\interval{\open{0.103}}{\open{0.195}}$, which means we are 95\% confident that the proportion of college-educated adults that answered the question correctly is between 10.3\% and 19.5\% Since $p=0.333$ is implausible, we reject the null hypothesis.
  \end{example}
\end{frame}

\begin{frame}
  \begin{note}
    In the last example, because the confidence interval $\interval{\open{0.103}}{\open{0.195}}$ is below p=0.333, we can conclude that college-educated adults do worse than random guessing on this question.
  \end{note}\pause

  \begin{note}
    This shows a general trend, in that many people are more pessimistic about progress than reality suggests.
  \end{note}\pause

  \begin{note}
    It is possible to come to difficult conclusions depending on the confidence level you use.\pause

    \vspace{1mm}
    A 99.9\% confidence interval for the last example is $\interval{\open{0.072}}{\open{0.227}}$, which is wider than the 95\% confidence interval.\pause

    \vspace{1mm}
    This means you always need to write what confidence level you used.
  \end{note}
\end{frame}

%% \begin{frame}
%% \begin{note}
%% If you are conducting a study and want to use a hypothesis test to \emph{support} your claim, your claim must be worded such that it becomes the alternative hypothesis and can be expressed using only the symbols $>$, $<$, or $\neq$.
%% \end{note}\pause

%% \begin{block}{Caution}
%% You \emph{never} support a claim that a parameter is equal to a specified value.
%% \end{block}
%% \end{frame}

%% \begin{frame}
%% \begin{definition}
%% The \textbf{significance level $\boldsymbol{\alpha}$} for a hypothesis test is the probability value used as the cutoff for determining when the sample evidence constitutes significant evidence against the null hypothesis. 
%% \end{definition}\pause

%% \begin{note}
%% By its nature, the significance level $\alpha$ is the probability of mistakely rejecting the null hypothesis when it is true:
%% \begin{equation*}
%% \text{Significance level $\alpha$} = \prob{\text{rejecting $H_0$ when $H_0$ is true}}
%% \end{equation*}
%% \end{note}\pause

%% \begin{note}
%% The significance level $\alpha$ is the same $\alpha$ we talked about in Chapter 7, when discussing confidence intervals.
%% \end{note}
%% \end{frame}

%% \begin{frame}
%% \begin{definition}
%% The \textbf{test statistic} is a value used in making a decision about the null hypotheses. It is found by converting the sample statistic to a score with the assumption that the null hypothesis is true.
%% \end{definition}\pause

%% \begin{block}{Test Statistic for Proportion $p$}
%% \begin{description}
%% \item[\textbf{Sampling Distribution:}] Normal ($z$)
%% \item[\textbf{Requirements:}] $np\geq5$ and $nq\geq5$
%% \item[\textbf{Test Statistic:}] $z=\dfrac{\hat{p}-p}{\sqrt{\dfrac{pq}{n}}}$
%% \end{description}
%% \end{block}
%% \end{frame}

%% \begin{frame}
%% \begin{example}
%% In Example~\ref{exp_drones} we made a claim about the population proportion $p$, where we have $n=1009$ and $x=545$.\pause

%% \vspace{2mm}
%% This means we have
%% \begin{equation*}
%% \hat{p}=\dfrac{x}{n}\pause
%% =\dfrac{545}{1009}\pause
%% =0.540
%% \end{equation*}\pause

%% The null hypotheses is $\nullhypothesis{p=0.5}$.\pause

%% \vspace{2mm}
%% Which means we are working form the assumption that $p=0.5$ and so $q=1-p=0.5$.\pause

%% \vspace{2mm}
%% The test statistic is then
%% \begin{equation*}
%% z=\dfrac{\hat{p}-p}{\sqrt{\dfrac{pq}{n}}}\pause
%% =\dfrac{0.540-0.5}{\sqrt{\dfrac{0.5\cdot 0.5}{1009}}}\pause
%% =2.54
%% \end{equation*}
%% \end{example}
%% \end{frame}

%% \begin{frame}
%% \begin{block}{Test Statistic for Mean $\mu$}
%% \begin{description}
%% \item[\textbf{Sampling Distribution:}] Student $t$
%% \item[\textbf{Requirements:}] Both of the following:
%% \begin{itemize}
%% \item $\sigma$ not known.
%% \item Normally distributed or $n>30$.
%% \end{itemize}
%% \item[\textbf{Test Statistic:}] $t=\dfrac{\bar{x}-\mu}{\dfrac{s}{\sqrt{n}}}$
%% \end{description}
%% \end{block}\pause

%% \begin{block}{Test Statistic for Mean $\mu$}
%% \begin{description}
%% \item[\textbf{Sampling Distribution:}] Normal ($z$)
%% \item[\textbf{Requirements:}] Both of the following:
%% \begin{itemize}
%% \item $\sigma$ known.
%% \item Normally distributed or $n>30$.
%% \end{itemize}
%% \item[\textbf{Test Statistic:}] $z=\dfrac{\bar{x}-\mu}{\dfrac{\sigma}{\sqrt{n}}}$
%% \end{description}
%% \end{block}
%% \end{frame}

%% \begin{frame}[fragile]
%% \begin{definition}
%% The \textbf{critical region} (or \textbf{rejection region}) is the area corresponding to all values of the test statistic that cause us to reject the null hypothesis.
%% \end{definition}\pause

%% \begin{definition}
%% In a hypothesis test, the \textbf{$\boldsymbol{P}$-value} is the probability of getting a value of the test statistic that is at least as extreme as the test statistic obtained from the sample data, assuming that the null hypothesis is true.
%% \end{definition}\pause

%% \begin{block}{Caution}
%% Be careful not to confuse the notation.
%% \begin{center}
%% \begin{tabular}{ll}
%% \textbf{$\boldsymbol{P}$-value} & The probability of a test statistic at least as extreme as \\
%% &the one obtained.\\
%% $\boldsymbol{p}$ & The population proportion.\\
%% $\boldsymbol{\hat{p}}$ & The sample proportion.
%% \end{tabular}
%% \end{center}
%% \end{block}
%% \end{frame}

%% \begin{frame}[fragile]
%% \begin{block}{Two-tailed Test $\left(\althypothesis{\neq}\right)$}
%% The critical region is in the two extreme regions under the curve.
%% \begin{center}
%% \begin{asy}
%% size(crit.x,crit.y,IgnoreAspect);
%% real xmin=-4; 
%% real xmax=4;

%% draw(normal_dist(0,1,xmin,xmax));
%% shade_between(0,1,xmin,a,orange);
%% shade_between(0,1,b,xmax,orange);

%% real y = nd_func(0,1,0)/5;
%% draw((a-0.25,y)--(b+0.25,y),Arrows(TeXHead));
%% label("$P$-value",(0,y),UnFill);

%% xaxis();
%% \end{asy}
%% \end{center}
%% \end{block}\pause

%% \begin{block}{Left-tailed Test $\left(\althypothesis{<}\right)$}
%% The critical region is in the extreme left region under the curve.
%% \begin{center}
%% \begin{asy}
%% size(crit.x,crit.y,IgnoreAspect);
%% real xmin=-4; 
%% real xmax=4;

%% draw(normal_dist(0,1,xmin,xmax));
%% shade_between(0,1,xmin,a,orange);

%% real y = nd_func(0,1,0)/5;
%% draw((a-0.25,y)--(0,y),Arrows(TeXHead));
%% label("$P$-value",(0,y),UnFill);

%% xaxis();
%% \end{asy}
%% \end{center}
%% \end{block}\pause

%% \begin{block}{Right-tailed Test $\left(\althypothesis{>}\right)$}
%% The critical region is in the extreme right region under the curve.
%% \begin{center}
%% \begin{asy}
%% size(crit.x,crit.y,IgnoreAspect);
%% real xmin=-4; 
%% real xmax=4;

%% draw(normal_dist(0,1,xmin,xmax));
%% shade_between(0,1,b,xmax,orange);

%% real y = nd_func(0,1,0)/5;
%% draw((0,y)--(b+0.25,y),Arrows(TeXHead));
%% label("$P$-value",(0,y),UnFill);

%% xaxis();
%% \end{asy}
%% \end{center}
%% \end{block}
%% \end{frame}

%% \begin{frame}
%% \begin{block}{Decision Criteria}
%% \begin{itemize}
%% \item If $P$-value $\leq\alpha$, reject $H_0$.
%% \item If $P$-value $\geq\alpha$, fail to reject $H_0$.
%% \end{itemize}
%% \end{block}\pause

%% \begin{example}
%% In Example~\ref{exp_drones} we made a claim about the population proportion $p$, where we have $n=1009$ and $x=545$.\pause

%% \vspace{2mm}
%% The alternative hypothesis is $\althypothesis{p>0.5}$, so this is a right-tailed test.\pause

%% \vspace{2mm}
%% The test statistic is $z=2.54$ and the area to the right of $z$ is 0.0055.\pause

%% \vspace{2mm}
%% If we are working with a significance level $\alpha=0.05$, so we reject the null hypothesis.
%% \end{example}\pause

%% \begin{note}
%% Technology will compute $P$-values for you.
%% \end{note}
%% \end{frame}

%% \begin{frame}
%% \begin{block}{Restate the Decision Using Nontechnical Terms}
%% After you have decided to reject or not reject the null hypothesis, you need to restate the decision in terms that a layperson can understand.
%% \end{block}\pause

%% \begin{example}
%% In Example~\ref{exp_drones} we restate the decision to reject the null hypothesis as:
%% \begin{center}
%% \textquote{There is sufficient evidence to support the claim that the majority of consumers are uncomfortable with drone deliveries.}
%% \end{center}
%% \end{example}
%% \end{frame}

%% \begin{frame}
%% \begin{block}{Helpful Wording}
%% Original claim does not include equality and you reject $H_0$:
%% \begin{center}\small
%% \textquote{There is sufficient evidence to support the claim that \ldots (claim).}
%% \end{center}\pause

%% Original claim does not include equality and you fail to reject $H_0$:
%% \begin{center}\small
%% \textquote{There is not sufficient evidence to support the claim that \ldots (claim).}
%% \end{center}\pause

%% Original claim includes equality and you reject $H_0$:
%% \begin{center}\small
%% \textquote{There is sufficient evidence to warrant rejection the claim that \ldots (claim).}
%% \end{center}\pause

%% Original claim includes equality and you fail to reject $H_0$:
%% \begin{center}\small
%% \textquote{There is not sufficient evidence to warrant rejection the claim that \ldots (claim).}
%% \end{center}
%% \end{block}\pause

%% \begin{block}{Caution}
%% We say \textquote{fail to reject the null hypothesis} instead of \textquote{accept the null hypothesis.}
%% \end{block}
%% \end{frame}

%% \begin{frame}
%% \begin{block}{Procedure for Hypothesis Tests Flow Chart}
%% Page 360 in your textbook contains a summary of all the steps.
%% \end{block}\pause

%% \begin{note}
%% A confidence interval estimate of a population parameter contains the likely values of that parameter. 

%% \vspace{2mm}
%% We should therefore reject a claim that the population parameter has a value that is not included in the confidence interval.
%% \end{note}
%% \end{frame}

%% \begin{frame}
%% \begin{definition}
%% A \textbf{type I error} is the mistake of rejecting the null hypothesis when it is actually true.

%% \vspace{2mm}
%% The symbol $\alpha$ is used to represent the probability of a type I error.
%% \begin{equation*}
%% \alpha = \prob{\text{type I error}} = \prob{\text{rejecting $H_0$ when $H_0$ is true}}
%% \end{equation*}
%% \end{definition}\pause

%% \begin{definition}
%% A \textbf{type II error} is the mistake of failing to reject the null hypothesis when it is actually false.

%% \vspace{2mm}
%% The symbol $\beta$ is used to represent the probability of a type II error.
%% \begin{equation*}
%% \beta = \prob{\text{type II error}} = \prob{\text{failing to reject $H_0$ when $H_0$ is false}}
%% \end{equation*}
%% \end{definition}\pause

%% \begin{block}{Describing Type I and Type II Errors}
%% When wording a statement representing a type I / II error, be sure that the conclusion addresses the original claim, which may or may not be $H_0$.
%% \end{block}
%% \end{frame}

%% \begin{frame}
%% \begin{example}
%% Consider the claim that a medical procedure designed to increase the likelihood of a baby girl is effective, which means the probability of a baby girl is $p>0.5$.\pause

%% \vspace{2mm}
%% Given the following null and alternative hypotheses
%% \begin{equation*}
%% \begin{aligned}
%% \nullhypothesis{p=0.5} \\
%% \althypothesis{p>0.5}
%% \end{aligned}
%% \end{equation*}
%% what is a statement that describes a type I error?\pause

%% \vspace{2mm}
%% In reality $p=0.5$, but sample evidence leads us to conclude the $p>0.5$. That is, we conclude that the medical procedure is effective when it reality it has no effect.\pause

%% \vspace{2mm}
%% What is a statement that describes a type II error?\pause

%% \vspace{2mm}
%% In reality $p>0.5$, but we fail to support that conclusion. That is, we conclude that the medical procedure has no effect, when it really is effective in increasing the likelihood of a baby girl.
%% \end{example}
%% \end{frame}
\end{document}
