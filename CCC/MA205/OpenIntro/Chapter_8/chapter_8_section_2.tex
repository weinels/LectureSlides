\documentclass{beamer}
\usepackage[utf8]{inputenc}
\usepackage[english]{babel}
\usepackage{helvet}
\usepackage[T1]{fontenc}
\usepackage[inline]{asymptote}
\usepackage{asy_helper}
\usepackage{slide_helper}
\usepackage{multirow}
\usepackage{cancel}
\usepackage{tikz}
\usetikzlibrary{shapes.geometric, arrows}
\usepackage{pgfplots}
\usepackage{pgfplotstable}
\pgfplotsset{compat=1.5} 
\usepgfplotslibrary{statistics}
\usetikzlibrary{external}
\tikzexternalize%

\pgfplotsset{%
  filter discard warning=false,
    discard if not/.style 2 args={%
        x filter/.code={%
            \edef\tempa{\thisrow{#1}}%
            \edef\tempb{#2}%
            \ifx\tempa\tempb%
            \else%
                \def\pgfmathresult{inf}%
            \fi% 
        }
    }
}

\title[MA205 - Section 8.2]{Least Squares Regression}

\begin{document}
\begin{frame}
  \titlepage
\end{frame}

\begin{frame}
  \begin{example}
    Gift aid is financial aid that does not need to be paid back.

    \vspace{1mm}
    A sample of 50 random freshmen at Elmhurst College is shown, comparing the student's family income against gift aid received.
    
    \begin{center}
      \begin{tikzpicture}
        \pgfkeys{/pgf/number format/.cd,
          fixed,
          precision=999,
          set thousands separator={},
          1000 sep in fractionals=false,
        }
        \begin{axis}[
            width=0.9\linewidth,
            height=4.9cm,
            xlabel={Family Income},
            ylabel={Gift Aid From School},
            %ymajorgrids=true,
            %xmajorgrids=true,
            %enlarge x limits=false,
            %enlarge y limits=false,
            xticklabel style={/pgf/number format/.cd,fixed,precision=1},
            xticklabel=\$\pgfmathprintnumber{\tick}k,
            yticklabel style={/pgf/number format/.cd,fixed,precision=1},
            yticklabel=\$\pgfmathprintnumber{\tick}k,
            xtick={0,50,100,...,1000},
            ytick={0,10,20,...,1000},
            ymin=5,
            ymax=35,
            xmin=-5,
            xmax=300,
            scatter/use mapped color={
              %draw=mapped color,
              fill=blue,
            },
          ]
          \addplot[scatter, only marks, blue, scatter src=y, mark size=1pt]
          table [x=family_income, y=gift_aid, col sep=comma] {elmhurst.csv};
          \only<2>{\addplot[domain=-5:300, color=red]{-0.0431 * x + 24.319};}
          \only<3->{\addplot[domain=-5:300, color=green!75!black]{-0.0431 * x + 24.319};}
          \only<2->{\addplot[domain=-5:300, color=red]{-0.04 * x + 24.7};}
          \only<2->{\addplot[domain=-5:300, color=red]{-0.05 * x + 24};}
          \only<2->{\addplot[domain=-5:300, color=red]{-0.0431 * x + 26};}
        \end{axis}
      \end{tikzpicture}
    \end{center}\pause

    \vspace{-4mm}
    \question{Which of the lines best fits the data?}\pause
    \answer{Without an objective definition of measure of \textquote{best}, the answer will vary from person to person.}
  \end{example}
\end{frame}

\begin{frame}
  \begin{block}{What does \textquote{best} mean?}
    A reasonable idea of best, is if we make the sum of the residuals as small as possible:

    \vspace{-4mm}
    \begin{equation*}
      \begin{aligned}
        \abs{e_1}+\abs{e_2}+\cdots+\abs{e_n}
      \end{aligned}
    \end{equation*}\pause

    \vspace{-3mm}
    A more common practice is to choose a line that minimizes the sum of the squared residuals:

    \vspace{-4mm}
    \begin{equation*}
      \begin{aligned}
        e^2_1+e^2_2+\cdots+e^2_n
      \end{aligned}
    \end{equation*}
  \end{block}\pause

  \begin{definition}
    The line that minimizes the sum of the squares of the residuals is called the \textbf{least squares line}.
  \end{definition}\pause

  \begin{note}
    In many applications, a residual twice as large as another is more than twice as bad. Squaring the residuals helps account for this discrepancy.
  \end{note}
\end{frame}

\begin{frame}
  \begin{block}{Conditions for the Least Squares Line}
    \begin{description}
    \item[Linearity:] The data should show a linear trend. If there is a non-linear trend a more advanced method is needed.\pause
    \item[Near Normal Residuals:] When this condition is found unreasonable, it is usually because of outliers or concerns about influential points.\pause
    \item[Constant Variability:] The variability of points around the least squares line remains roughly constant.\pause
    \item[Independent Observations:] Be careful about applying regression to \textbf{time series} data, which are sequential observations in time such as a stock price each day.
    \end{description}
  \end{block}
\end{frame}

\begin{frame}
  \begin{example}
    Scatter plot where linearity fails:
    \begin{center}
      \begin{tikzpicture}
        \pgfkeys{/pgf/number format/.cd,
          fixed,
          precision=999,
          set thousands separator={},
          1000 sep in fractionals=false,
        }
        \begin{axis}[
            width=0.7\linewidth,
            height=4cm,
            xlabel={},
            ylabel={},
            xtick=\empty,
            ytick=\empty,
            xticklabels={,,},
            yticklabels={,,},
            ymin=-2,
            ymax=11,
            xmin=0,
            xmax=1,
            scatter/use mapped color={
              %draw=mapped color,
              fill=blue,
            },
          ]
          \addplot[scatter, only marks, blue, scatter src=y, mark size=1pt]
          table [x=x, y=y, col sep=comma, discard if not={group}{20}] {simulated_scatter.csv};
          \addplot[domain=0:1, color=red]{5.2033747528001175 * x + 3.4442315838402804};
        \end{axis}
      \end{tikzpicture}
    \end{center}

    \vspace{-3mm}
    Residual plot:
    \begin{center}
      \begin{tikzpicture}
        \pgfkeys{/pgf/number format/.cd,
          fixed,
          precision=999,
          set thousands separator={},
          1000 sep in fractionals=false,
        }
        \begin{axis}[
            width=0.7\linewidth,
            height=4cm,
            xlabel={},
            ylabel={},
            xtick=\empty,
            ytick=\empty,
            xticklabels={,,},
            yticklabels={,,},
            ymin=-5.5,
            ymax=5,
            xmin=0,
            xmax=1,
            scatter/use mapped color={
              %draw=mapped color,
              fill=blue,
            },
          ]
          \addplot[scatter, only marks, blue, scatter src=y, mark size=1pt]
          table [x=x, y=y, col sep=comma, discard if not={group}{20}] {residuals.csv};
          \addplot[domain=0:4, color=red]{0 * x};
        \end{axis}
      \end{tikzpicture}
    \end{center}
  \end{example}
\end{frame}

\end{document}
