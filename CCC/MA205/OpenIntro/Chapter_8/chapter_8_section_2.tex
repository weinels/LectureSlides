\documentclass{beamer}
\usepackage[utf8]{inputenc}
\usepackage[english]{babel}
\usepackage{helvet}
\usepackage[T1]{fontenc}
\usepackage[inline]{asymptote}
\usepackage{asy_helper}
\usepackage{slide_helper}
\usepackage{multirow}
\usepackage{cancel}
\usepackage{tikz}
\usetikzlibrary{shapes.geometric, arrows}
\usepackage{pgfplots}
\usepackage{pgfplotstable}
\pgfplotsset{compat=1.5} 
\usepgfplotslibrary{statistics}
\usetikzlibrary{external}
\tikzexternalize%

\pgfplotsset{%
  filter discard warning=false,
    discard if not/.style 2 args={%
        x filter/.code={%
            \edef\tempa{\thisrow{#1}}%
            \edef\tempb{#2}%
            \ifx\tempa\tempb%
            \else%
                \def\pgfmathresult{inf}%
            \fi% 
        }
    }
}

\title[MA205 - Section 8.2]{Least Squares Regression}

\begin{document}
\begin{frame}
  \titlepage
\end{frame}

\begin{frame}
  \begin{example}
    Gift aid is financial aid that does not need to be paid back.

    \vspace{1mm}
    A sample of 50 random freshmen at Elmhurst College is shown, comparing the student's family income against gift aid received.
    
    \begin{center}
      \begin{tikzpicture}
        \pgfkeys{/pgf/number format/.cd,
          fixed,
          precision=999,
          set thousands separator={},
          1000 sep in fractionals=false,
        }
        \begin{axis}[
            width=0.9\linewidth,
            height=4.9cm,
            xlabel={Family Income},
            ylabel={Gift Aid From School},
            %ymajorgrids=true,
            %xmajorgrids=true,
            %enlarge x limits=false,
            %enlarge y limits=false,
            xticklabel style={/pgf/number format/.cd,fixed,precision=1},
            xticklabel=\$\pgfmathprintnumber{\tick}k,
            yticklabel style={/pgf/number format/.cd,fixed,precision=1},
            yticklabel=\$\pgfmathprintnumber{\tick}k,
            xtick={0,50,100,...,1000},
            ytick={0,10,20,...,1000},
            ymin=5,
            ymax=35,
            xmin=-5,
            xmax=300,
            scatter/use mapped color={
              %draw=mapped color,
              fill=blue,
            },
          ]
          \addplot[scatter, only marks, blue, scatter src=y, mark size=1pt]
          table [x=family_income, y=gift_aid, col sep=comma] {elmhurst.csv};
          \only<2>{\addplot[domain=-5:300, color=red]{-0.0431 * x + 24.319};}
          \only<3->{\addplot[domain=-5:300, color=green!75!black]{-0.0431 * x + 24.319};}
          \only<2->{\addplot[domain=-5:300, color=red]{-0.04 * x + 24.7};}
          \only<2->{\addplot[domain=-5:300, color=red]{-0.05 * x + 24};}
          \only<2->{\addplot[domain=-5:300, color=red]{-0.0431 * x + 26};}
        \end{axis}
      \end{tikzpicture}
    \end{center}\pause

    \vspace{-4mm}
    \question{Which of the lines best fits the data?}\pause
    \answer{Without an objective definition of measure of \textquote{best}, the answer will vary from person to person.}
  \end{example}
\end{frame}

\begin{frame}
  \begin{block}{What does \textquote{best} mean?}
    A reasonable idea of best, is if we make the sum of the residuals as small as possible:

    \vspace{-4mm}
    \begin{equation*}
      \begin{aligned}
        \abs{e_1}+\abs{e_2}+\cdots+\abs{e_n}
      \end{aligned}
    \end{equation*}\pause

    \vspace{-3mm}
    A more common practice is to choose a line that minimizes the sum of the squared residuals:

    \vspace{-4mm}
    \begin{equation*}
      \begin{aligned}
        e^2_1+e^2_2+\cdots+e^2_n
      \end{aligned}
    \end{equation*}
  \end{block}\pause

  \begin{definition}
    The line that minimizes the sum of the squares of the residuals is called the \textbf{least squares line}.
  \end{definition}\pause

  \begin{note}
    In many applications, a residual twice as large as another is more than twice as bad. Squaring the residuals helps account for this discrepancy.
  \end{note}
\end{frame}

\begin{frame}
  \begin{block}{Conditions for the Least Squares Line}
    \begin{description}
    \item[Linearity:] The data should show a linear trend. If there is a non-linear trend a more advanced method is needed.\pause
    \item[Near Normal Residuals:] When this condition is found unreasonable, it is usually because of outliers or concerns about influential points.\pause
    \item[Constant Variability:] The variability of points around the least squares line remains roughly constant.\pause
    \item[Independent Observations:] Be careful about applying regression to \textbf{time series} data, which are sequential observations in time such as a stock price each day.
    \end{description}
  \end{block}
\end{frame}

\begin{frame}
  \begin{example}
    Scatter plot where the linearity condition fails:
    \begin{center}
      \begin{tikzpicture}
        \pgfkeys{/pgf/number format/.cd,
          fixed,
          precision=999,
          set thousands separator={},
          1000 sep in fractionals=false,
        }
        \begin{axis}[
            width=0.7\linewidth,
            height=5.5cm,
            xlabel={},
            ylabel={},
            xtick=\empty,
            ytick=\empty,
            xticklabels={,,},
            yticklabels={,,},
            ymin=-2,
            ymax=11,
            xmin=0,
            xmax=1,
            scatter/use mapped color={
              %draw=mapped color,
              fill=blue,
            },
          ]
          \addplot[scatter, only marks, blue, scatter src=y, mark size=1pt]
          table [x=x, y=y, col sep=comma, discard if not={group}{20}] {simulated_scatter.csv};
          \addplot[domain=0:1, color=red]{5.2033747528001175 * x + 3.4442315838402804};
        \end{axis}
      \end{tikzpicture}
    \end{center}

    \vspace{-3mm}
    Residual plot:
    \begin{center}
      \begin{tikzpicture}
        \pgfkeys{/pgf/number format/.cd,
          fixed,
          precision=999,
          set thousands separator={},
          1000 sep in fractionals=false,
        }
        \begin{axis}[
            width=0.7\linewidth,
            height=4cm,
            xlabel={},
            ylabel={},
            xtick=\empty,
            ytick=\empty,
            xticklabels={,,},
            yticklabels={,,},
            ymin=-5.5,
            ymax=5,
            xmin=0,
            xmax=1,
            scatter/use mapped color={
              %draw=mapped color,
              fill=blue,
            },
          ]
          \addplot[scatter, only marks, blue, scatter src=y, mark size=1pt]
          table [x=x, y=y, col sep=comma, discard if not={group}{20}] {residuals.csv};
          \addplot[domain=0:1, color=red]{0 * x};
        \end{axis}
      \end{tikzpicture}
    \end{center}
  \end{example}
\end{frame}

\begin{frame}
  \begin{example}
    Scatter plot where there are clear outliers:
    \begin{center}
      \begin{tikzpicture}
        \pgfkeys{/pgf/number format/.cd,
          fixed,
          precision=999,
          set thousands separator={},
          1000 sep in fractionals=false,
        }
        \begin{axis}[
            width=0.7\linewidth,
            height=5.5cm,
            xlabel={},
            ylabel={},
            xtick=\empty,
            ytick=\empty,
            xticklabels={,,},
            yticklabels={,,},
            ymin=-16,
            ymax=26,
            xmin=-0.7,
            xmax=1.1,
            scatter/use mapped color={
              %draw=mapped color,
              fill=blue,
            },
          ]
          \addplot[scatter, only marks, blue, scatter src=y, mark size=1pt]
          table [x=x, y=y, col sep=comma, discard if not={group}{21}] {simulated_scatter.csv};
          \addplot[domain=-1:1.1, color=red]{24.033896862467678 * x + 0.5315380596430936};
        \end{axis}
      \end{tikzpicture}
    \end{center}

    \vspace{-3mm}
    Residual plot:
    \begin{center}
      \begin{tikzpicture}
        \pgfkeys{/pgf/number format/.cd,
          fixed,
          precision=999,
          set thousands separator={},
          1000 sep in fractionals=false,
        }
        \begin{axis}[
            width=0.7\linewidth,
            height=4cm,
            xlabel={},
            ylabel={},
            xtick=\empty,
            ytick=\empty,
            xticklabels={,,},
            yticklabels={,,},
            ymin=-3.5,
            ymax=9,
            xmin=-0.7,
            xmax=1.1,
            scatter/use mapped color={
              %draw=mapped color,
              fill=blue,
            },
          ]
          \addplot[scatter, only marks, blue, scatter src=y, mark size=1pt]
          table [x=x, y=y, col sep=comma, discard if not={group}{21}] {residuals.csv};
          \addplot[domain=-1:1.1, color=red]{0 * x};
        \end{axis}
      \end{tikzpicture}
    \end{center}
  \end{example}
\end{frame}

\begin{frame}
  \begin{example}
    Scatter plot where the variability around the line isn't constant:
    \begin{center}
      \begin{tikzpicture}
        \pgfkeys{/pgf/number format/.cd,
          fixed,
          precision=999,
          set thousands separator={},
          1000 sep in fractionals=false,
        }
        \begin{axis}[
            width=0.7\linewidth,
            height=5.5cm,
            xlabel={},
            ylabel={},
            xtick=\empty,
            ytick=\empty,
            xticklabels={,,},
            yticklabels={,,},
            ymin=0,
            ymax=6,
            xmin=0,
            xmax=1,
            scatter/use mapped color={
              %draw=mapped color,
              fill=blue,
            },
          ]
          \addplot[scatter, only marks, blue, scatter src=y, mark size=1pt]
          table [x=x, y=y, col sep=comma, discard if not={group}{22}] {simulated_scatter.csv};
          \addplot[domain=0:1, color=red]{4.939997747516614 * x + 0.020156186535077454};
        \end{axis}
      \end{tikzpicture}
    \end{center}

    \vspace{-3mm}
    Residual plot:
    \begin{center}
      \begin{tikzpicture}
        \pgfkeys{/pgf/number format/.cd,
          fixed,
          precision=999,
          set thousands separator={},
          1000 sep in fractionals=false,
        }
        \begin{axis}[
            width=0.7\linewidth,
            height=4cm,
            xlabel={},
            ylabel={},
            xtick=\empty,
            ytick=\empty,
            xticklabels={,,},
            yticklabels={,,},
            ymin=-2.5,
            ymax=1.5,
            xmin=0,
            xmax=1,
            scatter/use mapped color={
              %draw=mapped color,
              fill=blue,
            },
          ]
          \addplot[scatter, only marks, blue, scatter src=y, mark size=1pt]
          table [x=x, y=y, col sep=comma, discard if not={group}{22}] {residuals.csv};
          \addplot[domain=0:1, color=red]{0 * x};
        \end{axis}
      \end{tikzpicture}
    \end{center}
  \end{example}
\end{frame}

\begin{frame}
  \begin{example}
    Scatter plot using time series data:
    \begin{center}
      \begin{tikzpicture}
        \pgfkeys{/pgf/number format/.cd,
          fixed,
          precision=999,
          set thousands separator={},
          1000 sep in fractionals=false,
        }
        \begin{axis}[
            width=0.7\linewidth,
            height=5.5cm,
            xlabel={},
            ylabel={},
            xtick=\empty,
            ytick=\empty,
            xticklabels={,,},
            yticklabels={,,},
            ymin=0.86,
            ymax=1.03,
            xmin=0,
            xmax=61,
            scatter/use mapped color={
              %draw=mapped color,
              fill=blue,
            },
          ]
          \addplot[scatter, only marks, blue, scatter src=y, mark size=1pt]
          table [x=x, y=y, col sep=comma, discard if not={group}{23}] {simulated_scatter.csv};
          \addplot[domain=0:61, color=red]{-0.0013372904437959792 * x + 0.9840752827021301};
        \end{axis}
      \end{tikzpicture}
    \end{center}

    \vspace{-3mm}
    Residual plot:
    \begin{center}
      \begin{tikzpicture}
        \pgfkeys{/pgf/number format/.cd,
          fixed,
          precision=999,
          set thousands separator={},
          1000 sep in fractionals=false,
        }
        \begin{axis}[
            width=0.7\linewidth,
            height=4cm,
            xlabel={},
            ylabel={},
            xtick=\empty,
            ytick=\empty,
            xticklabels={,,},
            yticklabels={,,},
            ymin=-0.06,
            ymax=0.05,
            xmin=0,
            xmax=61,
            scatter/use mapped color={
              %draw=mapped color,
              fill=blue,
            },
          ]
          \addplot[scatter, only marks, blue, scatter src=y, mark size=1pt]
          table [x=x, y=y, col sep=comma, discard if not={group}{23}] {residuals.csv};
          \addplot[domain=0:61, color=red]{0 * x};
        \end{axis}
      \end{tikzpicture}
    \end{center}
  \end{example}
\end{frame}

\begin{frame}
  \begin{block}{Finding the Least Squares Line}
    The least squares regression line will have form:

    \vspace{-4mm}
    \begin{equation*}
      \begin{aligned}
        \hat{y} = \beta_{0} + \beta_{1} x
      \end{aligned}
    \end{equation*}\pause

    \vspace{-5mm}
    While technology is usually used to find $\beta_0$ and $\beta_1$, we can use the following properties to estimate them by hand:\pause
    \begin{itemize}
    \item The slope of the least squares line can be estimated by

      \vspace{-2mm}
      \begin{equation*}
        \begin{aligned}
          b_1 = \dfrac{s_y}{s_x} R
        \end{aligned}
      \end{equation*}\pause

      \vspace{-4mm}
      \item The point $(\bar{x},\bar{y})$ is on the least squares line.
    \end{itemize}
  \end{block}\pause

  \begin{note}
    Recall from Algebra that if we know the slope, $m$, of a line and a point, $(x_0,y_0)$, on that line, then:

    \vspace{-4mm}
    \begin{equation*}
      \begin{aligned}
        y-y_0=m(x-x_0)
      \end{aligned}
    \end{equation*}
  \end{note}
\end{frame}

\begin{frame}
  \begin{example}
    The summary statistics of the Elmhurst College data set are:
    \begin{center}
      \begin{tabular}{lrrrr}\hline
        & \multicolumn{2}{c}{Family Income ($x$)} & \multicolumn{2}{c}{Gift Aid ($y$)} \\\hline
        mean & $\bar{x}=$&$\$101,780$ & $\bar{y}=$&$\$19,940$ \\
        std. dev. & $s_x=$&$\$63,200$ & $s_y=$&$\$5,460$ \\\hline
      \end{tabular}
    \end{center}\pause

    The correlation of the data set is $R=-0.499$, so

    \vspace{-3mm}
    \begin{equation*}
      \begin{aligned}
        b_1 = \dfrac{s_y}{s_x} R\pause
        = \dfrac{5,460}{63,200} (-0.499)\pause
        = -0.0431 
      \end{aligned}
    \end{equation*}\pause

    \vspace{-3mm}
    Since $(\bar{x},\bar{y})=(101,780, 19,940)$ is on the least squares line, we have $x_0=101,780$ and $y_0=19,940$ which gives\pause

    \vspace{-3mm}
    \begin{equation*}
      \begin{aligned}
        y-y_0 &= m(x-x_0) \\\pause
        y-19,940 &= -0.0431(x-101,780) \\\pause
        y-19,940 &= -0.0431x + 4386.72\\\pause
        y &= -0.0431x + 4386.72 + 19,940\\\pause
        y &= 24,327 - 0.0431x
      \end{aligned}
    \end{equation*}
  \end{example}
\end{frame}

\begin{frame}
  \begin{block}{Process for estimating the least squares line}
    \begin{enumerate}
    \item Estimate the slope parameter: $b_1=\dfrac{s_y}{s_x} R$
    \item Since $(\bar{x}, \bar{y})$ is on the least squares line, use $x_0=\bar{x}$ and $y_0=\bar{y}$
    \item Using the point-slope form: $b_0=\bar{y}-b_1 x$
    \end{enumerate}
  \end{block}\pause

  \begin{note}
    The slope, $b_1$, describes the estimated difference in the $y$ variable if the explanatory variable $x$ for a case happened to be one unit larger.
  \end{note}\pause

  \begin{note}
    The intercept, $b_0$, describes the average outcome of $y$ if $x=0$ and the linear model is valid all the way to $x=0$.
  \end{note}
\end{frame}

\begin{frame}
  \begin{examplecont}
    The slope, $b_1=-0.0431$, means that for each \$1,000 family income, we would expect a student to receive a net difference of
    \begin{equation*}
      \begin{aligned}
        \$1,000 \cdot (-0.0431) = -\$43.10
      \end{aligned}
    \end{equation*}\pause
    Which means, on average, \$43.10 less in gift aid.\pause

    \vspace{2mm}
    The intercept, $b_0=\$24,319$, gives the gift aid, on average, a student would receive if their family had no income.
  \end{examplecont}\pause

  \begin{note}
    We must be cautious about interpreting a causal connection between these variables because this data is observational, not experimental.
  \end{note}
\end{frame}

\begin{frame}
  \begin{definition}
    When a regression is used to predict from a $x$ value in between the maximum and minimum values, it is called \textbf{interpolation}.
  \end{definition}\pause

  \begin{definition}
    When a regression is used to predict from a $x$ value bigger than the maximum or smaller than the minimum, it is called \textbf{extrapolation}. 
  \end{definition}\pause

  \begin{example}
    The largest family income in the Elmhurst data set is \$271,974.\pause

    \vspace{1mm}
    If we use the least squares line to estimate the aid of a student with a family income of \$1,000,000, we would get:
    \begin{equation*}
      \begin{aligned}
        24,319 - 0.0431 (1,000,000) = -18,781
      \end{aligned}
    \end{equation*}\pause
    The financial aid a school gives a student can never be less than zero!
  \end{example}
\end{frame}

\begin{frame}
  \begin{block}{Strength of Fit}
    We have used the correlation $R$ to describe the linear relationship between two variable, but it is more common to use $R^2$, called \textbf{R-squared}.\pause

    \vspace{2mm}
    The $R^2$ of a linear model describes what percent of the variation in the response that is explained by the least squares line.
  \end{block}
\end{frame}

\begin{frame}
  \begin{example}
    With the Elmhurst College data, the variance of the response variable is $s_{\text{aid}}^2={(5,460)}^2\approx 29.8~\text{million}$.\pause

    \vspace{1mm}
    If we apply our least squares line, then this model reduces our uncertainty in predicting aid using a student's family income.\pause

    \vspace{1mm}
    The variability in the residuals describes how much variation remains after using the model: $s_{residuals}^2\approx 22.4~\text{million}$.\pause

    \vspace{1mm}
    This means we have a reduction of
    \begin{equation*}
      \begin{aligned}
        \dfrac{s_{aid}^2 - s_{residuals}^2}{s_{aid}^2} =\pause
        \dfrac{29,800,000 - 22,400,000}{29,800,000}\pause
        = \dfrac{7,500,000}{29,800,000}\pause
        \approx 0.25
      \end{aligned}
    \end{equation*}\pause
    Which means a reduction of about 25\% by using information about family income for predicting aid.\pause

    \vspace{1mm}
    Note that for this data set we have $R=-0.499$ and

    \vspace{-4mm}
    \begin{equation*}
      \begin{aligned}
        R^2 = {(-0.4999)}^2 \approx 0.25
      \end{aligned}
    \end{equation*}
  \end{example}
\end{frame}
\end{document}
