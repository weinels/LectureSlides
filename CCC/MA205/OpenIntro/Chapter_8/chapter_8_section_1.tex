\documentclass{beamer}
\usepackage[utf8]{inputenc}
\usepackage[english]{babel}
\usepackage{helvet}
\usepackage[T1]{fontenc}
\usepackage[inline]{asymptote}
\usepackage{asy_helper}
\usepackage{slide_helper}
\usepackage{multirow}
\usepackage{cancel}
\usepackage{tikz}
\usetikzlibrary{shapes.geometric, arrows}
\usepackage{pgfplots}
\usepackage{pgfplotstable}
\pgfplotsset{compat=1.5} 
\usepgfplotslibrary{statistics}
\usetikzlibrary{external}
\tikzexternalize%

\pgfplotsset{%
  filter discard warning=false,
    discard if not/.style 2 args={%
        x filter/.code={%
            \edef\tempa{\thisrow{#1}}%
            \edef\tempb{#2}%
            \ifx\tempa\tempb%
            \else%
                \def\pgfmathresult{inf}%
            \fi% 
        }
    }
}

\title[MA205 - Section 8.1]{Fitting a Line, Residuals, and Correlation}

\begin{document}
\begin{frame}
  \titlepage
\end{frame}

\begin{frame}
  \begin{definition}
    \textbf{Linear Regression} is the statistical method for fitting a line to data where the relationship between two variables, $x$ and $y$, can be modeled by a straight line with some error:

    \vspace{-4mm}
    \begin{equation*}
      \begin{aligned}
        y = \beta_0 + \beta_1 x + \epsilon
      \end{aligned}
    \end{equation*}
  \end{definition}\pause

  \begin{note}
    We don't use $y=mx+b$ because the format $y=b_0+b_1 x$ can easily be expanded in include more variables:

    \vspace{-4mm}
    \begin{equation*}
      y=\beta_0+\beta_1 x_1 + \beta_2 x_2 + \beta_3 x_3 + \cdots
    \end{equation*}

    \vspace{-2mm}
    This is used when performing a multiple regression.
  \end{note}\pause

  \begin{definition}
    We call $x$ the \textbf{explanatory variable}or  \textbf{predictor variable}.
  \end{definition}\pause

  \begin{definition}
    We call $y$ the \textbf{response variable}.
  \end{definition}
\end{frame}

\begin{frame}
  \begin{example}
    \begin{center}
      \begin{tikzpicture}
        \pgfkeys{/pgf/number format/.cd,
          fixed,
          precision=999,
          set thousands separator={},
          1000 sep in fractionals=false,
        }
        \begin{axis}[
            width=0.9\linewidth,
            height=8cm,
            xlabel={},
            ylabel={},
            ymajorgrids=true,
            xmajorgrids=true,
            %enlarge x limits=false,
            %enlarge y limits=false,
            xticklabel style={/pgf/number format/.cd,fixed,precision=1},
            xticklabel=\pgfmathprintnumber{\tick},
            %xtick={0,2,...,14},
            %ytick={-5,0,...,35},
            ymin=-50,
            ymax=60,
            xmin=-60,
            xmax=60,
            scatter/use mapped color={
              %draw=mapped color,
              fill=blue,
            },
          ]
          \addplot[scatter, only marks, blue, scatter src=y, mark size=1pt]
          table [x=x, y=y, col sep=comma, discard if not={group}{1}] {simulated_scatter.csv};
          \addplot[domain=-60:60, color=red]{-0.8046264645323266 * x + 13.524779728906694};
        \end{axis}
      \end{tikzpicture}
    \end{center}
  \end{example}
\end{frame}

\begin{frame}
  \begin{example}
    \begin{center}
      \begin{tikzpicture}
        \pgfkeys{/pgf/number format/.cd,
          fixed,
          precision=999,
          set thousands separator={},
          1000 sep in fractionals=false,
        }
        \begin{axis}[
            width=0.9\linewidth,
            height=8cm,
            xlabel={},
            ylabel={},
            ymajorgrids=true,
            xmajorgrids=true,
            %enlarge x limits=false,
            %enlarge y limits=false,
            xticklabel style={/pgf/number format/.cd,fixed,precision=1},
            xticklabel=\pgfmathprintnumber{\tick},
            %xtick={0,2,...,14},
            %ytick={-5,0,...,35},
            scaled y ticks=base 10:0,
            ytick scale label code/.code={},
            ymin=-40,
            ymax=24000,
            xmin=520,
            xmax=1900,
            scatter/use mapped color={
              %draw=mapped color,
              fill=blue,
            },
          ]
          \addplot[scatter, only marks, blue, scatter src=y, mark size=1pt]
          table [x=x, y=y, col sep=comma, discard if not={group}{2}] {simulated_scatter.csv};
          \addplot[domain=520:1900, color=red]{7.481971646848163 * x + 1082.3905257569718};
        \end{axis}
      \end{tikzpicture}
    \end{center}
  \end{example}
\end{frame}

\begin{frame}
  \begin{example}    
    \begin{center}
      \begin{tikzpicture}
        \pgfkeys{/pgf/number format/.cd,
          fixed,
          precision=999,
          set thousands separator={},
          1000 sep in fractionals=false,
        }
        \begin{axis}[
            width=0.9\linewidth,
            height=8cm,
            xlabel={},
            ylabel={},
            ymajorgrids=true,
            xmajorgrids=true,
            %enlarge x limits=false,
            %enlarge y limits=false,
            xticklabel style={/pgf/number format/.cd,fixed,precision=1},
            xticklabel=\pgfmathprintnumber{\tick},
            %xtick={0,2,...,14},
            %ytick={-5,0,...,35},
            ymin=-195,
            ymax=380,
            xmin=-15,
            xmax=55,
            scatter/use mapped color={
              %draw=mapped color,
              fill=blue,
            },
          ]
          \addplot[scatter, only marks, blue, scatter src=y, mark size=1pt]
          table [x=x, y=y, col sep=comma, discard if not={group}{3}] {simulated_scatter.csv};
          \addplot[domain=-15:55, color=red]{-0.21040134542259065 * x + 133.07988554369706};
        \end{axis}
      \end{tikzpicture}
    \end{center}
    Even though this looks like just a cloud, the linear model may be useful.
  \end{example}
\end{frame}

\begin{frame}
  \begin{example}
    \begin{center}
      \begin{tikzpicture}
        \pgfkeys{/pgf/number format/.cd,
          fixed,
          precision=999,
          set thousands separator={},
          1000 sep in fractionals=false,
        }
        \begin{axis}[
            width=0.9\linewidth,
            height=7.5cm,
            xlabel={},
            ylabel={},
            ymajorgrids=true,
            xmajorgrids=true,
            %enlarge x limits=false,
            %enlarge y limits=false,
            xticklabel style={/pgf/number format/.cd,fixed,precision=1},
            xticklabel=\pgfmathprintnumber{\tick},
            %xtick={0,2,...,14},
            %ytick={-5,0,...,35},
            scaled y ticks=base 10:0,
            ytick scale label code/.code={},
            ymin=0,
            ymax=15,
            xmin=0,
            xmax=90,
            scatter/use mapped color={
              %draw=mapped color,
              fill=blue,
            },
          ]
          \addplot[scatter, only marks, blue, scatter src=y, mark size=1pt]
          table [x=x, y=y, col sep=comma, discard if not={group}{5}] {simulated_scatter.csv};
          \addplot[domain=0:90, color=red]{1.9596137434921878e-17 * x + 8.967409971878514};
        \end{axis}
      \end{tikzpicture}
    \end{center}
    Because, there is a clear non-linear pattern, the linear model is a poor choice for this data.
  \end{example}
\end{frame}

\begin{frame}
  \begin{example}
    Brushtail possums are a marsupial that lives in Australia.\pause

    \vspace{1mm}
    Researchers captured 104 of these animal and took body measurements before releasing the animals back into the wild.\pause

    \vspace{1mm}
    We will consider two measurements:
    \begin{itemize}
    \item The length of each possum from head to tail.
    \item The length of each possum's head.
    \end{itemize}

    \begin{center}
      \begin{tikzpicture}
        \pgfkeys{/pgf/number format/.cd,
          fixed,
          precision=999,
          set thousands separator={},
          1000 sep in fractionals=false,
        }
        \begin{axis}[
            width=0.9\linewidth,
            height=5cm,
            xlabel={Total Length (cm)},
            ylabel={Head Length (mm)},
            ymajorgrids=true,
            xmajorgrids=true,
            %enlarge x limits=false,
            %enlarge y limits=false,
            xticklabel style={/pgf/number format/.cd,fixed,precision=1},
            xticklabel=\pgfmathprintnumber{\tick},
            %xtick={0,2,...,14},
            %ytick={-5,0,...,35},
            scaled y ticks=base 10:0,
            ytick scale label code/.code={},
            ymin=80,
            ymax=105,
            xmin=73,
            xmax=98,
            scatter/use mapped color={
              %draw=mapped color,
              fill=blue,
            },
          ]
          \addplot[scatter, only marks, blue, scatter src=y, mark size=1pt]
          table [x=x, y=y, col sep=comma, discard if not={group}{31}] {simulated_scatter.csv};
          %\addplot[domain=0:90, color=red]{41 + 0.59x};
        \end{axis}
      \end{tikzpicture}
    \end{center}
  \end{example}
\end{frame}

\begin{frame}
  \begin{examplecont}
    We could fit the linear relationship by eye, giving the equation:
    \begin{equation*}
      \begin{aligned}
        \hat{y} &= 41 + 0.59x
      \end{aligned}
    \end{equation*}

    \vspace{-6mm}
    \begin{center}
      \begin{tikzpicture}
        \pgfkeys{/pgf/number format/.cd,
          fixed,
          precision=999,
          set thousands separator={},
          1000 sep in fractionals=false,
        }
        \begin{axis}[
            width=0.9\linewidth,
            height=4.5cm,
            xlabel={Total Length (cm)},
            ylabel={Head Length (mm)},
            ymajorgrids=true,
            xmajorgrids=true,
            %enlarge x limits=false,
            %enlarge y limits=false,
            xticklabel style={/pgf/number format/.cd,fixed,precision=1},
            xticklabel=\pgfmathprintnumber{\tick},
            %xtick={0,2,...,14},
            %ytick={-5,0,...,35},
            scaled y ticks=base 10:0,
            ytick scale label code/.code={},
            ymin=80,
            ymax=105,
            xmin=73,
            xmax=98,
            scatter/use mapped color={
              %draw=mapped color,
              fill=blue,
            },
          ]
          \addplot[scatter, only marks, blue, scatter src=y, mark size=1pt]
          table [x=x, y=y, col sep=comma, discard if not={group}{31}] {simulated_scatter.csv};
          \addplot[domain=73:98, color=red]{41 + 0.59 * x};
        \end{axis}
      \end{tikzpicture}
    \end{center}\pause

    \vspace{-4mm}
    This allows us to make estimates of the possum population.\pause

    \vspace{-4mm}
    \begin{equation*}
      \begin{aligned}
        \hat{y} = 41 + 0.59 (80) = 88.2
      \end{aligned}
    \end{equation*}

    \vspace{-1mm}
    We expect that a possum with a total length of 80cm would have a head length of about 88.2mm.
  \end{examplecont}
\end{frame}

\begin{frame}[fragile]
  \begin{definition}
    \textbf{Residuals} are the leftover variation in the data after accounting for the model fit:
    \begin{equation*}
      \begin{aligned}
        \text{Data} = \text{Fit} + \text{Residual}
      \end{aligned}
    \end{equation*}

    \begin{center}
      \begin{asy}
        size(240, 150, IgnoreAspect);

        real min_x = 0;
        real max_x = 9;
        real min_y = 0;
        real max_y = 20;

        real sc = 0.79;

        real m = 3;
        real b = -1;
        
        real f(real x) { return m*x+b; }

        draw(graph(f,min_x, max_x), red+1bp);
        label(scale(sc)*Label("$\hat{y}=\beta_0+\beta_1 x$"), (2,f(2)), SE, red);

        dot((5,f(5)), red);
        dot((5,19));

        void note(string st, string sb, pair top, pair bottom, real swap=1)
        {
	  real sl = 0.2*swap;
	  real sr = 0.5*swap;
	  real sh = 0.4;
	  real sn = 1.8*swap;
	  real sm = 0.25;
	  pen p = blue;
	  Label Lt = scale(sc)*Label(st);
	  Label Lb = scale(sc)*Label(sb);
	  real ls = 0; //0.5;

	  pair mid = ((top.x+bottom.x)/2, (top.y+bottom.y)/2);
	  
	  path b = (top.x + sl, top.y-0.1) -- (top.x + sr, top.y-sh) -- (top.x + sr, mid.y+sm) -- (top.x+sn,mid.y) -- (top.x+sr, mid.y-sm) -- (bottom.x + sr, bottom.y+sh) -- (bottom.x + sl, bottom.y+0.1);
	  if (swap > 0)
	  {
	    label(Lt,(mid.x+sn, mid.y+ls),E,p);
	    label(Lb,(mid.x+sn, mid.y-ls),E,p);
	  }
	  else
	  {
	    label(Lt,(mid.x+sn, mid.y+ls),W,p);
	    label(Lb,(mid.x+sn, mid.y-ls),W,p);	
	  }
	  
	  draw(b, p );
        }

        note("Residual ($e_i$)", "", (5,19), (5,f(5)));
        note("Fit ($\hat{y_i}$)", "", (5,f(5)), (5,0));
        note("Data ($y_i$)", "", (5,19), (5,0), -1);

        limits((min_x, min_y), (max_x, max_y), Crop);

        real[] tx = {2,4,6,8, 10, 12};
        real[] ty = {5,10,15,20, 25, 30};
        xaxis("$x$", YEquals(0), RightTicks(tx));
        yaxis("$y$", XEquals(0), LeftTicks(ty));
      \end{asy}
    \end{center}

    \vspace{-4mm}
    The residuals are calculated as:

    \vspace{-4mm}
    \begin{equation*}
      \begin{aligned}
        e_i = y_i - \hat{y}_i
      \end{aligned}
    \end{equation*}
  \end{definition}
\end{frame}

\begin{frame}
  \begin{examplecont}
    Let's calculate the residual for the observation $(85.0, 96.6)$.
    \begin{center}
      \begin{tikzpicture}
        \pgfkeys{/pgf/number format/.cd,
          fixed,
          precision=999,
          set thousands separator={},
          1000 sep in fractionals=false,
        }
        \begin{axis}[
            width=0.9\linewidth,
            height=4.5cm,
            xlabel={Total Length (cm)},
            ylabel={Head Length (mm)},
            ymajorgrids=true,
            xmajorgrids=true,
            %enlarge x limits=false,
            %enlarge y limits=false,
            xticklabel style={/pgf/number format/.cd,fixed,precision=1},
            xticklabel=\pgfmathprintnumber{\tick},
            %xtick={0,2,...,14},
            %ytick={-5,0,...,35},
            scaled y ticks=base 10:0,
            ytick scale label code/.code={},
            ymin=80,
            ymax=105,
            xmin=73,
            xmax=98,
            scatter/use mapped color={
              %draw=mapped color,
              fill=blue,
            },
          ]
          \addplot[scatter, only marks, blue, scatter src=y, mark size=1pt]
          table [x=x, y=y, col sep=comma, discard if not={group}{31}] {simulated_scatter.csv};
          \addplot[domain=73:98, color=red]{41 + 0.59 * x};
          \addplot[mark=x, thick, orange, mark size=4pt] coordinates { (85.0, 98.6) };
        \end{axis}
      \end{tikzpicture}
    \end{center}\pause

    \vspace{-4mm}
    We first need to find $\hat{y}$:

    \vspace{-4mm}
    \begin{equation*}
      \begin{aligned}
        \hat{y}_{\color{orange}\times} = \pause 41 + 0.59 (85)\pause = 91.15
      \end{aligned}
    \end{equation*}\pause

    \vspace{-4mm}
    Next, the residual:

    \vspace{-4mm}
    \begin{equation*}
      \begin{aligned}
        e_{\color{orange}\times} = \pause y_{\color{orange}\times} - \hat{y}_{\color{orange}\times}\pause = 96.6-91.15\pause = 7.45
      \end{aligned}
    \end{equation*}
  \end{examplecont}
\end{frame}

\begin{frame}
  \begin{definition}
    If the residual for each point is calculated, the corresponding graph is called a \textbf{residual plot}.
  \end{definition}\pause

  \begin{examplecont}
    \begin{center}
      \begin{tikzpicture}
        \pgfkeys{/pgf/number format/.cd,
          fixed,
          precision=999,
          set thousands separator={},
          1000 sep in fractionals=false,
        }
        \begin{axis}[
            width=0.9\linewidth,
            height=4.5cm,
            xlabel={Total Length (cm)},
            ylabel={Head Length (mm)},
            ymajorgrids=true,
            xmajorgrids=true,
            %enlarge x limits=false,
            %enlarge y limits=false,
            xticklabel style={/pgf/number format/.cd,fixed,precision=1},
            xticklabel=\pgfmathprintnumber{\tick},
            %xtick={0,2,...,14},
            %ytick={-5,0,...,35},
            scaled y ticks=base 10:0,
            ytick scale label code/.code={},
            ymin=-10,
            ymax=10,
            xmin=73,
            xmax=98,
            scatter/use mapped color={
              %draw=mapped color,
              fill=blue,
            },
          ]
          \addplot[scatter, only marks, blue, scatter src=y, mark size=1pt]
          table [x=x, y=y, col sep=comma, discard if not={group}{31}] {residuals.csv};
          \addplot[domain=73:98, color=red]{0 * x};
        \end{axis}
      \end{tikzpicture}
    \end{center}
  \end{examplecont}
\end{frame}

\begin{frame}
  \begin{example}
    Scatter plot with linear regression:
    \begin{center}
      \begin{tikzpicture}
        \pgfkeys{/pgf/number format/.cd,
          fixed,
          precision=999,
          set thousands separator={},
          1000 sep in fractionals=false,
        }
        \begin{axis}[
            width=0.9\linewidth,
            height=4cm,
            xlabel={},
            ylabel={},
            ymajorgrids=true,
            xmajorgrids=true,
            %enlarge x limits=false,
            %enlarge y limits=false,
            xticklabel style={/pgf/number format/.cd,fixed,precision=1},
            xticklabel=\pgfmathprintnumber{\tick},
            %xtick={0,2,...,14},
            %ytick={-5,0,...,35},
            scaled y ticks=base 10:0,
            ytick scale label code/.code={},
            ymin=-9,
            ymax=1,
            xmin=0,
            xmax=1,
            scatter/use mapped color={
              %draw=mapped color,
              fill=blue,
            },
          ]
          \addplot[scatter, only marks, blue, scatter src=y, mark size=1pt]
          table [x=x, y=y, col sep=comma, discard if not={group}{6}] {simulated_scatter.csv};
          \addplot[domain=0:1, color=red]{-8.53960933407856 * x + 0.2402480567744385};
        \end{axis}
      \end{tikzpicture}
    \end{center}

    \vspace{-3mm}
    Residual plot:
    \begin{center}
      \begin{tikzpicture}
        \pgfkeys{/pgf/number format/.cd,
          fixed,
          precision=999,
          set thousands separator={},
          1000 sep in fractionals=false,
        }
        \begin{axis}[
            width=0.9\linewidth,
            height=4cm,
            xlabel={},
            ylabel={},
            ymajorgrids=true,
            xmajorgrids=true,
            %enlarge x limits=false,
            %enlarge y limits=false,
            xticklabel style={/pgf/number format/.cd,fixed,precision=1},
            xticklabel=\pgfmathprintnumber{\tick},
            %xtick={0,2,...,14},
            %ytick={-5,0,...,35},
            scaled y ticks=base 10:0,
            ytick scale label code/.code={},
            ymin=-2,
            ymax=2,
            xmin=0,
            xmax=1.0,
            scatter/use mapped color={
              %draw=mapped color,
              fill=blue,
            },
          ]
          \addplot[scatter, only marks, blue, scatter src=y, mark size=1pt]
          table [x=x, y=y, col sep=comma, discard if not={group}{6}] {residuals.csv};
          \addplot[domain=0:1, color=red]{0 * x};
        \end{axis}
      \end{tikzpicture}
    \end{center}
   \end{example}
\end{frame}

\begin{frame}
  \begin{example}
    Scatter plot with linear regression:
    \begin{center}
      \begin{tikzpicture}
        \pgfkeys{/pgf/number format/.cd,
          fixed,
          precision=999,
          set thousands separator={},
          1000 sep in fractionals=false,
        }
        \begin{axis}[
            width=0.9\linewidth,
            height=3.5cm,
            xlabel={},
            ylabel={},
            ymajorgrids=true,
            xmajorgrids=true,
            %enlarge x limits=false,
            %enlarge y limits=false,
            xticklabel style={/pgf/number format/.cd,fixed,precision=1},
            xticklabel=\pgfmathprintnumber{\tick},
            %xtick={0,2,...,14},
            %ytick={-5,0,...,35},
            scaled y ticks=base 10:0,
            ytick scale label code/.code={},
            ymin=-29,
            ymax=1,
            xmin=0,
            xmax=4,
            scatter/use mapped color={
              %draw=mapped color,
              fill=blue,
            },
          ]
          \addplot[scatter, only marks, blue, scatter src=y, mark size=1pt]
          table [x=x, y=y, col sep=comma, discard if not={group}{7}] {simulated_scatter.csv};
          \addplot[domain=0:4, color=red]{-8.007341341280465 * x + 5.734798131432777};
        \end{axis}
      \end{tikzpicture}
    \end{center}

    \vspace{-3mm}
    Residual plot:
    \begin{center}
      \begin{tikzpicture}
        \pgfkeys{/pgf/number format/.cd,
          fixed,
          precision=999,
          set thousands separator={},
          1000 sep in fractionals=false,
        }
        \begin{axis}[
            width=0.9\linewidth,
            height=3.5cm,
            xlabel={},
            ylabel={},
            ymajorgrids=true,
            xmajorgrids=true,
            %enlarge x limits=false,
            %enlarge y limits=false,
            xticklabel style={/pgf/number format/.cd,fixed,precision=1},
            xticklabel=\pgfmathprintnumber{\tick},
            %xtick={0,2,...,14},
            %ytick={-5,0,...,35},
            scaled y ticks=base 10:0,
            ytick scale label code/.code={},
            ymin=-5,
            ymax=5,
            xmin=0,
            xmax=4,
            scatter/use mapped color={
              %draw=mapped color,
              fill=blue,
            },
          ]
          \addplot[scatter, only marks, blue, scatter src=y, mark size=1pt]
          table [x=x, y=y, col sep=comma, discard if not={group}{7}] {residuals.csv};
          \addplot[domain=0:4, color=red]{0 * x};
        \end{axis}
      \end{tikzpicture}
    \end{center}\pause
    
      \vspace{-2mm}
      Since there is a clear curve in the residual plot, we should not use a linear model. A more advanced method is needed.
  \end{example}
\end{frame}

\begin{frame}
  \begin{example}
    Scatter plot with linear regression:
    \begin{center}
      \begin{tikzpicture}
        \pgfkeys{/pgf/number format/.cd,
          fixed,
          precision=999,
          set thousands separator={},
          1000 sep in fractionals=false,
        }
        \begin{axis}[
            width=0.9\linewidth,
            height=4cm,
            xlabel={},
            ylabel={},
            ymajorgrids=true,
            xmajorgrids=true,
            %enlarge x limits=false,
            %enlarge y limits=false,
            xticklabel style={/pgf/number format/.cd,fixed,precision=1},
            xticklabel=\pgfmathprintnumber{\tick},
            %xtick={0,2,...,14},
            %ytick={-5,0,...,35},
            scaled y ticks=base 10:0,
            ytick scale label code/.code={},
            ymin=-2,
            ymax=2,
            xmin=0,
            xmax=1,
            scatter/use mapped color={
              %draw=mapped color,
              fill=blue,
            },
          ]
          \addplot[scatter, only marks, blue, scatter src=y, mark size=1pt]
          table [x=x, y=y, col sep=comma, discard if not={group}{8}] {simulated_scatter.csv};
          \addplot[domain=0:1, color=red]{0.6425701453508648 * x + -0.3747573610359961};
        \end{axis}
      \end{tikzpicture}
    \end{center}

    \vspace{-3mm}
    Residual plot:
    \begin{center}
      \begin{tikzpicture}
        \pgfkeys{/pgf/number format/.cd,
          fixed,
          precision=999,
          set thousands separator={},
          1000 sep in fractionals=false,
        }
        \begin{axis}[
            width=0.9\linewidth,
            height=4cm,
            xlabel={},
            ylabel={},
            ymajorgrids=true,
            xmajorgrids=true,
            %enlarge x limits=false,
            %enlarge y limits=false,
            xticklabel style={/pgf/number format/.cd,fixed,precision=1},
            xticklabel=\pgfmathprintnumber{\tick},
            %xtick={0,2,...,14},
            %ytick={-5,0,...,35},
            scaled y ticks=base 10:0,
            ytick scale label code/.code={},
            ymin=-2,
            ymax=2,
            xmin=0,
            xmax=1,
            scatter/use mapped color={
              %draw=mapped color,
              fill=blue,
            },
          ]
          \addplot[scatter, only marks, blue, scatter src=y, mark size=1pt]
          table [x=x, y=y, col sep=comma, discard if not={group}{8}] {residuals.csv};
          \addplot[domain=0:1, color=red]{0 * x};
        \end{axis}
      \end{tikzpicture}
    \end{center}
  \end{example}
\end{frame}

\begin{frame}
  \begin{definition}
    \textbf{Correlation}, which is always between -1 and 1, describes the strength of the linear relationship between two values. We denote the correlation by $R$.
  \end{definition}\pause

  \begin{note}
    While technology is often used, the formula for correlation is:
    \begin{equation*}
      \begin{aligned}
        R=\dfrac{1}{n-1} \sum_{i=1}^{n} \left(\dfrac{x_i-\bar{x}}{s_x}\cdot\dfrac{y_i-\bar{y}}{s_y}\right)
      \end{aligned}
    \end{equation*}
  \end{note}
\end{frame}

\newcommand{\corrgraph}[6]{%
  \begin{tikzpicture}
    \pgfkeys{/pgf/number format/.cd,
      fixed,
      precision=999,
      set thousands separator={},
      1000 sep in fractionals=false,
    }
    \begin{axis}[
        width=4.4cm,
        height=4.4cm,
        xlabel={$R=#2$},
        ylabel={},
        ymajorgrids=false,
        xmajorgrids=false,
        %enlarge x limits=false,
        %enlarge y limits=false,
        xticklabel style={/pgf/number format/.cd,fixed,precision=1},
        xticklabel=\pgfmathprintnumber{\tick},
        xtick=\empty,
        ytick=\empty,
        xticklabels={,,},
        yticklabels={,,},
        scaled y ticks=base 10:0,
        ytick scale label code/.code={},
        ymin=#5,
        ymax=#6,
        xmin=#3,
        xmax=#4,
        scatter/use mapped color={
          %draw=mapped color,
          fill=blue,
        },
      ]
      \addplot[scatter, only marks, blue, scatter src=y, mark size=1pt]
      table [x=x, y=y, col sep=comma, discard if not={group}{#1}] {simulated_scatter.csv};
    \end{axis}
  \end{tikzpicture}
}

\begin{frame}
  \begin{example}
    \begin{center}
      \corrgraph{9}{0.33}{-0.2}{4.2}{-9}{17}
      \corrgraph{10}{0.69}{-0.2}{4.2}{-2.5}{4.5}
      \corrgraph{11}{0.98}{-0.2}{4.2}{-2}{9.6}
      \corrgraph{12}{1.00}{-0.03}{1.03}{-0.1}{1.1}
    \end{center}

    \begin{center}
      \corrgraph{13}{0.08}{-0.2}{4.2}{-17}{14}
      \corrgraph{14}{-0.64}{-5.2}{5.2}{-12}{10}
      \corrgraph{15}{-0.92}{-0.03}{1.03}{-10}{2}
      \corrgraph{16}{-1.00}{-0.03}{1.03}{-1.2}{0.2}
    \end{center}
  \end{example}
\end{frame}

\begin{frame}
  \begin{example}
    \begin{center}
      \corrgraph{17}{-0.23}{-2}{2.2}{-49}{8}
      \corrgraph{18}{0.31}{-19}{9}{-1110}{792}
      \corrgraph{19}{0.50}{-1}{4}{-0.2}{2}
    \end{center}\pause

    Since each of these scatter plots has a clear non-linear pattern, a linear model is not appropriate and correlation shouldn't have been calculated.
  \end{example}\pause

  \begin{note}
    Given a table of $x$ and $y$ values, a computer will happily compute correlation. It is your job to determine if a linear model makes sense. 
  \end{note}
\end{frame}
\end{document}
