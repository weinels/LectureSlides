\documentclass{beamer}
\usepackage[utf8]{inputenc}
\usepackage[english]{babel}
\usepackage{helvet}
\usepackage[T1]{fontenc}
\usepackage{textcomp}
\usepackage[inline]{asymptote}
\usepackage{slide_helper}
\usepackage{tikz}
\usetikzlibrary{shapes.geometric, arrows}
\usepackage{pgfplots}
\pgfplotsset{compat=1.5} 
\usepgfplotslibrary{statistics}
\usetikzlibrary{external}
\tikzexternalize%

\title[MA205 - Section 2.1]{Examining Numerical Data}

\begin{document}
\begin{frame}
\titlepage
\end{frame}

\begin{frame}
\begin{example}
Let us consider a scatterplot of borrowers total income and the loan amount from the \texttt{loan50} data set.

\only<1| handout:0>{
\begin{center}
\begin{tikzpicture}
\tikzstyle{every pin}=[
fill=white,
draw=black,
font=\tiny,
]
\pgfkeys{/pgf/number format/.cd,
fixed,
precision=999,
set thousands separator={},
1000 sep in fractionals=false,
}
\begin{axis}[
clip mode=individual,
clip marker paths=true,
width=10cm,
height=5cm,
xlabel={Total Income},
ylabel={Loan Amount},
%ymajorgrids=true,
%xmajorgrids=true,
%enlarge x limits=false,
%enlarge y limits=false,
%xticklabel style={/pgf/number format/.cd,fixed,precision=0},
xticklabel=\$\pgfmathprintnumber{\tick}k,,
yticklabel=\$\pgfmathprintnumber{\tick}k,,
xtick={0, 50000, 100000, 150000, 200000, 250000, 300000},
ytick={0, 10000, 20000, 30000, 40000},
scaled x ticks=base 10:-3,
xtick scale label code/.code={},
scaled y ticks=base 10:-3,
ytick scale label code/.code={},
ymin=-5,
ymax=35000,
xmin=-5,
xmax=325000,
scatter/use mapped color={
%draw=mapped color,
fill=black,
},
]
\addplot [scatter, only marks, blue!50!black, scatter src=y, mark size=0.8pt] 
table [y=loan_amount, x=total_income, col sep=comma] {loan50.csv};
\end{axis}
\end{tikzpicture}
\end{center}
}
\only<2->{
\begin{center}
\begin{tikzpicture}
\tikzstyle{every pin}=[
fill=white,
draw=black,
font=\tiny,
]
\pgfkeys{/pgf/number format/.cd,
fixed,
precision=999,
set thousands separator={},
1000 sep in fractionals=false,
}
\begin{axis}[
clip mode=individual,
clip marker paths=true,
width=10cm,
height=5cm,
xlabel={Total Income},
ylabel={Loan Amount},
%ymajorgrids=true,
%xmajorgrids=true,
%enlarge x limits=false,
%enlarge y limits=false,
%xticklabel style={/pgf/number format/.cd,fixed,precision=0},
xticklabel=\$\pgfmathprintnumber{\tick}k,,
yticklabel=\$\pgfmathprintnumber{\tick}k,,
xtick={0, 50000, 100000, 150000, 200000, 250000, 300000},
ytick={0, 10000, 20000, 30000, 40000},
scaled x ticks=base 10:-3,
xtick scale label code/.code={},
scaled y ticks=base 10:-3,
ytick scale label code/.code={},
ymin=-5,
ymax=35000,
xmin=-5,
xmax=325000,
scatter/use mapped color={
%draw=mapped color,
fill=black,
},
]
\addplot [scatter, only marks, red, scatter src=y, mark size=1.0pt]
table [y=loan_amount, x=total_income, col sep=comma, x expr={\thisrow{total_income} / (\thisrow{total_income} <= 100000)}] {loan50.csv};
\addplot [scatter, only marks, blue!50!black, scatter src=y, mark size=0.8pt]
table [y=loan_amount, x=total_income, col sep=comma, x expr={\thisrow{total_income} / (\thisrow{total_income} > 100000)}] {loan50.csv};
\addplot [mark size=0pt, red] coordinates {(100000,0) (100000,35000)};
\end{axis}
\end{tikzpicture}
\end{center}
}\pause

We can see that the many of borrowers earn less than \$100,000 a year.
\end{example}
\end{frame}

\begin{frame}
\begin{example}
Let us consider a scatterplot of borrowers total income and the loan amount from the \texttt{loan50} data set.

\only<1| handout:0>{
\begin{center}
\begin{tikzpicture}
\tikzstyle{every pin}=[
fill=white,
draw=black,
font=\tiny,
]
\pgfkeys{/pgf/number format/.cd,
fixed,
precision=999,
set thousands separator={},
1000 sep in fractionals=false,
}
\begin{axis}[
clip mode=individual,
clip marker paths=true,
width=10cm,
height=5cm,
xlabel={Poverty Rate},
ylabel={Median Household Income},
%ymajorgrids=true,
%xmajorgrids=true,
%enlarge x limits=false,
%enlarge y limits=false,
%xticklabel style={/pgf/number format/.cd,fixed,precision=0},
xticklabel=\pgfmathprintnumber{\tick}\%,
yticklabel=\$\pgfmathprintnumber{\tick}k,
xtick={0, 10, 20, 30, 40, 50},
ytick={0, 20000, 40000, 60000, 80000, 100000, 120000},
%scaled x ticks=base 10:-3,
%xtick scale label code/.code={},
scaled y ticks=base 10:-3,
ytick scale label code/.code={},
ymin=-5,
ymax=125000,
xmin=-5,
xmax=60,
scatter/use mapped color={
%draw=mapped color,
fill=black,
},
]
\addplot [scatter, only marks, blue!50!black, scatter src=y, mark size=0.3pt] 
table [y=median_hh_income, x=poverty, col sep=comma,ignore chars={N,A}] {county.csv};
\end{axis}
\end{tikzpicture}
\end{center}
}
\only<2->{
\begin{center}
\begin{tikzpicture}
\tikzstyle{every pin}=[
fill=white,
draw=black,
font=\tiny,
]
\pgfkeys{/pgf/number format/.cd,
fixed,
precision=999,
set thousands separator={},
1000 sep in fractionals=false,
}
\begin{axis}[
clip mode=individual,
clip marker paths=true,
width=10cm,
height=5cm,
xlabel={Poverty Rate},
ylabel={Median Household Income},
%ymajorgrids=true,
%xmajorgrids=true,
%enlarge x limits=false,
%enlarge y limits=false,
%xticklabel style={/pgf/number format/.cd,fixed,precision=0},
xticklabel=\pgfmathprintnumber{\tick}\%,
yticklabel=\$\pgfmathprintnumber{\tick}k,
xtick={0, 10, 20, 30, 40, 50},
ytick={0, 20000, 40000, 60000, 80000, 100000, 120000},
%scaled x ticks=base 10:-3,
%xtick scale label code/.code={},
scaled y ticks=base 10:-3,
ytick scale label code/.code={},
ymin=-5,
ymax=125000,
xmin=-5,
xmax=60,
scatter/use mapped color={
%draw=mapped color,
fill=black,
},
]
\addplot [scatter, only marks, blue!50!black, scatter src=y, mark size=0.3pt] 
table [y=median_hh_income, x=poverty, col sep=comma,ignore chars={N,A}] {county.csv};
\addplot [domain=0:50,red, mark size=1.5pt] {84923*e^(-0.0342*x)};
\end{axis}
\end{tikzpicture}
\end{center}
}\pause
It is clear there is a \textbf{nonlinear} association between the median household income and the poverty rate.
\end{example}
\end{frame}
\end{document}
