\documentclass{beamer}
\usepackage[utf8]{inputenc}
\usepackage[english]{babel}
\usepackage[T1]{fontenc}
\usepackage{slide_helper}
\usepackage[inline]{asymptote}
\usepackage{asy_helper}
\usepackage[super]{nth}
\usepackage{array}
\usepackage{wasysym}
\usepackage{pgfplots}
\pgfplotsset{compat=1.5} 
\usepgfplotslibrary{statistics}

\DeclareSymbolFont{extraup}{U}{zavm}{m}{n}
\DeclareMathSymbol{\varheart}{\mathalpha}{extraup}{86}
\DeclareMathSymbol{\vardiamond}{\mathalpha}{extraup}{87}
\DeclareMathSymbol{\varclub}{\mathalpha}{extraup}{84} 
\DeclareMathSymbol{\varspade}{\mathalpha}{extraup}{85}

\newcommand{\suitheart}[1][]{{\color{red}\text{#1}\varheart}}
\newcommand{\suitspade}[1][]{{\color{black}\text{#1}\spadesuit}}
\newcommand{\suitdiamond}[1][]{{\color{red}\text{#1}\vardiamond}}
\newcommand{\suitclub}[1][]{{\color{black}\text{#1}\varclub}}

\newcommand{\prob}[1]{P\left(#1\right)}
\newcommand{\condprob}[2]{\prob{#1~\middle|~#2}}
\newcommand{\comb}[2]{{_{#1}C_{#2}}}
\newcommand{\perm}[2]{_{#1}P_{#2}}

\newcommand{\nullhypothesis}[1]{H_0:~{#1}}
\newcommand{\althypothesis}[1]{H_A:~{#1}}

\begin{asydef}
real nd_func(real mu, real sigma, real x) { return 1/sqrt(2*pi*sigma*sigma)*exp((-1*(x-mu)*(x-mu))/(2*sigma*sigma)); }

guide normal_dist(real mu, real sigma, real xmin, real xmax)
{
	real f(real x) { return nd_func(mu, sigma, x); }
	return graph(f, xmin, xmax);
}

void shade_between(real mu, real sigma, real a, real b, pen p=orange)
{
	real f(real x) { return nd_func(mu, sigma, x); }
	guide g = graph(f, a, b);
	
	filldraw((a,0) -- g -- (b,0) -- cycle, p, black);
}

void multiple_nd_curves_example(real std_dev)
{
	size(300, 190, IgnoreAspect);
    
    draw(normal_dist(0, std_dev, -6,6));
    shade_between(0,std_dev,-std_dev,std_dev);
    draw((0,0)--(0,0.45));
    
    label("$\sigma="+format("%#.2f", std_dev)+"$", (-4.2,0.45), Fill(paleyellow));
    
    xaxis(Bottom(), RightTicks(new real[] {-6,-4,-2,0,2,4,6}));
    yaxis(Left(), LeftTicks(size=nan),ymin = 0, ymax = 0.5);
}
\end{asydef}

\begin{asydef}
pair crit = (150, 33);
real a=-1.25;
real b=1.25;
\end{asydef}

\title[MA205 - Section 8.3]{Testing a Claim About a Mean}

\begin{document}
\begin{frame}
\titlepage
\end{frame}

\begin{frame}
\begin{block}{Requirements}
\begin{enumerate}
\item The sample is a simple random sample.\pause
\item Either or both of these conditions are satisfied:
\begin{itemize}
\item The population is normally distributed.
\item $n>30$
\end{itemize}
\end{enumerate}
\end{block}\pause

\begin{note}
It is very unlikely that you'll know a population standard deviation but not know the population mean. So, we will once again, work from the assumption that both are unknown.
\end{note}\pause

\begin{block}{Test Statistic}
\begin{equation*}
t=\dfrac{\bar{x}-\mu_{\bar{x}}}{\dfrac{s}{\sqrt{n}}}
\end{equation*}
\end{block}
\end{frame}

\begin{frame}
\begin{block}{Equivalent Methods}
\begin{itemize}
\item $P$-values is the most common method used.\pause
\item Critical values can be used when technology is unavailable.
\end{itemize}
\end{block}\pause

\begin{block}{Technology}
$P$-values are usually provided automatically by technology. Otherwise use the student $t$ distribution.
\end{block}\pause

\begin{block}{Recall}
The degrees of freedom are given by
\begin{equation*}
df=n-1
\end{equation*}
\end{block}
\end{frame}

\begin{frame}
\begin{block}{Important Properties of the Student $t$ Distribution}
\begin{itemize}[<+- | alert@+>]
\item The Student $t$ distribution is different for different sample sizes.
\item The Student $t$ distribution has the same general bell shape as the standard normal distribution; its wider shape reflects the greater variability that is expected when $s$ is used estimate $\sigma$.
\item The Student $t$ distribution has a mean of $t=0$.
\item The standard deviation of the Student $t$ varies with the sample size and is greater than 1.
\item As the sample size $n$ gets larger, the Student $t$ distribution gets closer to the standard normal distribution.
\end{itemize}
\end{block}
\end{frame}

\begin{frame}
\begin{example}
\begin{overprint}
\onslide<1-3>
The National Health and Nutrition Examination Study included the sleep times for randomly selected adult subjects:
\begin{center}
\begin{tabular}{cccccccccccc}
4&8&4&4&8&6&9&7&7&10&7&8
\end{tabular}
\end{center}
The unrounded statistics for this sample are
\begin{equation*}
n=12\qquad
\bar{x}=6.83333333\qquad
s=1.99240984
\end{equation*}
\visible<2->{A common recommendation is that adults should sleep between 7 and 9 hours each night.}

\vspace{2mm}
\visible<3->{Using $\alpha=0.05$ let us test the claim that the mean amount of sleep for adults is less than 7 hours.}

\onslide<4>
We first need to check that the requirements have been met. Since we have a sample size less than 30, we need to check for normality.
\begin{center}
\begin{tikzpicture}
\begin{axis}[
small,
height=3.3cm,
width=\linewidth,
enlarge x limits=true,
enlarge y limits=false,
ybar interval,
%ymajorgrids=true,
ylabel={Frequency},
xlabel={Population (in millions)},
x tick label style={rotate=0,anchor=center},
xtick=,
%xtick={0.64, 0.66,...,0.78},
%ytick={0,5,...,1000},
%ymin=0,
%ymax=25,
xmin=2,
xmax=12
%xticklabel style={/pgf/number format/.cd,fixed,precision=0},
%xticklabel=
%\pgfmathprintnumber\tick--\pgfmathprintnumber\nexttick
]
\addplot+ [hist={bins=5}]
table [y=nums] {sleep.dat};
\end{axis}
\end{tikzpicture}
\end{center}

\begin{center}
\begin{tikzpicture}[
declare function={inverf(\x)=\x/abs(\x) * sqrt( sqrt( (4.3307 + ln(1-\x^2)/2 )^2 - ln(1-\x^2)/0.147 ) - (4.3307 + ln(1-\x^2)/2);},
declare function={normq(\p)=1.4142 * inverf(2*\p-1);}
]
\begin{axis}[
height=4cm,
width=\linewidth,
ylabel={$z$ score},
xlabel={$x$ value},
]
\addplot [only marks, mark=*] table [x index=0, y index=0, y expr=normq(\coordindex/12), x=nums] {sleep.dat};
\addplot[red,  thick, domain=4:10] (x,0.47648*x-3.25594);
\end{axis}
\end{tikzpicture}
\end{center}
\onslide<5->
The National Health and Nutrition Examination Study included the sleep times for randomly selected adult subjects. The unrounded statistics are:
\vspace{-1mm}
\begin{equation*}
n=12\qquad
\bar{x}=6.83333333\qquad
s=1.99240984
\end{equation*}

\vspace{-2mm}
Using $\alpha=0.05$ let us test the claim that the mean amount of sleep for adults is less than 7 hours.

\begin{enumerate}
\visible<6->{\item The hypotheses are
\begin{equation*}
\begin{aligned}
\nullhypothesis{\mu=7} \\
\althypothesis{\mu<7}
\end{aligned}
\end{equation*}}
\vspace{-3mm}
\visible<7->{\item Because the claim is about the population mean, the sample statistic most relevant to this test is the sample mean $\bar{s}$.}
\visible<8->{\item Using technology we get $t=-0.289775$ and $P\text{-value}=0.388689$.}
\visible<9->{\item Since $0.388689>0.05$, we fail to reject the null hypothesis.}
\end{enumerate}
\visible<10->{We conclude that there is not sufficient to support the claim that the mean amount of adult sleep is less than 7 hours.}
\end{overprint}
\end{example}
\end{frame}

\begin{frame}
\begin{example}
Data Set 3 \textquote{Body Temperatures} includes measured body temperatures with these statistics for 12 AM on day 2:
\vspace{-1mm}
\begin{equation*}
n=106\qquad
\bar{x}=\degree{98.20}\text{F}\qquad
s=\degree{0.62}\text{F}\qquad
\end{equation*}\pause

\vspace{-5mm}
Let's test the belief that the population mean is $\degree{98.6}$F. (Using $\alpha=0.05$.)\pause
\begin{enumerate}
\item The hypotheses are
\begin{equation*}
\begin{aligned}
\nullhypothesis{\mu=98.6} \\
\althypothesis{\mu\neq98.6}
\end{aligned}
\end{equation*}\pause
\vspace{-3mm}
\item Because the claim is about the population mean, the sample statistic most relevant to this test is the sample mean $\bar{s}$.\pause
\item Using technology we get $t=-6.64234$ and $P\text{-value}=0.00000000140369$.\pause
\item Since $0.00000000140369<0.05$, we reject the null hypothesis.\pause
\end{enumerate}
We conclude that there is sufficient evidence to warrant rejection of the common belief that the population mean is $\degree{98.6}$F.
\end{example}
\end{frame}
\end{document}