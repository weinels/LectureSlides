\documentclass{beamer}
\usepackage[utf8]{inputenc}
\usepackage[english]{babel}
\usepackage[T1]{fontenc}
\usepackage{slide_helper}
\usepackage[super]{nth}
\usepackage{array}
\usepackage{wasysym}
\usepackage{pgfplots}
\pgfplotsset{compat=1.5} 
\usepgfplotslibrary{statistics}
\usepackage{asy_helper}
\usepackage{colortbl}

\DeclareSymbolFont{extraup}{U}{zavm}{m}{n}
\DeclareMathSymbol{\varheart}{\mathalpha}{extraup}{86}
\DeclareMathSymbol{\vardiamond}{\mathalpha}{extraup}{87}
\DeclareMathSymbol{\varclub}{\mathalpha}{extraup}{84} 
\DeclareMathSymbol{\varspade}{\mathalpha}{extraup}{85}

\begin{asydef}
real nd_func(real mu, real sigma, real x) { return 1/sqrt(2*pi*sigma*sigma)*exp((-1*(x-mu)*(x-mu))/(2*sigma*sigma)); }

guide normal_dist(real mu, real sigma, real xmin, real xmax)
{
	real f(real x) { return nd_func(mu, sigma, x); }
	return graph(f, xmin, xmax);
}

void shade_below(real mu, real sigma, real b, real xmin, real xmax, pen p=orange)
{		
	real f(real x) { return nd_func(mu, sigma, x); }
	guide g = graph(f, xmin, b);
	
	filldraw(g -- (b,0) -- cycle, p, black);
	
	draw((xmin,0)--(b,0));
}

void shade_above(real mu, real sigma, real b, real xmin, real xmax, pen p=orange)
{	
	real f(real x) { return nd_func(mu, sigma, x); }
	guide g = graph(f, xmin, b);
	
	filldraw(g -- (b,0) -- cycle, p, black);
	
	draw((xmin,0)--(b,0));
}

void shade_between(real mu, real sigma, real a, real b, pen p=orange)
{
	real f(real x) { return nd_func(mu, sigma, x); }
	guide g = graph(f, a, b);
	
	filldraw((a,0) -- g -- (b,0) -- cycle, p, black);
}

void multiple_nd_curves_example(real std_dev)
{
	size(300, 190, IgnoreAspect);
    
    draw(normal_dist(0, std_dev, -6,6));
    shade_between(0,std_dev,-std_dev,std_dev);
    draw((0,0)--(0,0.45));
    
    label("$\sigma="+format("%#.2f", std_dev)+"$", (-4.2,0.45), Fill(paleyellow));
    
    xaxis(Bottom(), RightTicks(new real[] {-6,-4,-2,0,2,4,6}));
    yaxis(Left(), LeftTicks(size=nan),ymin = 0, ymax = 0.5);
}
\end{asydef}

\newcommand{\prob}[1]{P\left(#1\right)}
\newcommand{\condprob}[2]{\prob{#1~\middle|~#2}}
\newcommand{\comb}[2]{_{#1}C_{#2}}

\title[MA205 - Section 6.5]{Assessing Normality}

\begin{document}
\begin{frame}
\titlepage
\end{frame}

\begin{frame}
\begin{definition}
A \textbf{normal quantile plot} (or \textbf{normal probability plot}) is a graph of points $(x,y)$ where:
\begin{itemize}
\item each $x$ value is from the original set of sample data
\item each $y$ value is the corresponding $z$ score that is expected from the standard normal distribution.
\end{itemize}
\end{definition}
\end{frame}

\begin{frame}
\begin{block}{Procedure}
To determine whether it is reasonable to assume that sample data from a population having a normal distribution follow the steps:\pause
\begin{enumerate}
\item Construct a histogram, if it departs dramatically from a bell shape, conclude that the data do not have a normal distribution.\pause
\item Identify outliers, if there is more than one present, conclude that the data night not have a normal distribution.\pause
\item Use technology to generate a normal quantile plot. \pause\begin{description}
\item[\textbf{Normal Distribution:}] The population is normal if the pattern of the points is reasonably close to a straight line ad the points do not show some systemic pattern that is not a straight-line pattern.\pause
\item[\textbf{Not a Normal Distribution:}] The population is not normal if either or both of these two conditions applies:\pause
\begin{itemize}
\item The points do not lie reasonably close to a straight line.\pause
\item The points show some systemic pattern that is not a straight-line pattern.
\end{itemize}
\end{description}
\end{enumerate}
\end{block}
\end{frame}

\begin{frame}
\begin{example}
Let us consider the IQ scores for 121 individuals.
\begin{center}
\begin{tikzpicture}
\begin{axis}[
small,
height=3.4cm,
width=\linewidth,
enlarge x limits=false,
enlarge y limits=false,
ybar interval,
%ymajorgrids=true,
ylabel={Frequency},
xlabel={IQ Score},
x tick label style={rotate=0,anchor=center},
xtick=,
%xtick={0.64, 0.66,...,0.78},
%ytick={0,5,...,1000},
%ymin=0,
%ymax=25,
%xmin=120,
%xmax=280,
%xticklabel style={/pgf/number format/.cd,fixed,precision=0},
%xticklabel=
%\pgfmathprintnumber\tick--\pgfmathprintnumber\nexttick
]
\addplot+ [hist={bins=9}]
table [y=nums] {iq.dat};
\end{axis}
\end{tikzpicture}
\end{center}

\begin{center}
\begin{tikzpicture}[
declare function={inverf(\x)=\x/abs(\x) * sqrt( sqrt( (4.3307 + ln(1-\x^2)/2 )^2 - ln(1-\x^2)/0.147 ) - (4.3307 + ln(1-\x^2)/2);},
declare function={normq(\p)=1.4142 * inverf(2*\p-1);}
]
\begin{axis}[
height=4.6cm,
width=\linewidth,
ylabel={$z$ score},
xlabel={$x$ value},
]
\addplot [only marks, mark=*] table [x index=0, y index=0, y expr=normq(\coordindex/121), x=nums] {iq.dat};
\addplot[red,  thick, domain=51:121] (x,0.073895*x-6.23346);
\end{axis}
\end{tikzpicture}
\end{center}
\end{example}
\end{frame}

\begin{frame}
\begin{example}
Let us consider 50 uniformly distributed values.
\begin{center}
\begin{tikzpicture}
\begin{axis}[
small,
height=3.4cm,
width=\linewidth,
enlarge x limits=false,
enlarge y limits=false,
ybar interval,
%ymajorgrids=true,
ylabel={Frequency},
xlabel={Value},
x tick label style={rotate=0,anchor=center},
xtick=,
%xtick={0.64, 0.66,...,0.78},
%ytick={0,5,...,1000},
ymin=0,
%ymax=25,
%xmin=120,
%xmax=280,
%xticklabel style={/pgf/number format/.cd,fixed,precision=0},
%xticklabel=
%\pgfmathprintnumber\tick--\pgfmathprintnumber\nexttick
]
\addplot+ [hist={bins=5}]
table [y=nums] {uniform.dat};
\end{axis}
\end{tikzpicture}
\end{center}

\begin{center}
\begin{tikzpicture}[
declare function={inverf(\x)=\x/abs(\x) * sqrt( sqrt( (4.3307 + ln(1-\x^2)/2 )^2 - ln(1-\x^2)/0.147 ) - (4.3307 + ln(1-\x^2)/2);},
declare function={normq(\p)=1.4142 * inverf(2*\p-1);}
]
\begin{axis}[
height=4.5cm,
width=\linewidth,
ylabel={$z$ score},
xlabel={$x$ value},
]
\addplot [only marks, mark=*] table [x index=0, y index=0, y expr=normq(\coordindex/50), x=nums] {uniform.dat};
\addplot[red,  thick, domain=0:50] (x,0.067176*x-1.6458);
\end{axis}
\end{tikzpicture}
\end{center}
\end{example}
\end{frame}

\begin{frame}
\begin{example}
Let us consider the populations of 23 countries.
\begin{center}
\begin{tikzpicture}
\begin{axis}[
small,
height=3.4cm,
width=\linewidth,
enlarge x limits=false,
enlarge y limits=false,
ybar interval,
%ymajorgrids=true,
ylabel={Frequency},
xlabel={Population (in millions)},
x tick label style={rotate=0,anchor=center},
xtick=,
%xtick={0.64, 0.66,...,0.78},
%ytick={0,5,...,1000},
%ymin=0,
%ymax=25,
%xmin=120,
%xmax=280,
%xticklabel style={/pgf/number format/.cd,fixed,precision=0},
%xticklabel=
%\pgfmathprintnumber\tick--\pgfmathprintnumber\nexttick
]
\addplot+ [hist={bins=30}]
table [y=nums] {pop.dat};
\end{axis}
\end{tikzpicture}
\end{center}

\begin{center}
\begin{tikzpicture}[
declare function={inverf(\x)=\x/abs(\x) * sqrt( sqrt( (4.3307 + ln(1-\x^2)/2 )^2 - ln(1-\x^2)/0.147 ) - (4.3307 + ln(1-\x^2)/2);},
declare function={normq(\p)=1.4142 * inverf(2*\p-1);}
]
\begin{axis}[
height=4.4cm,
width=\linewidth,
ylabel={$z$ score},
xlabel={$x$ value},
]
\addplot [only marks, mark=*] table [x index=0, y index=0, y expr=normq(\coordindex/23), x=nums] {pop.dat};
\addplot[red,  thick, domain=-30:1400] (x,0.002162*x-0.234002);
\end{axis}
\end{tikzpicture}
\end{center}
\end{example}
\end{frame}
\end{document}
