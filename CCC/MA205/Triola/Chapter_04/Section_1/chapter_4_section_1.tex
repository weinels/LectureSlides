\documentclass{beamer}
\usepackage[utf8]{inputenc}
\usepackage[english]{babel}
\usepackage[T1]{fontenc}
\usepackage{slide_helper}
\usepackage[super]{nth}
\usepackage{array}
\usepackage{wasysym}

\DeclareSymbolFont{extraup}{U}{zavm}{m}{n}
\DeclareMathSymbol{\varheart}{\mathalpha}{extraup}{86}
\DeclareMathSymbol{\vardiamond}{\mathalpha}{extraup}{87}
\DeclareMathSymbol{\varclub}{\mathalpha}{extraup}{84} 
\DeclareMathSymbol{\varspade}{\mathalpha}{extraup}{85}

\newcommand{\suitheart}[1][]{{\color{red}\text{#1}\varheart}}
\newcommand{\suitspade}[1][]{{\color{black}\text{#1}\spadesuit}}
\newcommand{\suitdiamond}[1][]{{\color{red}\text{#1}\vardiamond}}
\newcommand{\suitclub}[1][]{{\color{black}\text{#1}\varclub}}

\newcommand{\prob}[1]{P\left(#1\right)}
\newcommand{\condprob}[2]{\prob{#1~\middle|~#2}}
\newcommand{\comb}[2]{_{#1}C_{#2}}
\title[MA205 - Section 4.1]{Basic Concepts of Probability}

\begin{document}
\begin{frame}
\titlepage
\end{frame}

\begin{frame}
\begin{definition}
The result of an procedure is called an \textbf{outcome}.
\end{definition}\pause

\begin{example}
Rolling a die is an procedure, where the outcome is the number rolled.
\end{example}\pause

\begin{definition}
An \textbf{event} is any particular outcome or group of outcomes.

\vspace{2mm}
A \textbf{simple event} is an event that cannot be broken down further.
\end{definition}\pause

\begin{example}
Drawing a hand for poker is an event, where the outcome is the hand of cards you drew.

\vspace{2mm}
This is not a simple event, since we could break it down to five individual card draws. Where each card draw would be a simple event.
\end{example}
\end{frame}

\begin{frame}
\begin{definition}
The \textbf{sample space} of an procedure is the set of all possible simple events.
\end{definition}\pause

\begin{example}
If we roll a standard 6-sided die, what is the sample space?\pause

\vspace{2mm}
The sample space is $\set{1,2,3,4,5,6}$.
\end{example}\pause

\begin{block}{Classical Definition}
Given that all out outcomes are equally likely, we can compute the probability of an event $E$ using the formula
\begin{equation*}
\prob{E}=\dfrac{\text{Number of outcomes corresponding to the event $E$}}{\text{Total number of outcomes}}
\end{equation*}
\end{block}
\end{frame}

\begin{frame}
\begin{example}
What is the probability we will roll a 1 on a 6-sided die?\pause

\vspace{2mm}
There is only one outcome to \textquote{rolling a 1}, so
\begin{equation*}
\prob{\text{rolling a 1}} = \dfrac{1}{6}
\end{equation*}
\end{example}\pause

\begin{example}
What is the probability we will roll better than a 4 on a 6-sided die?\pause

\vspace{2mm}
There are two outcomes bigger than 4, so
\begin{equation*}
\prob{\text{roll better than a 4}} = \dfrac{2}{6} = \dfrac{1}{3}
\end{equation*}
\end{example}\pause

\begin{note}
Probabilities are ratios, and so can be reduced like fractions.
\end{note}
\end{frame}

\begin{frame}
\begin{example}
Suppose you have a bag with 20 cherries, 14 sweet and 6 sour. If you pick a cherry at random, what is the probability that it will sweet?\pause

\vspace{2mm}
The total number of outcomes is 20, one for each cherry in the bag. 14 of those outcomes are sweet cherries, so the probability is
\begin{equation*}
\prob{\text{sweet cherry}} = \dfrac{14}{20} = \dfrac{7}{10}
\end{equation*}
\end{example}\pause

\begin{note}
It is assumed that you cannot differentiate the cherries by touch. If, say, the sweet cherries were smaller than the sour ones, you could always pick a sweet cherry.
\end{note}
\end{frame}

\begin{frame}
\begin{block}{Relative Frequency Definition}
Conduct an procedure and count the number of times that event $A$ occurs. $P(A)$ is then \emph{approximated} as follows:
\begin{equation*}
P(A) \approx \dfrac{\text{number of times $A$ occurs}}{\text{number of times the procedure was repeated}}
\end{equation*}
\end{block}\pause

\begin{example}
To determine the probability that an individual car crashes in a year, we must examine past results to determine the number of cars in use in a year and the number of them that crashed. 
\end{example}\pause

\begin{block}{Note}
Relative frequency probabilities are often used in place of classical probabilities. The instruction \textquote{find the probability} often means \textquote{estimate the probability.}
\end{block}
\end{frame}

\begin{frame}
\begin{example}
In a recent year, there were about $3,000,000$ skydiving jumps and 21 of them result in deaths.\pause

\vspace{2mm}
The relative frequency probability is
\begin{equation*}
\begin{aligned}
\prob{\text{skydiving death}}
&=\dfrac{\text{number of skydiving deaths}}{\text{total number of skydiving jumps}} \\\pause
&=\dfrac{21}{3,000,000} \\\pause
&=0.000007 = 0.007\%
\end{aligned}
\end{equation*}
\end{example}\pause

\begin{block}{Note}
We cannot use the classical definition of probability because the outcomes \textquote{dying} and \textquote{not dying} are not equally likely.
\end{block}
\end{frame}

\begin{frame}
\begin{block}{Law of Large Numbers}
As a procedure is repeated again and again, the relative frequency probability of an event tends to approach the actual probability.
\end{block}\pause

\begin{block}{Cautions}
\begin{itemize}
\item The law of large numbers applies to behavior over a large number of trails, and it does not apply to any one individual outcome.\pause
\begin{itemize}
\item Gamblers sometimes foolishly lose large sums of money by incorrectly thinking that a string of losses increases the chances of a win on the next bet, or that a string of wins is likely to continue.
\end{itemize}\pause
\item If we know nothing about the likelihood of different possible outcomes, we should not assume that they are equally likely.\pause
\begin{itemize}
\item You should not think that the probability of passing the next exam is $\tfrac{1}{2}$, or 0.5. The actual probability depends on factors such as the amount of preparation and the difficulty of the exam.
\end{itemize}
\end{itemize}
\end{block}
\end{frame}

\begin{frame}
\begin{definition}
A standard deck of 52 playing cards consists of four \textbf{suits} in two colors: 
Hearts $\suitheart$, Spades $\suitspade$, Diamonds $\suitdiamond$, and Clubs $\suitclub$

\vspace{2mm}
Each suit contains 13 cards, each of a different \textbf{rank}: 

Ace, 2 through 10, Jack, Queen, and King.
\end{definition}\pause

\begin{example}
Let us compute the probability of drawing a card from a deck and getting an Ace.\pause

\vspace{2mm}
There are four aces in a deck of 52 cards. Which gives the probability
\begin{equation*}
\prob{\text{Ace}}=\dfrac{4}{52}=\dfrac{1}{13}\pause
= 0.0769=7.67\%
\end{equation*}
\end{example}
\end{frame}

\begin{frame}
\begin{example}
In a study of U.S.\@ high school drivers, it was found that 3785 texted while driving during the previous 30 days, and 4720 did not text while driving. What is the probability that a individual student texted while driving?\pause

\vspace{1mm}
We need to calculate the total number of high school drivers

\vspace{-7mm}
\begin{equation*}
3785~\text{texted while driving} + 4720~\text{did not text while driving} = 8505
\end{equation*}\pause

\vspace{-7mm}
We can now calculate the probability that a student texted while driving.

\vspace{-7mm}
\begin{equation*}
\begin{aligned}
\prob{\text{texting while driving}}
&= \dfrac{\text{number of drivers who texted while driving}}{\text{total number of drivers in the sample}} \\\pause
&= \dfrac{3785}{8505}\pause
=0.445 = 44.5\%
\end{aligned}
\end{equation*}
\end{example}\pause

\begin{block}{Note}
A common mistake is to use $\tfrac{3785}{4720}$ as the probability.\pause

\vspace{1mm}
Don't make the common mistake finding a probability by mindlessly dividing a smaller number by a larger number. You need to think carefully about the numbers involved and what they represent.
\end{block}
\end{frame}

\begin{frame}
\begin{definition}
An impossible event has probability 0.

\vspace{2mm}
A certain event has a probability of 1.

\vspace{2mm}
The probability of any event must be $0\leq\prob{E}\leq1$.
\end{definition}\pause

\begin{example}
If a year is selected at random, find the probability that Thanksgiving Day in the United States will be on a Wednesday.\pause

\vspace{2mm}
In the United States, Thanksgiving Day always falls on the fourth Thursday in November. This means it is impossible for Thanksgiving Day to fall on a Wednesday.
\begin{equation*}
\begin{aligned}
\prob{\text{Thanksgiving on Wednesday}}&=0\\
\prob{\text{Thanksgiving on Thursday}}&=1
\end{aligned}
\end{equation*}
\end{example}
\end{frame}

\begin{frame}
\begin{definition}
The \textbf{complement} of an event $E$ is the event \textquote{$E$ doesn't happen.}

\vspace{2mm}
The notation $\bar{E}$ is used to denote the complement of $E$.
\end{definition}\pause

\begin{example}
If we roll a 6-sided die, and
\begin{equation*}
\prob{\text{roll a six}} = \dfrac{1}{6}
\end{equation*}
The complement is
\begin{equation*}
\prob{\text{roll not a six}} = \prob{\text{roll a one, two, three, four, or five}} = \dfrac{5}{6}
\end{equation*}
\end{example}
\end{frame}

\begin{frame}
\begin{note}
Either an even occurs or it does not. This means the sum of $\prob{E}$ and $\prob{\bar{E}}$ always has to equal 1.

\vspace{2mm}
In other words,
\begin{equation*}
\prob{\bar{E}} = 1-\prob{E}
\quad\text{or}\quad
\prob{E} = 1-\prob{\bar{E}}
\end{equation*}
\end{note}\pause

\begin{example}
Let us calculate the probability of drawing a card from a deck and not getting a heart.\pause

\vspace{2mm}
There are 13 hearts in a deck, giving $\prob{\text{heart}}=\dfrac{13}{52}=\dfrac{1}{4}$.\pause

\vspace{2mm}
This means the probability of not drawing a heart is
\begin{equation*}
\prob{\text{not heart}} = 1 - \prob{\text{heart}} = 1-\dfrac{1}{4} = \dfrac{3}{4}
\end{equation*}
\end{example}
\end{frame}

\begin{frame}
\begin{block}{The Rare Even Rule for Inferential Statistics}
If, under a given assumption, the probability of a particular observed event is very small and the observed even occurs \emph{significantly less than} or \emph{significantly more than} what we typically expect with that assumption, we conclude that the assumption is probably not correct.\pause

\vspace{2mm}
Worded differently:
\begin{itemize}
\item $x$ successes among $n$ trails is a \textbf{significantly high} number of successes if the probability of $x$ or more successes is unlikely with a probability of 0.05 or less, i.e\@ if $\prob{x~\text{or more}}\leq 0.05$.
\item $x$ successes among $n$ trails is a \textbf{significantly low} number of successes if the probability of $x$ or fewer successes is unlikely with a probability of 0.05 or less, i.e\@ if $\prob{x~\text{or fewer}}\leq 0.05$.
\end{itemize}
\end{block}
\end{frame}

\begin{frame}
\begin{definition}
The \textbf{actual odds against} event $A$ occurring are the ratio $\tfrac{\prob{\bar{A}}}{\prob{A}}$.\pause
\begin{itemize}
\item Usually expressed in the form $a:b$ where $a$ and $b$ are integers. Reduce using the largest common factor; If $a=16$ and $b=4$, express the odds as $4:1$ instead of $16:4$.
\end{itemize}\pause

\vspace{1mm}
The \textbf{actual odds in favor} of event $A$ occurring are the ratio $\tfrac{\prob{A}}{\prob{\bar{A}}}$, which is the reciprocal of the actual odds against.\pause
\begin{itemize}
\item If the odds against an event are $a:b$, then the odds in favor are $b:a$.
\end{itemize}\pause

\vspace{1mm}
The \textbf{payoff odds} against event A occurring are the ratio of net profit (if you win) to the amount bet:

\vspace{-4mm}
\begin{equation*}
\text{Payoff odds against event $A$} = \left(\text{net profit}):~\right(\text{amount bet})
\end{equation*}
\end{definition}\pause

\vspace{-0.5mm}
\begin{block}{Note}
The actual odds reflect the actual likelihood of an event. The payoff odds describe the payout amount determined by the casino, lottery, or racetrack operators. (i.e.\@ based on how greedy the operators are.)
\end{block}
\end{frame}

\begin{frame}
\begin{example}
If you bet \$5 on the number 13 in roulette, your probability of winning is 1/38, but the payoff odds are given by the casino as $35:1$.\pause

\begin{itemize}
\item What are the actual odd against the outcome of 13?\pause
\begin{equation*}
\text{Actual odds against 13} = \dfrac{\prob{\text{not 13}}}{\prob{13}}\pause
= \dfrac{37/38}{1/38}\pause
= \dfrac{37}{1}\pause
\quad\text{or}\quad 37:1
\end{equation*}\pause
\item How much profit would a bet of $\$5$ make, if 13 won?\pause

\vspace{2mm}
For every $\$1$ bet, we would win $\$5$. So, betting $\$5$ would win $\$5\cdot 35 = \$175$.\pause\vspace{3mm}
\item If the casino was not run for profit, the payoff odds would be the same as the actual odds in favor, $37:1$. So, the profit would be $\$185$.
\end{itemize}
\end{example}
\end{frame}
\end{document}
