\documentclass{beamer}
\usepackage[utf8]{inputenc}
\usepackage[english]{babel}
\usepackage[T1]{fontenc}
\usepackage{slide_helper}
\usepackage[inline]{asymptote}
\usepackage{asy_helper}
\usepackage[super]{nth}
\usepackage{array}
\usepackage{wasysym}
\usepackage{pgfplots}
\pgfplotsset{compat=1.5} 
\usepgfplotslibrary{statistics}

\DeclareSymbolFont{extraup}{U}{zavm}{m}{n}
\DeclareMathSymbol{\varheart}{\mathalpha}{extraup}{86}
\DeclareMathSymbol{\vardiamond}{\mathalpha}{extraup}{87}
\DeclareMathSymbol{\varclub}{\mathalpha}{extraup}{84} 
\DeclareMathSymbol{\varspade}{\mathalpha}{extraup}{85}

\newcommand{\suitheart}[1][]{{\color{red}\text{#1}\varheart}}
\newcommand{\suitspade}[1][]{{\color{black}\text{#1}\spadesuit}}
\newcommand{\suitdiamond}[1][]{{\color{red}\text{#1}\vardiamond}}
\newcommand{\suitclub}[1][]{{\color{black}\text{#1}\varclub}}

\newcommand{\prob}[1]{P\left(#1\right)}
\newcommand{\condprob}[2]{\prob{#1~\middle|~#2}}
\newcommand{\comb}[2]{{_{#1}C_{#2}}}
\newcommand{\perm}[2]{_{#1}P_{#2}}

\newcommand{\nullhypothesis}[1]{H_0:~{#1}}
\newcommand{\althypothesis}[1]{H_A:~{#1}}

\begin{asydef}
real nd_func(real mu, real sigma, real x) { return 1/sqrt(2*pi*sigma*sigma)*exp((-1*(x-mu)*(x-mu))/(2*sigma*sigma)); }

guide normal_dist(real mu, real sigma, real xmin, real xmax)
{
	real f(real x) { return nd_func(mu, sigma, x); }
	return graph(f, xmin, xmax);
}

void shade_between(real mu, real sigma, real a, real b, pen p=orange)
{
	real f(real x) { return nd_func(mu, sigma, x); }
	guide g = graph(f, a, b);
	
	filldraw((a,0) -- g -- (b,0) -- cycle, p, black);
}

void multiple_nd_curves_example(real std_dev)
{
	size(300, 190, IgnoreAspect);
    
    draw(normal_dist(0, std_dev, -6,6));
    shade_between(0,std_dev,-std_dev,std_dev);
    draw((0,0)--(0,0.45));
    
    label("$\sigma="+format("%#.2f", std_dev)+"$", (-4.2,0.45), Fill(paleyellow));
    
    xaxis(Bottom(), RightTicks(new real[] {-6,-4,-2,0,2,4,6}));
    yaxis(Left(), LeftTicks(size=nan),ymin = 0, ymax = 0.5);
}
\end{asydef}

\begin{asydef}
pair crit = (150, 33);
real a=-1.25;
real b=1.25;
\end{asydef}

\title[MA205 - Section 8.1]{Basics of Hypothesis Testing}

\begin{document}
\begin{frame}
\titlepage
\end{frame}

\begin{frame}
\begin{definition}
In statistics, a \textbf{hypothesis} is a claim or statement about a property of a population.
\end{definition}\pause

\begin{definition}
A \textbf{hypothesis test} (or \textbf{test of significance}) is a procedure for testing a claim about a property of a population.
\end{definition}\pause

\begin{note}
the \textquote{property of a population} is often the value of a population parameter.
\end{note}
\end{frame}

\begin{frame}
\begin{example}\label{exp_drones}
In a Pitney Bowes survey, 1009 consumers were asked if they are comfortable with having drones deliver their purchases, and 54\% of them responded \textquote{no.}\pause

\vspace{2mm}
Consider the claim that \textquote{the majority of consumers are not comfortable with drone deliveries.}\pause

\vspace{2mm}
Using $p$ to denote the proportion of consumers not comfortable with drone deliveries, the \textquote{majority} claim is equivalent to the claim that the proportion is greater than half, or $p>0.5$. The expression $p>0.5$ is the symbolic form of the original claim.\pause

\vspace{2mm}
We need to know if 545 is significantly high. In other words, do we have $\prob{\text{545 or more consumers}}<0.05$?\pause

\vspace{2mm}
Using technology we have 
\begin{equation*}
\prob{\text{545 or more consumers}}\approx0.005386
\end{equation*}
\end{example}
\end{frame}

\begin{frame}
\begin{definition}
The \textbf{null hypothesis} is a statement that the value of the population parameter (such as proportion, mean, or standard deviation) is equal to some claimed value.
\end{definition}\pause

\begin{definition}
The \textbf{alternative hypotheses} is a statement that the parameter has a value that somehow differs from the null hypotheses. 
\end{definition}\pause

\begin{note}
For the methods of this chapter, the symbolic form of the alternative hypotheses must use one of these symbols: $<$, $>$, $\neq$.
\end{note}\pause

\begin{block}{Notation}
The null hypotheses is denoted by $\boldsymbol{H_0}$.

\vspace{2mm}
The alternative hypotheses is denoted $\boldsymbol{H_1}$ or $\boldsymbol{H_a}$ or $\boldsymbol{H_A}$.
\end{block}
\end{frame}

\begin{frame}
\begin{example}
Here is an example of a null hypothesis involving a proportion:
\begin{equation*}
\nullhypothesis{p=0.5}
\end{equation*}
\end{example}\pause

\begin{example}
Here are different examples of alternative hypotheses involving proportions:
\begin{equation*}
\althypothesis{p>0.5},\quad
\althypothesis{p<0.5},\quad
\althypothesis{p\neq0.5}
\end{equation*}
\end{example}\pause

\begin{example}
Returning to Example~\ref{exp_drones}, for
\begin{center}
\textquote{the majority of consumers are not comfortable with drone deliveries.}
\end{center}
we have the hypotheses:
\begin{equation*}
\begin{aligned}
\nullhypothesis{p=0.5}\\
\althypothesis{p>0.5}
\end{aligned}
\end{equation*}
\end{example}
\end{frame}

\begin{frame}
\begin{note}
If you are conducting a study and want to use a hypothesis test to \emph{support} your claim, your claim must be worded such that it becomes the alternative hypothesis and can be expressed using only the symbols $>$, $<$, or $\neq$.
\end{note}\pause

\begin{block}{Caution}
You \emph{never} support a claim that a parameter is equal to a specified value.
\end{block}
\end{frame}

\begin{frame}
\begin{definition}
The \textbf{significance level $\boldsymbol{\alpha}$} for a hypothesis test is the probability value used as the cutoff for determining when the sample evidence constitutes significant evidence against the null hypothesis. 
\end{definition}\pause

\begin{note}
By its nature, the significance level $\alpha$ is the probability of mistakely rejecting the null hypothesis when it is true:
\begin{equation*}
\text{Significance level $\alpha$} = \prob{\text{rejecting $H_0$ when $H_0$ is true}}
\end{equation*}
\end{note}\pause

\begin{note}
The significance level $\alpha$ is the same $\alpha$ we talked about in Chapter 7, when discussing confidence intervals.
\end{note}
\end{frame}

\begin{frame}
\begin{definition}
The \textbf{test statistic} is a value used in making a decision about the null hypotheses. It is found by converting the sample statistic to a score with the assumption that the null hypothesis is true.
\end{definition}\pause

\begin{block}{Test Statistic for Proportion $p$}
\begin{description}
\item[\textbf{Sampling Distribution:}] Normal ($z$)
\item[\textbf{Requirements:}] $np\geq5$ and $nq\geq5$
\item[\textbf{Test Statistic:}] $z=\dfrac{\hat{p}-p}{\sqrt{\dfrac{pq}{n}}}$
\end{description}
\end{block}
\end{frame}

\begin{frame}
\begin{example}
In Example~\ref{exp_drones} we made a claim about the population proportion $p$, where we have $n=1009$ and $x=545$.\pause

\vspace{2mm}
This means we have
\begin{equation*}
\hat{p}=\dfrac{x}{n}\pause
=\dfrac{545}{1009}\pause
=0.540
\end{equation*}\pause

The null hypotheses is $\nullhypothesis{p=0.5}$.\pause

\vspace{2mm}
Which means we are working form the assumption that $p=0.5$ and so $q=1-p=0.5$.\pause

\vspace{2mm}
The test statistic is then
\begin{equation*}
z=\dfrac{\hat{p}-p}{\sqrt{\dfrac{pq}{n}}}\pause
=\dfrac{0.540-0.5}{\sqrt{\dfrac{0.5\cdot 0.5}{1009}}}\pause
=2.54
\end{equation*}
\end{example}
\end{frame}

\begin{frame}
\begin{block}{Test Statistic for Mean $\mu$}
\begin{description}
\item[\textbf{Sampling Distribution:}] Student $t$
\item[\textbf{Requirements:}] Both of the following:
\begin{itemize}
\item $\sigma$ not known.
\item Normally distributed or $n>30$.
\end{itemize}
\item[\textbf{Test Statistic:}] $t=\dfrac{\bar{x}-\mu}{\dfrac{s}{\sqrt{n}}}$
\end{description}
\end{block}\pause

\begin{block}{Test Statistic for Mean $\mu$}
\begin{description}
\item[\textbf{Sampling Distribution:}] Normal ($z$)
\item[\textbf{Requirements:}] Both of the following:
\begin{itemize}
\item $\sigma$ known.
\item Normally distributed or $n>30$.
\end{itemize}
\item[\textbf{Test Statistic:}] $z=\dfrac{\bar{x}-\mu}{\dfrac{\sigma}{\sqrt{n}}}$
\end{description}
\end{block}
\end{frame}

\begin{frame}[fragile]
\begin{definition}
The \textbf{critical region} (or \textbf{rejection region}) is the area corresponding to all values of the test statistic that cause us to reject the null hypothesis.
\end{definition}\pause

\begin{definition}
In a hypothesis test, the \textbf{$\boldsymbol{P}$-value} is the probability of getting a value of the test statistic that is at least as extreme as the test statistic obtained from the sample data, assuming that the null hypothesis is true.
\end{definition}\pause

\begin{block}{Caution}
Be careful not to confuse the notation.
\begin{center}
\begin{tabular}{ll}
\textbf{$\boldsymbol{P}$-value} & The probability of a test statistic at least as extreme as \\
&the one obtained.\\
$\boldsymbol{p}$ & The population proportion.\\
$\boldsymbol{\hat{p}}$ & The sample proportion.
\end{tabular}
\end{center}
\end{block}
\end{frame}

\begin{frame}[fragile]
\begin{block}{Two-tailed Test $\left(\althypothesis{\neq}\right)$}
The critical region is in the two extreme regions under the curve.
\begin{center}
\begin{asy}
size(crit.x,crit.y,IgnoreAspect);
real xmin=-4; 
real xmax=4;

draw(normal_dist(0,1,xmin,xmax));
shade_between(0,1,xmin,a,orange);
shade_between(0,1,b,xmax,orange);

real y = nd_func(0,1,0)/5;
draw((a-0.25,y)--(b+0.25,y),Arrows(TeXHead));
label("$P$-value",(0,y),UnFill);

xaxis();
\end{asy}
\end{center}
\end{block}\pause

\begin{block}{Left-tailed Test $\left(\althypothesis{<}\right)$}
The critical region is in the extreme left region under the curve.
\begin{center}
\begin{asy}
size(crit.x,crit.y,IgnoreAspect);
real xmin=-4; 
real xmax=4;

draw(normal_dist(0,1,xmin,xmax));
shade_between(0,1,xmin,a,orange);

real y = nd_func(0,1,0)/5;
draw((a-0.25,y)--(0,y),Arrows(TeXHead));
label("$P$-value",(0,y),UnFill);

xaxis();
\end{asy}
\end{center}
\end{block}\pause

\begin{block}{Right-tailed Test $\left(\althypothesis{>}\right)$}
The critical region is in the extreme right region under the curve.
\begin{center}
\begin{asy}
size(crit.x,crit.y,IgnoreAspect);
real xmin=-4; 
real xmax=4;

draw(normal_dist(0,1,xmin,xmax));
shade_between(0,1,b,xmax,orange);

real y = nd_func(0,1,0)/5;
draw((0,y)--(b+0.25,y),Arrows(TeXHead));
label("$P$-value",(0,y),UnFill);

xaxis();
\end{asy}
\end{center}
\end{block}
\end{frame}

\begin{frame}
\begin{block}{Decision Criteria}
\begin{itemize}
\item If $P$-value $\leq\alpha$, reject $H_0$.
\item If $P$-value $\geq\alpha$, fail to reject $H_0$.
\end{itemize}
\end{block}\pause

\begin{example}
In Example~\ref{exp_drones} we made a claim about the population proportion $p$, where we have $n=1009$ and $x=545$.\pause

\vspace{2mm}
The alternative hypothesis is $\althypothesis{p>0.5}$, so this is a right-tailed test.\pause

\vspace{2mm}
The test statistic is $z=2.54$ and the area to the right of $z$ is 0.0055.\pause

\vspace{2mm}
If we are working with a significance level $\alpha=0.05$, so we reject the null hypothesis.
\end{example}\pause

\begin{note}
Technology will compute $P$-values for you.
\end{note}
\end{frame}

\begin{frame}
\begin{block}{Restate the Decision Using Nontechnical Terms}
After you have decided to reject or not reject the null hypothesis, you need to restate the decision in terms that a layperson can understand.
\end{block}\pause

\begin{example}
In Example~\ref{exp_drones} we restate the decision to reject the null hypothesis as:
\begin{center}
\textquote{There is sufficient evidence to support the claim that the majority of consumers are uncomfortable with drone deliveries.}
\end{center}
\end{example}
\end{frame}

\begin{frame}
\begin{block}{Helpful Wording}
Original claim does not include equality and you reject $H_0$:
\begin{center}\small
\textquote{There is sufficient evidence to support the claim that \ldots (claim).}
\end{center}\pause

Original claim does not include equality and you fail to reject $H_0$:
\begin{center}\small
\textquote{There is not sufficient evidence to support the claim that \ldots (claim).}
\end{center}\pause

Original claim includes equality and you reject $H_0$:
\begin{center}\small
\textquote{There is sufficient evidence to warrant rejection the claim that \ldots (claim).}
\end{center}\pause

Original claim includes equality and you fail to reject $H_0$:
\begin{center}\small
\textquote{There is not sufficient evidence to warrant rejection the claim that \ldots (claim).}
\end{center}
\end{block}\pause

\begin{block}{Caution}
We say \textquote{fail to reject the null hypothesis} instead of \textquote{accept the null hypothesis.}
\end{block}
\end{frame}

\begin{frame}
\begin{block}{Procedure for Hypothesis Tests Flow Chart}
Page 360 in your textbook contains a summary of all the steps.
\end{block}\pause

\begin{note}
A confidence interval estimate of a population parameter contains the likely values of that parameter. 

\vspace{2mm}
We should therefore reject a claim that the population parameter has a value that is not included in the confidence interval.
\end{note}
\end{frame}

\begin{frame}
\begin{definition}
A \textbf{type I error} is the mistake of rejecting the null hypothesis when it is actually true.

\vspace{2mm}
The symbol $\alpha$ is used to represent the probability of a type I error.
\begin{equation*}
\alpha = \prob{\text{type I error}} = \prob{\text{rejecting $H_0$ when $H_0$ is true}}
\end{equation*}
\end{definition}\pause

\begin{definition}
A \textbf{type II error} is the mistake of failing to reject the null hypothesis when it is actually false.

\vspace{2mm}
The symbol $\beta$ is used to represent the probability of a type II error.
\begin{equation*}
\beta = \prob{\text{type II error}} = \prob{\text{failing to reject $H_0$ when $H_0$ is false}}
\end{equation*}
\end{definition}\pause

\begin{block}{Describing Type I and Type II Errors}
When wording a statement representing a type I / II error, be sure that the conclusion addresses the original claim, which may or may not be $H_0$.
\end{block}
\end{frame}

\begin{frame}
\begin{example}
Consider the claim that a medical procedure designed to increase the likelihood of a baby girl is effective, which means the probability of a baby girl is $p>0.5$.\pause

\vspace{2mm}
Given the following null and alternative hypotheses
\begin{equation*}
\begin{aligned}
\nullhypothesis{p=0.5} \\
\althypothesis{p>0.5}
\end{aligned}
\end{equation*}
what is a statement that describes a type I error?\pause

\vspace{2mm}
In reality $p=0.5$, but sample evidence leads us to conclude the $p>0.5$. That is, we conclude that the medical procedure is effective when it reality it has no effect.\pause

\vspace{2mm}
What is a statement that describes a type II error?\pause

\vspace{2mm}
In reality $p>0.5$, but we fail to support that conclusion. That is, we conclude that the medical procedure has no effect, when it really is effective in increasing the likelihood of a baby girl.
\end{example}
\end{frame}
\end{document}
