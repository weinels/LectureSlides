\documentclass{beamer}
\usepackage[utf8]{inputenc}
\usepackage[english]{babel}
\usepackage[T1]{fontenc}
\usepackage{slide_helper}
\usepackage[inline]{asymptote}
\usepackage{asy_helper}
\usepackage[super]{nth}
\usepackage{array}
\usepackage{wasysym}
\usepackage{pgfplots}
\pgfplotsset{compat=1.5} 
\usepgfplotslibrary{statistics}

\DeclareSymbolFont{extraup}{U}{zavm}{m}{n}
\DeclareMathSymbol{\varheart}{\mathalpha}{extraup}{86}
\DeclareMathSymbol{\vardiamond}{\mathalpha}{extraup}{87}
\DeclareMathSymbol{\varclub}{\mathalpha}{extraup}{84} 
\DeclareMathSymbol{\varspade}{\mathalpha}{extraup}{85}

\newcommand{\suitheart}[1][]{{\color{red}\text{#1}\varheart}}
\newcommand{\suitspade}[1][]{{\color{black}\text{#1}\spadesuit}}
\newcommand{\suitdiamond}[1][]{{\color{red}\text{#1}\vardiamond}}
\newcommand{\suitclub}[1][]{{\color{black}\text{#1}\varclub}}

\newcommand{\prob}[1]{P\left(#1\right)}
\newcommand{\condprob}[2]{\prob{#1~\middle|~#2}}
\newcommand{\comb}[2]{{_{#1}C_{#2}}}
\newcommand{\perm}[2]{_{#1}P_{#2}}

\newcommand{\nullhypothesis}[1]{H_0&:~{#1}}
\newcommand{\althypothesis}[1]{H_A&:~{#1}}


\begin{asydef}
real nd_func(real mu, real sigma, real x) { return 1/sqrt(2*pi*sigma*sigma)*exp((-1*(x-mu)*(x-mu))/(2*sigma*sigma)); }

guide normal_dist(real mu, real sigma, real xmin, real xmax)
{
	real f(real x) { return nd_func(mu, sigma, x); }
	return graph(f, xmin, xmax);
}

void shade_between(real mu, real sigma, real a, real b, pen p=orange)
{
	real f(real x) { return nd_func(mu, sigma, x); }
	guide g = graph(f, a, b);
	
	filldraw((a,0) -- g -- (b,0) -- cycle, p, black);
}

void multiple_nd_curves_example(real std_dev)
{
	size(300, 190, IgnoreAspect);
    
    draw(normal_dist(0, std_dev, -6,6));
    shade_between(0,std_dev,-std_dev,std_dev);
    draw((0,0)--(0,0.45));
    
    label("$\sigma="+format("%#.2f", std_dev)+"$", (-4.2,0.45), Fill(paleyellow));
    
    xaxis(Bottom(), RightTicks(new real[] {-6,-4,-2,0,2,4,6}));
    yaxis(Left(), LeftTicks(size=nan),ymin = 0, ymax = 0.5);
}
\end{asydef}

\begin{asydef}
pair crit = (150, 33);
real a=-1.25;
real b=1.25;
\end{asydef}

\title[MA205 - Section 10.3]{Prediction Intervals and Variation}

\begin{document}
\begin{frame}
\titlepage
\end{frame}

\begin{frame}
\begin{definition}
A \textbf{prediction interval} is a range of values used to estimate a variable.
\end{definition}\pause

\begin{definition}
A \textbf{confidence interval} is a range of values used to estimate a population parameter.
\end{definition}\pause

\begin{block}{Formula}
Given a fixed and known value $x_0$, the prediction interval for an individual $y$ value is
\begin{equation*}
\hat{y} - E < y < \hat{y} + E
\end{equation*}
where 
\begin{equation*}
E=t_{\alpha/2} s_e\sqrt{1+\dfrac{1}{n}+\dfrac{n{(x_0-\bar{x})}^2}{n(\sum x^2)-{(\sum x)}^2}}
~\text{and}~
s_e=\sqrt{\dfrac{\sum{(y-\hat{y})}^2}{n-2}}
\end{equation*}
and $t_{\alpha/2}$ has $n-2$ degrees of freedom.
\end{block}
\end{frame}

\begin{frame}
\begin{example}
For the paired data in Data Set 16 comparing chocolate consumption and Nobel laureate rate has linear correlation and the regression equation is $\hat{y}=-3.37+2.49x$.\pause

\vspace{2mm}
Using the regression equation to predict the Nobel laureate rate for a country that consumes 10kg per capita is 21.5 Nobel laureates per 10 million people.\pause

\vspace{2mm}
Technology gives a 95\% prediction interval of $7.8<y<35.3$.\pause

\vspace{2mm}
This means that if we select some country with a chocolate consumption rate of 10kg per capita, we have a 95\% confidence that the limits of 7.8 and 35.3 contain the Nobel laureate rate.\pause
\end{example}

\begin{note}
We could narrow down the interval by using a much larger set of data.
\end{note}
\end{frame}

\begin{frame}\small
\begin{note}
For the following definitions, we assume that we have a collection of paired data containing the sample point $(x,y)$, that $\hat{y}$ is the predicted value of $y$ (obtained by using the regression equation), and that the mean of the sample $y$ values is $\bar{y}$.
\end{note}\pause

\begin{definition}
The \textbf{total deviation} of $(x,y)$ is the vertical distance $(y-\bar{y})$, which is the distance between the point $(x,y)$ and the horizontal line passing through the sample mean $\bar{y}$.
\end{definition}\pause

\begin{definition}
The \textbf{explained deviation} is the vertical distance $(\hat{y}-\bar{y})$, which is the distance between the predicted $y$ value and the horizontal line passing through the sample mean $\bar{y}$.
\end{definition}\pause

\begin{definition}
The \textbf{unexplained deviation} is the vertical distance $(y-\hat{y})$, which is the vertical distance between the point $(x,y)$ and the regression line.
\end{definition}
\end{frame}

\begin{frame}[fragile]
\begin{example}
\begin{columns}
\begin{column}{0.5\linewidth}
\begin{overprint}
\onslide<1>
\begin{center}
\begin{asy}
size(190, 200, IgnoreAspect);

real min_x = 0;
real max_x = 9;
real min_y = 0;
real max_y = 20;

draw((1,1)--(max_x,max_y),invisible);

limits((min_x, min_y), (max_x, max_y), Crop);

real[] tx = {2,4,6,8, 10, 12};
real[] ty = {5,10,15,20, 25, 30};
xaxis("$x$", YEquals(0), RightTicks(tx));
yaxis("$y$", XEquals(0), LeftTicks(ty));
\end{asy}
\end{center}
\onslide<2>
\begin{center}
\begin{asy}
size(190, 200, IgnoreAspect);

real min_x = 0;
real max_x = 9;
real min_y = 0;
real max_y = 20;

real sc = 0.79;

draw((0,3)--(9,21), red+1bp);
label(scale(sc)*Label("$\hat{y}=3+2x$"), (2,4), red);

limits((min_x, min_y), (max_x, max_y), Crop);

real[] tx = {2,4,6,8, 10, 12};
real[] ty = {5,10,15,20, 25, 30};
xaxis("$x$", YEquals(0), RightTicks(tx));
yaxis("$y$", XEquals(0), LeftTicks(ty));
\end{asy}
\end{center}
\onslide<3>
\begin{center}
\begin{asy}
size(190, 200, IgnoreAspect);

real min_x = 0;
real max_x = 9;
real min_y = 0;
real max_y = 20;

real sc = 0.79;

draw((0,3)--(9,21), red+1bp);
label(scale(sc)*Label("$\hat{y}=3+2x$"), (2,4), red);

draw((0,9)--(9,9), black+1bp);
label(scale(sc)*Label("$\bar{y}=9$"), (8,8), black);

dot((5,9));

limits((min_x, min_y), (max_x, max_y), Crop);

real[] tx = {2,4,6,8, 10, 12};
real[] ty = {5,10,15,20, 25, 30};
xaxis("$x$", YEquals(0), RightTicks(tx));
yaxis("$y$", XEquals(0), LeftTicks(ty));
\end{asy}
\end{center}
\onslide<4>
\begin{center}
\begin{asy}
size(190, 200, IgnoreAspect);

real min_x = 0;
real max_x = 9;
real min_y = 0;
real max_y = 20;

real sc = 0.79;

draw((0,3)--(9,21), red+1bp);
label(scale(sc)*Label("$\hat{y}=3+2x$"), (2,4), red);

draw((0,9)--(9,9), black+1bp);
label(scale(sc)*Label("$\bar{y}=9$"), (8,8), black);

dot((5,9));
dot((5,19));

limits((min_x, min_y), (max_x, max_y), Crop);

real[] tx = {2,4,6,8, 10, 12};
real[] ty = {5,10,15,20, 25, 30};
xaxis("$x$", YEquals(0), RightTicks(tx));
yaxis("$y$", XEquals(0), LeftTicks(ty));
\end{asy}
\end{center}
\onslide<5>
\begin{center}
\begin{asy}
size(190, 200, IgnoreAspect);

real min_x = 0;
real max_x = 9;
real min_y = 0;
real max_y = 20;

real sc = 0.79;

draw((0,3)--(9,21), red+1bp);
label(scale(sc)*Label("$\hat{y}=3+2x$"), (2,4), red);

draw((0,9)--(9,9), black+1bp);
label(scale(sc)*Label("$\bar{y}=9$"), (8,8), black);

dot((5,9));
dot((5,13));
dot((5,19));

limits((min_x, min_y), (max_x, max_y), Crop);

real[] tx = {2,4,6,8, 10, 12};
real[] ty = {5,10,15,20, 25, 30};
xaxis("$x$", YEquals(0), RightTicks(tx));
yaxis("$y$", XEquals(0), LeftTicks(ty));
\end{asy}
\end{center}
\onslide<6->
\begin{center}
\begin{asy}
size(190, 200, IgnoreAspect);

real min_x = 0;
real max_x = 9;
real min_y = 0;
real max_y = 20;

real sc = 0.79;

real f(real x) { return (1/4) * x * x; }

draw((0,3)--(9,21), red+1bp);
label(scale(sc)*Label("$\hat{y}=3+2x$"), (2,4), red);

draw((0,9)--(9,9), black+1bp);
label(scale(sc)*Label("$\bar{y}=9$"), (8,8), black);

dot((5,9));
dot((5,13));
dot((5,19));

void note(string st, string sb, pair top, pair bottom, real swap=1)
{
	real sl = 0.2*swap;
	real sr = 0.5*swap;
	real sh = 0.4;
	real sn = 1.8*swap;
	real sm = 0.25;
	pen p = blue;
	Label Lt = scale(sc)*Label(st);
	Label Lb = scale(sc)*Label(sb);
	real ls = 0.5;

	pair mid = ((top.x+bottom.x)/2, (top.y+bottom.y)/2);
	
	path b = (top.x + sl, top.y-0.1) -- (top.x + sr, top.y-sh) -- (top.x + sr, mid.y+sm) -- (top.x+sn,mid.y) -- (top.x+sr, mid.y-sm) -- (bottom.x + sr, bottom.y+sh) -- (bottom.x + sl, bottom.y+0.1);
	if (swap > 0)
	{
		label(Lt,(mid.x+sn, mid.y+ls),E,p);
		label(Lb,(mid.x+sn, mid.y-ls),E,p);
	}
	else
	{
		label(Lt,(mid.x+sn, mid.y+ls),W,p);
		label(Lb,(mid.x+sn, mid.y-ls),W,p);	
	}
	
	draw(b, p );
}

note("Unexplained", "$(y-\hat{y})$", (5,19), (5,13));
note("Explained", "$(\hat{y}-\bar{y})$", (5,13), (5,9));
note("Total Deviation", "$(y-\bar{y})$", (5,19), (5,9), -1);

limits((min_x, min_y), (max_x, max_y), Crop);

real[] tx = {2,4,6,8, 10, 12};
real[] ty = {5,10,15,20, 25, 30};
xaxis("$x$", YEquals(0), RightTicks(tx));
yaxis("$y$", XEquals(0), LeftTicks(ty));
\end{asy}
\end{center}
\end{overprint}
\end{column}
\begin{column}{0.45\linewidth}
We assume the following:
\begin{itemize}[<+- | alert@+>]
\item There is sufficient evidence to support the claim of a linear correlation between $x$ and $y$.
\item The equation of the regression is $\hat{y}=3+2x$.
\item The mean of the $y$ values is given by $\bar{y}=9$.
\item One of the pairs of sample data is $x=5$ and $y=19$.
\item The point $(5,13)$ is one of the points on the regression line, because substitution $x=5$ into the regression equation yields $\hat{y}=13$.
\end{itemize}
\end{column}
\end{columns}
\end{example}
\end{frame}

\begin{frame}
\begin{definition}
The \textbf{coefficient of determination} is the proportion of the variation in $y$ that is explained by the regression line. It is computed as
\begin{equation*}
r^2 = \dfrac{\text{explained variation}}{\text{total variation}}
\end{equation*}
\end{definition}\pause

\begin{note}
We can either compute $r^2$ using the variation or we can square the linear correlation coefficient $r$.
\end{note}
\end{frame}

\begin{frame}
\begin{example}
For the paired data in Data Set 16 comparing chocolate consumption and Nobel laureate rate has linear correlation coefficient $r=0.801$.\pause

\vspace{2mm} 
The coefficient of determination is $r^2=0.641$.\pause

\vspace{2mm}
We conclude that $64.2\%$ of the total variation in the Nobel rate can be explained by chocolate consumption, and the other 35.8\% cannot be explained by chocolate consumption.\pause
\end{example}

\begin{note}
The other 35.8\% might be explained by some other factors. But it is pretty silly to seriously think that chocolate consumption in a country is going to directly effect the country's rate of Nobel Laureates.
\end{note}
\end{frame}
\end{document}