\documentclass{beamer}
\usepackage[utf8]{inputenc}
\usepackage[english]{babel}
\usepackage[T1]{fontenc}
\usepackage[inline]{asymptote}
\usepackage{slide_helper}

\title[MA205 - Section 2.1]{Frequency Distributions}

\newcommand{\textsep}{\vspace{0.5mm}}

\begin{document}
\begin{frame}
\titlepage
\end{frame}

\begin{frame}
\begin{definition}
A \textbf{frequency distribution} (or \textbf{frequency table}) shows how data are partitioned among several categories by listing the categories (or classes) along with the number (frequency) of data values in each of them.
\end{definition}\pause

\begin{example}
The table contains drive-through service times, in seconds, for McDonald's.
\begin{center}
\begin{tabular}{rrrrrrrrrrrr}
107 & 139 & 197 & 209 & 281 & 254 & 163 & 150 & 127 & 308 & 206 & 187 \\
169 &  83 & 127 & 133 & 140 & 143 & 130 & 144 &  91 & 113 & 153 & 255 \\
252 & 200 & 117 & 167 & 148 & 184 & 123 & 153 & 155 & 154 & 100 & 117 \\
101 & 138 & 186 & 196 & 146 &  90 & 144 & 119 & 135 & 151 & 197 & 171 \\
\end{tabular}
\end{center}\pause

A frequency table for this data is
\begin{center}
\begin{tabular}{cc|cc}
\textbf{Time} & \textbf{Frequency} & \textbf{Time} & \textbf{Frequency} \\
75-124  & 11 & 225-274 &  3\\
125-174 & 23 & 275-324 &  2\\
175-224 &  9 
\end{tabular}
\end{center}
\end{example}
\end{frame}

\begin{frame}
\begin{definition}
\textbf{Lower class limits} are the smallest numbers that can belong to each of the different classes.
\end{definition}\pause

\begin{definition}
\textbf{Upper class limits} are the largest numbers that can belong to each of the different classes.
\end{definition}\pause

\begin{definition}
\textbf{Class boundaries} are the numbers used to separate the classes, but without the gaps created by class limits.
\end{definition}\pause

\begin{definition}
\textbf{Class midpoints} are the values in the middle of the classes.
\end{definition}\pause

\begin{definition}
\textbf{Class width} is the difference between two consecutive lower class limits (or two consecutive lower class bounds) in a frequency distribution.
\end{definition}
\end{frame}

\begin{frame}
\begin{example}
Recall the frequency table for McDonalds.
\begin{center}
\begin{tabular}{cc|cc}
\textbf{Time} & \textbf{Frequency} & \textbf{Time} & \textbf{Frequency} \\
75-124  & 11 & 225-274 &  3\\
125-174 & 23 & 275-324 &  2\\
175-224 &  9 
\end{tabular}
\end{center}

What are the lower class limits?\pause

\textsep%
\textbf{75, 125, 175, 225, 275}\pause

\textsep%
What are the upper class limits?\pause

\textsep%
\textbf{124, 174, 224, 274, 324}\pause

\textsep%
What are the class boundaries?\pause

\textsep%
\textbf{74.5, 124.5, 174.5, 224.5, 274.5, 324.5}\pause

\textsep%
What are the class midpoints?\pause

\textsep%
\textbf{99.5, 149.5, 199.5, 249.5, 299.5}\pause

\textsep%
What is the class width?\pause

\textsep%
\textbf{50}
\end{example}
\end{frame}

\begin{frame}
\begin{example}
The table lists data for the highest seven sources of injuries resulting in a visit to a hospital emergency room visit in a recent year. (Based on CDC data.)
\begin{center}
\begin{tabular}{lr}
\textbf{Activity} & \textbf{Frequency} \\
Bicycling & 26,212 \\
Football & 25,376 \\
Playground & 16,706 \\
Basketball & 13,987 \\
Soccer & 10,436 \\
Baseball & 9,634 \\
All-terrain vehicle & 6,337
\end{tabular}
\end{center}\pause

Notice that the activity names are categorical data at the nominal level of measurement, but we can still create a frequency distribution.\pause

It should also be noted that just because \textquote{bicycling} is at the top of the table, doesn't mean it is the most dangerous. Many more people ride bicycles in a day than play football.
\end{example}
\end{frame}

\begin{frame}
\begin{definition}
A \textbf{relative frequency distribution} is a variation of the basic frequency distribution in which each class frequency is replaced by a relative frequency.

\vspace{-5mm}
\begin{equation*}
\text{Relative frequency for a class} = \dfrac{\text{frequency for a class}}{\text{sum of all frequencies}}
\end{equation*}
\end{definition}\pause

\begin{example}
Let us look at the relative frequency for the McDonald's data.

\vspace{-3mm}
\begin{center}
\begin{tabular}{ccc}
\textbf{Time} & \textbf{Frequency} & \textbf{Relative Frequency}\\
75-124  & 11 & 23\% \\
125-174 & 23 & 48\% \\
175-224 &  9 & 19\% \\
225-274 &  3 & \phantom{0}6\%\\
275-324 &  2 & \phantom{0}4\%\\
\end{tabular}
\end{center}
\end{example}\pause

\begin{block}{Note}
The sum of the percentages must be 100\% or very close to 100\%.
\end{block}
\end{frame}

\begin{frame}
\begin{definition}
A \textbf{cumulative frequency distribution} is a variation of the basic frequency distribution in which each class frequency is replaced by the sum of frequencies for that class and all previous.
\end{definition}\pause

\begin{example}
\begin{center}
\begin{tabular}{cc}
\textbf{Time} & \textbf{Cumulative Frequency}\\\
\text{Less Than }125 & 11 \\
\text{Less Than }175 & 35 \\
\text{Less Than }225 & 45 \\
\text{Less Than }275 & 48 \\
\text{Less Than }325 & 50 \\
\end{tabular}
\end{center}
\end{example}
\end{frame}

\begin{frame}
\begin{definition}
Data have an approximately normal distributions when
\begin{itemize}
\item The frequencies start low, then increase to one or two high frequencies, and then the decrease to a low frequency.
\item The frequency distribution is approximately symmetric.
\end{itemize}
\end{definition}\pause

\begin{example}
\begin{center}
\begin{tabular}{ccl}
\textbf{Time} & \textbf{Frequency} & \\
75-124 & 2 & Frequencies start low \\
125-174 & 8 & \\
175-224 & 30 & Increase to maximum \\
225-274 & 8 & \\
275-324 & 2 & Frequencies become low again
\end{tabular}
\end{center}
\end{example}\pause

\begin{block}{Note}
Real data are never this perfect, so some discretion is needed to judge if the distribution is \textquote{close enough} to satisfying these two conditions.
\end{block}
\end{frame}

\begin{frame}
\begin{block}{Gaps}
The presence of gaps can suggest that the data are from two or more different populations.
\end{block}\pause

\begin{example}
\begin{columns}
\begin{column}{0.5\textwidth}
The table contains a frequency distribution of the weights, in grams, of randomly selected pennies.

\vspace{1mm}
\visible<3->{Examination reveals a large gap between the lightest pennies and the heaviest pennies.}

\vspace{1mm}
\visible<4->{This difference is explained by:
\begin{itemize}
\item Pennies made before 1983 are 95\% copper and 5\% zinc.
\item Pennies made after 1983 are 2.5\% copper and 97.5\% zinc.
\end{itemize}
(Zinc is less dense than copper.)}
\end{column}
\begin{column}{0.5\textwidth}  %%<--- here
\begin{center}
\begin{tabular}{cc}
\textbf{Weight (grams)} & \textbf{Frequency} \\
2.40-2.49 & 18 \\
2.50-2.59 & 19 \\
2.60-2.69 & 0 \\
2.70-2.79 & 0 \\
2.80-2.89 & 0 \\
2.90-2.99 & 2 \\
3.00-3.09 & 25 \\
3.10-3.19 & 8
\end{tabular}
\end{center}
\end{column}
\end{columns}
\end{example}
\end{frame}

\begin{frame}
\begin{example}
Let us compare the service times for McDonald's and Dunkin' Donuts.
\begin{center}
\begin{tabular}{ccc}
\textbf{Time} & \textbf{McDonald's} & \textbf{Dunkin' Donuts}\\
25-74   &      & 22\% \\
75-124  & 23\% & 44\% \\
125-174 & 48\% & 28\% \\
175-224 & 19\% & \phantom{0}6\% \\
225-274 & \phantom{0}6\% & \\
275-324 & \phantom{0}4\% &
\end{tabular}
\end{center}\pause

We might expect that the difference in menus would lead to very different service times. There are differences, but not a large as you might expect.
\end{example}
\end{frame}
\end{document}
