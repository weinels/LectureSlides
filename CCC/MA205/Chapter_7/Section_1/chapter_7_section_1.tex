\documentclass{beamer}
\usepackage[utf8]{inputenc}
\usepackage[english]{babel}
\usepackage[T1]{fontenc}
\usepackage{slide_helper}
\usepackage[super]{nth}
\usepackage{array}
\usepackage{wasysym}
\usepackage{pgfplots}
\pgfplotsset{compat=1.5} 
\usepgfplotslibrary{statistics}

\DeclareSymbolFont{extraup}{U}{zavm}{m}{n}
\DeclareMathSymbol{\varheart}{\mathalpha}{extraup}{86}
\DeclareMathSymbol{\vardiamond}{\mathalpha}{extraup}{87}
\DeclareMathSymbol{\varclub}{\mathalpha}{extraup}{84} 
\DeclareMathSymbol{\varspade}{\mathalpha}{extraup}{85}

\newcommand{\suitheart}[1][]{{\color{red}\text{#1}\varheart}}
\newcommand{\suitspade}[1][]{{\color{black}\text{#1}\spadesuit}}
\newcommand{\suitdiamond}[1][]{{\color{red}\text{#1}\vardiamond}}
\newcommand{\suitclub}[1][]{{\color{black}\text{#1}\varclub}}

\newcommand{\prob}[1]{P\left(#1\right)}
\newcommand{\condprob}[2]{\prob{#1~\middle|~#2}}
\newcommand{\comb}[2]{{_{#1}C_{#2}}}
\newcommand{\perm}[2]{_{#1}P_{#2}}

\title[MA205 - Section 6.1]{Estimating a Population Proportion}

\begin{document}
\begin{frame}
\titlepage
\end{frame}

\begin{frame}
\begin{definition}
A \textbf{point estimate} is a single value used to estimate a population parameter.
\end{definition}\pause

\begin{block}{Notation}
The sample proportion $\hat{p}$ is the best point estimate of the population proportion $p$.
\end{block}\pause

\begin{block}{Note}
We use the sample proportion $\hat{p}$ to estimate the population portion $p$ because it is unbiased and the most consistent of the estimators that could be used.
\end{block}\pause

\begin{block}{Recall}
An unbiased estimator is a statistic that targets the value of the corresponding population parameter in the sense that the sampling distribution of the statistic has a mean that is equal to the corresponding population parameter.
\end{block}
\end{frame}

\begin{frame}
\begin{example}
A Gallup poll, in which 1487 adults were surveyed, 43\% of them said that they have a Facebook page. Based on this result, find the best point estimate of the proportion of \emph{all} adults who have a Facebook page.\pause

\vspace{2mm}
The sample proportion is the best point estimate of the population proportion.\pause

\vspace{2mm}
The sample proportion is 0.43, so the best estimate of $p$ is 0.43.
\end{example}\pause

\begin{block}{Note}
We have no indication of how \emph{good} of an estimate 0.43 is, just that it is the best of the available options.
\end{block}
\end{frame}

\begin{frame}
\begin{definition}
A \textbf{confidence interval} is a range of values used to estimate the true value of a population parameter. A confidence interval is sometimes abbreviated as CI\@.
\end{definition}\pause

\begin{definition}
The \textbf{confidence level} is the probability $1-\alpha$ that the confidence interval actually does contain the population parameter, assuming that the estimation process is repeated a large number of times.
\end{definition}\pause

\begin{block}{Note}
The confidence is sometimes called the \textbf{degree of confidence} or the \textbf{confidence coefficient}.
\end{block}\pause

\begin{block}{Common Confidence Intervals}
\begin{center}
\begin{tabular}{cc}
%\textbf{Confidence Level} & \textbf{Value of $\boldsymbol{\alpha}$} \\\hline
90\% (or 0.90) confidence level: & $\alpha=0.10$ \\
95\% (or 0.95) confidence level: & $\alpha=0.05$ \\
99\% (or 0.99) confidence level: & $\alpha=0.01$ \\
\end{tabular}
\end{center}
\end{block}
\end{frame}
\end{document}
