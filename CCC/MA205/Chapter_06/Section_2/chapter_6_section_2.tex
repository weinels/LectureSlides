\documentclass{beamer}
\usepackage[utf8]{inputenc}
\usepackage[english]{babel}
\usepackage[T1]{fontenc}
\usepackage{slide_helper}
\usepackage[super]{nth}
\usepackage{array}
\usepackage{wasysym}
\usepackage{pgfplots}
\pgfplotsset{compat=1.5} 
\usepgfplotslibrary{statistics}
\usepackage{asy_helper}
\usepackage{colortbl}

\DeclareSymbolFont{extraup}{U}{zavm}{m}{n}
\DeclareMathSymbol{\varheart}{\mathalpha}{extraup}{86}
\DeclareMathSymbol{\vardiamond}{\mathalpha}{extraup}{87}
\DeclareMathSymbol{\varclub}{\mathalpha}{extraup}{84} 
\DeclareMathSymbol{\varspade}{\mathalpha}{extraup}{85}

\begin{asydef}
real nd_func(real mu, real sigma, real x) { return 1/sqrt(2*pi*sigma*sigma)*exp((-1*(x-mu)*(x-mu))/(2*sigma*sigma)); }

guide normal_dist(real mu, real sigma, real xmin, real xmax)
{
	real f(real x) { return nd_func(mu, sigma, x); }
	return graph(f, xmin, xmax);
}

void shade_below(real mu, real sigma, real b, real xmin, real xmax, pen p=orange)
{		
	real f(real x) { return nd_func(mu, sigma, x); }
	guide g = graph(f, xmin, b);
	
	filldraw(g -- (b,0) -- cycle, p, black);
	
	draw((xmin,0)--(b,0));
}

void shade_above(real mu, real sigma, real b, real xmin, real xmax, pen p=orange)
{	
	real f(real x) { return nd_func(mu, sigma, x); }
	guide g = graph(f, xmin, b);
	
	filldraw(g -- (b,0) -- cycle, p, black);
	
	draw((xmin,0)--(b,0));
}

void shade_between(real mu, real sigma, real a, real b, pen p=orange)
{
	real f(real x) { return nd_func(mu, sigma, x); }
	guide g = graph(f, a, b);
	
	filldraw((a,0) -- g -- (b,0) -- cycle, p, black);
}

void multiple_nd_curves_example(real std_dev)
{
	size(300, 190, IgnoreAspect);
    
    draw(normal_dist(0, std_dev, -6,6));
    shade_between(0,std_dev,-std_dev,std_dev);
    draw((0,0)--(0,0.45));
    
    label("$\sigma="+format("%#.2f", std_dev)+"$", (-4.2,0.45), Fill(paleyellow));
    
    xaxis(Bottom(), RightTicks(new real[] {-6,-4,-2,0,2,4,6}));
    yaxis(Left(), LeftTicks(size=nan),ymin = 0, ymax = 0.5);
}
\end{asydef}

\newcommand{\prob}[1]{P\left(#1\right)}
\newcommand{\condprob}[2]{\prob{#1~\middle|~#2}}
\newcommand{\comb}[2]{_{#1}C_{#2}}

\title[MA205 - Section 6.2]{Applications of Normal Distributions}

\begin{document}
\begin{frame}
\titlepage
\end{frame}

\begin{frame}
\begin{block}{Nonstandard Normal Distribution}
To work with a nonstandard normal distribution we need to standardize the distribution by transforming $x$ values into $z$ scores.
\begin{equation*}
z=\dfrac{x-\mu}{\sigma}
\end{equation*}
\end{block}\pause

\begin{note}
If your calculator or software lets you enter values for $\mu$ and $\sigma$, you do not need to perform this conversion yourself.
\end{note}
\end{frame}

\begin{frame}[fragile]
\begin{block}{Procedure for Finding Areas with a Nonstandard Normal Distribution}
\begin{enumerate}
\item<1-> Sketch a normal curve, label the mean and any specific $x$ values, and then shade the region representing the desimediumyellow probability.
\item<2-> For each relevant $x$ value that is a boundary for the shaded region, convert that value to the equivalent $z$ score.
\item<3-> Use technology to find the area of the shaded region.
\end{enumerate}
\begin{overprint}
\onslide<1>
\begin{center}
\begin{asy}
size(300, 140, IgnoreAspect);

shade_between(0,1,-0.7,1.3);
draw(normal_dist(0, 1, -4.5, 4.5));

label("$x_1$",(-0.7,0),NW);
dot((-0.7,0));
label("$x_2$",(0.4,0),NE);
dot((0.4,0));
label("$x_3$",(1.3,0),NE);
dot((1.3,0));
label("$x_4$",(2.3,0),NW);
dot((2.3,0));

xaxis("$x$",Bottom(), RightTicks(new real [] {-4,-3,-2,-1,0,1,2,3,4}, size=nan, ticklabel = new string(real x) { if (x == 0) return "$\mu$"; else return "$" + string(x) + "\sigma$";}));
\end{asy}
\end{center}
\onslide<2>
\begin{center}
\begin{asy}
size(300, 140, IgnoreAspect);

shade_between(0,1,-0.7,1.3);
draw(normal_dist(0, 1, -4.5, 4.5));

label("$z_1$",(-0.7,0),NW,blue);
dot((-0.7,0));
label("$z_3$",(1.3,0),NE,blue);
dot((1.3,0));


xaxis("$z$",Bottom(), RightTicks(new real [] {-4,-3,-2,-1,0,1,2,3,4}, size=nan));
\end{asy}
\end{center}
\onslide<3>
\begin{center}
\begin{asy}
size(300, 140, IgnoreAspect);

shade_between(0,1,-0.7,1.3);
draw(normal_dist(0, 1, -4.5, 4.5));

label("$z_1$",(-0.7,0),NW,blue);
dot((-0.7,0));
label("$z_3$",(1.3,0),NE,blue);
dot((1.3,0));

label("$P(z_1\leq z\leq z_2)$",(0.3,0.4*nd_func(0,1,0)));

xaxis("$z$",Bottom(), RightTicks(new real [] {-4,-3,-2,-1,0,1,2,3,4}, size=nan));
\end{asy}
\end{center}
\end{overprint}
\end{block}
\end{frame}

\begin{frame}[fragile]
\begin{example}
From Data Set 1 \textquote{Body Data} we see that heights of men are normally distributed with a mean of 68.6 in\@. and a standard deviation of 2.8 in. Find the percentage of men who are taller than a showerhead at 72 in\@.

\vspace{2mm}
\begin{overprint}
\onslide<2>
\begin{center}
\begin{asy}
size(300, 120, IgnoreAspect);

real xmin=57.4; real xmax=79.8;

shade_between(68.6,2.8,72,xmax);
draw(normal_dist(68.6, 2.8, xmin, xmax));

label("72 in.",(72,0),NW);
dot((72,0));

xaxis("$x$",Bottom(), RightTicks(new real [] {60.2, 63, 65.8, 68.6, 71.4, 74.2, 77.0}, size=nan));
\end{asy}
\end{center}
\onslide<3->
\begin{center}
\begin{asy}
size(300, 120, IgnoreAspect);

real xmin=-4; real xmax=4;

shade_between(0,1,1.21,xmax);
draw(normal_dist(0, 1, xmin, xmax));

label("$1.21$",(1.21,0),NW);
dot((1.21,0));

xaxis("$z$",Bottom(), RightTicks(new real [] {-3,-2,-1,0,1,2,3}, size=nan));
\end{asy}
\end{center}
\end{overprint}
\onslide<4->
We then need to compute
\begin{equation*}
\prob{z\geq 1.21} \onslide<5->= 0.1123\quad\text{(rounded)}
\end{equation*}
\onslide<5->
So, we see that 11.23\% of men are taller than the showerhead.
\end{example}
\end{frame}

\begin{frame}[fragile]
\begin{example}\label{airforce_womens_example} 
\vspace{-2mm}
The U.S.\@ Air Force requires that pilots have heights between 64 and 77 in.\@ From Data Set 1 \textquote{Body Data} we see that heights of women have a mean of 63.7 in.\@ and a standard deviation of 2.9 in.\@ What percentage of women meet that requirement?

\vspace{2mm}
\begin{overprint}
\onslide<2>
\begin{center}
\begin{asy}
size(300, 105, IgnoreAspect);

real xmin=54; real xmax=79;

shade_between(63.7,2.9,64,77);
draw(normal_dist(63.7, 2.9, xmin, xmax));

label("64 in.",(64,0),NW);
dot((64,0));

label("77 in.",(77,0),NW);
dot((77,0));

xaxis("$x$",Bottom(), RightTicks(new real [] {55.0, 57.9, 60.8, 63.7, 66.6, 69.5, 72.4, 75.3, 78.2}, size=nan));
\end{asy}
\end{center}
\onslide<3->
\begin{center}
\begin{asy}
size(300, 105, IgnoreAspect);

real xmin=-3.34483; real xmax=5.27586;

shade_between(0,1,0.1,4.59);
draw(normal_dist(0, 1, xmin, xmax));

label("$0.10$",(0.1,0),NW);
dot((0.1,0));

label("$4.59$",(4.59,0),NW);
dot((4.59,0));

xaxis("$z$",Bottom(), RightTicks(new real [] {-3,-2,-1,0,1,2,3,4,5}, size=nan));
\end{asy}
\end{center}
\end{overprint}
\onslide<4->
We then need to compute
\begin{equation*}
\prob{0.10\leq z\leq 4.59} \onslide<5->= 0.4601\quad\text{(rounded)}
\end{equation*}
\onslide<6->
So, we see that about 46\% of women meet the U.S.\@ Air Force requirements.
\end{example}
\end{frame}

\begin{frame}
\begin{block}{When Finding Values from Known Areas}
\begin{itemize}[<+- | alert@+>]
\item Draw a sketch of the graph.
\item Don't confuse $z$ scores and areas.
\item Choose the correct side of the graph.
\item A $z$ score must be negative whenever it is located in the left half of the normal distribution.
\item Areas are always between 0 and 1, and are never negative.
\end{itemize}
\end{block}

\onslide<+->
\begin{block}{Procedure}
\begin{enumerate}[<+- | alert@+>]
\item<.-> Sketch the normal distribution curve, write the given probability or percentage in the appropriate region of the graph, and identify the $x$ values being sought.
\item Either use technology or a table to identify the $z$ scores corresponding to that area.
\item Convert to $x$ values: $x=\mu + z\cdot\sigma$
\item Use your sketch to verify that the solution makes sense.
\end{enumerate}
\end{block}
\end{frame}

\begin{frame}[fragile]
\begin{example}\label{airforce_womens_lower_example}
\vspace{-2mm}
When designing equipment, one common criterion is to use a design that accommodates 95\% of the population. In Example~\ref{airforce_womens_example} we saw that only 46\% of women satisfy the height requirement for U.S.\@ Air Force pilots. 

\onslide<2->
\vspace{1mm}
What would be the maximum acceptable height of a woman if the requirements were changed to allow the shortest 95\% of women?

\vspace{1mm}
\begin{overprint}
\onslide<3>
\begin{center}
\begin{asy}
size(300, 80, IgnoreAspect);

real xmin=-3.5; real xmax=3.5;

shade_between(0,1,xmin,1.645);
draw(normal_dist(0, 1, xmin, xmax));

label("$95\%$", (0,0.5*nd_func(0,1,0)));

label("$z=??$",(1.645,0),NW);
dot((1.645,0));

xaxis("$z$",Bottom(), RightTicks(new real [] {-3,-2,-1,0,1,2,3}, size=nan));
\end{asy}
\end{center}
\onslide<4>
\begin{center}
\begin{asy}
size(300, 80, IgnoreAspect);

real xmin=-3.5; real xmax=3.5;

shade_between(0,1,xmin,1.645);
draw(normal_dist(0, 1, xmin, xmax));

label("$95\%$", (0,0.5*nd_func(0,1,0)));

label("$z=1.645$",(1.645,0),NW);
dot((1.645,0));

xaxis("$z$",Bottom(), RightTicks(new real [] {-3,-2,-1,0,1,2,3}, size=nan));
\end{asy}
\end{center}
\vspace{-5mm}
Using either technology or a table, we find that $z=1.645$.

\vspace{1mm}
\onslide<5->
\begin{center}
\begin{asy}
size(300, 80, IgnoreAspect);

real xmin=53.55; real xmax=73.85;

shade_between(63.7,2.9,xmin,68.4705);
draw(normal_dist(63.7, 2.9, xmin, xmax));

label("$95\%$", (63.7,0.5*nd_func(63.7,2.9,63.7)));

label("$x=68.4705$",(68.4705,0),NW);
dot((68.4705,0));

xaxis("$x$",Bottom(), RightTicks(new real [] {55.0, 57.9, 60.8, 63.7, 66.6, 69.5, 72.4}, size=nan));
\end{asy}
\end{center}
\vspace{-5mm}
Using either technology or a table, we find that $z=1.645$.

\vspace{1mm}
We then need to convert to the $x$ value.
\vspace{-2mm}
\begin{equation*}
x = \mu + z\cdot\sigma = 63.7 + 1.645\cdot 2.9 = 68.4705
\end{equation*}

\vspace{-2mm}
\visible<6->{A requirement of a height less than or equal to 68.5 in.\@ would allow 95\% of women to be eligible.}
\end{overprint}
\end{example}
\end{frame}

\begin{frame}[fragile]
\begin{example}\label{airforce_womens_middle_example}
What would be the maximum and minimum acceptable heights of a woman if the U.S.\@ Air Force requirements were changed to allow the middle 90\% of women?

\vspace{2mm}
\begin{overprint}
\onslide<2>
\begin{center}
\begin{asy}
size(300, 95, IgnoreAspect);

real mu = 0;
real sigma = 1;

real xmin=mu-3.5*sigma; real xmax=mu+3.5*sigma;

shade_between(mu,sigma,-1.645,1.645);
draw(normal_dist(mu, sigma, xmin, xmax));

label("$90\%$", (mu,0.5*nd_func(mu,sigma,mu)));

label("$z_2=??$",(1.645,0),NW);
dot((1.645,0));

label("$z_1=??$",(-1.645,0),NE);
dot((-1.645,0));

xaxis("$z$",Bottom(), RightTicks(new real [] {-3,-2,-1,0,1,2,3}, size=nan));
\end{asy}
\end{center}
\onslide<3>
\begin{center}
\begin{asy}
size(300, 95, IgnoreAspect);

real mu = 0;
real sigma = 1;

real xmin=mu-3.5*sigma; real xmax=mu+3.5*sigma;

shade_between(mu,sigma,xmin,-1.645,mediumyellow);
shade_between(mu,sigma,1.645,xmax,mediumyellow);
shade_between(mu,sigma,-1.645,1.645);
draw(normal_dist(mu, sigma, xmin, xmax));

label("$90\%$", (mu,0.5*nd_func(mu,sigma,mu)));
label("$5\%$", (-1.645,0),NW);
label("$5\%$", (1.645,0),NE);

label("$z_2=??$",(1.645,0),NW);
dot((1.645,0));

label("$z_1=??$",(-1.645,0),NE);
dot((-1.645,0));

xaxis("$z$",Bottom(), RightTicks(new real [] {-3,-2,-1,0,1,2,3}, size=nan));
\end{asy}
\end{center}
\onslide<4>
\begin{center}
\begin{asy}
size(300, 95, IgnoreAspect);

real mu = 0;
real sigma = 1;

real xmin=mu-3.5*sigma; real xmax=mu+3.5*sigma;

shade_between(mu,sigma,xmin,-1.645,mediumyellow);
shade_between(mu,sigma,1.645,xmax,mediumyellow);
shade_between(mu,sigma,-1.645,1.645);
draw(normal_dist(mu, sigma, xmin, xmax));

label("$90\%$", (mu,0.5*nd_func(mu,sigma,mu)));
label("$5\%$", (-1.645,0),NW);
label("$5\%$", (1.645,0),NE);

label("$z_2=1.645$",(1.645,0),NW);
dot((1.645,0));

label("$z_1=-1.645$",(-1.645,0),NE);
dot((-1.645,0));

xaxis("$z$",Bottom(), RightTicks(new real [] {-3,-2,-1,0,1,2,3}, size=nan));
\end{asy}
\end{center}
\vspace{-5mm}
Using either technology or a table, we find that $z_1=-1.645$ and $z_2=1.645$.
\onslide<5->
\begin{center}
\begin{asy}
size(300, 95, IgnoreAspect);

real mu = 63.7;
real sigma = 2.9;

real xmin=mu-3.5*sigma; real xmax=mu+3.5*sigma;

shade_between(mu,sigma,xmin,58.9295,mediumyellow);
shade_between(mu,sigma,68.4705,xmax,mediumyellow);
shade_between(mu,sigma,58.9295,68.4705);
draw(normal_dist(mu, sigma, xmin, xmax));

label("$90\%$", (mu,0.5*nd_func(mu,sigma,mu)));
label("$5\%$", (58.9295,0),NW);
label("$5\%$", (68.4705,0),NE);

label("$x_2=68.4705$",(68.4705,0),NW);
dot((68.4705,0));

label("$x_1=58.9295$",(58.9295,0),NE);
dot((58.9295,0));

xaxis("$x$",Bottom(), RightTicks(new real [] {mu-3*sigma, mu-2*sigma, mu-sigma, mu, mu+sigma, mu+2*sigma, mu+3*sigma}, size=nan));
\end{asy}
\end{center}
\vspace{-5mm}
Using either technology or a table, we find that $z_1=-1.645$ and $z_2=1.645$.

\vspace{1mm}
We then need to convert to the $x$ values.
\vspace{-3mm}
\begin{equation*}
\begin{aligned}
x_1 &= 63.7 - 1.645\cdot 2.9 = 58.9295 \\
x_2 &= 63.7 + 1.645\cdot 2.9 = 68.4705
\end{aligned}
\end{equation*}

\vspace{-4.5mm}
\visible<6->{A requirement of a heights between 58.9 in.\@ and 68.5 in.\@ would allow 90\% of women to be eligible.}
\end{overprint}
\end{example}
\end{frame}

\begin{frame}
\begin{block}{Recall}
\begin{description}
\item[\textbf{Significantly high:}] The value $x$ is significantly high if 
\begin{equation*}
\prob{\text{$x$ or greater}}\leq 0.05
\end{equation*}
\item[\textbf{Significantly low:}] The value $x$ is significantly low if 
\begin{equation*}
\prob{\text{$x$ or less}}\leq 0.05
\end{equation*}
\end{description}
\end{block}\pause

\begin{block}{Note}
The value of $0.05$ is not absolutely rigid, and other values such as $0.01$ may make more sense for a given situation.
\end{block}
\end{frame}

\begin{frame}[fragile]
\begin{example}
Based on Data Set 1, the pulse rates of women have mean 74.0 bpm and standard deviation 12.5 bpm. What pulse rates of women are significantly low or significantly high?

\vspace{2mm}
\begin{overprint}
\onslide<2>
\begin{center}
\begin{asy}
size(300, 80, IgnoreAspect);

real mu = 0;
real sigma = 1;

real left = -1.645;
real right = 1.645;

real xmin=mu-2.5*sigma; real xmax=mu+2.5*sigma;

shade_between(mu,sigma,xmin,left);
shade_between(mu,sigma,right,xmax);
draw(normal_dist(mu, sigma, xmin, xmax));

label("$5\%$", (left,0),NW);
label("$5\%$", (right,0),NE);

label("$z_1=??$",(left,0),NE);
dot((left,0));

label("$z_2=??$",(right,0),NW);
dot((right,0));

xaxis("$z$",Bottom(), RightTicks(new real [] {mu-2*sigma, mu-sigma, mu, mu+sigma, mu+2*sigma}, size=nan));
\end{asy}
\end{center}
\onslide<3>
\begin{center}
\begin{asy}
size(300, 80, IgnoreAspect);

real mu = 0;
real sigma = 1;

real left = -1.645;
real right = 1.645;

real xmin=mu-2.5*sigma; real xmax=mu+2.5*sigma;

shade_between(mu,sigma,left,right, mediumyellow);
shade_between(mu,sigma,xmin,left);
shade_between(mu,sigma,right,xmax);
draw(normal_dist(mu, sigma, xmin, xmax));

label("$90\%$", (mu,0.5*nd_func(mu,sigma,mu)));
label("$5\%$", (left,0),NW);
label("$5\%$", (right,0),NE);

label("$z_1=??$",(left,0),NE);
dot((left,0));

label("$z_2=??$",(right,0),NW);
dot((right,0));

xaxis("$z$",Bottom(), RightTicks(new real [] {mu-2*sigma, mu-sigma, mu, mu+sigma, mu+2*sigma}, size=nan));
\end{asy}
\end{center}
\onslide<4>
\begin{center}
\begin{asy}
size(300, 80, IgnoreAspect);

real mu = 0;
real sigma = 1;

real left = -1.645;
real right = 1.645;

real xmin=mu-2.5*sigma; real xmax=mu+2.5*sigma;

shade_between(mu,sigma,left,right, mediumyellow);
shade_between(mu,sigma,xmin,left);
shade_between(mu,sigma,right,xmax);
draw(normal_dist(mu, sigma, xmin, xmax));

label("$90\%$", (mu,0.5*nd_func(mu,sigma,mu)));
label("$5\%$", (left,0),NW);
label("$5\%$", (right,0),NE);

label("$z_1="+string(left)+"$",(left,0),NE);
dot((left,0));

label("$z_2="+string(right)+"$",(right,0),NW);
dot((right,0));

xaxis("$z$",Bottom(), RightTicks(new real [] {mu-2*sigma, mu-sigma, mu, mu+sigma, mu+2*sigma}, size=nan));
\end{asy}
\end{center}
\vspace{-5mm}
Using either technology or a table, we find that $z_1=-1.645$ and $z_2=1.645$.
\onslide<5->
\begin{center}
\begin{asy}
size(300, 80, IgnoreAspect);

real mu = 74;
real sigma = 12.5;

real left = 53.4;
real right = 94.6;

real xmin=mu-2.5*sigma; real xmax=mu+2.5*sigma;

shade_between(mu,sigma,left,right, mediumyellow);
shade_between(mu,sigma,xmin,left);
shade_between(mu,sigma,right,xmax);
draw(normal_dist(mu, sigma, xmin, xmax));

label("$90\%$", (mu,0.5*nd_func(mu,sigma,mu)));
label("$5\%$", (left,0),NW);
label("$5\%$", (right,0),NE);

label("$x_1="+string(left)+"$",(left,0),NE);
dot((left,0));

label("$x_2="+string(right)+"$",(right,0),NW);
dot((right,0));

xaxis("$x$",Bottom(), RightTicks(new real [] {mu-2*sigma, mu-sigma, mu, mu+sigma, mu+2*sigma}, size=nan));
\end{asy}
\end{center}
\vspace{-5mm}
Using either technology or a table, we find that $z_1=-1.645$ and $z_2=1.645$.

\vspace{1mm}
We then need to convert to the $x$ values.
\vspace{-2.5mm}
\begin{equation*}
\begin{aligned}
x_1 &= 74 - 1.645\cdot 12.5 = 53.4 \\
x_2 &= 74 + 1.645\cdot 12.5 = 94.6
\end{aligned}
\end{equation*}

\vspace{-3.5mm}
\visible<6->{The pulse rates of women that are significant:
\begin{itemize}
\item Significantly low: 53.4 bpm or lower.
\item Significantly high: 94.6 bpm or higher.
\end{itemize}}
\end{overprint}
\end{example}
\end{frame}
\end{document}
