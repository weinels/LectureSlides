\documentclass{beamer}
\usepackage[utf8]{inputenc}
\usepackage[english]{babel}
\usepackage[T1]{fontenc}
\usepackage{slide_helper}
\usepackage[super]{nth}
\usepackage{array}
\usepackage{wasysym}
\usepackage{pgfplots}
\pgfplotsset{compat=1.5} 
\usepgfplotslibrary{statistics}
\usepackage{asy_helper}
\usepackage{colortbl}

\DeclareSymbolFont{extraup}{U}{zavm}{m}{n}
\DeclareMathSymbol{\varheart}{\mathalpha}{extraup}{86}
\DeclareMathSymbol{\vardiamond}{\mathalpha}{extraup}{87}
\DeclareMathSymbol{\varclub}{\mathalpha}{extraup}{84} 
\DeclareMathSymbol{\varspade}{\mathalpha}{extraup}{85}

\begin{asydef}
real nd_func(real mu, real sigma, real x) { return 1/sqrt(2*pi*sigma*sigma)*exp((-1*(x-mu)*(x-mu))/(2*sigma*sigma)); }

guide normal_dist(real mu, real sigma, real xmin, real xmax)
{
	real f(real x) { return nd_func(mu, sigma, x); }
	return graph(f, xmin, xmax);
}

void shade_below(real mu, real sigma, real b, real xmin, real xmax, pen p=orange)
{		
	real f(real x) { return nd_func(mu, sigma, x); }
	guide g = graph(f, xmin, b);
	
	filldraw(g -- (b,0) -- cycle, p, black);
	
	draw((xmin,0)--(b,0));
}

void shade_above(real mu, real sigma, real b, real xmin, real xmax, pen p=orange)
{	
	real f(real x) { return nd_func(mu, sigma, x); }
	guide g = graph(f, xmin, b);
	
	filldraw(g -- (b,0) -- cycle, p, black);
	
	draw((xmin,0)--(b,0));
}

void shade_between(real mu, real sigma, real a, real b, pen p=orange)
{
	real f(real x) { return nd_func(mu, sigma, x); }
	guide g = graph(f, a, b);
	
	filldraw((a,0) -- g -- (b,0) -- cycle, p, black);
}

void multiple_nd_curves_example(real std_dev)
{
	size(300, 190, IgnoreAspect);
    
    draw(normal_dist(0, std_dev, -6,6));
    shade_between(0,std_dev,-std_dev,std_dev);
    draw((0,0)--(0,0.45));
    
    label("$\sigma="+format("%#.2f", std_dev)+"$", (-4.2,0.45), Fill(paleyellow));
    
    xaxis(Bottom(), RightTicks(new real[] {-6,-4,-2,0,2,4,6}));
    yaxis(Left(), LeftTicks(size=nan),ymin = 0, ymax = 0.5);
}
\end{asydef}

\newcommand{\prob}[1]{P\left(#1\right)}
\newcommand{\condprob}[2]{\prob{#1~\middle|~#2}}
\newcommand{\comb}[2]{_{#1}C_{#2}}

\title[MA205 - Section 6.4]{Central Limit Theorem}

\begin{document}
\begin{frame}
\titlepage
\end{frame}

\begin{frame}
\begin{definition}
For all samples of the same size $n$ with $n>30$, the sampling distribution of $\bar{x}$ can be approximated by a normal distribution with mean $\mu$ and standard deviation $\dfrac{\sigma}{\sqrt{n}}$.
\end{definition}\pause

\begin{note}
The Central Limit Theorem holds for \emph{any} population with \emph{any} distribution.
\end{note}\pause

\begin{note}
If the population is normally distributed, then samples of any size will yield means that are normally distributed.
\end{note}\pause

\begin{note}
There are some rare cases where the requirement $n>30$ isn't quite enough.
\end{note}
\end{frame}

\begin{frame}
\begin{example}
If we look at depths of the 600 earthquakes in Data Set 21 we see that this distribution is not normal.
\begin{center}
\begin{tikzpicture}
\begin{axis}[
small,
height=3.4cm,
width=\linewidth,
enlarge x limits=false,
enlarge y limits=false,
ybar interval,
%ymajorgrids=true,
ylabel={Frequency},
xlabel={Earthquake Depth (km)},
x tick label style={rotate=0,anchor=center},
xtick=,
%xtick={0.64, 0.66,...,0.78},
%ytick={0,5,...,1000},
%ymin=0,
%ymax=25,
%xmin=120,
%xmax=280,
%xticklabel style={/pgf/number format/.cd,fixed,precision=0},
%xticklabel=
%\pgfmathprintnumber\tick--\pgfmathprintnumber\nexttick
]
\addplot+ [hist={bins=50}]
table [y=nums] {earthquakes.dat};
\end{axis}
\end{tikzpicture}
\end{center}\pause

\vspace{-5mm}
Looking at 200 sample means, each includes 50 randomly selected earthquake depths, we see that the distribution is approximately normal.
\begin{center}
\begin{tikzpicture}
\begin{axis}[
small,
height=3.4cm,
width=\linewidth,
enlarge x limits=false,
enlarge y limits=false,
ybar interval,
%ymajorgrids=true,
ylabel={Frequency},
xlabel={Sample Mean (km)},
x tick label style={rotate=0,anchor=center},
xtick=,
%xtick={0.64, 0.66,...,0.78},
%ytick={0,5,...,1000},
%ymin=0,
%ymax=25,
%xmin=120,
%xmax=280,
%xticklabel style={/pgf/number format/.cd,fixed,precision=0},
%xticklabel=
%\pgfmathprintnumber\tick--\pgfmathprintnumber\nexttick
]
\addplot+ [hist={bins=15}]
table [y=nums] {earthquake_samples.dat};
\end{axis}
\end{tikzpicture}
\end{center}
\vspace{-4mm}
\end{example}
\end{frame}

\begin{frame}
\begin{block}{Universal Truth}
The Central Limit Theorem describes a rule of nature that works throughout the universe. 

\vspace{2mm}
If we could send a spaceship to a distant planet and collect samples of rocks and weight them, the sample means would have a distribution that is approximately normal.
\end{block}\pause

\begin{block}{Notation}
If all possible simple random samples of size $n$ are selected from a population with mean $\mu$ and standard deviation $\sigma$, the mean of all sample means is denoted by:
\begin{center}
\begin{tabular}{ll}
Mean of all values of $\bar{x}$: & $\mu_{\bar{x}}=\mu$ \\
Standard deviation of all values of $\bar{x}$: & $\sigma_{\bar{x}}=\dfrac{\sigma}{\sqrt{n}}$
\end{tabular}
\end{center}
\end{block}\pause

\begin{note}
$\sigma_{\bar{x}}$ is called the \textbf{standard error of the mean}.
\end{note}
\end{frame}

\begin{frame}[fragile]
\begin{example}
An elevator in a building has a sign stating the maximum capacity is \\\textquote{4000 lb---27 passengers.} Because $4000/27=148$, this converts to a mean passenger weight of 148 lb when the elevator is full.

\onslide<2->
\vspace{1mm}
We will assume a worst-case scenario in which the elevator is filled with 27 adult males. Based on Data Set 1, assume that adult males have weights that are normally distributed with a mean of 189 lb and a standard deviation of 39 lb.

\vspace{1mm}
\begin{overprint}
\onslide<3>
Let us find the probability that 1 randomly selected adult male has weight greater than 148 lb.
\onslide<4-5>
Let us find the probability that 1 randomly selected adult male has weight greater than 148 lb.
\begin{center}
\begin{asy}
size(300, 70, IgnoreAspect);

real mu = 189;
real sigma = 39;

real xmin=mu-2.5*sigma; real xmax=mu+2.5*sigma;

real left = 148;
real right = xmax;

shade_between(mu,sigma,left,right);
draw(normal_dist(mu, sigma, xmin, xmax));

//label("$90\%$", (mu,0.5*nd_func(mu,sigma,mu)));
//label("$5\%$", (left,0),NW);
//label("$5\%$", (right,0),NE);

int i = 1;
if (left != xmin)
{
	label("$x_"+string(i) + "="+string(left)+"$",(left,0),NE);
	dot((left,0));
	i += 1;
}

if (right != xmax)
{
	label("$x_"+string(i) + "="+string(right)+"$",(right,0),NW);
	dot((right,0));
	i += 1;
}

xaxis("$x$",Bottom(), RightTicks(new real [] {mu-2*sigma, mu-sigma, mu, mu+sigma, mu+2*sigma}, size=nan));
\end{asy}
\end{center}
\vspace{-4mm}
\visible<5->{Using technology we find that 0.8531 or 85.31\% of males have weights greater than 148 lb.}
\onslide<6>
Let us find the probability that 27 randomly selected adult male has weight greater than 148 lb.
\onslide<7>
Let us find the probability that 27 randomly selected adult male has weight greater than 148 lb.
\begin{equation*}
\begin{aligned}
\mu_{\bar{x}} &= \mu = 189 \\
\sigma_{\bar{x}} &=\sigma/\sqrt{n} = 39/\sqrt{27} = 7.51
\end{aligned}
\end{equation*}
\onslide<8->
Let us find the probability that 27 randomly selected adult male has weight greater than 148 lb.
\begin{center}
\begin{asy}
size(300, 70, IgnoreAspect);

real mu = 189;
real sigma = 7.51;

real xmin=mu-2.5*sigma; real xmax=mu+2.5*sigma;

real left = 148;
real right = xmax;

shade_between(mu,sigma,left,right);
draw(normal_dist(mu, sigma, xmin, xmax));

//label("$90\%$", (mu,0.5*nd_func(mu,sigma,mu)));
//label("$5\%$", (left,0),NW);
//label("$5\%$", (right,0),NE);

int i = 1;
if (left != xmin)
{
	label("$x_"+string(i) + "="+string(left)+"$",(left,0),NE);
	dot((left,0));
	i += 1;
}

if (right != xmax)
{
	label("$x_"+string(i) + "="+string(right)+"$",(right,0),NW);
	dot((right,0));
	i += 1;
}

xaxis("$x$",Bottom(), RightTicks(new real [] {mu-2*sigma, mu-sigma, mu, mu+sigma, mu+2*sigma}, size=nan));
\end{asy}
\end{center}
\vspace{-4mm}
\visible<9->{Using technology we find that 0.9999 or 99.99\% of the time 27 randomly selected males have a mean weight greater than 148 lb.}
\end{overprint}
\end{example}
\end{frame}

\begin{frame}
\begin{block}{Rare Event Rule for Inferential Statistics}
If, under a given assumption the probability of a particular observed event is very small and the observed event occurs \emph{significantly less than} or \emph{significantly greater than} what we typically expect with that assumption, we conclude that the assumption is probably not correct. 
\end{block}
\end{frame}

\begin{frame}[fragile]
\begin{example}
Assume that the population of human body temperatures has a mean of $\degree{98.6}$F, as is commonly believed. Also assume that the population standard deviation is $\degree{0.62}$F.\pause

\vspace{1mm}
If a sample of size $n=106$ is randomly selected, let's find the probability of getting a mean of $\degree{98.2}$F or lower.\pause

\vspace{-3mm}
\begin{equation*}
\begin{aligned}
\mu_{\bar{x}} &= \mu = 98.6 \\
\sigma_{\bar{x}} &=\sigma/\sqrt{n} = 0.62/\sqrt{106} = 0.0602197
\end{aligned}
\end{equation*}\pause
\vspace{-4mm}
\begin{center}
\begin{asy}
size(300, 49, IgnoreAspect);

real mu = 98.6;
real sigma = 0.06;

real xmin=mu-2.5*sigma; real xmax=mu+2.5*sigma;

real left = 98.1;
real right = 98.2;

shade_between(mu,sigma,left,right);
draw(normal_dist(mu, sigma, min(left,xmin), max(right,xmax)));

//label("$90\%$", (mu,0.5*nd_func(mu,sigma,mu)));
//label("$5\%$", (left,0),NW);
//label("$5\%$", (right,0),NE);

int i = 1;
if (left > xmin)
{
	label("$x_"+string(i) + "="+string(left)+"$",(left,0),NE);
	dot((left,0));
	i += 1;
}

if (right < xmax)
{
	label("$x_"+string(i) + "="+string(right)+"$",(right,0),NW);
	dot((right,0));
	i += 1;
}

xaxis("$x$",Bottom(), RightTicks(new real [] {mu-2*sigma, mu-sigma, mu, mu+sigma, mu+2*sigma}, size=nan));
\end{asy}
\end{center}\pause
\vspace{-4mm}
Using technology we see that the probability is 0.0000000000155. \pause

\vspace{1mm}
If $\degree{98.6}$F is the mean of our body temperature, there is a very small probability of getting a sample mean of $\degree{98.2}$F.\pause

\vspace{1mm}
It is then reasonable to conclude that the population mean is lower than $\degree{98.6}$F. (It's really closer to $\degree{98.2}$F.)
\end{example}
\end{frame}

\begin{frame}
\begin{note}
In applying the Central Limit Theorem, we assume that the population has infinitely many members. When we sample with replacement, this is effectively true.
\end{note}\pause

\begin{block}{Correction for a Finite Population}
When sampling without replacement and the sample size $n$ is greater than 5\% of the finite population size $N$ (that is $n> 0.05N$), adjust the standard deviation of sample means $\sigma_{\bar{x}}$ by multiplying it by this \textbf{finite population correction factor}:
\vspace{-2mm}
\begin{equation*}
\sqrt{\dfrac{N-n}{N-1}}
\end{equation*}
\end{block}\pause

\begin{note}
You \emph{do not} use the finite population correction factor when:
\begin{itemize}
\item You sample with replacement.
\item The population if infinite.
\item Sample size does not exceed 5\% of the population size.
\end{itemize}
\end{note}
\end{frame}
\end{document}
