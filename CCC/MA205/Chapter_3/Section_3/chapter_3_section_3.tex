\documentclass{beamer}
\usepackage[utf8]{inputenc}
\usepackage[english]{babel}
\usepackage[T1]{fontenc}
\usepackage[inline]{asymptote}
\usepackage{pgfplots}
\pgfplotsset{compat=1.5} 
\usepgfplotslibrary{statistics}
\usepackage{tikz}
\usetikzlibrary{shapes,arrows}
\usepackage{slide_helper}

\title[MA205 - Section 3.3]{Measures of Relative Standing and Boxplots}

%% Define block styles
%\tikzstyle{decision} = [diamond, draw, fill=yellow!20, 
%    text width=4.5em, text badly centered, node distance=3cm, inner sep=0pt]
%\tikzstyle{block} = [rectangle, draw, fill=blue!20, 
%    text width=5em, text centered, rounded corners, minimum height=4em]
%\tikzstyle{line} = [draw, -latex']
%\tikzstyle{cloud} = [draw, ellipse,fill=red!20, node distance=3cm,
%    minimum height=2em]
%    
%\begin{tikzpicture}[node distance = 2cm, auto]
%    % Place nodes
%    \node [block] (init) {initialize model};
%    \node [cloud, left of=init] (expert) {expert};
%    \node [cloud, right of=init] (system) {system};
%    \node [block, below of=init] (identify) {identify candidate models};
%    \node [block, below of=identify] (evaluate) {evaluate candidate models};
%    \node [block, left of=evaluate, node distance=3cm] (update) {update model};
%    \node [decision, below of=evaluate] (decide) {is best candidate better?};
%    \node [block, below of=decide, node distance=3cm] (stop) {stop};
%    % Draw edges
%    \path [line] (init) -- (identify);
%    \path [line] (identify) -- (evaluate);
%    \path [line] (evaluate) -- (decide);
%    \path [line] (decide) -| node [near start] {yes} (update);
%    \path [line] (update) |- (identify);
%    \path [line] (decide) -- node {no}(stop);
%    \path [line,dashed] (expert) -- (init);
%    \path [line,dashed] (system) -- (init);
%    \path [line,dashed] (system) |- (evaluate);
%\end{tikzpicture}
    
\begin{document}
\begin{frame}
\titlepage
\end{frame}

\begin{frame}
\begin{definition}
A \textbf{$z$-score} is the number of standard deviations that a given value is above or below the mean.
\end{definition}\pause

\begin{block}{Formula}
The $z$-score for a given value $x$ is calculated as follows.
\begin{description}
\item[\textbf{Sample:}] $z=\dfrac{x-\bar{x}}{s}$
\item[\textbf{Population:}] $z=\dfrac{x-\mu}{\sigma}$
\end{description}
\end{block}\pause

\begin{block}{Properties}
\begin{itemize}[<+- | alert@+>]
\item $z$-scores are expressed as numbers with no units.
\item A data value is \emph{significantly low} if $z\leq -2$.
\item A data value is \emph{significantly high} if $z\geq 2$.
\item If a data value is less than the mean, its $z$-score will be negative.
\end{itemize}
\end{block}
\end{frame}

\begin{frame}
\begin{example}
The weights of a sample of 400 newborn baby weights has mean $\bar{x}=\textcolor<3-4>{blue}{3152.0}$ g and standard deviation $s=\textcolor<3-4>{red}{693.4}$ g. What is the $z$-score of a \textcolor<3-4>{cyan}{4000} g baby?\pause

\vspace{-3mm}
\begin{equation*}
z=\dfrac{x-\bar{x}}{s}\pause
=\dfrac{\textcolor<3-4>{cyan}{4000}-\textcolor<3-4>{blue}{3152.0}}{\textcolor<3-4>{red}{693.4}}\pause
=1.22
\end{equation*}
\end{example}\pause

\begin{example}
The weights of a sample of 106 adult temperature has mean $\bar{x}=\degree{\textcolor<7-8>{blue}{98.20}}$F and standard deviation $s=\degree{\textcolor<7-8>{red}{0.62}}$F. What is the $z$-score of an adult with temperature $\degree{\textcolor<7-8>{cyan}{96.5}}$F?\pause

\vspace{-3mm}
\begin{equation*}
z=\dfrac{x-\bar{x}}{s}\pause
=\dfrac{\textcolor<7-8>{cyan}{96.5}-\textcolor<7-8>{blue}{98.20}}{\textcolor<7-8>{red}{0.62}}\pause
=-2.74
\end{equation*}
\end{example}\pause

\begin{block}{Rounding}
Round $z$-scores to two decimal places.
\end{block}
\end{frame}

\begin{frame}
\begin{definition}
\textbf{Percentiles} are measures of location, denoted $P_{1},P_{2},\ldots,P_{99}$, which divide a set of data into 100 groups with about $1\%$ of the values in each group.
\end{definition}\pause

\begin{block}{Formula}
The process of finding the percentile that corresponds to a particular data value $x$ is given by the following:
\begin{equation*}
\text{Percentile of value $x$} = \dfrac{\text{number of values less than $x$}}{\text{total number of values}}\cdot 100
\end{equation*}
\end{block}\pause

\begin{block}{Rounding}
Round percentiles to the nearest whole number.
\end{block}
\end{frame}

\begin{frame}
\begin{example}
The table lists the 50 Verizon airport data speeds, in Mbps, from\\ Data Set 32 in Appendix B.
\begin{center}
\begin{tabular}{rrrrrrrrrr}
38.5 & 55.6 & 22.4 & 14.1 & 23.1 & 24.5 & \textcolor<2->{blue}{6.5} & 21.5 & 25.7 & 14.7 \\
77.8 & 71.3 & 43.0 & 20.2 & 15.5 & 13.7 & \textcolor<2->{blue}{11.1} & 13.5 & \textcolor<2->{blue}{10.2} & 21.1 \\
15.1 & 14.2 & \textcolor<2->{blue}{4.5} & \textcolor<2->{blue}{7.9} & \textcolor<2->{blue}{9.9} & \textcolor<2->{blue}{10.3} & \textcolor<2->{blue}{6.2} & 17.5 & 22.2 & 13.1 \\
18.2 & 28.5 & 15.8 & 15.0 & \textcolor<2->{blue}{11.1} & \textcolor<2->{red}{11.8} & 16.0 & \textcolor<2->{blue}{10.9} & \textcolor<2->{blue}{1.8} & 34.6 \\
\textcolor<2->{blue}{4.6} & 12.0 & \textcolor<2->{blue}{11.6} & \textcolor<2->{blue}{3.6} & \textcolor<2->{blue}{1.9} & \textcolor<2->{blue}{7.7} & \textcolor<2->{blue}{0.8} & \textcolor<2->{blue}{4.5} & \textcolor<2->{blue}{1.4} & \textcolor<2->{blue}{3.2} \\
\end{tabular}
\end{center}
What percentile is the data value 11.8 Mbps in?\pause

There are 20 \textcolor{blue}{data values less than} \textcolor{red}{11.8} Mbps.\pause 

\vspace{-3mm}
\begin{equation*}
\text{Percentile of}~\textcolor{red}{11.8}=\dfrac{20}{50}\cdot 100\pause = 40
\end{equation*}

\vspace{-5mm}
A data speed of 11.8 Mbps is in the 40th percentile.
\end{example}\pause

\begin{block}{Note}
This can be interpreted loosely as 40\% of Verizon data speeds are slower than 11.8 Mbps and 60\% of Verizon data speeds are faster than 11.8 Mbps.
\end{block}
\end{frame}

\begin{frame}
\begin{block}{Converting a Percentile to a Data Value}
Notation:
\begin{itemize}
\item $n$ is the total number of values in the data set.
\item $k$ is the percentile being used.
\item $L$ is the locator that gives the position of a value.
\item $P_k$ is the $k$th percentile.
\end{itemize}\pause
To find which data value is in the $P_k$ percentile:
\begin{enumerate}
\item Sort the data from lowest to highest.\pause
\item Compute $L=\left(\dfrac{k}{100}\right)n$\pause
\item \begin{itemize}
\item If $L$ is a whole number, the value of the $k$th percentile is midway between the $L$th value and the next value in the sorted data. Add the $L$th value and $(L+1)$th value, then divide by 2.\pause
\item If $L$ is not a whole number, round $L$ up to the nearest whole number. $P_k$is the $L$th data value. 
\end{itemize}
\end{enumerate}
\end{block}
\end{frame}

\begin{frame}
\begin{example}
The table lists the \textcolor<4-5>{red}{50} Verizon airport data speeds, in Mbps, from\\ Data Set 32 in Appendix B.
\alt<2->{%
\begin{center}
\begin{tabular}{|rrrrrrrrrr|}\hline
\textcolor<7->{red}{0.8} & \textcolor<7->{red}{1.4} & \textcolor<7->{red}{1.8} & \textcolor<7->{red}{1.9} & \textcolor<7->{red}{3.2} & \textcolor<7->{red}{3.6} & \textcolor<7->{red}{4.5} & \textcolor<7->{red}{4.5} & \textcolor<7->{red}{4.6} & \textcolor<7->{red}{6.2} \\
\textcolor<7->{red}{6.5} & \textcolor<7->{red}{7.7} & \textcolor<7->{blue}{7.9} & 9.9 & 10.2 & 10.3 & 10.9 & 11.1 & 11.1 & 11.6 \\
11.8 & 12.0 & 13.1 & 13.5 & 13.7 & 14.1 & 14.2 & 14.7 & 15.0 & 15.1 \\
15.5 & 15.8 & 16.0 & 17.5 & 18.2 & 20.2 & 21.1 & 21.5 & 22.2 & 22.4 \\
23.1 & 24.5 & 25.7 & 28.5 & 34.6 & 38.5 & 43.0 & 55.6 & 71.3 & 77.8 \\\hline
\end{tabular}
\end{center}
}{%
\begin{center}
\begin{tabular}{|rrrrrrrrrr|}\hline
38.5 & 55.6 & 22.4 & 14.1 & 23.1 & 24.5 & 6.5 & 21.5 & 25.7 & 14.7 \\
77.8 & 71.3 & 43.0 & 20.2 & 15.5 & 13.7 & 11.1 & 13.5 & 10.2 & 21.1 \\
15.1 & 14.2 & 4.5 & 7.9 & 9.9 & 10.3 & 6.2 & 17.5 & 22.2 & 13.1 \\
18.2 & 28.5 & 15.8 & 15.0 & 11.1 & 11.8 & 16.0 & 10.9 & 1.8 & 34.6 \\
4.6 & 12.0 & 11.6 & 3.6 & 1.9 & 7.7 & 0.8 & 4.5 & 1.4 & 3.2 \\\hline
\end{tabular}
\end{center}
}
What is the value in the \textcolor<4-5>{blue}{25}th percentile, $P_{25}$?\pause

First, sort the data.\pause

We next need to compute
\begin{equation*}
L=\dfrac{k}{100}\cdot n\pause
=\dfrac{\textcolor<4-5>{blue}{25}}{100}\cdot \textcolor<4-5>{red}{50}\pause
=12.5\pause
\end{equation*}
Since $L=12.5$ is not a whole number, we round up to $13$.\pause

So, $P_{25}$ is the 13th data value, \textcolor<7->{blue}{7.9} Mbps.
\end{example}
\end{frame}

\begin{frame}
\begin{definition}
\textbf{Quartiles} are measures of location, denoted $Q_1$, $Q_2$, and $Q_3$, which divide a set of data into four groups with about 25\% of the values in each group.
\end{definition}\pause

\begin{block}{Quartile Descriptions}
\begin{description}
\item[\textbf{First Quartile, $\boldsymbol{Q_1}$:}] Same value as $P_{25}$. It separates the bottom 25\% of the sorted values from the top 75\%.\pause
\item[\textbf{Second Quartile, $\boldsymbol{Q_2}$:}] Same as the $P_{50}$ and the median. It separates the bottom 50\% of the sorted values from the top 50\%\pause
\item[\textbf{Third Quartile, $\boldsymbol{Q_3}$}:] Same as $P_{75}$. It separates the bottom 75\% of the sorted values from the top 25\%.
\end{description}
\end{block}\pause

\begin{block}{Note}
Use the same procedure for calculating percentiles to calculate quartiles.
\end{block}
\end{frame}
\end{document}
