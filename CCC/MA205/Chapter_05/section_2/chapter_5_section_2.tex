\documentclass{beamer}
\usepackage[utf8]{inputenc}
\usepackage[english]{babel}
\usepackage[T1]{fontenc}
\usepackage{slide_helper}
\usepackage[super]{nth}
\usepackage{array}
\usepackage{wasysym}
\usepackage{pgfplots}
\pgfplotsset{compat=1.5} 
\usepgfplotslibrary{statistics}

\DeclareSymbolFont{extraup}{U}{zavm}{m}{n}
\DeclareMathSymbol{\varheart}{\mathalpha}{extraup}{86}
\DeclareMathSymbol{\vardiamond}{\mathalpha}{extraup}{87}
\DeclareMathSymbol{\varclub}{\mathalpha}{extraup}{84} 
\DeclareMathSymbol{\varspade}{\mathalpha}{extraup}{85}

\newcommand{\suitheart}[1][]{{\color{red}\text{#1}\varheart}}
\newcommand{\suitspade}[1][]{{\color{black}\text{#1}\spadesuit}}
\newcommand{\suitdiamond}[1][]{{\color{red}\text{#1}\vardiamond}}
\newcommand{\suitclub}[1][]{{\color{black}\text{#1}\varclub}}

\newcommand{\prob}[1]{P\left(#1\right)}
\newcommand{\condprob}[2]{\prob{#1~\middle|~#2}}
\newcommand{\comb}[2]{{_{#1}C_{#2}}}
\newcommand{\perm}[2]{_{#1}P_{#2}}

\title[MA205 - Section 5.2]{Binomial Probability Distributions}

\begin{document}
\begin{frame}
\titlepage
\end{frame}

\begin{frame}
\begin{block}{Types of Discrete Probability Distributions}
In this section we will talk about the Binomial Probability Distribution, which is a discrete probability distribution. 

\vspace{1mm}
Other types of discrete probability distributions include:
\begin{itemize}
\item Poisson distributions
\item Geometric distributions
\item Hypergeometric distributions
\end{itemize}
\end{block}\pause

\begin{definition}
A \textbf{binomial probability distribution} results from a procedure that meets these four requirements:\pause
\begin{itemize}
\item The procedure has a fixed number of trials.\pause
\item The trials must be independent.\pause
\item Each trial must have all outcomes classified into exactly two categories, commonly referred to as \textbf{success} and \text{failure}.\pause
\item The probability of a success remains the same in all trails.
\end{itemize}
\end{definition}
\end{frame}

\begin{frame}
\begin{block}{Notation}
\begin{center}
\begin{tabular}{ll}
\multicolumn{2}{l}{$S$ (success) and $F$ (failure) denote the two possible categories.}\\\pause
$\prob{S}=p$ & ($p$ is the probability of a success) \\\pause
$\prob{F}=q$ & ($q=1-p$ is the probability of a failure) \\\pause
$n$ & the fixed number of trails\\\pause
$x$ & a specific number of successes in $n$ trails, so $x$ is a\\
& whole number between 0 and $n$, inclusive.\\\pause
$\prob{x}$ & probability of getting exactly $x$ successes among the $n$ trials.
\end{tabular}
\end{center}
\end{block}\pause

\begin{block}{Caution}
When using a binomial probability distribution, \emph{always} be sure that $x$ and $p$ are consistent in the sense that they both refer to the \emph{same} category being called a success.
\end{block}\pause

\begin{block}{Caution}
The probability $p$ is the probability of getting a success on just \emph{one} individual trail.
\end{block}
\end{frame}

\begin{frame}
\begin{example}
When an adult is randomly selected (with replacement), there is a 0.85 probability that this person knows what Twitter is (based on results from a Pew Research Center survey). Suppose that we want to find the probability that exactly three of five randomly selected adults know what Twitter is.\pause

\vspace{1mm}
Does this procedure result in a binomial distribution?\pause
\begin{itemize}
\item The procedure has 5 trials.\pause
\item The 5 trials are independent because the probability of any adult knowing Twitter is not affected by results from other selected adults.\pause
\item The two categories are \textquote{Knows Twitter} and \textquote{Doesn't know Twitter.}\pause
\item For each randomly selected adult, the probability they know Twitter is 0.85 and that probability remains the same for each selected person.\pause
\end{itemize}
We see that this is a binomial distribution.\pause

\vspace{1mm}
In this example we have
\begin{center}
\begin{tabular}{lcl}
$n = 5$ &\quad& $x = 3$ \\
$p = 0.85$ &\quad& $q = 1-0.85=0.15$
\end{tabular}
\end{center}
\end{example}
\end{frame}

\begin{frame}
\begin{block}{Note}
Sampling without replacement violates the requirements for a binomial distribution.
\end{block}\pause

\begin{block}{5\% Guideline for Cumbersome Calculations}
When sampling without replacement and the same size is no more than $5\%$ of the size of the population, treat the selections as being independent (even though they are actually dependent).
\end{block}\pause

\begin{block}{Methods of Finding Binomial Probabilities}
\begin{enumerate}
\item Using the Binomial Probability Formula.
\item Using technology.
\item Using a table.
\end{enumerate}
\end{block}\pause

\begin{block}{Note}
Technology is most often used to calculate binomial probabilities.
\end{block}
\end{frame}

\begin{frame}
\begin{block}{Binomial Probability Formula}
\begin{equation*}
\prob{x}=\dfrac{n!}{{(n-x)}!x!}\cdot p^x\cdot q^{n-x}
\qquad\text{for}~x=0,1,\ldots,n
\end{equation*}
where
\begin{center}
\begin{tabular}{ll}
$n$ & is the number of trails. \\
$x$ & is the number of successes among $n$ trials. \\
$p$ & is the probability of success in any one trial. \\
$q$ & is the probability of failure in any one trial. ($q=1-p$)
\end{tabular}
\end{center}
\end{block}\pause

\begin{block}{Note}
We can also write this formula as 
\begin{equation*}
\prob{x}=\comb{n}{x}\cdot p^x\cdot q^{n-x}
\qquad\text{for}~x=0,1,\ldots,n
\end{equation*}
where $x$ items identical to themselves, and $n-x$ other items identical to themselves, the number of permutations is $\comb{n}{x}$.
\end{block}
\end{frame}

\begin{frame}
\begin{example}
When an adult is randomly selected (with replacement), there is a 0.85 probability that this person knows what Twitter is (based on results from a Pew Research Center survey). Suppose that we want to find the probability that exactly three of five randomly selected adults know what Twitter is.\pause

\vspace{1mm}
We have
\begin{center}
\begin{tabular}{lcl}
$n = 5$ &\quad& $x = 3$ \\
$p = 0.85$ &\quad& $q = 1-0.85=0.15$
\end{tabular}
\end{center}\pause
Applying the formula gives
\begin{equation*}
\begin{aligned}
\prob{3}&=\dfrac{5!}{{(5-3)}!3!}\cdot {(0.85)}^{3}\cdot {(0.15)}^{5-3}\\\pause
&= 0.138~\text{(rounded)}
\end{aligned}
\end{equation*}\pause
We then round to three significant digits to get the probability that exactly three out of five randomly selected adults know Twitter is 0.138.
\end{example}
\end{frame}

\begin{frame}
\begin{block}{Technology}
If you have a programmable calculator that does not have a binomial distribution function, you can create a program from the formula.
\end{block}\pause

\begin{example}
When an adult is randomly selected (with replacement), there is a 0.85 probability that this person knows what Twitter is (based on results from a Pew Research Center survey). Suppose that we want to find the probability that exactly three of five randomly selected adults know what Twitter is.\pause

\vspace{1mm}
We have
\begin{center}
\begin{tabular}{lcl}
$n = 5$ &\quad& $x = 3$ \\
$p = 0.85$ &\quad& $q = 0.15$
\end{tabular}
\end{center}\pause

\begin{center}
\begin{tabular}{ll}
Calculator gives: & 0.138178125 \\
Statdisk gives: & 0.1381781 \\
Table gives: & 0.139
\end{tabular}
\end{center}
\end{example}
\end{frame}

\begin{frame}
\begin{example}
Between 1974 and 2011, there were 460 NFL football games decided in overtime, and 252 of them were won by the team that won the overtime coin toss. Is the result of 252 wins in the 460 games equivalent to random chance, or is the 252 wins significantly high?\pause

\vspace{1mm}
Using the notation for binomial probabilities, we have $n=460$, $p=0.5$, and $q=0.5$. What is $x$?\pause

\vspace{1mm}
Significantly high means we need to check if the probability of 252 or more wins in 460 games is less than 0.05.\pause

\vspace{1mm}
To calculate this with the formula, you would need to calculate

\vspace{-3mm}
\begin{equation*}
\begin{aligned}
\prob{\text{252 or more}} &= \prob{\text{252 or 253 or $\ldots$ or 459 or 460}}\\
&= \prob{252}+\prob{253}+\cdots+\prob{459}+\prob{460}
\end{aligned}
\end{equation*}\pause
Instead, it is better to use technology.

\vspace{-3mm}
\begin{equation*}
\prob{252\text{ or more}}=0.0224
\end{equation*}\pause
Since $0.0224<0.05$ we see that it is unlikely we would get 252 or more wins by chance.
\end{example}
\end{frame}

\begin{frame}
\begin{block}{Mean and Standard Deviation}
For a binomial distribution, the formulas for mean and standard deviation can be rewritten as:
\begin{center}
\begin{tabular}{rll}
Mean: & $\mu^{\phantom{2}}=np$ \\
Variance: & $\sigma^2=npq$ \\
Standard Deviation: & $\sigma^{\phantom{2}}=\sqrt{npq}$
\end{tabular}
\end{center}
\end{block}\pause

\begin{block}{Range Rule of Thumb}
\textbf{Significantly low} values $\leq (\mu-2\sigma)$

\vspace{1mm}
\textbf{Significantly high} values $\geq (\mu+2\sigma)$

\vspace{1mm}
\textbf{Values not significant}: Between $(\mu-2\sigma)$ and $(\mu+2\sigma)$
\end{block}
\end{frame}

\begin{frame}
\begin{example}
Between 1974 and 2011, there were 460 NFL football games decided in overtime, and 252 of them were won by the team that won the overtime coin toss. \pause

\vspace{1mm}
Using the notation for binomial probabilities, we have $n=460$, $p=0.5$, and $q=0.5$. We then get:
\begin{equation*}
\begin{aligned}
\mu &= np = (460)(0.5) = 230~\text{games} \\
\sigma &= \sqrt{npq} = \sqrt{(460)(0.5)(0.5)} = 10.7~\text{games (rounded)}
\end{aligned}
\end{equation*}\pause
The number of wins that are significantly low are:
\begin{equation*}
\mu-2\sigma = 230-2(10.7) = 208.6~\text{games or fewer}
\end{equation*}\pause
The number of wins that are significantly high are:
\begin{equation*}
\mu+2\sigma = 230+2(10.7) = 251.4~\text{games or more}
\end{equation*}\pause
Since $252>251.4$, we see that 252 wins is significantly high.
\end{example}
\end{frame}
\end{document}
