\documentclass{beamer}
\usepackage[utf8]{inputenc}
\usepackage[english]{babel}
\usepackage[T1]{fontenc}
\usepackage[inline]{asymptote}
\usepackage{slide_helper}

\title[MA205 - Section 1.1]{Statistical and Critical Thinking}

\begin{document}
\begin{frame}
\titlepage
\end{frame}

\begin{frame}
\begin{definition}
\textbf{Data} are collections of observations, such as measurements, genders, or survey responses.
\end{definition}\pause

\begin{definition}
\textbf{Statistics} is the science of planning studies and experiments; obtaining data; and organizing, summarizing, presenting, analyzing, and interpreting those data and then drawing conclusions based on them.
\end{definition}\pause

\begin{definition}
A \textbf{population} is the complete collection of all measurements or data that are being considered.
\end{definition}\pause

\begin{definition}
A \textbf{census} is the collection of data from every member of the population.
\end{definition}\pause

\begin{definition}
A \textbf{sample} is a subcollection of members selected from a population.
\end{definition}
\end{frame}

\begin{frame}
\begin{example}
In the journal article \textquote{Residential Carbon Monoxide Detector Failure Rates in the United States} (by Ryan and Arnold, \emph{American Journal of Public Health}, Vol. 101, No. 10) it was stated that there are 38 million carbon monoxide detectors installed in the United States. When 30 of them were randomly selected and tested, it was found that 12 of them failed an alarm in hazardous carbon monoxide conditions.

\vspace{5mm}
\textbf{What is the population?}\pause 

All 38 million carbon monoxide detectors in the United states. \pause

\vspace{5mm}
\textbf{What is the sample?}\pause 

The 30 carbon monoxide detectors that were selected and tested.
\end{example}\pause

\begin{block}{Note}
Performing a full census is typically very hard. So, the goal is to use sample data as a basis for drawing conclusions about the full population. The methods of statistics are helpful in drawing such conclusions.
\end{block}
\end{frame}

\begin{frame}
\begin{block}{General Process of a Statistical Study}
\begin{enumerate}
\item Prepare
\begin{itemize}
\item Context. (What do the data represent? What is the goal of the study?)
\item Source of the Data. (Is there pressure to obtain results favorable to the source?) 
\item Sampling Method. (Were there any biases in data collection?)
\end{itemize}\pause
\item Analyze
\begin{itemize}
\item Graph the Data
\item Explore the Data
\begin{itemize}
\item Are there any outliers?
\item What important statistics summarize the data?
\item How are the data distributed?
\item Are there missing data?
\item Did many selected subjects refuse to respond?
\end{itemize}
\item Apply Statistical Methods. (Technology is often used.)
\end{itemize}\pause
\item Conclude
\begin{itemize}
\item Significance
\begin{itemize}
\item Do the results have statistical significance?
\item Do the results have practical significance?
\end{itemize}
\end{itemize}
\end{enumerate}
\end{block}
\end{frame}

\begin{frame}
\begin{block}{Bias}
When collecting data there are many types of bias:
\begin{itemize}
\item A \textbf{voluntary response sample} (or \textbf{self-selected sample}) is one which the respondents themselves decide whether to be included.\pause
\begin{itemize}
\item Internet polls, such as on Twitter, where people decided whether or not to participate.
\item Call-in polls, where people are asked to call a special number to register an opinion.
\end{itemize}\pause
\item A \textbf{loaded question} is one that is intentionally worded to elicit a desired response.\pause
\begin{itemize}
\item When asked \textquote{Should the President have the line item veto to eliminate waste?} 97\% of respondents said \textquote{yes.}
\item When asked \textquote{Should the President have the line item veto?} 57\% of respondents said \textquote{yes.}
\end{itemize}\pause
\item A \textbf{nonresponse} occurs when someone refuses to respond or is not available.\pause
\begin{itemize}
\item Many telemarketers have been disguising their sales pitch as an opinion poll, causing the nonresponse problems to increase in recent years.
\end{itemize}
\end{itemize}
\end{block}
\end{frame}

\begin{frame}
\begin{block}{Percentages}
\begin{itemize}
\item \textbf{Percentage of:} To find a percentage of an amount, replace the \% symbol with division by 100 and multiply by the amount.
\begin{description}
\item[Example:] 6\% of 1200 responses is $\frac{6}{100}\cdot 1200=72$
\end{description}
\item \textbf{Decimal to Percentage:} To convert from a decimal to a percentage, multiply by 100\%.
\begin{description}
\item[Example:] $0.25\rightarrow 0.25\cdot100\%=25\%$
\end{description}
\item \textbf{Fraction to Percentage:} To convert from a fraction to a percentage, divide the denominator into the numerator and multiply by 100\%.
\begin{description}
\item[Example:] $\frac{3}{4}=0.75\rightarrow 0.75\cdot 100\%=75\%$
\end{description}
\item \textbf{Percentage to Decimal:} To convert from a percentage to a decimal number, replace the \% by division by 100. 
\begin{description}
\item[Example:] $85\%\rightarrow \frac{85}{100}=0.85$
\end{description}
\end{itemize}
\end{block}
\end{frame}
\end{document}
