\documentclass{beamer}
\usepackage[utf8]{inputenc}
\usepackage[english]{babel}
\usepackage[T1]{fontenc}
\usepackage{slide_helper}
\usepackage[inline]{asymptote}
\usepackage{asy_helper}
\usepackage[super]{nth}
\usepackage{array}
\usepackage{wasysym}
\usepackage{pgfplots}
\pgfplotsset{compat=1.5} 
\usepgfplotslibrary{statistics}

\DeclareSymbolFont{extraup}{U}{zavm}{m}{n}
\DeclareMathSymbol{\varheart}{\mathalpha}{extraup}{86}
\DeclareMathSymbol{\vardiamond}{\mathalpha}{extraup}{87}
\DeclareMathSymbol{\varclub}{\mathalpha}{extraup}{84} 
\DeclareMathSymbol{\varspade}{\mathalpha}{extraup}{85}

\newcommand{\suitheart}[1][]{{\color{red}\text{#1}\varheart}}
\newcommand{\suitspade}[1][]{{\color{black}\text{#1}\spadesuit}}
\newcommand{\suitdiamond}[1][]{{\color{red}\text{#1}\vardiamond}}
\newcommand{\suitclub}[1][]{{\color{black}\text{#1}\varclub}}

\newcommand{\prob}[1]{P\left(#1\right)}
\newcommand{\condprob}[2]{\prob{#1~\middle|~#2}}
\newcommand{\comb}[2]{{_{#1}C_{#2}}}
\newcommand{\perm}[2]{_{#1}P_{#2}}

\newcommand{\nullhypothesis}[1]{H_0:~{#1}}
\newcommand{\althypothesis}[1]{H_A:~{#1}}

\begin{asydef}
real nd_func(real mu, real sigma, real x) { return 1/sqrt(2*pi*sigma*sigma)*exp((-1*(x-mu)*(x-mu))/(2*sigma*sigma)); }

guide normal_dist(real mu, real sigma, real xmin, real xmax)
{
	real f(real x) { return nd_func(mu, sigma, x); }
	return graph(f, xmin, xmax);
}

void shade_between(real mu, real sigma, real a, real b, pen p=orange)
{
	real f(real x) { return nd_func(mu, sigma, x); }
	guide g = graph(f, a, b);
	
	filldraw((a,0) -- g -- (b,0) -- cycle, p, black);
}

void multiple_nd_curves_example(real std_dev)
{
	size(300, 190, IgnoreAspect);
    
    draw(normal_dist(0, std_dev, -6,6));
    shade_between(0,std_dev,-std_dev,std_dev);
    draw((0,0)--(0,0.45));
    
    label("$\sigma="+format("%#.2f", std_dev)+"$", (-4.2,0.45), Fill(paleyellow));
    
    xaxis(Bottom(), RightTicks(new real[] {-6,-4,-2,0,2,4,6}));
    yaxis(Left(), LeftTicks(size=nan),ymin = 0, ymax = 0.5);
}
\end{asydef}

\begin{asydef}
pair crit = (150, 33);
real a=-1.25;
real b=1.25;
\end{asydef}

\title[MA205 - Section 8.2]{Testing a Claim About a Proportion}

\begin{document}
\begin{frame}
\titlepage
\end{frame}

\begin{frame}
\begin{block}{Requirements}
\begin{enumerate}
\item The sample observations are a simple random sample.\pause
\item The conditions of the binomial distribution are satisfied:
\begin{itemize}
\item There is a fixed number of trials.
\item The trials are independent.
\item Each trial has two categories of \textquote{success} and \textquote{failure.}
\item The probability of a success remains the same in all trials.
\end{itemize}\pause
\item The conditions of $np\geq 5$ and $nq\geq 5$ are both satisfied.
\end{enumerate}
\end{block}\pause

\begin{block}{Note}
When these requirements are met, we can approximate a binomial distribution with a normal distribution using $\mu=np$ and $\sigma=\sqrt{npq}$
\end{block}
\end{frame}

\begin{frame}
\begin{example}\label{exp_drones}
In a Pitney Bowes survey in which 1009 consumers were asked if they are comfortable with having drones deliver their purchases, and 54\% of them responded with \textquote{no.}\pause

\vspace{1mm}
Let us check that requirements are met for the claim that most (more than half) of consumers are uncomfortable with drone deliveries.\pause

\vspace{1mm}
Requirements check:
\begin{enumerate}
\item The 1009 consumers were randomly selected.\pause
\item There are a fixed number of independent trials where the two categories are wether the subject is comfortable with drone deliveries or not.\pause
\item We have $n=1009$, $p=0.5$, and $q=0.5$ and so
\begin{equation*}
\begin{aligned}
np&=1009\cdot 0.5 = 504.5\geq 5 \\
nq&=1009\cdot 0.5 = 504.5\geq 5
\end{aligned}
\end{equation*}
\end{enumerate}
\end{example}
\end{frame}

\begin{frame}
\begin{block}{Test Statistic}
\begin{equation*}
z=\dfrac{\hat{p}-p}{\sqrt{\dfrac{pq}{n}}}
\end{equation*}
\end{block}\pause

\begin{block}{Equivalent Methods}
\begin{itemize}
\item $P$-values is the most common method used.\pause
\item Critical values can be used when technology is unavailable.
\end{itemize}
\end{block}\pause

\begin{block}{Confidence Intervals}
Confidence intervals use an estimated standard deviation based on the sample proportion, which means this method could result in a different conclusion.
\end{block}\pause

\begin{block}{Technology}
$P$-values are usually provided automatically by technology. Otherwise use the standard normal distribution.
\end{block}
\end{frame}

\begin{frame}
\begin{example}
In a Pitney Bowes survey in which 1009 consumers were asked if they are comfortable with having drones deliver their purchases, and 54\% of them responded with \textquote{no.}\pause

\vspace{1mm}
Let us test the claim that most (more than half) of consumers are uncomfortable with drone deliveries.\pause

\vspace{1mm}
The hypotheses for this example are:
\begin{equation*}
\begin{aligned}
\nullhypothesis{p=0.5} \\
\althypothesis{p>0.5}
\end{aligned}
\end{equation*}\pause

Using technology we get $P$-value of $0.005386<0.05$ which means we reject the null hypothesis.\pause

\vspace{1mm}
Because we reject the null hypothesis we conclude that there is sufficient sample evidence to support the claim that more than half of consumers are uncomfortable with drone deliveries.
\end{example}
\end{frame}

\begin{frame}
\begin{block}{Caution}
Be careful not to confuse the notation.
\begin{center}
\begin{tabular}{ll}
\textbf{$\boldsymbol{P}$-value} & The probability of a test statistic at least as extreme as \\
&the one obtained.\\
$\boldsymbol{p}$ & The population proportion.\\
$\boldsymbol{\hat{p}}$ & The sample proportion.
\end{tabular}
\end{center}
\end{block}
\end{frame}

\begin{frame}
\begin{note}
When using technology, you often need to enter the sample size $n$ and the number of successes $x$. But, in real applications the sample proportion $\hat{p}$ is often given instead of $x$.\pause
\begin{equation*}
\hat{p} = \dfrac{x}{n} \pause
\quad\Rightarrow\quad
x = n\cdot\hat{p}
\end{equation*}
\end{note}\pause

\begin{example}
A study of sleepwalking of \textquote{nocturnal wandering} was described in \emph{Neurology} magazine, and it included information that 29.2\% of $19,136$ American adults have sleepwalked.\pause

\vspace{1mm}
The number of adults who have sleepwalked is
\begin{equation*}
0.292\cdot 19136\pause=5587.712
\end{equation*}\pause
But, the result must be a whole number, so we round to the nearest whole number of 5588.
\end{example}
\end{frame}

\begin{frame}
\begin{example}\label{exp_sleepwalking}
A study of sleepwalking of \textquote{nocturnal wandering} was described in \emph{Neurology} magazine, and it included information that 29.2\% of $19,136$ American adults have sleepwalked. Let us test the claim that \textquote{fewer than 30\% of adults have sleepwalked} using $\alpha=0.05$.\pause

\vspace{1mm}
\begin{enumerate}
\item The original claim is $p<0.30$ and the opposite claim is $p\geq0.30$.\pause
\item Because $p<0.30$ does not contain equality, we have the hypotheses:
\vspace{-2mm}
\begin{equation*}
\nullhypothesis{p=0.30} 
\quad\text{and}\quad
\althypothesis{p<0.30}
\end{equation*}\pause
\vspace{-8mm}
\item The relevant statistic is $\hat{p}$ and the sampling distribution of sample proportions can be approximated by the normal distribution.\pause
\item Using technology we get the test statistic $z=-2.41039$ and $P$-value of $0.007968$.\pause
\item Because $0.007968\leq0.05$ we reject the null hypothesis.\pause
\end{enumerate}
We conclude that there is sufficient evidence to support the claim that fewer than 30\% of adults have sleepwalked.
\end{example}
\end{frame}

\begin{frame}
\begin{block}{Critical Value Method}
If we had used critical values instead of $P$-values for Example~\ref{exp_sleepwalking} we would have used a critical value of $z=-1.645$.\pause

\vspace{1mm}
We would reject the null hypothesis because $z=-2.41039$ falls in the critical region bounded by $z=-1.645$.
\end{block}\pause
\begin{note}
This is the same conclusion we reached using $P$-values.
\end{note}
\end{frame}

\begin{frame}
\begin{block}{Confidence Interval Method}
If we were to repeat Example~\ref{exp_sleepwalking} using confidence intervals we would get a 90\% confidence interval of $0.287<p<0.297$.\pause

\vspace{1mm}
Since the entire confidence interval is less than 0.30, there is sufficient evidence to support the claim that fewer than 30\% of adults have sleepwalked.
\end{block}\pause

\begin{note}
While this is the same result, remember that we have no guarantee that confidence intervals will give the same result as $P$-values.
\end{note}
\end{frame}
\end{document}
