\documentclass{beamer}

\usepackage[english]{babel}
\usepackage[utf8x]{inputenc}
\usepackage{slide_helper}
\usepackage{graphicx}

\usepackage[customcolors, beamer]{hf-tikz}
\tikzset{style green/.style={
    set fill color=green!60!lime!50,
    set border color=green!100,
  },
  style cyan/.style={
    set fill color=cyan!50!blue!40,
    set border color=cyan!100,
  },
  style orange/.style={
    set fill color=orange!50!red!60,
    set border color=red!80,
  },
  hor/.style={
    above left offset={-0.15,0.35},
    below right offset={0.15,-0.125},
    #1
  },
  ver/.style={
    above left offset={-0.38,0.35},
    below right offset={0.15,-0.15},
    #1
  },
  ver2/.style={
    above left offset={-0.15,0.35},
    below right offset={0.15,-0.15},
    #1
  }
}

\title[MATH 1080 - Section 10.4]{Matrix Algebra}

\begin{document}

\begin{frame}
  \titlepage
 
\end{frame}

\begin{frame}
\begin{block}{Matrix}
A \textbf{matrix} is a rectangular array of \textbf{elements} or \textbf{entries} (numbers or functions) arranged in \textbf{rows} (horizontal) and \textbf{columns} (vertical).
\begin{equation*}
\mat{A}=\begin{bmatrix}
a_{11}&a_{12}&\cdots&a_{1n}\\
a_{21}&a_{22}&\cdots&a_{2n}\\
\vdots&\vdots&\ddots&\vdots\\
a_{m1}&a_{m1}&\cdots&a_{mn}
\end{bmatrix}
\end{equation*}
The \textbf{order} of $\mat{A}$ is $m\times n$. If $m=n$, we call the matrix \textbf{square}.
\end{block}\pause
\begin{block}{Equal Matrices}
Two matrices of the same order are \textbf{equal} if their corresponding entries are equal. If matrices $\mat{A}=[a_{ij}]$ and $\mat{B}=[a_{ij}]$ are both $m\times n$, then
\[\mat{A}=\mat{B} \Leftrightarrow a_{ij}=b_{ij},\quad 1\leq i\leq m,\> 1\leq j\leq n\]
\end{block}
\end{frame}

\begin{frame}
\begin{block}{Special Matrices}
\begin{itemize}
\item<+-> The $m\times n$ \textbf{zero matrix}, denoted $\textbf{0}_{mn}$, has all its entries equal to zero.
\item<+->A \textbf{diagonal matrix} is:
\begin{equation*}
\mat{D}=\begin{bmatrix}
a_{11}&0&\cdots&0\\
0&a_{22}&\cdots&0\\
\vdots&\vdots&\ddots&\vdots\\
0&0&\cdots&a_{mn}
\end{bmatrix}
\end{equation*}
\item<+->The $n\times n$ \textbf{identity matrix}, denoted $\identmat{n}$ is:
 \begin{equation*}
 \identmat{n}=\begin{bmatrix}
1&0&\cdots&0\\
0&1&\cdots&0\\
\vdots&\vdots&\ddots&\vdots\\
0&0&\cdots&1
\end{bmatrix}
\end{equation*}
\end{itemize}
\end{block}
\end{frame}

\begin{frame}
\begin{block}{Matrix Addition}
Two matrices of the same order are added (or subtracted) by adding (or subtracting) corresponding entries and recording the results in a matrix of the same size. Using matrix notation, if $A=[a_{ij}]$ and $B=[b_{ij}]$ are both $m\times n$.
\begin{equation*}
\begin{split}
\mat{A}+\mat{B}&=[a_{ij}]+[b_{ij}]=[a_{ij}+b_{ij}]\\
\mat{A}-\mat{B}&=[a_{ij}]-[b_{ij}]=[a_{ij}-b_{ij}]
\end{split}
\end{equation*}
\end{block}\pause
\begin{block}{Multiplication by a Scalar}
To find the product of a matrix and a scalar (a complex number), multiply each entry of the matrix by that number. This is called \textbf{multiplication by a scalar}. Using matrix notation, if $\mat{A}=[a_{ij}]$, then
\[c\cdot \mat{A}=[c\cdot a_{ij}] = [a_{ij}\cdot c] = \mat{A}\cdot c\]
\end{block}
\end{frame}

\begin{frame}
\begin{example}
Suppose that
\begin{equation*}
\mat{A}=
\begin{bmatrix}[rrr]
3 & 1 & 5 \\
-2 & 0 & 6 \\
\end{bmatrix}\qquad
\mat{B}=
\begin{bmatrix}[rrr]
4 & 1 & 0 \\
8 & 1 & -3 \\
\end{bmatrix}\qquad
\mat{C}=
\begin{bmatrix}[rr]
9 & 0 \\
-3 & 6 \\
\end{bmatrix}
\end{equation*}\pause

\vspace{-5mm}
What is $\mat{A}+\mat{B}$?\pause

\vspace{-4mm}
\begin{equation*}
\begin{bmatrix}
3+4 & 1+1 & 5+0 \\
-2+8 & 0+1 & 6+-3 \\
\end{bmatrix}\pause
=
\begin{bmatrix}
7 & 2 & 5 \\
6 & 1 & 3 \\
\end{bmatrix}
\end{equation*}\pause
What is $\mat{A}+\mat{C}$?\pause

\vspace{-3mm}
\begin{center}
$\mat{A}$ and $\mat{C}$ have different dimensions, so the sum cannot be performed.\pause
\end{center}

\vspace{-2mm}
What is $3\mat{C}$?\pause
\begin{equation*}
\begin{bmatrix}[rr]
3\cdot 9 & 3\cdot 0 \\
3\cdot (-3) & 3\cdot 6 \\
\end{bmatrix}\pause
=
\begin{bmatrix}[rr]
27 & 0 \\
-9 & 18 \\
\end{bmatrix}
\end{equation*}\pause

\vspace{-3mm}
\begin{overprint}
\onslide<-10>
\onslide<11>
What is $3\mat{A}-2\mat{B}$?
\onslide<12>
What is $3\mat{A}-2\mat{B}$?
\begin{equation*}
3
\begin{bmatrix}[rrr]
3 & 1 & 5 \\
-2 & 0 & 6 \\
\end{bmatrix}
-2
\begin{bmatrix}[rrr]
4 & 1 & 0 \\
8 & 1 & -3 \\
\end{bmatrix}
\end{equation*}
\onslide<13>
What is $3\mat{A}-2\mat{B}$?
\begin{equation*}
\begin{bmatrix}[ccc]
3\cdot 3 & 3\cdot 1 & 3\cdot 5 \\
3\cdot (-2) & 3\cdot 0 & 3\cdot 6 \\
\end{bmatrix}
-
\begin{bmatrix}[ccc]
2\cdot 4 & 2\cdot 1 & 2\cdot 0 \\
2\cdot 8 & 2\cdot 1 & 2\cdot (-3) \\
\end{bmatrix}
\end{equation*}
\onslide<14>
What is $3\mat{A}-2\mat{B}$?
\begin{equation*}
\begin{bmatrix}[rrr]
9 & 3 & 15 \\
-6 & 0 & 18 \\
\end{bmatrix}
-
\begin{bmatrix}[rrr]
8 & 2 & 0 \\
16 & 2 & -6 \\
\end{bmatrix}
\end{equation*}
\onslide<15>
What is $3\mat{A}-2\mat{B}$?
\begin{equation*}
\begin{bmatrix}[ccc]
9-8 & 3-2 & 15-0 \\
(-6)-16 & 0-2 & 18-
(-6) \\
\end{bmatrix}
\end{equation*}
\onslide<16>
What is $3\mat{A}-2\mat{B}$?

\begin{equation*}
\begin{bmatrix}[rrr]
1 & 1 & 15 \\
-22 & -2 & 24 \\
\end{bmatrix}
\end{equation*}
\end{overprint}
\end{example}
\end{frame}

\begin{frame}
\begin{block}{Properties of Matrix Addition and Scalar Multiplication}
Suppose $\mat{A}$, $\mat{B}$, and $\mat{C}$ are $m\times n$ matrices and $c$ and $k$ are scalars. Then the following properties hold:
\begin{itemize}
\item<+->$\mat{A}+\mat{B}=\mat{B}+\mat{A}$\hfill(Commutativity)
\item<+->$\mat{A}+(\mat{B}+\mat{C})=(\mat{A}+\mat{B})+\mat{C}$\hfill(Associativity)
\item<+->$c(k\mat{A})=(ck)\mat{A}$\hfill(Associativity)
\item<+->$\mat{A}+\textbf{0}=\mat{A}$\hfill(Zero Element)
\item<+->$\mat{A}+(-\mat{A})=\textbf{0}$\hfill(Inverse Element)
\item<+->$c(\mat{A}+\mat{B})=c\mat{A}+c\mat{B}$\hfill(Distributivity)
\item<+->$(c+k)\mat{A}=c\mat{A}+k\mat{A}$\hfill(Distributivity)
\end{itemize}
\end{block}
\end{frame}

\begin{frame}
\begin{block}{Vectors}
A vector $\vect{v}=< v_1,\dots,v_n >$ can be represented by either by a $1\times n$ row matrix, or a $n\times 1$ column matrix.
\end{block}\pause
\begin{block}{Vector addition and Scalar Multiplication}
Let
\[
\vect{x}=\begin{bmatrix}x_1\\\vdots\\x_n\end{bmatrix}
\quad\text{and}\quad
\vect{y}=\begin{bmatrix}y_1\\\vdots\\y_n\end{bmatrix}
\]
be vectors in  $\mathbb{R}^n$ and $c$ be any scalar. Then, we have:
\[
\begin{bmatrix}x_1\\\vdots\\x_n\end{bmatrix}+
\begin{bmatrix}y_1\\\vdots\\y_n\end{bmatrix}=
\begin{bmatrix}x_1+y_1\\\vdots\\x_n+y_n\end{bmatrix}
\quad\text{and}\quad
c\cdot\begin{bmatrix}x_1\\\vdots\\x_n\end{bmatrix}=
\begin{bmatrix} c\cdot x_1\\\vdots\\ c\cdot x_n\end{bmatrix}
\]
\end{block}
\end{frame}

\begin{frame}
\begin{block}{Properties of Vector Addition and Multiplication}
For vectors $\vect{u}$, $\vect{v}$, and $\vect{w}$ in $\mathbb{R}^n$ and scalars $c$ and $k$.\begin{itemize}
\item $\vect{u}+\vect{v}=\vect{v}+\vect{u}$\hfill(Commutativity)
\item $\vect{u}+(\vect{v}+\vect{w})=(\vect{u}+\vect{v})+\vect{w}$\hfill(Associativity)
\item $c(k\vect{v})=(ck)\vect{v}$\hfill(Associativity)
\item $\vect{u}+\vect{0}=\vect{u}$\hfill(Zero Element)
\item $\vect{u}+(-\vect{u})=\vect{0}$\hfill(Inverse Element)
\item $c(\vect{u}+\vect{v})=c\vect{u}+c\vect{v}$\hfill(Distributivity)
\item $(c+k)\vect{u}=c\vect{u}+k\vect{u}$\hfill(Distributivity)
\end{itemize}
\end{block}
\end{frame}

\begin{frame}
\begin{block}{Dot Product}
The \textbf{dot product} of a row vector $\vect{x}$ and a column vector $\vect{y}$ of equal length $n$ is the result of adding the products of the corresponding entries as follows:

\vspace{-10mm}
\begin{equation*}
\begin{split}
\vect{x}\pdot\vect{y}&=\begin{bmatrix}x_1&\cdots&x_n\end{bmatrix}\pdot
\begin{bmatrix}y_1\\\vdots\\y_n\end{bmatrix}\\
&=x_1\cdot y_1+x_2\cdot y_2+\cdots +x_n\cdot y_n\\
\end{split}
\end{equation*}
\end{block}\pause
\begin{example}
Consider

\vspace{-5mm}
\begin{equation*}
\vect{r}=
\begin{bmatrix}
3 & -5 & 2
\end{bmatrix}
\quad\text{and}\quad
\vect{c}=\bvector{\+3,\+4,-5}
\end{equation*}

\vspace{-6mm}
What is $\vect{r}\pdot\vect{c}$?\pause
\begin{equation*}
\begin{bmatrix}
3 & -5 & 2
\end{bmatrix}
\pdot\bvector{\+3,\+4,-5}\pause
=3\cdot 3+(-5)\cdot 4+2\cdot(-5)\pause
=9-20-10\pause
=-21
\end{equation*}

\vspace{-4mm}
\end{example}
\end{frame}

\begin{frame}
\begin{block}{Matrix Product}
The \textbf{matrix product} of a $m\times r$ matrix $\mat{A}$ and a $r\times n$ matrix $\mat{B}$ is denoted
\begin{equation*}
\mat{C}=\mat{A}\cdot \mat{B}=\mat{A}\mat{B}
\end{equation*}
where the $ij$th entry of $\mat{C}$ is the dot product of the $i$th row vector of $A$ and the $j$th column vector of $\mat{B}$:
\begin{equation*}
\begin{split}
c_{ij}&=\begin{bmatrix}a_{i1}&a_{2j}&\cdots&a_{ir}\end{bmatrix}\pdot
\begin{bmatrix}b_{1j}\\\vdots\\b_{rj}\end{bmatrix}\\
\end{split}
\end{equation*}
The matrix $\mat{C}$ has order $m\times n$.
\end{block}
\end{frame}

\begin{frame}
\large
\begin{example}
Perform $\mat{A}\mat{B}$ where
\begin{equation*}
\mat{A}=
	\begin{bmatrix}[rrr]
		1 & -1 & 3 \\
		0 & 4 & 2
	\end{bmatrix}
\quad\text{and}\quad
\mat{B}=
	\begin{bmatrix}[rr]
		3 & 1 \\
		2 & -4 \\
		-1 & 0 
	\end{bmatrix}
\end{equation*}
\visible<2->{
\begin{center}
\begin{tabular}{r|l}
&
\begin{minipage}{2.4cm}
%	\begin{equation*}
		$\left[
			\begin{array}{rr}
				\tikzmarkin<3,5>[ver=style cyan]{col-a-ex1}3  &\tikzmarkin<4,6>[ver=style cyan]{col-b-ex1} 1 \\
				2 &\ \ \ \ -4 \\
				-1\tikzmarkend{col-a-ex1} & 0\tikzmarkend{col-b-ex1} \\
			\end{array}
		\right]$
%	\end{equation*}
\end{minipage}\\\\
\hline\\
\begin{minipage}{2.3cm}
%	\begin{equation*}
		$\left[
			\begin{array}{rrr}
				\tikzmarkin<3-4>[hor=style orange]{row-a-ex1}1  & -1 & 3\tikzmarkend{row-a-ex1} \\
				\tikzmarkin<5-6>[hor=style orange]{row-b-ex1}0 & 4 & 2 \tikzmarkend{row-b-ex1}\\
			\end{array}
		\right]$
%	\end{equation*}
\end{minipage}
&
\begin{minipage}{2.4cm}
%	\begin{equation*}
		$\left[
			\begin{array}{rr}
				\tikzmarkin<3>[hor=style green]{end-a-ex1}\onslide<3->{-2} \tikzmarkend{end-a-ex1}
				& \tikzmarkin<4>[hor=style green]{end-b-ex1}\onslide<4->{5}\tikzmarkend{end-b-ex1}   \\
				 \tikzmarkin<5>[hor=style green]{end-c-ex1}\onslide<5->{6}\tikzmarkend{end-c-ex1} 
				 & \tikzmarkin<6>[hor=style green]{end-d-ex1}\onslide<6>{-16}\tikzmarkend{end-d-ex1}  \\
			\end{array}
		\right]$
%	\end{equation*}
\end{minipage}
\end{tabular}
\end{center}}
\vspace{0.05cm}
\end{example}
\end{frame}

\begin{frame}
\large
\begin{example}
Perform $\mat{A}\mat{B}$ where
\begin{equation*}
\mat{A}=
	\begin{bmatrix}[rrr]
		2 & 4 & -1 \\
		5 & 8 & 0
	\end{bmatrix}
\quad\text{and}\quad
\mat{B}=
	\begin{bmatrix}[rrrr]
		2 & 5 & 1 & 4 \\
		4 & 8 & 0 & 6 \\
		-3 & 1 & -2 & -1 \\
	\end{bmatrix}
\end{equation*}
\visible<2->{
\begin{center}
\begin{tabular}{r|l}
&
\begin{minipage}{2.4cm}
%	\begin{equation*}
		$\left[
			\begin{array}{rrrr}
				\tikzmarkin<3,7>[ver=style cyan]{col-a-ex2}2  &\tikzmarkin<4,8>[ver=style cyan]{col-b-ex2}5 & \tikzmarkin<5,9>[ver=style cyan]{col-c-ex2}1 & \tikzmarkin<6,10>[ver=style cyan]{col-d-ex2}4 \\
		4 & 8 & 0 & 6 \\
				-3\tikzmarkend{col-a-ex2} & \+1\tikzmarkend{col-b-ex2} & -2\tikzmarkend{col-c-ex2} & -1\tikzmarkend{col-d-ex2} \\
			\end{array}
		\right]$
%	\end{equation*}
\end{minipage}\\\\
\hline\\
\begin{minipage}{2.3cm}
%	\begin{equation*}
		$\left[
			\begin{array}{rrr}
				\tikzmarkin<3-6>[hor=style orange]{row-a-ex2}2 & 4 & -1\tikzmarkend{row-a-ex2} \\
				\tikzmarkin<7-10>[hor=style orange]{row-b-ex2}5 & 8 & 0 \tikzmarkend{row-b-ex2}\\
			\end{array}
		\right]$
%	\end{equation*}
\end{minipage}
&
\begin{minipage}{2.4cm}
%	\begin{equation*}
	$\left[
		\begin{array}{rrrr}
			\hspace{1.9mm}\tikzmarkin<3>[hor=style green]{end-a-ex2}\visible<3->{23}   \tikzmarkend{end-a-ex2}
			& \hspace{0.9mm}\tikzmarkin<4>[hor=style green]{end-b-ex2}\visible<4->{41}  \tikzmarkend{end-b-ex2} 
			& \hspace{3.4mm}\tikzmarkin<5>[hor=style green]{end-c-ex2}\visible<5->{4} \tikzmarkend{end-c-ex2}
			& \hspace{1.5mm}\tikzmarkin<6>[hor=style green]{end-d-ex2}\visible<6->{33}  \tikzmarkend{end-d-ex2}   \\
			  \tikzmarkin<7>[hor=style green]{end-e-ex2}\visible<7->{42}  \tikzmarkend{end-e-ex2} 
			& \tikzmarkin<8>[hor=style green]{end-f-ex2}\visible<8->{89} \tikzmarkend{end-f-ex2}
			& \tikzmarkin<9>[hor=style green]{end-g-ex2}\visible<9->{5}  \tikzmarkend{end-g-ex2} 
			& \tikzmarkin<10>[hor=style green]{end-h-ex2}\visible<10->{68} \tikzmarkend{end-h-ex2}  \\
		\end{array}
	\right]$
%	\end{equation*}
\end{minipage}
\end{tabular}
\end{center}}
\vspace{0.05cm}
\end{example}
\end{frame}

\begin{frame}
\begin{block}{Properties of Matrix Multiplication}
\begin{itemize}
\item $(\mat{A}\mat{B})\mat{C}=\mat{A}(\mat{B}\mat{C})$\hfill (Associativity)
\item $\mat{A}(\mat{B}+\mat{C})=\mat{A}\mat{B}+\mat{A}\mat{C}$\hfill (Distributivity)
\item $(\mat{B}+\mat{C})\mat{A}=\mat{B}\mat{A}+\mat{C}\mat{A}$\hfill (Distributivity)\pause
\item $\mat{A}\mat{B}\neq \mat{B}\mat{A}$\hfill (Generally Noncommutative)
\end{itemize}
\end{block}\pause
\begin{block}{Properties of Identity Matrices}
For a $m\times n$ matrix $\mat{A}$:
\begin{itemize}
\item $\mat{A}\cdot I_n = \mat{A}$\quad\text{and}\quad$I_m\cdot \mat{A} = \mat{A}$
\item $\mat{A}\cdot\zeromat{n} = \zeromat{mn}$\quad\text{and}\quad$\zeromat{m}\cdot \mat{A}=\zeromat{mn}$
\end{itemize}
\end{block}
\end{frame}
\end{document}
