\documentclass{beamer}
\usepackage[utf8]{inputenc}
\usepackage[english]{babel}
\usepackage[T1]{fontenc}
\usepackage{amsmath}
\usepackage[inline]{asymptote}
\newcommand{\dx}{\hspace{2pt}dx}
\newcommand{\dd}[1]{\hspace{2pt}d#1}
\usepackage{multicol}
\usepackage{slide_helper}

\newcommand*{\thead}[1]{\multicolumn{1}{c}{\bfseries #1}}

\title[MATH 2250 - Section 1.4]{Approximation Methods\\Numerical Analysis}

\begin{document}
\begin{frame}
\titlepage
\end{frame}

\begin{frame}[fragile]
\begin{example}
Consider the IVP
\[y^\prime = t - y,\ y(0)=1\]
\begin{multistepslide}
\begin{center}
\begin{asy}[height=6.3cm]
import graph;
import slopefield;
import fontsize;
defaultpen(fontsize(9pt));
real dy(real x,real y) {return x-y;}
real xmin=-.1, xmax=2.4;
real ymin=-.1, ymax=1.5;

add(slopefield(dy,(xmin,ymin),(xmax,ymax),20,grey+0.4bp,Arrow));

real h=0.5;
real EulerStep(real x,real y) {return y+h*dy(x,y);}
real t0=0.0;
real y0=1.0;
dot((t0,y0));

real t_old=t0;
real y_old=y0;
real t_new=0.0;
real y_new=0.0;

//for(int i=1; i<2; ++i)
//{
//	t_new=t_old+h;
//	y_new=EulerStep(t_old, y_old);
//	draw((t_old, y_old)--(t_new, y_new), blue+1bp);
//    dot((t_new,y_new));
//    t_old=t_new;
//    y_old=y_new;
//}

//pair P=(t0,y0);
//draw(curve(P,dy,(xmin,ymin),(xmax,ymax)),red+1bp);

limits((xmin,ymin),(xmax,ymax),Crop);

xaxis(YEquals(ymin),xmin,xmax,LeftTicks());
xaxis(YEquals(ymax),xmin,xmax);
yaxis(XEquals(xmin),ymin,ymax,RightTicks());
yaxis(XEquals(xmax),ymin,ymax);
\end{asy}
\end{center}
\nextstep
\begin{center}
\begin{asy}[height=6.3cm]
import graph;
import slopefield;
import fontsize;
defaultpen(fontsize(9pt));
real dy(real x,real y) {return x-y;}
real xmin=-.1, xmax=2.4;
real ymin=-.1, ymax=1.5;

add(slopefield(dy,(xmin,ymin),(xmax,ymax),20,grey+0.4bp,Arrow));

real h=0.5;
real EulerStep(real x,real y) {return y+h*dy(x,y);}
real t0=0.0;
real y0=1.0;
dot((t0,y0));

real t_old=t0;
real y_old=y0;
real t_new=0.0;
real y_new=0.0;

for(int i=1; i<2; ++i)
{
	t_new=t_old+h;
	y_new=EulerStep(t_old, y_old);
	draw((t_old, y_old)--(t_new, y_new), blue+1bp);
    dot((t_new,y_new));
    t_old=t_new;
    y_old=y_new;
}

//pair P=(t0,y0);
//draw(curve(P,dy,(xmin,ymin),(xmax,ymax)),red+1bp);

limits((xmin,ymin),(xmax,ymax),Crop);

xaxis(YEquals(ymin),xmin,xmax,LeftTicks());
xaxis(YEquals(ymax),xmin,xmax);
yaxis(XEquals(xmin),ymin,ymax,RightTicks());
yaxis(XEquals(xmax),ymin,ymax);
\end{asy}
\end{center}
\nextstep
\begin{center}
\begin{asy}[height=6.3cm]
import graph;
import slopefield;
import fontsize;
defaultpen(fontsize(9pt));
real dy(real x,real y) {return x-y;}
real xmin=-.1, xmax=2.4;
real ymin=-.1, ymax=1.5;

add(slopefield(dy,(xmin,ymin),(xmax,ymax),20,grey+0.4bp,Arrow));

real h=0.5;
real EulerStep(real x,real y) {return y+h*dy(x,y);}
real t0=0.0;
real y0=1.0;
dot((t0,y0));

real t_old=t0;
real y_old=y0;
real t_new=0.0;
real y_new=0.0;

for(int i=1; i<3; ++i)
{
	t_new=t_old+h;
	y_new=EulerStep(t_old, y_old);
	draw((t_old, y_old)--(t_new, y_new), blue+1bp);
    dot((t_new,y_new));
    t_old=t_new;
    y_old=y_new;
}

//pair P=(t0,y0);
//draw(curve(P,dy,(xmin,ymin),(xmax,ymax)),red+1bp);

limits((xmin,ymin),(xmax,ymax),Crop);

xaxis(YEquals(ymin),xmin,xmax,LeftTicks());
xaxis(YEquals(ymax),xmin,xmax);
yaxis(XEquals(xmin),ymin,ymax,RightTicks());
yaxis(XEquals(xmax),ymin,ymax);
\end{asy}
\end{center}
\nextstep
\begin{center}
\begin{asy}[height=6.3cm]
import graph;
import slopefield;
import fontsize;
defaultpen(fontsize(9pt));
real dy(real x,real y) {return x-y;}
real xmin=-.1, xmax=2.4;
real ymin=-.1, ymax=1.5;

add(slopefield(dy,(xmin,ymin),(xmax,ymax),20,grey+0.4bp,Arrow));

real h=0.5;
real EulerStep(real x,real y) {return y+h*dy(x,y);}
real t0=0.0;
real y0=1.0;
dot((t0,y0));

real t_old=t0;
real y_old=y0;
real t_new=0.0;
real y_new=0.0;

for(int i=1; i<4; ++i)
{
	t_new=t_old+h;
	y_new=EulerStep(t_old, y_old);
	draw((t_old, y_old)--(t_new, y_new), blue+1bp);
    dot((t_new,y_new));
    t_old=t_new;
    y_old=y_new;
}

//pair P=(t0,y0);
//draw(curve(P,dy,(xmin,ymin),(xmax,ymax)),red+1bp);

limits((xmin,ymin),(xmax,ymax),Crop);

xaxis(YEquals(ymin),xmin,xmax,LeftTicks());
xaxis(YEquals(ymax),xmin,xmax);
yaxis(XEquals(xmin),ymin,ymax,RightTicks());
yaxis(XEquals(xmax),ymin,ymax);
\end{asy}
\end{center}
\nextstep
\begin{center}
\begin{asy}[height=6.3cm]
import graph;
import slopefield;
import fontsize;
defaultpen(fontsize(9pt));
real dy(real x,real y) {return x-y;}
real xmin=-.1, xmax=2.4;
real ymin=-.1, ymax=1.5;

add(slopefield(dy,(xmin,ymin),(xmax,ymax),20,grey+0.4bp,Arrow));

real h=0.5;
real EulerStep(real x,real y) {return y+h*dy(x,y);}
real t0=0.0;
real y0=1.0;
dot((t0,y0));

real t_old=t0;
real y_old=y0;
real t_new=0.0;
real y_new=0.0;

for(int i=1; i<5; ++i)
{
	t_new=t_old+h;
	y_new=EulerStep(t_old, y_old);
	draw((t_old, y_old)--(t_new, y_new), blue+1bp);
    dot((t_new,y_new));
    t_old=t_new;
    y_old=y_new;
}

//pair P=(t0,y0);
//draw(curve(P,dy,(xmin,ymin),(xmax,ymax)),red+1bp);

limits((xmin,ymin),(xmax,ymax),Crop);

xaxis(YEquals(ymin),xmin,xmax,LeftTicks());
xaxis(YEquals(ymax),xmin,xmax);
yaxis(XEquals(xmin),ymin,ymax,RightTicks());
yaxis(XEquals(xmax),ymin,ymax);
\end{asy}
\end{center}
\nextstep
\begin{center}
\begin{asy}[height=6.3cm]
import graph;
import slopefield;
import fontsize;
defaultpen(fontsize(9pt));
real dy(real x,real y) {return x-y;}
real xmin=-.1, xmax=2.4;
real ymin=-.1, ymax=1.5;

add(slopefield(dy,(xmin,ymin),(xmax,ymax),20,grey+0.4bp,Arrow));

real h=0.5;
real EulerStep(real x,real y) {return y+h*dy(x,y);}
real t0=0.0;
real y0=1.0;
dot((t0,y0));

real t_old=t0;
real y_old=y0;
real t_new=0.0;
real y_new=0.0;

for(int i=1; i<5; ++i)
{
	t_new=t_old+h;
	y_new=EulerStep(t_old, y_old);
	draw((t_old, y_old)--(t_new, y_new), blue+1bp);
    dot((t_new,y_new));
    t_old=t_new;
    y_old=y_new;
}

pair P=(t0,y0);
draw(curve(P,dy,(xmin,ymin),(xmax,ymax)),red+1bp);

limits((xmin,ymin),(xmax,ymax),Crop);

xaxis(YEquals(ymin),xmin,xmax,LeftTicks());
xaxis(YEquals(ymax),xmin,xmax);
yaxis(XEquals(xmin),ymin,ymax,RightTicks());
yaxis(XEquals(xmax),ymin,ymax);
\end{asy}
\end{center}
\nextstep
\begin{center}
\begin{asy}[height=6.3cm]
import graph;
import slopefield;
import fontsize;
defaultpen(fontsize(9pt));
real dy(real x,real y) {return x-y;}
real xmin=-.1, xmax=2.4;
real ymin=-.1, ymax=1.5;

add(slopefield(dy,(xmin,ymin),(xmax,ymax),20,grey+0.4bp,Arrow));

real h=0.25;
real EulerStep(real x,real y) {return y+h*dy(x,y);}
real t0=0.0;
real y0=1.0;
dot((t0,y0));

real t_old=t0;
real y_old=y0;
real t_new=0.0;
real y_new=0.0;

for(int i=1; i<9; ++i)
{
	t_new=t_old+h;
	y_new=EulerStep(t_old, y_old);
	draw((t_old, y_old)--(t_new, y_new), blue+1bp);
    dot((t_new,y_new));
    t_old=t_new;
    y_old=y_new;
}

pair P=(t0,y0);
draw(curve(P,dy,(xmin,ymin),(xmax,ymax)),red+1bp);

limits((xmin,ymin),(xmax,ymax),Crop);

xaxis(YEquals(ymin),xmin,xmax,LeftTicks());
xaxis(YEquals(ymax),xmin,xmax);
yaxis(XEquals(xmin),ymin,ymax,RightTicks());
yaxis(XEquals(xmax),ymin,ymax);
\end{asy}
\end{center}
\nextstep
\begin{center}
\begin{asy}[height=6.3cm]
import graph;
import slopefield;
import fontsize;
defaultpen(fontsize(9pt));
real dy(real x,real y) {return x-y;}
real xmin=-.1, xmax=2.4;
real ymin=-.1, ymax=1.5;

add(slopefield(dy,(xmin,ymin),(xmax,ymax),20,grey+0.4bp,Arrow));

real h=0.125;
real EulerStep(real x,real y) {return y+h*dy(x,y);}
real t0=0.0;
real y0=1.0;
dot((t0,y0));

real t_old=t0;
real y_old=y0;
real t_new=0.0;
real y_new=0.0;

for(int i=1; i<17; ++i)
{
	t_new=t_old+h;
	y_new=EulerStep(t_old, y_old);
	draw((t_old, y_old)--(t_new, y_new), blue+1bp);
    dot((t_new,y_new));
    t_old=t_new;
    y_old=y_new;
}

pair P=(t0,y0);
draw(curve(P,dy,(xmin,ymin),(xmax,ymax)),red+1bp);

limits((xmin,ymin),(xmax,ymax),Crop);

xaxis(YEquals(ymin),xmin,xmax,LeftTicks());
xaxis(YEquals(ymax),xmin,xmax);
yaxis(XEquals(xmin),ymin,ymax,RightTicks());
yaxis(XEquals(xmax),ymin,ymax);
\end{asy}
\end{center}
\nextstep
\begin{center}
\begin{asy}[height=6.3cm]
import graph;
import slopefield;
import fontsize;
defaultpen(fontsize(9pt));
real dy(real x,real y) {return x-y;}
real xmin=-.1, xmax=2.4;
real ymin=-.1, ymax=1.5;

add(slopefield(dy,(xmin,ymin),(xmax,ymax),20,grey+0.4bp,Arrow));

real h=0.0625;
real EulerStep(real x,real y) {return y+h*dy(x,y);}
real t0=0.0;
real y0=1.0;
dot((t0,y0));

real t_old=t0;
real y_old=y0;
real t_new=0.0;
real y_new=0.0;

for(int i=1; i<33; ++i)
{
	t_new=t_old+h;
	y_new=EulerStep(t_old, y_old);
	draw((t_old, y_old)--(t_new, y_new), blue+1bp);
    dot((t_new,y_new));
    t_old=t_new;
    y_old=y_new;
}

pair P=(t0,y0);
draw(curve(P,dy,(xmin,ymin),(xmax,ymax)),red+1bp);

limits((xmin,ymin),(xmax,ymax),Crop);

xaxis(YEquals(ymin),xmin,xmax,LeftTicks());
xaxis(YEquals(ymax),xmin,xmax);
yaxis(XEquals(xmin),ymin,ymax,RightTicks());
yaxis(XEquals(xmax),ymin,ymax);
\end{asy}
\end{center}
\nextstep
\begin{center}
\begin{asy}[height=6.3cm]
import graph;
import slopefield;
import fontsize;
defaultpen(fontsize(9pt));
real dy(real x,real y) {return x-y;}
real xmin=-.1, xmax=2.4;
real ymin=-.1, ymax=1.5;

add(slopefield(dy,(xmin,ymin),(xmax,ymax),20,grey+0.4bp,Arrow));

real h=0.03125;
real EulerStep(real x,real y) {return y+h*dy(x,y);}
real t0=0.0;
real y0=1.0;
dot((t0,y0));

real t_old=t0;
real y_old=y0;
real t_new=0.0;
real y_new=0.0;

for(int i=1; i<65; ++i)
{
	t_new=t_old+h;
	y_new=EulerStep(t_old, y_old);
	draw((t_old, y_old)--(t_new, y_new), blue+1bp);
    dot((t_new,y_new));
    t_old=t_new;
    y_old=y_new;
}

pair P=(t0,y0);
draw(curve(P,dy,(xmin,ymin),(xmax,ymax)),red+1bp);

limits((xmin,ymin),(xmax,ymax),Crop);

xaxis(YEquals(ymin),xmin,xmax,LeftTicks());
xaxis(YEquals(ymax),xmin,xmax);
yaxis(XEquals(xmin),ymin,ymax,RightTicks());
yaxis(XEquals(xmax),ymin,ymax);
\end{asy}
\end{center}
\end{multistepslide}
\end{example}
\end{frame}

\begin{frame}
\begin{block}{Consider}
Given the IVP
\begin{equation*}
y^\prime = f(t,y),\ y(t_0)=y_0
\end{equation*}
We want to compute  approximate values for $y(t_n)$ at the (finite) set of points $t_1, t_2, t_3, \dots, t_k$.

\vspace{2mm}
\begin{overprint}
\onslide<1>
\onslide<2-4>
We can calculate the $t$-values, for $1,2,3,\dots,k$, with $t_n=t_0 + n\cdot h$.

\vspace{2mm}
Where $h$, called the \textbf{step size}, is the common difference between successive points.

\visible<3->{\vspace{2mm}
So, starting at $(t_0, y_0)$ we want to follow the tangent line determined by
\begin{equation*}
y-y_0 = (t - t_0)f(t_0, y_0)
\end{equation*}}
\visible<4->{%

\vspace{-5mm}
to find the approximate solution $\left(t_1, y(t_1)\right)$:
\begin{equation*}
y_1 = y_0 + h\cdot f(t_0, y_0)
\end{equation*}}

\onslide<5->
We can extend this process to find all $k$ points.
\begin{equation*}
\begin{aligned}
y_1 &= y_0 + h\cdot f(t_0, y_0)\\
\visible<6->{y_2 &= y_1 + h\cdot f(t_1, y_1)\\}
\visible<7->{y_3 &= y_2 + h\cdot f(t_2, y_2)\\}
\visible<8->{&\vdots\\
y_k &= y_{k-1} + h\cdot f(t_{k-1}, y_{k-1})\\}
\end{aligned}
\end{equation*}

\vspace{-5mm}
\visible<9->{The resulting piecewise-linear function (i\@.e\@. play connect-the-dots) is called the \textbf{Euler-approximate} solution.}
\end{overprint}
\vspace{-3mm}
\end{block}
\end{frame}

\begin{frame}
\begin{block}{Euler's Method}
For the Initial-value problem
\[y^\prime=f(t,y),\ y(t_0)=y_0\]
use the formulas
\begin{align*}
t_{n+1} &= t_n + h\\
y_{n+1} &= y_n+h\cdot f(t_n, y_n)
\end{align*}
to iteratively compute the points, using step size $h$, 
\[(t_1, y_1), (t_2, y_2), \dots, (t_k, y_k).\]
The piecewise-linear function connecting these points is the Euler approximtion to the solution $y(t)$ of the IVP for $t_0\leq t\leq t_k$.
\end{block}
\end{frame}

\begin{frame}[fragile]
\begin{example}
\begin{overprint}
\onslide<1-2>
Let us obtain the Euler-approximate solution of the IVP
\[y^\prime = -2ty+t,\ y(0)=-1\]
with step size $0.1$ on $[0,0.4]$.

\vspace{2mm}
\visible<2>{%
In other words:
\begin{align*}
f(t,y) &= -2ty+t = t(1-2y)\\
t_0 &= 0\\
y_0 &= -1\\
h&=0.1\\
k&=1,2,3,4
\end{align*}}
\onslide<3-6>
\begin{equation*}
\begin{aligned}
t_1&=t_0+h=0+0.1=0.1\\
y_1&=y_0+h\cdot f(t_0, y_0)= -1+(0.1)(0)(1-2(-1))=-1\\\\
\visible<4->{%
t_2&=t_1+h=0.1+0.1=0.2\\
y_2&=y_1+h\cdot f(t_1, y_1) \\ 
&= -1+(0.1)(0.1)(1-2(-1))=-0.97\\\\}
\visible<5->{%
t_3&=t_2+h=0.2+0.1=0.3\\
y_3&=y_0+h\cdot f(t_2, y_2) \\
&= -0.97+(0.1)(0.2)(1-2(-0.97))=-0.9112\\\\}
\visible<6->{%
t_4&=t_3+h=0.3+0.1=0.4\\
y_4&=y_3+h\cdot f(t_3, y_3) \\
&= -0.9112+(0.1)(0.3)(1-2(-0.9112))=-0.82652\\}
\end{aligned}
\end{equation*}

\onslide<7-8>
How does this compare to the exact solution $y(t)=0.5 - 1.5e^{-t^2}$?

\visible<8->{%
\begin{center}
\begin{tabular}{rrrrr}
\thead{$n$}&\thead{$t_n$}&\thead{$y_n$}&\thead{$y(t_n)$}&\thead{Error}\\
\hline
0&0.0&-1.000000&-1.000000&\textbf{0.000000}\\
1&0.1&-1.000000&-0.985075&\textbf{-0.014925}\\
2&0.2&-0.970000&-0.941184&\textbf{-0.028815}\\
3&0.3&-0.911200&-0.870897&\textbf{-0.040303}\\
4&0.4&-0.826528&-0.778216&\textbf{-0.048312}
\end{tabular}
\end{center}

Notice how the error grows rapidly.}

\onslide<9>
\begin{center}
\begin{asy}[height=6.3cm]
import graph;
import slopefield;
import fontsize;
defaultpen(fontsize(9pt));
size(300);
real dy(real x,real y) {return -2.0*x*y+x;}
real xmin=-.1, xmax=0.5;
real ymin=-1.1, ymax=-0.6;

add(slopefield(dy,(xmin,ymin),(xmax,ymax),20,grey+0.4bp,Arrow));

xaxis(YEquals(ymin),xmin,xmax,LeftTicks());
xaxis(YEquals(ymax),xmin,xmax);
yaxis(XEquals(xmin),ymin,ymax,RightTicks());
yaxis(XEquals(xmax),ymin,ymax);

real h=0.1;
real EulerStep(real x,real y) {return y+h*dy(x,y);}
real t0=0.0;
real y0=-1.0;

dot((t0,y0));

real t_old=t0;
real y_old=y0;
real t_new=0.0;
real y_new=0.0;

for(int i=1; i<5; ++i)
{
	t_new=t_old+h;
	y_new=EulerStep(t_old, y_old);
	draw((t_old, y_old)--(t_new, y_new), blue+1bp);
    dot((t_new,y_new));
    t_old=t_new;
    y_old=y_new;
}

pair P=(t0,y0);
draw(curve(P,dy,(xmin,ymin),(xmax,ymax)),red+1bp);
\end{asy}
\end{center}
\end{overprint}
\vspace{-69mm}
\end{example}
\end{frame}

\begin{frame}[fragile]
\begin{example}
Find the Euler-approximation of 
\begin{equation*}
y^\prime=-2ty,\ y(0)=1
\end{equation*}
using a step size of 0.2 over the range of $[0,2]$.

\begin{overprint}
\onslide<1-12>
\vspace{0mm}
\begin{center}
\begin{tabular}{rrrrr}
\thead{$n$}&\thead{$t_n$}&\thead{$y_n$}&\thead{$y(t_n)$}&\thead{Error}\\
\hline
\visible<2->{0 &0.0&1.0000000&1.0000000&\textbf{0.000000}\\}
\visible<3->{1 &0.2&1.0000000&0.9607894&\textbf{-0.039211}\\}
\visible<4->{2 &0.4&0.9200000&0.8521437&\textbf{-0.067856}\\}
\visible<5->{3 &0.6&0.7728000&0.6976763&\textbf{-0.075124}\\}
\visible<6->{4 &0.8&0.5873280&0.5272925&\textbf{-0.060036}\\}
\visible<7->{5 &1.0&0.3993830&0.3678794&\textbf{-0.031504}\\}
\visible<8->{6 &1.2&0.2396298&0.2369277&\textbf{-0.002702}\\}
\visible<9->{7 &1.4&0.1246075&0.1408584&\textbf{0.016251}\\}
\visible<10->{8 &1.6&0.0548273&0.0773047&\textbf{0.022477}\\}
\visible<11->{9 &1.8&0.0197378&0.0391639&\textbf{0.019426}\\}
\visible<12->{10&2.0&0.0055265&0.0183156&\textbf{0.012789}}
\end{tabular}
\end{center}

\onslide<13->

\vspace{1mm}
\begin{center}
\begin{asy}
import graph;
import slopefield;
import fontsize;
defaultpen(fontsize(9pt));

size(280,0);

real dy(real x,real y) {return -2.0*x*y;}
real xmin=-.1, xmax=2.1;
real ymin=-0.1, ymax=1.1;

add(slopefield(dy,(xmin,ymin),(xmax,ymax),20,grey+0.4bp,Arrow));

xaxis(YEquals(ymin),xmin,xmax,LeftTicks());
xaxis(YEquals(ymax),xmin,xmax);
yaxis(XEquals(xmin),ymin,ymax,RightTicks());
yaxis(XEquals(xmax),ymin,ymax);

real h=0.2;
real EulerStep(real x,real y) {return y+h*dy(x,y);}
real t0=0.0;
real y0=1.0;

dot((t0,y0));

real t_old=t0;
real y_old=y0;
real t_new=0.0;
real y_new=0.0;

for(int i=1; i<11; ++i)
{
	t_new=t_old+h;
	y_new=EulerStep(t_old, y_old);
	draw((t_old, y_old)--(t_new, y_new), blue+1bp);
    dot((t_new,y_new));
    t_old=t_new;
    y_old=y_new;
}

pair P=(t0,y0);
draw(curve(P,dy,(xmin,ymin),(xmax,ymax)),red+1bp);
\end{asy}
\end{center}
\end{overprint}
\vspace{-1mm}
\end{example}
\end{frame}

\begin{frame}
\begin{block}{}
There are two types of error:

\pause
\begin{itemize}
\item<2-> \textbf{Roundoff error} is the discrepancy arising from rounding numbers. This tends to snowball pretty fast when you have a great many calculations.
\item<3-> \textbf{Discretization error} is the error that results from the approximation method itself. For Euler's method this is cause by using the linear tangent lines to approximate a nonlinear curve. 
\end{itemize}
\onslide<4->
It can be shown, using Taylor series expansions, that the error is proportional to the square of the step size.
\begin{equation*}
\abs{y_i-y(t_i)}\leq C\cdot h^2
\end{equation*}
Where the constant $C$ depends of the size of the second derivative of the exact solution.
\onslide<5->

\vspace{2mm}
We call this error the \textbf{local discretization error} because it estimates the error for a single step only. After $n$ steps, we have $n$ times the error. Which we call the \textbf{global discretization error}. 
\end{block}
\end{frame}

\begin{frame}
\begin{block}{Global Discretization Error in Euler's Method}
If the solution of the IVP $y^\prime=f(t,y)$, $y(t_0)=y_0$ has a continuous second derivative on the interval $[t_0, t_k]$, and $y_n$ is the value of the Euler approximation at $t_n$, $t_0<t_1<\cdots<t_n\cdots<t_k$, then there exists a constant $C$ such that 
\[\abs{y_n-y(t_n)}\leq C\cdot h,\ \ \ \ n=1,2,\dots,k.\]
where step size $h=t_n-t_{n-1}$.
\end{block}
\end{frame}

\begin{frame}
\begin{block}{Second-Order Runge-Kutta Method}
For the IVP $y^\prime=f(t,y)$, $y(t_0)=y_0$, use the following formulas to compute the points $(t_1,y_1), (t_2,y_2),\dots$ of the approximate solution, using step size $h$:
\begin{align*}
t_{n+1}&=t_n+h\\
y_{n+1}&=y_n+h\cdot k_{n_1}
\end{align*}
where
\begin{align*}
k_{n_1}&=f(t_n,y_n)\\
k_{n_2}&=f\left(t_n+\dfrac{h}{2},y_n+\dfrac{h}{2}\cdot k_{n_1}\right)
\end{align*}
\end{block}
\end{frame}

%\setbeamerfont{block body}{size=\small}
\begin{frame}
\begin{block}{Fourth-Order Runge-Kutta Method}
For the IVP $y^\prime=f(t,y)$, $y(t_0)=y_0$, use the following formulas to compute the points $(t_1,y_1), (t_2,y_2),\dots$ of the approximate solution, using step size $h$:
\begin{align*}
t_{n+1}&=t_n+h\\
y_{n+1}&=y_n+\dfrac{h}{6}(k_{n_1}+2k_{n_2}+2k_{n_3}+k_{n_4})
\end{align*}
where
\begin{equation*}
\begin{aligned}
k_{n_1}&=f(t_n,y_n)\\
k_{n_2}&=f\left(t_n+\dfrac{h}{2},y_n+\dfrac{h}{2}\cdot k_{n_1}\right)\\
k_{n_3}&=f\left(t_n+\dfrac{h}{2},y_n+\dfrac{h}{2}\cdot k_{n_2}\right)\\
k_{n_2}&=f\left(t_n+h,y_n+h\cdot k_{n_3}\right)
\end{aligned}
\end{equation*}
\end{block}
\end{frame}
\end{document}
