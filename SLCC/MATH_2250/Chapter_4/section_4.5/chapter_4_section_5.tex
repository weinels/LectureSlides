\documentclass{beamer}

\usepackage[english]{babel}
\usepackage[utf8x]{inputenc}
\usepackage[inline]{asymptote}
\usepackage{slide_helper}

\title[MATH 2250 - Section 4.5]{Variation of Parameters}

\begin{document}

\begin{frame}
  \titlepage
\end{frame}

\begin{frame}
\begin{block}{Variation of Parameters}
\begin{overprint}
\onslide<1-4>
Let us extend Variation of Parameters to solve
\begin{equation*}
y^{\prime\prime}+p(t)y^{\prime}+q(t)y=f(t)
\end{equation*}
\visible<2->{We first need to find two linearly independent solutions to the associated homogeneous equation
\begin{equation*}
y^{\prime\prime}+p(t)y^{\prime}+q(t)y=0
\end{equation*}}

\visible<3->{\vspace{-5mm}
Which gives the general solution
\begin{equation*}
y_h=c_1 y_1(t) + c_2 y_2(t)
\end{equation*}
where $c_1$ and $c_2$ are arbitrary constants.}

\visible<4->{\vspace{2mm}
Just like with single order equations, we want to perturb the homogeneous solution into a particular solution to the nonhomogeneous DE\@.}

\onslide<5-9>
\visible<5->{\vspace{2mm}
We do so by replacing the constants $c_1$ and $c_2$ with unknown functions. 
\begin{equation*}
y_p=v_1(t) y_1(t) + v_2(t) y_2(t)
\end{equation*}}

\visible<6->{\vspace{-5mm}
To find $v_1(t)$ and $v_2(t)$ we substitute $y_p$ into the nonhomogeneous DE\@.}

\visible<7->{\vspace{2mm}
But, we need two equations and we only have $L(y)=f$. Thus, we must choose an auxiliary condition.}

\visible<8->{\vspace{2mm}
Let us calculate
\begin{equation*}
y_p^\prime=v_1 y_1^\prime + v_2 y_2^\prime + v_1^\prime y_1 + v_2^\prime y_2
\end{equation*}}

\visible<9->{\vspace{-5mm}
So, we can choose $v_1 y_1^\prime + v_2 y_2^\prime=0$ as our auxiliary condition, which reduces $y_p^\prime$ to:
\begin{equation*}
y_p^\prime=v_1^\prime y_1 + v_2^\prime y_2
\end{equation*}}

\onslide<10-15>
We can then obtain
\begin{equation*}
y_p^{\prime\prime}=v_1 y_1^{\prime\prime} + v_2 y_2^{\prime\prime} + v_1^\prime y_1^\prime + v_2^\prime y_2^\prime
\end{equation*}
\visible<11->{We then substitute $y_p$, $y_p^\prime$, and $y_p^{\prime\prime}$ into $L(y)=f$.
\begin{equation*}
\begin{aligned}
(v_1 y_1^{\prime\prime} + v_2 y_2^{\prime\prime} + v_1^\prime y_1^\prime + v_2^\prime y_2^\prime) + p\cdot (v_1^\prime y_1 + v_2^\prime y_2) + q\cdot (v_1 y_1 + v_2 y_2) &= f \\
\visible<12->{v_1(y_1^{\prime\prime}+p y_1^\prime+q y_1) + v_2(y_2^{\prime\prime}+p y_2^\prime+q y_2) +
(v_1^\prime y_1^\prime + v_2^\prime y_2^\prime) &= f \\}
\visible<13->{v_1\cdot 0 + v_2\cdot 0 + v_1^\prime y_1^\prime + v_2^\prime y_2^\prime &= f \\}
\visible<14->{v_1^\prime y_1^\prime + v_2^\prime y_2^\prime &= f}
\end{aligned}
\end{equation*}}

\visible<15->{So, we have the system
\begin{equation*}
\begin{aligned}
v_1^\prime y_1^\prime + v_2^\prime y_2^\prime &=f\\
v_1^\prime y_1 + v_2^\prime y_2&=0
\end{aligned}
\end{equation*}}

\onslide<16->
Using Cramer's Rule, the system
\begin{equation*}
\begin{aligned}
v_1^\prime y_1^\prime + v_2^\prime y_2^\prime &=f\\
v_1^\prime y_1 + v_2^\prime y_2&=0
\end{aligned}
\end{equation*}
has solution
\begin{equation*}
v_1^\prime=\dfrac{%
\begin{vmatrix}
0 & y_2        \\
f & y_2^\prime \\
\end{vmatrix}}{%
\begin{vmatrix}
y_1        & y_2        \\
y_1^\prime & y_2^\prime \\
\end{vmatrix}}
\quad\text{and}\quad
v_2^\prime=\dfrac{%
\begin{vmatrix}
y_1        & 0 \\
y_1^\prime & f \\
\end{vmatrix}}{%
\begin{vmatrix}
y_1        & y_2        \\
y_1^\prime & y_2^\prime \\
\end{vmatrix}}
\end{equation*}
\visible<17->{The denominator is just the Wronskian $W(y_1, y_2)=y_1 y_2^\prime-y_2 y_1^\prime\neq 0$, because $y_1$ and $y_2$ are linearly independent.}

\visible<18->{\vspace{2mm}
Thus, we can integrate to find $v_1$ and $v_2$.}

\visible<19->{\vspace{-3mm}
\begin{equation*}
v_1=-\int\dfrac{y_2 f}{W(y_1, y_2)}
\quad\text{and}\quad
v_2=\int\dfrac{y_1 f}{W(y_1, y_2)}
\end{equation*}}
\end{overprint}
\end{block}
\end{frame}



\begin{frame}
\begin{example}
Consider
\begin{equation*}
y^{\prime\prime}+y=\sec[t] \quad \abs{t}<\dfrac{\pi}{2}
\end{equation*}\pause
The characteristic equation is $r^2+1=0$, so the characteristic roots are $r=\pm i$. Which means $y_1=\cos[t]$ and $y_2=\sin[t]$.\pause

\vspace{2mm}
The Wronskian is

\vspace{-4mm}
\begin{equation*}
W(y_1, y_2)=
\begin{vmatrix}
\+\cos[t] & \sin[t] \\
 -\sin[t] & \cos[t] \\
\end{vmatrix}
=\cos[t][2]+\sin[t][2]=1
\end{equation*}\pause

\vspace{-3mm}
So,

\vspace{-5mm}
\begin{equation*}
\begin{aligned}
v_1^\prime &= -\dfrac{y_2 f}{1} = -\sin[t]\sec[t]=-\dfrac{\sin[t]}{\cos[t]} \\
v_2^\prime &= \dfrac{y_1 f}{1} = \cos[t]\sec[t] = 1
\end{aligned}
\end{equation*}\pause

\vspace{-2mm}
Integrating gives $v_1=\ln[\cos[t]]$ and $v_2=t$.\pause

\vspace{2mm}
Thus, the general solution is

\vspace{-4mm}
\begin{equation*}
y=c_1\cos[t]+c_2\sin[t]+\ln[\cos[t]]\cos[t]+t\sin[t]
\end{equation*}
\end{example}
\end{frame}

\begin{frame}
\begin{example}
Consider
\begin{equation*}
y^{\prime\prime}+y=4\sin[t]
\end{equation*}\pause
The characteristic equation is $r^2+1=0$, so the characteristic roots are $r=\pm i$. Which means $y_1=\cos[t]$ and $y_2=\sin[t]$.\pause

\vspace{2mm}
The Wronskian is $W(y_1,y_2)=1$.\pause

\vspace{2mm}
So,
\begin{equation*}
v_1^\prime = -\dfrac{y_2 f}{1} = -2(1-\cos[2t])
\quad\text{and}\quad
v_2^\prime = \dfrac{y_1 f}{1} = 2\sin[2t]
\end{equation*}\pause
Integrating gives $v_1=-2t+2\sin[t]\cos[t]$ and $v_2=1-2\cos[t][2]$.\pause

\vspace{2mm}
Thus, the particular solution is
\begin{equation*}
\begin{aligned}
y_p&=(-2t+2\sin[t]\cos[t])(\cos[t])+(1-2\cos[t][2])(\sin[t])\\\pause
&=-2t\cos[t]+\sin[t]
\end{aligned}
\end{equation*}
\end{example}
\end{frame}

\begin{frame}
\begin{block}{Method for Determining a General Solution of $L(y)=f(t)$}
\begin{enumerate}[<+- | alert@+>]
\item Determine two linearly independent solutions, $y_1$ and $y_2$, of $L(y)=0$.
\item Solve, for $v_1^\prime$, and $v_2^\prime$, the system \begin{equation*}
\begin{aligned}
v_1^\prime y_1^\prime + v_2^\prime y_2^\prime &=f\\
v_1^\prime y_1 + v_2^\prime y_2&=0
\end{aligned}
\end{equation*}
(Cramer's Rule is recommended.)
\item Integrate to determine $v_1$ and $v_2$.
\item Compute $y=y_h+y_p=c_1 y_1+c_2 y_2 + v_1 y_1 + v_2 y_2$.
\end{enumerate}
\onslide<+->
\end{block}
\begin{block}{Note}
This method can be extended to higher orders.
\end{block}
\end{frame}

\begin{frame}
\begin{example}
Consider
\begin{equation*}
y^{\prime\prime}-2y^\prime+y=\dfrac{e^t}{t^2+1}
\end{equation*}\pause
The associated homogeneous system has solutions $y_1=e^t$ and $y_2=te^t$.\pause

\vspace{2mm}
The Wronskian is $W(y_1,y_2)=
\begin{vmatrix}
e^t & te^t     \\
e^t & (t+1)e^t \\
\end{vmatrix}
=(t+1)e^{2t}-te^{2t}=e^{2t}$\pause

\vspace{2mm}
So, using the Cramer's Rule formulas from before
\begin{equation*}
v_1^\prime=-\dfrac{y_2 f}{W(y_1,y_2)}=-\dfrac{t}{t^2+1}
\quad\text{and}\quad
v_2^\prime=\dfrac{y_1 f}{W(y_1,y_2)}=\dfrac{1}{t^2+1}
\end{equation*}\pause
Integrating gives
\begin{equation*}
v_1=-\tfrac{1}{2}\ln[t^2+1]
\quad\text{and}\quad
v_2=\tan[t][-1]
\end{equation*}\pause
Thus, the general solution is

\vspace{-3mm}
\begin{equation*}
y=c_1e^t+c_2te^t-\tfrac{1}{2}\ln[t^2+1]e^t+\tan[t][-1]te^t
\end{equation*}
\end{example}
\end{frame}

\begin{frame}
\begin{example}
Consider
\begin{equation*}
\begin{aligned}
t^2 y^{\prime\prime}-2t y^\prime+2y&=t\ln[t]
,\qquad t>0\\\pause
y^{\prime\prime}-\dfrac{2}{t}y^\prime+\dfrac{2}{t^2}y&=\dfrac{\ln[t]}{t}
\end{aligned}
\end{equation*}\pause
Two solutions of $L(y)=0$ are $y_1=t$ and $y_2=t^2$.\pause

\vspace{2mm}
The Wronskian is $W(y_1,y_2)=
\begin{vmatrix}
t & t^2 \\
1 & 2t  \\
\end{vmatrix}
=2t^2-t^2=t^2$\pause

\vspace{2mm}
So, using Cramer's Rule
\begin{equation*}
\begin{aligned}
v_1^\prime &= -\dfrac{y_2 f}{W(y_1, y_2)}=-\dfrac{\ln[t]}{t} 
&\rightarrow v_1 &= -\dfrac{1}{2}\ln[t][2]\\
v_2^\prime &= \dfrac{y_1 f}{W(y_1, y_2)}=\dfrac{\ln[t]}{t^2}
&\rightarrow v_2 &= -\dfrac{\ln[t]+1}{t}
\end{aligned}
\end{equation*}\pause
The general solution is

\vspace{-3mm}
\begin{equation*}
y=c_1t+c_2t^2-\tfrac{t}{2}\ln[t][2]-t\ln[t]-t
\end{equation*}
\end{example}
\end{frame}
\end{document}