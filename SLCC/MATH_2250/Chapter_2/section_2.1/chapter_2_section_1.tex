\documentclass{beamer}
\usepackage[utf8]{inputenc}
\usepackage[english]{babel}
\usepackage[T1]{fontenc}
\usepackage[inline]{asymptote}
\usepackage{slide_helper}
\usepackage{asy_helper}
\usepackage{subcaption}


\title[MATH 2250 - Section 2.1]{Linear Equations: The Nature of Their Solutions}

\begin{document}
\begin{frame}
\titlepage
\end{frame}

\begin{frame}
\begin{block}{Definition}
An equation $F(x_1, x_2, \ldots, x_n)=C$ is \textbf{linear} if it is of the form
\begin{equation*}
a_1 x_1+a_2 x_2+\cdots+a_n x_n=C
\end{equation*}
where $a_1, a_2, \ldots, a_n$ and $C$ are constants.

\vspace{2mm}
If $C=0$, the equation is said to be \textbf{homogeneous}.
\end{block}\pause

\begin{example}
Which of the following are linear equations?
\begin{equation*}
\begin{aligned}
4x - 3e^x &= 15 & \visible<3->{\text{No}} \\
4x - 2y + 3\sqrt{z} &= 12 & \visible<4->{\text{No}} \\
2x - 3y +4z + 3 &= w & \visible<5->{\text{Yes}}
\end{aligned}
\end{equation*}
\end{example}
\end{frame}

\begin{frame}
\begin{block}{Definition}
A differential equation $F(y, y^\prime, y^{\prime\prime},\ldots,y^{(n)})=f(t)$ is \textbf{linear} if it is of the form
\begin{equation*}
a_n(t)\dfrac{d^n y}{dt^n}+a_{n-1}(t)\dfrac{d^{n-1} y}{dt^{n-1}}+\cdots+a_1(t)\dfrac{dy}{dt}+a_0(t)y=f(t)
\end{equation*}
where all functions of $t$ are assumed to be defined over some common interval $I$.

\vspace{2mm}
If $f(t)=0$ over the interval $I$, the differential equation is said to be \textbf{homogeneous}.
\end{block}\pause

\begin{block}{First and Second Order Notation}
It is common to write first-order differential equations as

\vspace{-3mm}
\begin{equation*}
y^\prime+p(t)y=f(t)
\end{equation*}
and second-order differential equations as

\vspace{-3mm}
\begin{equation*}
y^{\prime\prime}+p(t)y^\prime+q(t)y=f(t)
\end{equation*}
\end{block}
\end{frame}

\begin{frame}
\begin{example}
Let us classify the following differential equations.

\vspace{-2mm}
\begin{center}
\begin{tabular}{ccccc}
Differential Equation & Order & Linear? & Homogeneous? & Coefficients \\\hline\\
\visible<+->{$y^{\prime}+ty= 1$} & \visible<+->{1} & \visible<+->{Yes} & \visible<+->{No} & \visible<+->{Variable} \\\\
\visible<+->{$y^{\prime\prime}+y y^\prime +y= t$} & \visible<+->{2} & \visible<+->{No & --- & ---}\\\\
\visible<+->{$y^{\prime\prime} + ty^\prime +y^2 = 0$} & \visible<+->{2} & \visible<+->{No & --- & ---}\\\\
\visible<+->{$y^{\prime\prime} +3 y^\prime + 2y = 0$} & \visible<+->{2} & \visible<+->{Yes}& \visible<+->{Yes} & \visible<+->{Constant}\\\\
\visible<+->{$y^{\prime\prime} + y = \sin[y]$} & \visible<+->{2} & \visible<+->{No & --- & ---}\\\\
\visible<+->{$y^{\prime\prime\prime} + 3y^\prime + y= \sin[t]$} & \visible<+->{3} & \visible<+->{Yes}& \visible<+->{No} & \visible<+->{Variable}
\end{tabular}
\end{center}
\end{example}
\end{frame}

\begin{frame}
\begin{block}{Notation}
We will use a \textbf{vector} notation to represent a whole set of variables:
\begin{description}
\item[Linear Algebraic Equations:] 
\begin{equation*}
\vect{x}=[x_1, x_2, \ldots, x_n]
\end{equation*}
\item[Linear Differential Equations:] 
\begin{equation*}
\vect{y}=[y^{(n)}, y^{(n-1)}, \ldots, y^\prime, y]
\end{equation*}
\end{description}
\end{block}\pause

\begin{block}{Definition}
A \textbf{linear operator} $L$ is an entire operation performed on a set of variables.
\begin{description}
\item[Linear Algebraic Equations:]
\begin{equation*}
L(\vect{x}) = a_1 x_1 + a_2 x_2 + \cdots + a_n x_n
\end{equation*}
\item[Linear Differential Equations:] 
\begin{equation*}
L(\vect{y})= a_n(t)\dfrac{d^n y}{dt^n}+a_{n-1}(t)\dfrac{d^{n-1} y}{dt^{n-1}}+\cdots+a_1(t)\dfrac{dy}{dt}+a_0(t)y
\end{equation*}
\end{description}
\end{block}
\end{frame}

\begin{frame}
\begin{example}
What is the linear operator for the following linear differential equations?
\begin{center}
\begin{tabular}{ccc}
$y^{\prime}+ty=1$ & \visible<2->{$\rightarrow$ & $L(\vect{y})=y^\prime+ty$} \\ 
\visible<3->{\alt<3>{$y^{\prime\prime} + 2y = 3y^\prime +t$}{$y^{\prime\prime} - 3 y^\prime + 2y=t$}} & \visible<5->{$\rightarrow$ & $L(\vect{y})=y^{\prime\prime} - 3 y^\prime + 2y$} \\
\visible<6->{$y^{(4)} + 3y = \sin[t]$} & \visible<7->{$\rightarrow$ & $L(\vect{y})=y^{(4)}+3y$}
\end{tabular}
\end{center}
\end{example}

\onslide<8->
\begin{block}{Linear Operator Properties}
\begin{equation*}
\begin{aligned}
L(k\vect{u})&=kL(\vect{u}),\quad k\in\R \\
L(\vect{u}+\vect{w}) &= L(\vect{u})+L(\vect{w})
\end{aligned}
\end{equation*}
\end{block}

\onslide<9->
\begin{block}{Proof}
The properties can be proved directly for algebraic operators.

\vspace{2mm}
For differential operators, the proof follows from the derivative properties:
\begin{itemize}
\item ${(kf)}^\prime = k f^\prime$
\item ${(f+g)}^\prime = f^\prime+g^\prime$
\end{itemize}
\end{block}
\end{frame}

\begin{frame}
\begin{block}{Superposition Principle for Linear Homogeneous Equations}
Let $\vect{u}_1$ and $\vect{u}_2$ be any solutions of the \emph{homogeneous linear} equation
\begin{equation*}
L(\vect{u})=0
\end{equation*}

\vspace{-3mm}
\begin{itemize}
\item The sum $\vect{u}=\vect{u}_1+\vect{u}_2$ is also a solution.
\item For any constant $k$, $\vect{u}=k\vect{u}_1$ is also a solution.
\end{itemize}
\end{block}\pause

\begin{block}{Proof}
The proof of the Superposition Principle follows directly from the properties of linear operators from the previous slides.
\begin{equation*}
\begin{aligned}
L(\vect{u}) = L(\vect{u_1}+\vect{u_2})
= L(\vect{u_1}) + L(\vect{u_2})
= 0 + 0 = 0
\end{aligned}
\end{equation*}
\begin{equation*}
\begin{aligned}
L(\vect{u}) = L(k \vect{u_1})
= k L(\vect{u_1})
= k\cdot 0 = 0
\end{aligned}
\end{equation*}
\end{block}
\end{frame}

\begin{frame}
\begin{example}
The point $(1,3)$ is on the line $y=3x$.\pause~So is the point $(2,6) = (2\cdot 1, 2\cdot 3)$.

\vspace{2mm}\pause
Additionally, the point $(3,9)=(1+2,3+6)=(1,3)+(2,6)$ is on the line.
\end{example}\pause

\begin{example}
The differential equation
\begin{equation*}
y^{\prime\prime}-4y=0
\end{equation*}
has the solutions $y=e^{2t}$ and $y=e^{-2t}$.\pause

\vspace{2mm}
By superposition, $y=2e^{2t}+3e^{-2t}$ must also be a solution.\pause

\vspace{2mm}
This is easily verified:
\begin{equation*}
\begin{aligned}
y^\prime &= 4e^{2t}-6e^{-2t} \\\pause
y^{\prime\prime} &= 8e^{2t}+12e^{-2t} \\\pause
y^{\prime\prime}-4y &= \left(8e^{2t}+12e^{-2t}\right) -4\left(2e^{2t}+3e^{-2t}\right)\pause \\
& = 8e^{2t}+12e^{-2t} -8e^{2t}-12e^{-2t}\pause = 0
\end{aligned}
\end{equation*}
\end{example}
\end{frame}

\begin{frame}
\begin{block}{Nonhomogeneous Principle}
Let $\vect{u}_p$ be any solution (called a particular solution) to \emph{linear nonhomogeneous} equation
\begin{equation*}
L(\vect{u})=C\qquad\text{(algebraic)} 
\end{equation*}
or
\begin{equation*}
L(\vect{u})=f(t)\qquad\text{(differential)}
\end{equation*}
Then,
\begin{equation*}
\vect{u}=\vect{u}_h+\vect{u}_p
\end{equation*}
is also a solution, here $\vect{u}_h$ is a solution to the \textbf{associated homogeneous} equation
\begin{equation*}
L(\vect{u})=0
\end{equation*}
Furthermore, \emph{every solution of the nonhomogeneous equation must be of the form $\vect{u}=\vect{u}_h+\vect{u}_p$.}
\end{block}
\end{frame}

\begin{frame}
\begin{block}{Proof}
It is easy to show that $\vect{u}=\vect{u}_h+\vect{u}_p$ is a solution.
\begin{equation*}
L(\vect{u}) = L(\vect{u}_h+\vect{u}_p) = L(\vect{u}_h) + L(\vect{u}_p) = 0 + f(t) = f(t)
\end{equation*}\pause

\vspace{-5mm}
To show that every solution has to be of this form, suppose that $\vect{u}_q$ is any solution. Note that $\vect{u}_q=\vect{u}_p + (\vect{u}_q - \vect{u}_p)$.\pause

\vspace{2mm}
We can then show that $\vect{u}_q-\vect{u}_p$ is also a solution to $L(\vect{u})=0$:
\begin{equation*}
\begin{aligned}
L(\vect{u}_q-\vect{u}_p) &=\pause  L(\vect{u}_q) + L(-\vect{u}_p)\pause \\
&= L(\vect{u}_q)-L(\vect{u}_p)\pause \\
&= f(t) - f(t) = 0
\end{aligned}
\end{equation*}
\end{block}\pause
\begin{block}{Process for Solving Nonhomogeneous Linear Equations}
\begin{description}
\item[Step 1:] Find all solutions $\vect{u}_h$ of $L(\vect{u})=0$.
\item[Step 2:] Find any solution $\vect{u}_p$ of $L(\vect{u})=f$.
\item[Step 3:] Add $\vect{u}_h+\vect{u}_p=\vect{u}$ to find all solutions of $L(\vect{u})=f$.
\end{description}
\end{block}
\end{frame}

\begin{frame}[fragile]
\begin{example}
Consider
\begin{equation*}
y^\prime-y=t
\end{equation*}
\begin{overprint}
\onslide<1-4>
To solve using superposition we need to complete three steps.
\begin{description}
\item[Step 1:] \mbox{}\visible<2->{Solve the associated homogeneous equation $y^\prime-y=0$, or $y^\prime = y$. (Note: first-order homogeneous linear differential equations are always separable.)
\begin{equation*}
y_{h}=c e^{t},\quad\text{for any}~c\in\R
\end{equation*}}
\item[Step 2:] \mbox{}\visible<3->{We can verify by differentiation and substitution that $y_{p}=-t-1$ is a particular solution.}
\item[Step 3:] \mbox{}\visible<4->{Superposition tells us that 
\begin{equation*}
y=y_h+y_p=c e^t-t-1
\end{equation*}
is a solution for any $c\in\R$.}
\end{description}
\onslide<5->
\begin{figure}[h]
\begin{subfigure}{.29\linewidth}
%\centering
\begin{asy}
size(103);

real min_x=-3, max_x=3;
real min_y=-3, max_y=3;

pair start=(min_x,min_y);
pair end=(max_x,max_y);

draw_grid_lines(min_x, max_x, min_y, max_y);

real yp (real t, real y) { return y; }

add(slopefield(yp, start, end, 10));

draw(curve((0,0.5), yp, start, end), red+1bp);
draw(curve((0,2), yp, start, end), red+1bp);
draw(curve((0,-0.5), yp, start, end), red+1bp);
draw(curve((0,-2), yp, start, end), red+1bp);
draw(curve((0,0.1), yp, start, end), red+1bp);
draw(curve((0,-0.1), yp, start, end), red+1bp);

limits(start,end,Crop);

xaxis("$t$",YEquals(min_y),min_x,max_x,LeftTicks());
xaxis(YEquals(max_y),min_x,max_x);
yaxis(XEquals(min_x),min_y,max_y,LeftTicks());
yaxis(XEquals(max_x),min_y,max_y);
\end{asy}
\caption*{$\{y_h\}$}
\end{subfigure}
\begin{subfigure}{1em}
+
\end{subfigure}
\begin{subfigure}{.29\linewidth}
\centering
\begin{asy}
size(103);

real min_x=-3, max_x=3;
real min_y=-3, max_y=3;

pair start=(min_x,min_y);
pair end=(max_x,max_y);

draw_grid_lines(min_x, max_x, min_y, max_y);

real yp (real t, real y) { return y+t; }

add(slopefield(yp, start, end, 10));

draw(curve((0,-1), yp, start, end), blue+1bp);

limits(start,end,Crop);

xaxis("$t$",YEquals(min_y),min_x,max_x,LeftTicks());
xaxis(YEquals(max_y),min_x,max_x);
yaxis(XEquals(min_x),min_y,max_y,LeftTicks());
yaxis(XEquals(max_x),min_y,max_y);
\end{asy}
\caption*{$y_p$}
\end{subfigure}
\begin{subfigure}{1em}
=
\end{subfigure}
\begin{subfigure}{.29\linewidth}
\centering
\begin{asy}
size(103);

real min_x=-3, max_x=3;
real min_y=-3, max_y=3;

pair start=(min_x,min_y);
pair end=(max_x,max_y);

draw_grid_lines(min_x, max_x, min_y, max_y);

real yp (real t, real y) { return y+t; }

add(slopefield(yp, start, end, 10));

draw(curve((0,-0.51), yp, start, end), heavymagenta+1bp);
draw(curve((0,0.9), yp, start, end), heavymagenta+1bp);
draw(curve((0,2.3), yp, start, end), heavymagenta+1bp);
draw(curve((-1.55,0), yp, start, end), heavymagenta+1bp);
draw(curve((-2.1,0), yp, start, end), heavymagenta+1bp);
draw(curve((-1.3,0), yp, start, end), heavymagenta+1bp);

draw(curve((0,-1), yp, start, end), blue+1bp);

limits(start,end,Crop);

xaxis("$t$",YEquals(min_y),min_x,max_x,LeftTicks());
xaxis(YEquals(max_y),min_x,max_x);
yaxis(XEquals(min_x),min_y,max_y,LeftTicks());
yaxis(XEquals(max_x),min_y,max_y);
\end{asy}
\caption*{$\{y_h\}+y_p$}
\end{subfigure}
\end{figure}
\end{overprint}
\end{example}
\end{frame}

\begin{frame}
\begin{block}{Note}
Many of you may be wondering how exactly we go about finding particular solutions?\pause

\vspace{2mm}
In the next section we will start talking about different methods to do just that.\pause

\vspace{2mm}
But, sometimes a particular solution may be staring us in the face.
\end{block}
\end{frame}

\begin{frame}[fragile]
\begin{example}
Let us solve
\begin{equation*}
y^\prime + ay = b
\end{equation*}
where $a$ and $b$ are constants.
\begin{overprint}
\onslide<1-4>
\begin{description}
\item[Step 1:] \mbox{}\visible<2->{The associated homogeneous equation $y^\prime+ay=0$ will soon become an old friend. \\It has the solution $y_h=c e^{-at}$, where $c\in\R$.}
\item[Step 2:] \mbox{}\visible<3->{Looking at the DE we can see that $y_p=\dfrac{b}{a}$ will satisfy this equation. (Recall that that derivative of a constant is zero.)}
\item[Step 3:] \mbox{}\visible<4->{Superposition tells us that 
\begin{equation*}
y=y_h+y_p=c e^{-at}+\dfrac{b}{a}
\end{equation*}
is a solution for any $c\in\R$.}
\end{description}
\onslide<5->
Alternatively, we can look for a horizontal line in the direction field.
\begin{center}
\begin{asy}
size(150);

real min_x=-5, max_x=5;
real min_y=-5, max_y=5;

pair start=(min_x,min_y);
pair end=(max_x,max_y);

real yp (real t, real y) { return 1-2*y; }

add(slopefield(yp, start, end, 20));
unfill(box((min_x,0.6),(max_x, 0.4)));

draw_grid_lines(min_x, max_x, min_y, max_y);

draw((min_x, 0.5)--(max_x, 0.5),dashed+blue);

limits(start,end,Crop);

label("$\dfrac{b}{a}$", (-5.5,1), blue);

xaxis("$t$",YEquals(min_y),min_x,max_x,NoTicks());
xaxis(YEquals(max_y),min_x,max_x);
yaxis("$y$", XEquals(min_x),min_y,max_y,NoTicks());
yaxis(XEquals(max_x),min_y,max_y);
\end{asy}
\end{center}
\end{overprint}
\end{example}
\end{frame}
\end{document}
