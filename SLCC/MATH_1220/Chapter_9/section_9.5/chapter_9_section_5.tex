\documentclass{beamer}
\usepackage[utf8]{inputenc}
\usepackage[english]{babel}
\usepackage[T1]{fontenc}
\usepackage[inline]{asymptote}
\usepackage{slide_helper}
\usepackage{asy_helper}
\usepackage{subcaption}


\title[MATH 1220 - Section 9.5]{Linear Equations}

\begin{document}
\begin{frame}
\titlepage
\end{frame}

\begin{frame}
\begin{block}{Definition}
A first-order linear differential equation is one that can be put into the form
\begin{equation*}
\dfrac{dy}{dt}+P(t)\cdot y = Q(t)
\end{equation*}
\end{block}
\onslide<2->
\begin{block}{Integrating Factor Method (Constant Coefficient)}
Let us look at the first-order linear differential equation
\begin{equation*}
y^\prime+ ay = f(t),\quad a\in\R
\end{equation*}
\onslide<3->
This method uses a simple observation made by Euler:
\begin{equation*}
e^{at}\left(y^\prime+ay\right) = \dfrac{d}{dt}\left(e^{at} y\right)
\end{equation*}
\begin{overprint}
\onslide<4>
Let us start with the differential equation.
\begin{equation*}
y^\prime+ ay = f(t)
\end{equation*}
\onslide<5>
We first multiply both sides of the equation by $e^{at}$.
\begin{equation*}
e^{at}\left(y^\prime+ay\right) = e^{at} f(t) 
\end{equation*}
\onslide<6>
We then apply Euler's observation to the left-hand side.
\begin{equation*}
\dfrac{d}{dt}\left(e^{at} y\right) = e^{at} f(t) 
\end{equation*}
\onslide<7>
Next we integrate both sides.
\begin{equation*}
e^{at} y = \int e^{at} f(t)  dt + c
\end{equation*}
\onslide<8->
Solving for $y$ gives:
\begin{equation*}
y(t) = e^{-at} \int e^{at} f(t)  dt + c e^{-at}
\end{equation*}
\end{overprint}
\end{block}
\end{frame}

\begin{frame}
\begin{block}{Integrating Factor Method (Variable Coefficient)}
Now let us look at the more general first-order differential equation
\begin{equation*}
y^\prime + p(t) y = f(t)
\end{equation*}
\begin{overprint}
\onslide<2>
We seek a function $\mu(t)$ that satisfies Euler's observation, i.e.
\begin{equation*}
\mu(t)\cdot\left(y^\prime + p(t) y\right) = \dfrac{d}{dt}\left(\mu(t)\cdot y\right)
\end{equation*}
\onslide<3>
Let us carry out the differentiation on the right-hand side
\begin{equation*}
\mu(t) y^\prime + p(t)\mu(t) y = \mu^\prime (t) y + \mu(t) y^\prime
\end{equation*}
\onslide<4>
If we assume $y(t)\neq 0$, this simplifies to
\begin{equation*}
\mu^\prime (t) = p(t)\mu(t)
\end{equation*}
\onslide<5>
We can find a solution $\mu(t)>0$ by Separation of Variables.
\begin{equation*}
\dfrac{\mu^\prime (t)}{\mu(t)} = p(t)
\end{equation*}
\onslide<6>
We can find a solution $\mu(t)>0$ by Separation of Variables.
\begin{equation*}
\ln\abs{\mu(t)} = \int p(t) dt
\end{equation*}
\onslide<7->
We can find a solution $\mu(t)>0$ by Separation of Variables.
\begin{equation*}
\mu(t) = e^{\int p(t) dt}
\end{equation*}
\end{overprint}
\begin{overprint}
\onslide<8>
We now know the integrating factor, and perform the same steps as before.
\begin{equation*}
y^\prime + p(t) y = f(t)
\end{equation*}
\onslide<9>
Multiply both sides by the integrating factor.
\begin{equation*}
\mu(t)\cdot\left(y^\prime + p(t) y\right) = \mu(t)\cdot f(t)
\end{equation*}
\onslide<10>
Apply the property $\mu(t)\cdot\left(y^\prime + p(t) y\right) = {\left(\mu(t)\cdot y\right)}^\prime$ to the left-hand side.
\begin{equation*}
{\left(\mu(t) y\right)}^\prime = \mu(t) f(t)
\end{equation*}
\onslide<11>
Integrate both sides.
\begin{equation*}
\mu(t) y(t) = \int \mu(t) f(t) dt + c
\end{equation*}
\onslide<12->
Assuming $\mu(t)\neq 0$, we can solve for $y$.
\begin{equation*}
y(t) = \dfrac{1}{\mu(t)} \int \mu(t) f(t) dt + \dfrac{c}{\mu(t)}
\end{equation*}
\end{overprint}
\end{block}
\end{frame}

\begin{frame}
\begin{block}{Integrating Factor Method for First-Order Linear DEs}
\small
To solve the linear first-order DE, where $p$ and $f$ are continuous on a domain $I$.
\begin{equation*}
y^\prime + p(t) y = f(t)
\end{equation*}
\begin{description}
\item[Step 1.] Find the integrating factor $\mu(t) = e^{\int p(t) dt}$, where $\int p(t) dt$ represents \emph{any} anti-derivative of $p(t)$.
\item[Step 2.] Multiply both sides of the DE by $mu(t)$, which always simplifies to:

\vspace{-2mm}
\begin{equation*}
{\left(e^{\int p(t) dt} y(t)\right)}^\prime = e^{\int p(t) dt} f(t)
\end{equation*}
\item[Step 3.] Find the anti-derivative to get:

\vspace{-2mm}
\begin{equation*}
e^{\int p(t) dt} y(t) = \int e^{\int p(t) dt} f(t) dt +c
\end{equation*}
\item[Step 4.] Solve algebraically for $y$.

\vspace{-2mm}
\begin{equation*}
y = e^{-\int p(t) dt} \int e^{\int p(t) dt} f(t) dt + c e^{-\int p(t) dt}
\end{equation*}
\item[Step 5.] For IVPs, substitute the initial conditions in to find $c$.
\end{description}
\end{block}
\end{frame}

\begin{frame}
\begin{example}
Consider the IVP
\begin{equation*}
y^\prime - y = t,\quad y(0) = 1
\end{equation*}

\vspace{2mm}
\begin{overprint}
\onslide<2-6>
\textbf{Step 1.} Find the integrating factor:
\begin{equation*}
\mu(t) = \alt<3->{e^{\int (-1) dt}}{e^{\int p(t) dt}} \visible<4->{=e^{-t}}
\end{equation*}
\visible<5->{\textbf{Step 2.} Multiply both sides of the DE by $\mu(t)$:
\begin{equation*}
e^{-t}\left(y^\prime - y\right) = e^{-t}
\end{equation*}}
\visible<6->{Which reduces to:
\begin{equation*}
{\left(e^{-t} y\right)}^\prime = t e^{-t}
\end{equation*}}
\onslide<7->
\textbf{Step 3.} Find the antiderivative:
\begin{equation*}
e^{-t} y = \int t e^{-t} dt \visible<8->{= e^{-t}(-t-1)+c}
\end{equation*}
\visible<9->{\textbf{Step 4.} Solve for $y$:
\begin{equation*}
y(t) = e^t \left(e^{-t}\right)\left(-t-1\right) + c e^{t} \visible<10->{=-t-1+ce^{t}}
\end{equation*}}
\visible<11->{%

\vspace{-5mm}
\textbf{Step 5.} Plug in the initial conditions to find the solution to the IVP\@:
\begin{equation*}
1 = y(0) = -0-1+c e^{0} \visible<12->{\Rightarrow c = 2}
\end{equation*}}
\visible<13->{Thus, the solution to the IVP is $y(t) = -t -1 + 2 e^t$}
\end{overprint}
\end{example}
\end{frame}

\begin{frame}
\begin{example}
Consider the IVP
\begin{equation*}
y^\prime + \dfrac{1}{t}y = \dfrac{1}{t^2}\quad(\text{assume}~t>0),\quad y(1) = 3
\end{equation*}

\vspace{2mm}
\begin{overprint}
\onslide<2-6>
\textbf{Step 1.} Find the integrating factor:
\begin{equation*}
\mu(t) = \alt<3->{e^{\int (\tfrac{1}{t}) dt}}{e^{\int p(t) dt}} \visible<4->{=e^{\ln[t]}=t}
\end{equation*}
\visible<5->{\textbf{Step 2.} Multiply both sides of the DE by $\mu(t)$:
\begin{equation*}
t\left(y^\prime - y\right) = \dfrac{1}{t^2}\cdot t
\end{equation*}}
\visible<6->{Which reduces to:
\begin{equation*}
{\left(t \cdot y\right)}^\prime = \dfrac{1}{t}
\end{equation*}}
\onslide<7->
\textbf{Step 3.} Find the antiderivative:
\begin{equation*}
t y = \int \dfrac{1}{t} dt  \visible<8->{= \ln[t] + c}
\end{equation*}
\visible<9->{\textbf{Step 4.} Solve for $y$:
\begin{equation*}
y(t) = \dfrac{\ln[t] + c}{t}
\end{equation*}}
\visible<10->{%

\vspace{-5mm}
\textbf{Step 5.} Plug in the initial conditions to find the solution to the IVP\@:
\begin{equation*}
3 = y(1) = \dfrac{\ln[1] + c}{1} \visible<11->{\Rightarrow c = 3}
\end{equation*}}

\vspace{-4mm}
\visible<12->{Thus, the solution to the IVP is $y(t) = \dfrac{\ln[t] + 3}{t}$}
\end{overprint}
\end{example}
\end{frame}

\end{document}
