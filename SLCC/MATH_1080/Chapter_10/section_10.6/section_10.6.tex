\documentclass{beamer}

\usepackage[english]{babel}
\usepackage[utf8x]{inputenc}
\usepackage{tabu}
\usepackage[inline]{asymptote}
\usepackage{subcaption}
\usepackage{slide_helper}

\usepackage[customcolors, beamer]{hf-tikz}
\tikzset{style green/.style={
    set fill color=green!60!lime!50,
    set border color=green!60!lime!50,
  },
  style cyan/.style={
    set fill color=cyan!50!blue!40,
    set border color=cyan!50!blue!40,
  },
  style orange/.style={
    set fill color=orange!50!red!60,
    set border color=red!80,
  },
  hor/.style={
    above left offset={-0.0,0.35},
    below right offset={0.03,-0.15},
    #1
  },
  ver/.style={
    above left offset={-0.0,0.35},
    below right offset={0.05,-0.15},
    #1
  },
  ver2/.style={
    above left offset={-0.15,0.35},
    below right offset={0.15,-0.15},
    #1
  }
}


\title[MATH 1080 - Section 10.6]{Determinants and Cramer's Rule}

\begin{document}

\begin{frame}
  \titlepage
\end{frame}

\begin{frame}
\begin{block}{Determinant of a $2 \by 2$ Matrix}
The \textbf{determinant of a $2 \by 2$ matrix} is defined to be:
\begin{equation*}
\abs{\mat{A}} = 
\begin{vmatrix}
a & b \\
c & d
\end{vmatrix}
=ad-bc
\end{equation*}
\end{block}\pause
\begin{example}
\begin{equation*}
\begin{vmatrix}
\+3 & \+8\\
\+5 & -1
\end{vmatrix}\pause
=3\cdot (-1) - 8\cdot 5\pause
=43
\end{equation*}
\end{example}\pause
\begin{example}
\begin{equation*}
\begin{vmatrix}
\+3 &  -2\\
\+6 & \+1
\end{vmatrix}\pause
=3\cdot (1) - (-2)\cdot 6\pause
=15
\end{equation*}
\end{example}
\end{frame}

\begin{frame}
\begin{block}{Minors of a Matrix}
For very element $a_{ij}$ of a $n \by n$ matrix $\mat{A}$, the \textbf{minor} $\mat{M_{ij}}$ is an $(n-1) \by (n-1)$ matrix obtained by deleting the $i$th row and the $j$th column of $\mat{A}$.
\end{block}\pause
\begin{example}
\begin{equation*}
\mat{A}=
\begin{bmatrix}
	\tikzmarkin<3>[hor=style green]{row-x}\+5 & \tikzmarkin<3>[ver=style green]{col-x}\+4 &  -3\tikzmarkend{row-x}\\
	\+2 &  -8 & \+1\\
	\+9 & \+3\tikzmarkend{col-x} & \+0\\
\end{bmatrix}
\qquad
\mat{M_{12}}=\pause
\begin{bmatrix}
		2 & 1\\
		9 & 0\\
\end{bmatrix}
\end{equation*}
\end{example}\pause
\begin{block}{Cofactors of a Matrix}
For every element $a_{ij}$ of a $n \by n$ matrix $\mat{A}$, the \textbf{cofactor} of $a_{ij}$ is the scalar
\begin{equation*}
C_{ij}={(-1)}^{(i+j)}\abs{\mat{M_{ij}}}
\end{equation*}
\end{block}
\end{frame}

\begin{frame}
\begin{block}{Determinants of a $n \by n$ Matrix}
For a $n \by n$ matrix $\mat{A}$, choose any row or column.
\begin{description}
\item<2->[\textbf{Expansion by the $\boldsymbol{i}$th row:}]
\begin{equation*}
\abs{\mat{A}} = \sum_{j=1}^{n} a_{ij}C_{ij}
              = \sum_{j=1}^{n} a_{ij}{(-1)}^{(i+j)}\abs{\mat{M_{ij}}}
\end{equation*}
\item<3->[\textbf{Expansion by the $\boldsymbol{j}$th column:}]
\begin{equation*}
\abs{\mat{A}} = \sum_{i=1}^{n} a_{ij}C_{ij}
              = \sum_{i=1}^{n} a_{ij}{(-1)}^{(i+j)}\abs{\mat{M_{ij}}}
\end{equation*}
\end{description}
\onslide<4->
I recommend you always expand across the first row.
\end{block}
\end{frame}

\begin{frame}
\begin{example}
Compute the determinant
\begin{equation*}
\begin{vmatrix}
\+3 & \+1 &  -1 \\
\+2 & \+1 & \+3 \\
\+0 & \+1 & \+2
\end{vmatrix}
\end{equation*}\pause
Expanding across the first row gives:
\begin{equation*}
\begin{split}
\begin{vmatrix}[rrr]
\tikzmarkin<2-4>[hor=style green]{row-a}\tikzmarkin<2>[ver=style green]{col-a}\tikzmarkin<2>[hor=style orange]{end-a}\+3\tikzmarkend{end-a} & \tikzmarkin<3>[ver=style green]{col-b}\tikzmarkin<3>[hor=style orange]{end-b} \+1 \tikzmarkend{end-b}&  \tikzmarkin<4>[ver=style green]{col-c}\tikzmarkin<4>[hor=style orange]{end-c}-1\tikzmarkend{end-c} \tikzmarkend{row-a}\\
\+2 & \+1 & \+3 \\
\+0\tikzmarkend{col-a} & \+1 \tikzmarkend{col-b} & \+2 \tikzmarkend{col-c}
\end{vmatrix}
&= 
\visible<2->{\tikzmarkin<2>[hor=style cyan]{end-aaa} + \tikzmarkend{end-aaa}\  \tikzmarkin<2>[hor=style orange]{end-aa} (+3)\tikzmarkend{end-aa}\begin{vmatrix}1&3\\1&2\end{vmatrix}}
\visible<3->{\tikzmarkin<3>[hor=style cyan]{end-bbb} - \tikzmarkend{end-bbb}\  \tikzmarkin<3>[hor=style orange]{end-bb} (+1) \tikzmarkend{end-bb} \begin{vmatrix}2&3\\0&2\end{vmatrix}} 
\visible<4->{\tikzmarkin<4>[hor=style cyan]{end-ccc} + \tikzmarkend{end-ccc}\ \tikzmarkin<4>[hor=style orange]{end-cc} (-1)\tikzmarkend{end-cc} \begin{vmatrix}2&1\\0&1\end{vmatrix}}\\
\visible<5->{&= 3(1\cdot 2-3\cdot 1) - (2\cdot 2 - 3\cdot 0) - (2\cdot 1 - 1\cdot 0)}\\
\visible<6->{&= -3 - 4 - 2}\\
\visible<7->{&= -9}
\end{split}
\end{equation*}
\end{example}
\end{frame}

\begin{frame}
\begin{example}
Compute the determinant
\begin{equation*}
\begin{vmatrix}
\+3 & \+0 &  -1 \\
\+4 & \+6 & \+2 \\
\+8 &  -2 & \+3
\end{vmatrix}
\end{equation*}\pause
Expanding across the first row gives:
\begin{equation*}
\begin{split}
\begin{vmatrix}[rrr]
\tikzmarkin<2-4>[hor=style green]{row-a-ex2}\tikzmarkin<2>[ver=style green]{col-a-ex2}\tikzmarkin<2>[hor=style orange]{end-a-ex2}\+3\tikzmarkend{end-a-ex2} & \tikzmarkin<3>[ver=style green]{col-b-ex2}\tikzmarkin<3>[hor=style orange]{end-b-ex2} \+0 \tikzmarkend{end-b-ex2}&  \tikzmarkin<4>[ver=style green]{col-c-ex2}\tikzmarkin<4>[hor=style orange]{end-c-ex2}-1\tikzmarkend{end-c-ex2} \tikzmarkend{row-a-ex2}\\
\+4 & \+6 & \+2 \\
\+8\tikzmarkend{col-a-ex2} & -2 \tikzmarkend{col-b-ex2} & \+3 \tikzmarkend{col-c-ex2}
\end{vmatrix}
&= 
\visible<2->{\tikzmarkin<2>[hor=style cyan]{end-aaa-ex2} + \tikzmarkend{end-aaa-ex2}\  \tikzmarkin<2>[hor=style orange]{end-aa-ex2} (+3)\tikzmarkend{end-aa-ex2}\begin{vmatrix}\+6&\+2\\-2&\+3\end{vmatrix}}
\visible<3->{\tikzmarkin<3>[hor=style cyan]{end-bbb-ex2} - \tikzmarkend{end-bbb-ex2}\  \tikzmarkin<3>[hor=style orange]{end-bb-ex2} (+0) \tikzmarkend{end-bb-ex2} \begin{vmatrix}4&2\\8&3\end{vmatrix}} 
\visible<4->{\tikzmarkin<4>[hor=style cyan]{end-ccc-ex2} + \tikzmarkend{end-ccc-ex2}\ \tikzmarkin<4>[hor=style orange]{end-cc-ex2} (-1)\tikzmarkend{end-cc-ex2} \begin{vmatrix}\+4&\+6\\\+8&-2\end{vmatrix}}\\
\visible<5->{&= 3(6\cdot 3-2\cdot (-2)) + 0(4\cdot 3-2\cdot 8) - (4\cdot (-2) - 6\cdot 8)}\\
\visible<6->{&= 66 + 0 + 56 }\\
\visible<7->{&= 122}
\end{split}
\end{equation*}
\end{example}
\end{frame}

\begin{frame}
\begin{block}{Properties of Determinants}
\begin{itemize}[<+- | alert@+>]
\item If any row (or any column) of $\mat{A}$ contains all zeros, then $\abs{\mat{A}}=0$
\item If any two rows (or any two columns) of $\mat{A}$ are equal, then $\abs{\mat{A}}=0$
\item If $\mat{A}$ is an diagonal, upper triangular, or lower triangular matrix, the determinant is the product of the diagonal elements.
\item The value of $\abs{\mat{A}}$ changes sign if any two rows (or columns) are interchanged.
\item If any row (or any column) of $\abs{\mat{A}}$ is multiplied by a nonzero number $k$, the value of $\abs{\mat{A}}$ is also changed by a factor of $k$.
\item If the entries of any row (or any column) of $\abs{\mat{A}}$ are multiplied by a nonzero number $k$ and the result is added to another row, the value of $\abs{\mat{A}}$ remains unchanged.
\end{itemize}
\end{block}

\onslide<+->
\begin{block}{Invertibility Criterion}
If $\mat{A}$ is a square matrix, then $A$ has an inverse if and only if $\abs{\mat{A}}\neq 0$.
\end{block}
\end{frame}

\begin{frame}
\begin{block}{Cramer's Rule}
Consider the system:
\begin{center}
\begin{tabu} to 4cm{*5{X[$r]}}
ax & + & by & = & s\\
cx & + & dy & = & t
\end{tabu}
\end{center}\pause
We then have the following three determinants:
\begin{equation*}
\abs{\mat{D}}=
\begin{vmatrix}
a & b \\
c & d \\
\end{vmatrix}\qquad
\abs{\mat{D_x}}=
\begin{vmatrix}
s & b \\
t & d \\
\end{vmatrix}\qquad
\abs{\mat{D_y}}=
\begin{vmatrix}
a & s \\
c & t \\
\end{vmatrix}
\end{equation*}\pause
The solutions is:

\vspace{-3mm}
\begin{equation*}
x=\dfrac{\abs{\mat{D_x}}}{\abs{\mat{D}}}=\dfrac{%
\begin{vmatrix}
s & b \\
t & d \\
\end{vmatrix}}{%
\begin{vmatrix}
a & b \\
c & d \\
\end{vmatrix}}
\qquad
y=\dfrac{\abs{\mat{D_y}}}{\abs{\mat{D}}}=\dfrac{%
\begin{vmatrix}
a & s \\
c & t \\
\end{vmatrix}}{%
\begin{vmatrix}
a & b \\
c & d \\
\end{vmatrix}}
\end{equation*}
\end{block}\pause
\begin{block}{Note}
This method can be extended to any size system of equations.
\end{block}
\end{frame}

\begin{frame}
\begin{example}
Consider the system
\begin{center}
\begin{tabu} to 4cm{*5{X[$r]}}
x & + & 2y & = & 5\\
2x & + & 3y & = & 8
\end{tabu}
\end{center}\pause
Let us solve this system using Cramer's Rule.

This means we need to calculate the following determinants.
\begin{center}
\begin{tabu}{*3{X[$c]}}
\abs{\mat{D}}=
\begin{vmatrix}[rr]
1 & 2 \\
2 & 3
\end{vmatrix}\visible<3->{=-1}
&
\abs{\mat{D_x}}=
\begin{vmatrix}[rr]
5 & 2 \\
8 & 3
\end{vmatrix}\visible<4->{=-1}
&
\abs{\mat{D_y}}=
\begin{vmatrix}[rr]
1 & 5 \\
2 & 8
\end{vmatrix}\visible<5->{=-2}\\
\end{tabu}
\end{center}
\onslide<6->
We can now find $x$ \visible<9->{and $y$}
\begin{equation*}
\visible<6->{x=\dfrac{\abs{\mat{D_x}}}{\abs{\mat{D}}}\visible<7->{=\dfrac{-1}{-1}\visible<8->{=1}}}
\qquad
\visible<9->{y=\dfrac{\abs{\mat{D_y}}}{\abs{\mat{D}}}\visible<10->{=\dfrac{-2}{-1}\visible<11->{=2}}}
\end{equation*}
\end{example}
\end{frame}

\begin{frame}
\begin{example}
Consider the system
\begin{center}
\begin{tabu} to 4cm{*5{X[$r]}}
3x & - & 2y & = & 4\\
6x & + & y & = & 13
\end{tabu}
\end{center}\pause
Let us solve this system using Cramer's Rule.

This means we need to calculate the following determinants.
\begin{center}
\begin{tabu} to 12.3cm{*3{X[$c]}}
\abs{\mat{D}}=
\begin{vmatrix}[rr]
3 & -2 \\
6 & 1
\end{vmatrix}\visible<3->{=15}
&
\abs{\mat{D_x}}=
\begin{vmatrix}[rr]
4 & -2 \\
13 & 1
\end{vmatrix}\visible<4->{=\small 30}
&
\abs{\mat{D_y}}=
\begin{vmatrix}[rr]
3 & 4 \\
6 & 13
\end{vmatrix}\visible<5->{=15}\\
\end{tabu}
\end{center}
\onslide<6->
We can now find $x$ \visible<9->{and $y$}
\begin{equation*}
\visible<6->{x=\dfrac{\abs{\mat{D_x}}}{\abs{\mat{D}}}\visible<7->{=\dfrac{30}{15}\visible<8->{=2}}}
\qquad
\visible<9->{y=\dfrac{\abs{\mat{D_y}}}{\abs{\mat{D}}}\visible<10->{=\dfrac{15}{15}\visible<11->{=1}}}
\end{equation*}
\end{example}
\end{frame}
\end{document}
