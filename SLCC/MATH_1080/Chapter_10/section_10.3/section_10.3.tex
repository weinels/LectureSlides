\documentclass{beamer}

\usepackage[english]{babel}
\usepackage[utf8x]{inputenc}
\usepackage{slide_helper}

\title[MATH 1080 - Section 10.3]{Matrices and Systems of Linear Equations}

\begin{document}

\begin{frame}
  \titlepage
\end{frame}

\begin{frame}
\begin{block}{Matrices and System of Linear Equations}
A $m\times n$ \textbf{system of linear equations} is a set of $m$ equations in $n$ variables:

\vspace{-3mm}
\begin{center}
\begin{tabular}{ccccccccc}
$a_{11}x_1$&$+$&$a_{12}x_2$&$+$&$\dots$&$+$&$a_{1n}x_n$&$=$&$b_1$\\
$a_{21}x_1$&$+$&$a_{22}x_2$&$+$&$\dots$&$+$&$a_{2n}x_n$&$=$&$b_2$\\
$a_{31}x_1$&$+$&$a_{32}x_2$&$+$&$\dots$&$+$&$a_{3n}x_n$&$=$&$b_3$\\
$\vdots$&&$\vdots$&&&&$\vdots$&&$\vdots$\\
$a_{m1}x_1$&$+$&$a_{m2}x_2$&$+$&$\dots$&$+$&$a_{mn}x_n$&$=$&$b_m$\\
\end{tabular}
\end{center}
\end{block}\pause
\begin{block}{Matricies}
A \textbf{matrix} is a rectangular array of numbers

\vspace{-2mm}
\begin{equation*}
\begin{bmatrix}
a_{11} & a_{12} & \cdots & a_{1j} & \cdots & a_{1n} \\
a_{21} & a_{22} & \cdots & a_{2j} & \cdots & a_{2n} \\
\vdots & \vdots &        & \vdots &        & \vdots \\
a_{i1} & a_{i2} & \cdots & a_{ij} & \cdots & a_{in} \\
\vdots & \vdots &        & \vdots &        & \vdots \\
a_{m1} & a_{m2} & \cdots & a_{mj} & \cdots & a_{mn} \\
\end{bmatrix}
\end{equation*}
\end{block}
\end{frame}

\begin{frame}
\begin{block}{}
There are two primary ways of writing a linear systems using matrices.
\end{block}\pause
\begin{block}{An Augmented Matrix}
\begin{equation*}
\begin{bmatrix}[cccc|c]
a_{11}&a_{12}&\cdots&a_{1n}&b_1\\
\vdots&\vdots&\ddots&\vdots&\\
a_{m1}&a_{m2}&\cdots&a_{mn}&b_m
\end{bmatrix}
\end{equation*}
\end{block}\pause
\begin{block}{A Matrix Equation (We will look at these in section 10.4)}
As the matrix equation $A\vect{x}=\vect{b}$, where:
\begin{equation*}
\underbrace{\begin{bmatrix}
a_{11}&a_{12}&\cdots&a_{1n}\\
\vdots&\vdots&\ddots&\vdots\\
a_{m1}&a_{m2}&\cdots&a_{mn}
\end{bmatrix}}_A
\underbrace{\begin{bmatrix}
x_1\\\vdots\\x_n
\end{bmatrix}}_\vect{x}
=
\underbrace{\begin{bmatrix}
b_1\\\vdots\\b_m
\end{bmatrix}}_\vect{b}
\end{equation*}
\end{block}
\end{frame}

\begin{frame}
\begin{example}
Consider the system of linear equations:

\vspace{-2mm}
\begin{equation*}
\begin{aligned}
3x -4y &= -6 \\
2x -3y &= -5 \\
\end{aligned}
\end{equation*}\pause
The augmented matrix for this system is:

\vspace{-2mm}
\begin{equation*}
\begin{bmatrix}[rr|r]
3 & -4 & -6 \\
2 & -3 & -5 \\
\end{bmatrix}
\end{equation*}
\end{example}\pause
\begin{example}
Consider the augmented matrix:

\vspace{-2mm}
\begin{equation*}
\begin{bmatrix}[rr|r]
5 & 2 & 13 \\
-3 & 1 & -10 \\
\end{bmatrix}
\end{equation*}\pause
This matrix corresponds to the system of linear equations:

\vspace{-2mm}
\begin{equation*}
\begin{aligned}
\+5x + 2y           &= \+13 \\
 -3x + \phantom{2}y &=  -10 \\
\end{aligned}
\end{equation*}
\end{example}
\end{frame}

\begin{frame}
\begin{example}
Consider the system of linear equations:
\begin{equation*}
\begin{aligned}
2x           - \phantom{2}y + z &= 0 \\
\phantom{2}x + \phantom{2}z - 1 &= 0 \\
\phantom{2}x + 2y           - 8 &= 0 \\
\end{aligned}
\end{equation*}\pause
{\small The system must be in standard form before we can write the augmented matrix.}\pause
\begin{equation*}
\begin{aligned}
2x           - \phantom{2}y + \phantom{0}z &= 0 \\\pause
\phantom{2}x + 0y           + \phantom{0}z &= 1 \\\pause
\phantom{2}x + 2y           + 0z           &= 8 \\
\end{aligned}
\end{equation*}\pause
Thus, the augmented matrix is:
\begin{equation*}
\begin{bmatrix}[rrr|r]
2 & -1 & 1 & 0 \\
1 &  0 & 1 & 1 \\
1 &  2 & 0 & 8 \\
\end{bmatrix}
\end{equation*}
\end{example}
\end{frame}

\begin{frame}
\begin{block}{Notation} 
\begin{itemize}
\item<+-> $r_i$ denotes row $i$ \emph{before} the row operation is applied
\item<.-> $R_i$ denotes row $i$ \emph{after} the row operation is applied
\end{itemize}
\end{block}
\onslide<+->
\begin{block}{Elementary Row Operations} 
\begin{itemize}
\item<.- | alert@.> Swap row $i$ and row $j$:
\begin{equation*}
R_i\leftrightarrow R_j \quad\left(\text{or}\ R_i=r_j\text{, } R_j=r_i\right)
\end{equation*}
\item<+- | alert@+> Multiply row $i$ by a nonzero constant:
\begin{equation*}
R_i=c\cdot r_i
\end{equation*}
\item<+- | alert@+> Add row $j$ to row $i$ (leaving row $j$ unchanged):
\begin{equation*}
R_i=r_i+r_j
\end{equation*}
\end{itemize}
\end{block}
\end{frame}

\begin{frame}
\begin{example}
Let us apply the row operation $R_2=-3r_1+r_2$ to the matrix

\vspace{-2mm}
\begin{equation*}
\begin{bmatrix}[rr|r]
1 & -2 & 2 \\
3 & -5 & 9 \\
\end{bmatrix}
\end{equation*}
\onslide<2->

\vspace{-4mm}
We want to work one column at a time:
\begin{equation*}
\begin{bmatrix}[rr|r]
\phantom{(}1 & -2\phantom{)} & 2 \\  							% chktex 9 chktex 10
\visible<3-4>{-3(1)}\visible<4>{+}\only<1-4>{\visible<4>{3}}\only<5->{0} & 	% chktex 9 chktex 10
\visible<6-7>{(-3)(-2)}\visible<7>{+(-}\only<1-7>{\visible<7>{5}}\only<8->{1}\visible<7>{)} & 					% chktex 9 chktex 10
\visible<9-10>{-3(2)}\visible<10>{+}\only<1-10>{\visible<10>{9}}\only<11->{3} \\ 							% chktex 9 chktex 10
\end{bmatrix}
\end{equation*}
\end{example}
\onslide<12->
\begin{example}
Let us apply the row operation $R_1=2r_2+r_1$ to the matrix

\vspace{-2mm}
\begin{equation*}
\begin{bmatrix}[rr|r]
2 & -2 & 1 \\
3 & 1 & 4 \\
\end{bmatrix}
\end{equation*}
\onslide<13->

\vspace{-4mm}
We want to work one column at a time:
\begin{equation*}
\begin{bmatrix}[rr|r]
\visible<14-15>{2(3)}\visible<15>{+}\only<12-15>{\visible<15>{2}}\only<16->{8} & % chktex 9 chktex 10
\visible<17-18>{2(1)}\visible<18>{+(-}\only<12-18>{\visible<18>{2}}\only<19->{0}\visible<18>{)} & % chktex 9 chktex 10
\visible<20-21>{2(4)}\visible<21>{+}\only<12-21>{\visible<21>{1}}\only<22->{9} \\ % chktex 9 chktex 10
\phantom{(}3 & 1\phantom{)} & 4 \\ % chktex 9 chktex 10
\end{bmatrix}
\end{equation*}
\end{example}
\end{frame}

\begin{frame}
\begin{block}{Gaussian Elimination}
\small
Use row operations until in \textbf{Row Echelon Form}\@:
\begin{equation*}
\begin{bmatrix}[ccccc|c]
1         & c_{12} & c_{13} & \cdots & c_{1n} & d_1\\
0         & 1         & c_{23} & \cdots & c_{2n} & d_2\\
0         & 0         & 1         & \cdots & c_{3n} & d_2\\
\vdots & \vdots  & \vdots &\ddots & \vdots & \vdots\\
0         & 0         & 0         & \cdots  & 1         & d_m
\end{bmatrix}
\end{equation*}\pause
Then back solve the system:
\begin{equation*}
\begin{aligned}
x_1 + c_{12}x_2+c_{13}x_3+\cdots+c_{1n}x_n  &= d_1\\
                    x_2+c_{23}x_3+\cdots+c_{2n}x_n  &= d_2\\
                                                                          &\vdots\\
                                                                   x_n &= d_m          
\end{aligned}
\end{equation*}
\end{block}
\end{frame}

\begin{frame}
\begin{example}
\begin{overprint}
\onslide<1-2>
Consider the system
\begin{center}
\begin{tabular}{rcrcrcr}
$x$&$+$&$y$&$+$&$z$&$=$&$3$\\
$2x$&$-$&$3y$&$-$&$z$&$=$&$-8$\\
$-x$&$+$&$2y$&$+$&$2z$&$=$&$3$
\end{tabular}
\end{center}
\visible<2>{
We can write this as the augmented matrix:
\begin{equation*}
\begin{bmatrix}[rrr|r]
 1 &  1 &  1 &  3\\
 2 & -3 & -1 & -8\\
-1 &  2 &  2 &  3
\end{bmatrix}
\end{equation*}
We now want to use row operations to transform this augmented matrix into Row Echelon Form.}
\onslide<3-5>%
\LARGE
\begin{equation*}
	\begin{aligned}
		&	\begin{bmatrix}[rrr|r]
				 \+1 & \+1 & \+1 & \+3\\
				 \+2 &  -3 &  -1 &  -8\\
				  -1 & \+2 & \+2 & \+3
			\end{bmatrix}
			\visible<4-5>{\begin{aligned}
				& \phantom{R_1}\\
				& R_2=r_2+2r_3\\
				& \phantom{R_2}
			\end{aligned}}\\
		\visible<5>{\Rightarrow
		&	\begin{bmatrix}[rrr|r]
				 \+1 & \+1 & \+1 & \+3\\
				 \+0 & \+1 & \+3 &  -2\\
				  -1 & \+2 & \+2 & \+3
			\end{bmatrix}}
	\end{aligned}
\end{equation*}
\onslide<6-8>%
\LARGE
\begin{equation*}
	\begin{aligned}
		&	\begin{bmatrix}[rrr|r]
				 \+1 & \+1 & \+1 & \+3\\
				 \+0 & \+1 & \+3 & -2\\
				   -1 & \+2 & \+2 & \+3
			\end{bmatrix}
			\visible<7-8>{\begin{aligned}
				& \phantom{R_1}\\
				& \phantom{R_2}\\
				& R_3=r_1+r_3
			\end{aligned}}\\
		\visible<8>{\Rightarrow
		&	\begin{bmatrix}[rrr|r]
				 \+1 & \+1 & \+1 & \+3\\
				 \+0 & \+1 & \+3 &  -2\\
				 \+0 & \+3 & \+3 & \+6
			\end{bmatrix}}
	\end{aligned}
\end{equation*}
\onslide<9-11>%
\LARGE
\begin{equation*}
	\begin{aligned}
		&	\begin{bmatrix}[rrr|r]
				 \+1 & \+1 & \+1 & \+\phantom{1}3\\
				 \+0 & \+1 & \+3 &  -2\\
				 \+0 & \+3 & \+3 & \+6
			\end{bmatrix}
			\visible<10-11>{\begin{aligned}
				& \phantom{R_1}\\
				& \phantom{R_2}\\
				& R_3=r_3-3r_2
			\end{aligned}}\\
		\visible<11>{\Rightarrow
		&	\begin{bmatrix}[rrr|r]
				 \+1 & \+1 & \+1 & \+3\\
				 \+0 & \+1 & \+3 &  -2\\
				 \+0 & \+0 &  -6 & \+12
			\end{bmatrix}}
	\end{aligned}
\end{equation*}
\onslide<12-14>%
\LARGE
\begin{equation*}
	\begin{aligned}
		&	\begin{bmatrix}[rrr|r]
				 \+1 & \+1 & \+1 & \+3\\
				 \+0 & \+1 & \+3 &  -2\\
				 \+0 & \+0 &  -6 & \+12
			\end{bmatrix}
			\visible<13-14>{\begin{aligned}
				& \phantom{R}\\
				& \phantom{R}\\
				& R_3=-\tfrac{1}{6}r_3
			\end{aligned}}\\
		\visible<14>{\Rightarrow
		&	\begin{bmatrix}[rrr|r]
				 \+1 & \+1 & \+1 & \+\phantom{1}3\\
				 \+0 & \+1 & \+3 &  -2\\
				 \+0 & \+0 &  1   &  -2
			\end{bmatrix}}
	\end{aligned}
\end{equation*}
\onslide<15->
Now, back solve the system
\begin{center}
\begin{tabular}{rcrcrcr}
$x$&$+$&$y$&$+$&$  z$&$=$&$3$\\
      &      &$y$&$+$&$3z$&$=$&$-2$\\
      &      &      &      &$  z$&$=$&$-2$
\end{tabular}
\end{center}
\visible<16->{Start with the third equation: $z=-2$}

\visible<17->{Plug it into the second equation and solve for $y$:
\begin{equation*}
y+3(-2)=-2\quad\Rightarrow\quad y=4
\end{equation*}}
\visible<18->{Plug both into the first equation and solve for $x$:
\begin{equation*}
x+(4)+(-2)=3\quad\Rightarrow\quad x=1
\end{equation*}}
\end{overprint}
\end{example}
\end{frame}

\begin{frame}
\begin{example}
\begin{overprint}
\onslide<1-2>
Consider the system
\begin{center}
\begin{tabular}{rcrcrcr}
$2x$ & $+$ & $2y$ &     &     & $=$ &  $6$ \\
 $x$ & $+$ &  $y$ & $+$ & $z$ & $=$ &  $1$ \\
$3x$ & $+$ & $4y$ & $-$ & $z$ & $=$ &  $13$
\end{tabular}
\end{center}
\visible<2>{
We can write this as the augmented matrix:
\begin{equation*}
\begin{bmatrix}[rrr|r]
 2 &  2 &  0 &  6\\
 1 &  1 &  1 &  1\\
 \+3 &  \+4 & -1 &  13
\end{bmatrix}
\end{equation*}
We now want to use row operations to transform this augmented matrix into Row Echelon Form.}
\onslide<3-5>%
\LARGE
\begin{equation*}
	\begin{aligned}
		&	\begin{bmatrix}[rrr|r]
				\+2 &  \+2 &  \+0 &  6\\
 				1 &  1 &  1 &  1\\
				3 &  4 &  -1 &  13
			\end{bmatrix}
			\visible<4-5>{\begin{aligned}
				& R_1=r_2\\
				& R_2=r_1\\
				& \phantom{R_2}
			\end{aligned}}\\
		\visible<5>{\Rightarrow
		&	\begin{bmatrix}[rrr|r]
				 \+1 &  \+1 &  \+1 &  1\\
				 2 &  2 &  0 &  6\\
				 3 &  4 &  -1 &  13
			\end{bmatrix}}
	\end{aligned}
\end{equation*}
\onslide<6-8>%
\LARGE
\begin{equation*}
	\begin{aligned}
		&	\begin{bmatrix}[rrr|r]
				 \+1 &  \+1 &  \+1 &  1\\
				 2 &  2 &  0 &  6\\
				 3 &  4 &  -1 &  13
			\end{bmatrix}
			\visible<7-8>{\begin{aligned}
				& \phantom{R_1}\\
				& R_2=-2r_1+r_2\\
				& \phantom{R_3}
			\end{aligned}}\\
		\visible<8>{\Rightarrow
		&	\begin{bmatrix}[rrr|r]
				 \+1 &  \+1 &  \+1 &  1\\
				 0 &  0 &  -2 &  4\\
				 3 &  4 &  -1 &  13
			\end{bmatrix}}
	\end{aligned}
\end{equation*}
\onslide<9-11>%
\LARGE
\begin{equation*}
	\begin{aligned}
		&	\begin{bmatrix}[rrr|r]
				 \+1 &  \+1 &  \+1 &  1\\
				 0 &  0 &  -2 &  4\\
				 3 &  4 &  -1 &  13
			\end{bmatrix}
			\visible<10-11>{\begin{aligned}
				& \phantom{R_1}\\
				& \phantom{R_2}\\
				& R_3=-3r_1+r_3
			\end{aligned}}\\
		\visible<11>{\Rightarrow
		&	\begin{bmatrix}[rrr|r]
				 \+1 &  \+1 &  \+1 &  1\\
				 0 &  0 &  -2 &  4\\
				 0 &  1 &  -4 &  10
			\end{bmatrix}}
	\end{aligned}
\end{equation*}
\onslide<12-14>%
\LARGE
\begin{equation*}
	\begin{aligned}
		&	\begin{bmatrix}[rrr|r]
				 \+1 &  \+1 &  \+1 &  1\\
				 0 &  0 &  -2 &  4\\
				 0 &  1 &  -4 &  10
			\end{bmatrix}
			\visible<13-14>{\begin{aligned}
				& \phantom{R_1}\\
				& R_2=r_3\\
				& R_3=r_2
			\end{aligned}}\\
		\visible<14>{\Rightarrow
		&	\begin{bmatrix}[rrr|r]
				 \+1 &  \+1 &  \+1 &  1\\
				 0 &  1 &  -4 &  10\\
				 0 &  0 &  -2 &  4
			\end{bmatrix}}
	\end{aligned}
\end{equation*}
\onslide<15-17>%
\LARGE
\begin{equation*}
	\begin{aligned}
		&	\begin{bmatrix}[rrr|r]
				 \+1 &  \+1 &  \+1 &  1\\
				 0 &  1 &  -4 &  10\\
				 0 &  0 &  -2 &  4
			\end{bmatrix}
			\visible<16-17>{\begin{aligned}
				& \phantom{R_1}\\
				& \phantom{R_2}\\
				& R_3=-\tfrac{1}{2}r_3
			\end{aligned}}\\
		\visible<17>{\Rightarrow
		&	\begin{bmatrix}[rrr|r]
				 \+1 &  \+1 &  \+1 &  1\\
				 0 &  1 &  -4 &  10\\
				 0 &  0 &  1 &  -2
			\end{bmatrix}}
	\end{aligned}
\end{equation*}
\onslide<18->
Now, back solve the system
\begin{center}
\begin{tabular}{rcrcrcr}
$x$&$+$&$y$&$+$&$  z$&$=$&$1$\\
      &      &$y$&$-$&$4z$&$=$&$10$\\
      &      &      &      &$  z$&$=$&$-2$
\end{tabular}
\end{center}
\visible<19->{Start with the third equation: $z=-2$}

\visible<20->{Plug it into the second equation and solve for $y$:
\begin{equation*}
y-4(-2)=10\quad\Rightarrow\quad y=2
\end{equation*}}
\visible<21->{Plug both into the first equation and solve for $x$:
\begin{equation*}
x+(2)+(-2)=1\quad\Rightarrow\quad x=1
\end{equation*}}
\end{overprint}
\end{example}
\end{frame}

\begin{frame}
\begin{example}
\begin{overprint}
\onslide<1-2>
Consider the system
\begin{center}
\begin{tabular}{rcrcrcr}
  $6x$ & $-$ &  $y$ & $-$ &  $z$ & $=$ &  $4$ \\
$-12x$ & $+$ & $2y$ & $+$ & $2z$ & $=$ & $-8$ \\
  $5x$ & $+$ &  $y$ & $-$ &  $z$ & $=$ &  $3$
\end{tabular}
\end{center}
\visible<2>{
We can write this as the augmented matrix:
\begin{equation*}
\begin{bmatrix}[rrr|r]
  6 &   -1 &  -1 &  4\\
-12 &    2 &   2 & -8\\
  5 &  \+1 &  -1 &  3
\end{bmatrix}
\end{equation*}
We now want to use row operations to transform this augmented matrix into Row Echelon Form.}
\onslide<3-5>%
\LARGE
\begin{equation*}
	\begin{aligned}
		&	\begin{bmatrix}[rrr|r]
				  6 &   -1 &  -1 &  4\\
				-12 &    2 &   2 & -8\\
				  5 &  \+1 &  -1 &  3
			\end{bmatrix}
			\visible<4-5>{\begin{aligned}
				& R_1=-r_3+r_1\\
				& \phantom{R_2}\\
				& \phantom{R_3}
			\end{aligned}}\\
		\visible<5>{\Rightarrow
		&	\begin{bmatrix}[rrr|r]
				  1 &   -2 &   0 &  1\\
				-12 &    2 &   2 & -8\\
				  5 &  \+1 &  -1 &  3
			\end{bmatrix}}
	\end{aligned}
\end{equation*}
\onslide<6-8>%
\LARGE
\begin{equation*}
	\begin{aligned}
		&	\begin{bmatrix}[rrr|r]
				  1 &   -2 &   0 &  1\\
				-12 &    2 &   2 & -8\\
				  5 &  \+1 &  -1 &  3
			\end{bmatrix}
			\visible<7-8>{\begin{aligned}
				& \phantom{R_1}\\
				& R_2=12r_1+r_2\\
				& \phantom{R_3}
			\end{aligned}}\\
		\visible<8>{\Rightarrow
		&	\begin{bmatrix}[rrr|r]
				\+1 &   -2 &   0 &\+1\\
				  0 &  -22 &   2 &  4\\
				  5 &  \+1 &  -1 &  3
			\end{bmatrix}}
	\end{aligned}
\end{equation*}
\onslide<9-11>%
\LARGE
\begin{equation*}
	\begin{aligned}
		&	\begin{bmatrix}[rrr|r]
				\+1 &   -2 &   0 &\+1\\
				  0 &  -22 &   2 &  4\\
				  5 &  \+1 &  -1 &  3
			\end{bmatrix}
			\visible<10-11>{\begin{aligned}
				& \phantom{R_1}\\
				& \phantom{R_2}\\
				& R_3=-5r_1+r_3
			\end{aligned}}\\
		\visible<11>{\Rightarrow
		&	\begin{bmatrix}[rrr|r]
				\+1 &   -2 &   0 &\+1\\
				  0 &  -22 &   2 &  4\\
				  0 &   11 &  -1 & -2
			\end{bmatrix}}
	\end{aligned}
\end{equation*}
\onslide<12-14>%
\LARGE
\begin{equation*}
	\begin{aligned}
		&	\begin{bmatrix}[rrr|r]
				\+1 &   -2 &   0 &\+1\\
				  0 &  -22 &   2 &  4\\
				  0 &   11 &  -1 & -2
			\end{bmatrix}
			\visible<13-14>{\begin{aligned}
				& \phantom{R_1}\\
				& R_2=-\tfrac{1}{22}r_2\\
				& \phantom{R_3}
			\end{aligned}}\\
		\visible<14>{\Rightarrow
		&	\begin{bmatrix}[rrr|r]
				\+1 &   -2 &   0 &\+1\\
				  0 &    1 &   -\tfrac{1}{11} &  -\tfrac{2}{11}\\
				  0 &   11 &  -1 & -2
			\end{bmatrix}}
	\end{aligned}
\end{equation*}
\onslide<15-17>%
\LARGE
\begin{equation*}
	\begin{aligned}
		&	\begin{bmatrix}[rrr|r]
				 \+1 &   -2 &   0 &\+1\\
				  0 &    1 &   -\tfrac{1}{11} &  -\tfrac{2}{11}\\
				  0 &   11 &  -1 & -2
			\end{bmatrix}
			\visible<16-17>{\begin{aligned}
				& \phantom{R_1}\\
				& \phantom{R_2}\\
				& R_3=-11r_2+r_3
			\end{aligned}}\\
		\visible<17>{\Rightarrow
		&	\begin{bmatrix}[rrr|r]
				 \+1 &   -2 &   0 &\+1\\
				  0 &    1 &   -\tfrac{1}{11} &  -\tfrac{2}{11}\\
				  0 &    0 &   0 & 0
			\end{bmatrix}}
	\end{aligned}
\end{equation*}
\onslide<18->
Now, back solve the system
\begin{center}
\begin{tabular}{rcrcrcr}
$x$&$-$&$2y$&&&$=$&$1$\\
      &      &$y$&$-$&$\tfrac{1}{11}z$&$=$&$-\tfrac{2}{11}$\\
      &      &      &      &$  0$&$=$&$0$
\end{tabular}
\end{center}
\visible<19->{This system of equations has an infinite number of solutions.}
\visible<20->{%

\vspace{2mm}
For any choice of $z$, we can calculate values for $x$ and $y$ that work:
\begin{equation*}
\begin{aligned}
x&=\tfrac{2}{11}z+\tfrac{7}{11}\\
y&=\tfrac{1}{11}z-\tfrac{2}{11}
\end{aligned}
\end{equation*}}
\end{overprint}
\end{example}
\end{frame}

\begin{frame}
\begin{example}
\begin{overprint}
\onslide<1-2>
Consider the system
\begin{center}
\begin{tabular}{rcrcrcr}
  $x$ & $+$ &  $y$ & $+$ &  $z$ & $=$ & $6$ \\
 $2x$ & $-$ &  $y$ & $-$ &  $z$ & $=$ & $3$ \\
  $x$ & $+$ & $2y$ & $+$ & $2z$ & $=$ & $0$
\end{tabular}
\end{center}
\visible<2>{
We can write this as the augmented matrix:
\begin{equation*}
\begin{bmatrix}[rrr|r]
1 & 1 &  1 &  6\\
2 & -1 &  -1 & 3\\
1 & 2 &  2 &  0
\end{bmatrix}
\end{equation*}
We now want to use row operations to transform this augmented matrix into Row Echelon Form.}
\onslide<3-5>%
\LARGE
\begin{equation*}
	\begin{aligned}
		&	\begin{bmatrix}[rrr|r]
				\+1 & \+1 &  \+1 &  \+6\\
				2 & -1 &  -1 & 3\\
				1 & 2 &  2 &  0
			\end{bmatrix}
			\visible<4-5>{\begin{aligned}
				& \phantom{R_1}\\
				& R_2=-2r_1+r_2\\
				& \phantom{R_3}
			\end{aligned}}\\
		\visible<5>{\Rightarrow
		&	\begin{bmatrix}[rrr|r]
				\+1 & \+1 &  \+1 &  \+6\\
				0 & -3 &  -3 & -9\\
				1 & 2 &  2 &  0
			\end{bmatrix}}
	\end{aligned}
\end{equation*}
\onslide<6-8>%
\LARGE
\begin{equation*}
	\begin{aligned}
		&	\begin{bmatrix}[rrr|r]
				\+1 & \+1 &  \+1 &  \+6\\
				0 & -3 &  -3 & -9\\
				1 & 2 &  2 &  0
			\end{bmatrix}
			\visible<7-8>{\begin{aligned}
				& \phantom{R_1}\\
				& \phantom{R_2}\\
				& R_3=-r_1+r_3
			\end{aligned}}\\
		\visible<8>{\Rightarrow
		&	\begin{bmatrix}[rrr|r]
				\+1 & \+1 &  \+1 &  \+6\\
				0 & -3 &  -3 & -9\\
				0 & 1 &  1 &  -6
			\end{bmatrix}}
	\end{aligned}
\end{equation*}
\onslide<9-11>%
\LARGE
\begin{equation*}
	\begin{aligned}
		&	\begin{bmatrix}[rrr|r]
				\+1 & \+1 &  \+1 &  \+6\\
				0 & -3 &  -3 & -9\\
				0 & 1 &  1 &  -6
			\end{bmatrix}
			\visible<10-11>{\begin{aligned}
				& \phantom{R_1}\\
				& R_2=r_3\\
				& R_3=r_2
			\end{aligned}}\\
		\visible<11>{\Rightarrow
		&	\begin{bmatrix}[rrr|r]
				\+1 & \+1 &  \+1 &  \+6\\
				0 & 1 &  1 &  -6\\
				0 & -3 &  -3 & -9
			\end{bmatrix}}
	\end{aligned}
\end{equation*}
\onslide<12-14>%
\LARGE
\begin{equation*}
	\begin{aligned}
		&	\begin{bmatrix}[rrr|r]
				\+1 & \+1 &  \+1 &  \+6\\
				0 & 1 &  1 &  -6\\
				0 & -3 &  -3 & -9
			\end{bmatrix}
			\visible<13-14>{\begin{aligned}
				& \phantom{R_1}\\
				& \phantom{R_2}\\
				& R_3=3r_2+r_3
			\end{aligned}}\\
		\visible<14>{\Rightarrow
		&	\begin{bmatrix}[rrr|r]
				\+1 & \+1 &  \+1 &  \+6\\
				0 & 1 &  1 &  -6\\
				0 & 0 &  0 & -27
			\end{bmatrix}}
	\end{aligned}
\end{equation*}
\onslide<15->
Now, back solve the system
\begin{center}
\begin{tabular}{rcrcrcr}
$x$&$+$&$y$&$+$&$z$&$=$&$6$\\
      &      &$y$&$+$&$z$&$=$&$-6$\\
      &      &      &      &$  0$&$=$&$-27$
\end{tabular}
\end{center}
\visible<16->{This system of equations has no solutions.}
\end{overprint}
\end{example}
\end{frame}

\begin{frame}
\begin{block}{Gauss-Jordan Elimination}
Use row operations until in \textbf{Reduced Row Echelon Form}\@:
\begin{equation*}
\begin{bmatrix}[ccccc|c]
1         & 0 &0 & \cdots & 0 & d_1\\
0         & 1         & 0 & \cdots & 0 & d_2\\
0         & 0         & 1         & \cdots & 0 & d_2\\
\vdots & \vdots  & \vdots &\ddots & \vdots & \vdots\\
0         & 0         & 0         & \cdots  & 1         & d_m
\end{bmatrix}
\end{equation*}\pause
Then the solution is:
\begin{equation*}
\left(d_1, d_2, \ldots, d_m\right)
\end{equation*}\pause
\end{block}
\begin{block}{Note}
Gaussian Elimination is often preferred when working by hand. Gauss-Jordan works better when programming a computer to solve a system of equations.
\end{block}
\end{frame}

\begin{frame}
\begin{block}{The Solutions of a Linear System of Equations}
During Gaussian or Gauss-Jordan Elimination:\pause
\begin{itemize}
\item<+->If a row is of the form
\begin{equation*}
\begin{bmatrix}[ccc|c]
0 & \cdots & 0 & k\neq 0
\end{bmatrix}
\end{equation*}
is encountered, then the system has \emph{no solutions}.
\item<+->
If a row is of the form
\begin{equation*}
\begin{bmatrix}[ccc|c]
0 & \cdots & 0 & 0
\end{bmatrix}
\end{equation*}
is encountered, then the system has \emph{infinitely many solutions}.
\end{itemize}
\onslide<+->
Some vocabulary:
\begin{itemize}
\item<.-> If a system has no solutions, it is called \textbf{inconsistent}.
\item<+-> If a system has at least one solution, it is called \textbf{consistent}.
\begin{itemize}
\item<+-> A system with exactly one solution is called \textbf{independent}.
\item<+-> A system with more than one solution is called \textbf{dependent}.
\end{itemize}
\end{itemize}
\end{block}
\end{frame}
\end{document}
