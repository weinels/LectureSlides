\documentclass{beamer}
\usepackage[utf8]{inputenc}
\usepackage[english]{babel}
\usepackage[T1]{fontenc}
\usepackage[inline]{asymptote}
\usepackage{slide_helper}
\usepackage{asy_helper}
\usepackage{subcaption}


\title[MATH 1220 - Section 9.5]{Linear Equations}

\begin{document}
\begin{frame}
\titlepage
\end{frame}

\begin{frame}
\begin{block}{Definition}
An equation $F(x_1, x_2, \ldots, x_n)=C$ is \textbf{linear} if it is of the form
\begin{equation*}
a_1 x_1+a_2 x_2+\cdots+a_n x_n=C
\end{equation*}
where $a_1, a_2, \ldots, a_n$ and $C$ are constants.

\vspace{2mm}
If $C=0$, the equation is said to be \textbf{homogeneous}.
\end{block}\pause

\begin{example}
Which of the following are linear equations?
\begin{equation*}
\begin{aligned}
4x - 3e^x &= 15 & \visible<3->{\text{No}} \\
4x - 2y + 3\sqrt{z} &= 12 & \visible<4->{\text{No}} \\
2x - 3y +4z + 3 &= w & \visible<5->{\text{Yes}}
\end{aligned}
\end{equation*}
\end{example}
\end{frame}

\begin{frame}
\begin{block}{Definition}
A differential equation $F(y, y^\prime, y^{\prime\prime},\ldots,y^{(n)})=f(t)$ is \textbf{linear} if it is of the form
\begin{equation*}
a_n(t)\dfrac{d^n y}{dt^n}+a_{n-1}(t)\dfrac{d^{n-1} y}{dt^{n-1}}+\cdots+a_1(t)\dfrac{dy}{dt}+a_0(t)y=f(t)
\end{equation*}
where all functions of $t$ are assumed to be defined over some common interval $I$.

\vspace{2mm}
If $f(t)=0$ over the interval $I$, the differential equation is said to be \textbf{homogeneous}.
\end{block}\pause

\begin{block}{First and Second Order Notation}
It is common to write first-order differential equations as

\vspace{-3mm}
\begin{equation*}
y^\prime+p(t)y=f(t)
\end{equation*}
and second-order differential equations as

\vspace{-3mm}
\begin{equation*}
y^{\prime\prime}+p(t)y^\prime+q(t)y=f(t)
\end{equation*}
\end{block}
\end{frame}

\begin{frame}
\begin{example}
Let us classify the following differential equations.

\vspace{-2mm}
\begin{center}
\begin{tabular}{ccccc}
Differential Equation & Order & Linear? & Homogeneous? & Coefficients \\\hline\\
\visible<+->{$y^{\prime}+ty= 1$} & \visible<+->{1} & \visible<+->{Yes} & \visible<+->{No} & \visible<+->{Variable} \\\\
\visible<+->{$y^{\prime\prime}+y y^\prime +y= t$} & \visible<+->{2} & \visible<+->{No & --- & ---}\\\\
\visible<+->{$y^{\prime\prime} + ty^\prime +y^2 = 0$} & \visible<+->{2} & \visible<+->{No & --- & ---}\\\\
\visible<+->{$y^{\prime\prime} +3 y^\prime + 2y = 0$} & \visible<+->{2} & \visible<+->{Yes}& \visible<+->{Yes} & \visible<+->{Constant}\\\\
\visible<+->{$y^{\prime\prime} + y = \sin[y]$} & \visible<+->{2} & \visible<+->{No & --- & ---}\\\\
\visible<+->{$y^{\prime\prime\prime} + 3y^\prime + y= \sin[t]$} & \visible<+->{3} & \visible<+->{Yes}& \visible<+->{No} & \visible<+->{Variable}
\end{tabular}
\end{center}
\end{example}
\end{frame}

\begin{frame}
\begin{block}{Notation}
We will use a \textbf{vector} notation to represent a whole set of variables:
\begin{description}
\item[Linear Algebraic Equations:] 
\begin{equation*}
\vect{x}=[x_1, x_2, \ldots, x_n]
\end{equation*}
\item[Linear Differential Equations:] 
\begin{equation*}
\vect{y}=[y^{(n)}, y^{(n-1)}, \ldots, y^\prime, y]
\end{equation*}
\end{description}
\end{block}\pause

\begin{block}{Definition}
A \textbf{linear operator} $L$ is an entire operation performed on a set of variables.
\begin{description}
\item[Linear Algebraic Equations:]
\begin{equation*}
L(\vect{x}) = a_1 x_1 + a_2 x_2 + \cdots + a_n x_n
\end{equation*}
\item[Linear Differential Equations:] 
\begin{equation*}
L(\vect{y})= a_n(t)\dfrac{d^n y}{dt^n}+a_{n-1}(t)\dfrac{d^{n-1} y}{dt^{n-1}}+\cdots+a_1(t)\dfrac{dy}{dt}+a_0(t)y
\end{equation*}
\end{description}
\end{block}
\end{frame}

\begin{frame}
\begin{example}
What is the linear operator for the following linear differential equations?
\begin{center}
\begin{tabular}{ccc}
$y^{\prime}+ty=1$ & \visible<2->{$\rightarrow$ & $L(\vect{y})=y^\prime+ty$} \\ 
\visible<3->{\alt<3>{$y^{\prime\prime} + 2y = 3y^\prime +t$}{$y^{\prime\prime} - 3 y^\prime + 2y=t$}} & \visible<5->{$\rightarrow$ & $L(\vect{y})=y^{\prime\prime} - 3 y^\prime + 2y$} \\
\visible<6->{$y^{(4)} + 3y = \sin[t]$} & \visible<7->{$\rightarrow$ & $L(\vect{y})=y^{(4)}+3y$}
\end{tabular}
\end{center}
\end{example}

\onslide<8->
\begin{block}{Linear Operator Properties}
\begin{equation*}
\begin{aligned}
L(k\vect{u})&=kL(\vect{u}),\quad k\in\R \\
L(\vect{u}+\vect{w}) &= L(\vect{u})+L(\vect{w})
\end{aligned}
\end{equation*}
\end{block}

\onslide<9->
\begin{block}{Proof}
The properties can be proved directly for algebraic operators.

\vspace{2mm}
For differential operators, the proof follows from the derivative properties:
\begin{itemize}
\item ${(kf)}^\prime = k f^\prime$
\item ${(f+g)}^\prime = f^\prime+g^\prime$
\end{itemize}
\end{block}
\end{frame}

\begin{frame}
\begin{block}{Superposition Principle for Linear Homogeneous Equations}
Let $\vect{u}_1$ and $\vect{u}_2$ be any solutions of the \emph{homogeneous linear} equation
\begin{equation*}
L(\vect{u})=0
\end{equation*}

\vspace{-3mm}
\begin{itemize}
\item The sum $\vect{u}=\vect{u}_1+\vect{u}_2$ is also a solution.
\item For any constant $k$, $\vect{u}=k\vect{u}_1$ is also a solution.
\end{itemize}
\end{block}\pause

\begin{block}{Proof}
The proof of the Superposition Principle follows directly from the properties of linear operators from the previous slides.
\begin{equation*}
\begin{aligned}
L(\vect{u}) = L(\vect{u_1}+\vect{u_2})
= L(\vect{u_1}) + L(\vect{u_2})
= 0 + 0 = 0
\end{aligned}
\end{equation*}
\begin{equation*}
\begin{aligned}
L(\vect{u}) = L(k \vect{u_1})
= k L(\vect{u_1})
= k\cdot 0 = 0
\end{aligned}
\end{equation*}
\end{block}
\end{frame}

\begin{frame}
\begin{example}
The point $(1,3)$ is on the line $y=3x$.\pause~So is the point $(2,6) = (2\cdot 1, 2\cdot 3)$.

\vspace{2mm}\pause
Additionally, the point $(3,9)=(1+2,3+6)=(1,3)+(2,6)$ is on the line.
\end{example}\pause

\begin{example}
The differential equation
\begin{equation*}
y^{\prime\prime}-4y=0
\end{equation*}
has the solutions $y=e^{2t}$ and $y=e^{-2t}$.\pause

\vspace{2mm}
By superposition, $y=2e^{2t}+3e^{-2t}$ must also be a solution.\pause

\vspace{2mm}
This is easily verified:
\begin{equation*}
\begin{aligned}
y^\prime &= 4e^{2t}-6e^{-2t} \\\pause
y^{\prime\prime} &= 8e^{2t}+12e^{-2t} \\\pause
y^{\prime\prime}-4y &= \left(8e^{2t}+12e^{-2t}\right) -4\left(2e^{2t}+3e^{-2t}\right)\pause \\
& = 8e^{2t}+12e^{-2t} -8e^{2t}-12e^{-2t}\pause = 0
\end{aligned}
\end{equation*}
\end{example}
\end{frame}

\begin{frame}
\begin{block}{Nonhomogeneous Principle}
Let $\vect{u}_p$ be any solution (called a particular solution) to \emph{linear nonhomogeneous} equation
\begin{equation*}
L(\vect{u})=C\qquad\text{(algebraic)} 
\end{equation*}
or
\begin{equation*}
L(\vect{u})=f(t)\qquad\text{(differential)}
\end{equation*}
Then,
\begin{equation*}
\vect{u}=\vect{u}_h+\vect{u}_p
\end{equation*}
is also a solution, here $\vect{u}_h$ is a solution to the \textbf{associated homogeneous} equation
\begin{equation*}
L(\vect{u})=0
\end{equation*}
Furthermore, \emph{every solution of the nonhomogeneous equation must be of the form $\vect{u}=\vect{u}_h+\vect{u}_p$.}
\end{block}
\end{frame}

\begin{frame}
\begin{block}{Proof}
It is easy to show that $\vect{u}=\vect{u}_h+\vect{u}_p$ is a solution.
\begin{equation*}
L(\vect{u}) = L(\vect{u}_h+\vect{u}_p) = L(\vect{u}_h) + L(\vect{u}_p) = 0 + f(t) = f(t)
\end{equation*}\pause

\vspace{-5mm}
To show that every solution has to be of this form, suppose that $\vect{u}_q$ is any solution. Note that $\vect{u}_q=\vect{u}_p + (\vect{u}_q - \vect{u}_p)$.\pause

\vspace{2mm}
We can then show that $\vect{u}_q-\vect{u}_p$ is also a solution to $L(\vect{u})=0$:
\begin{equation*}
\begin{aligned}
L(\vect{u}_q-\vect{u}_p) &=\pause  L(\vect{u}_q) + L(-\vect{u}_p)\pause \\
&= L(\vect{u}_q)-L(\vect{u}_p)\pause \\
&= f(t) - f(t) = 0
\end{aligned}
\end{equation*}
\end{block}\pause
\begin{block}{Process for Solving Nonhomogeneous Linear Equations}
\begin{description}
\item[Step 1:] Find all solutions $\vect{u}_h$ of $L(\vect{u})=0$.
\item[Step 2:] Find any solution $\vect{u}_p$ of $L(\vect{u})=f$.
\item[Step 3:] Add $\vect{u}_h+\vect{u}_p=\vect{u}$ to find all solutions of $L(\vect{u})=f$.
\end{description}
\end{block}
\end{frame}

\begin{frame}[fragile]
\begin{example}
Consider
\begin{equation*}
y^\prime-y=t
\end{equation*}
\begin{overprint}
\onslide<1-4>
To solve using superposition we need to complete three steps.
\begin{description}
\item[Step 1:] \mbox{}\visible<2->{Solve the associated homogeneous equation $y^\prime-y=0$, or $y^\prime = y$. (Note: first-order homogeneous linear differential equations are always separable.)
\begin{equation*}
y_{h}=c e^{t},\quad\text{for any}~c\in\R
\end{equation*}}
\item[Step 2:] \mbox{}\visible<3->{We can verify by differentiation and substitution that $y_{p}=-t-1$ is a particular solution.}
\item[Step 3:] \mbox{}\visible<4->{Superposition tells us that 
\begin{equation*}
y=y_h+y_p=c e^t-t-1
\end{equation*}
is a solution for any $c\in\R$.}
\end{description}
\onslide<5->
\begin{figure}[h]
\begin{subfigure}{.29\linewidth}
%\centering
\begin{asy}
size(103);

real min_x=-3, max_x=3;
real min_y=-3, max_y=3;

pair start=(min_x,min_y);
pair end=(max_x,max_y);

draw_grid_lines(min_x, max_x, min_y, max_y);

real yp (real t, real y) { return y; }

add(slopefield(yp, start, end, 10));

draw(curve((0,0.5), yp, start, end), red+1bp);
draw(curve((0,2), yp, start, end), red+1bp);
draw(curve((0,-0.5), yp, start, end), red+1bp);
draw(curve((0,-2), yp, start, end), red+1bp);
draw(curve((0,0.1), yp, start, end), red+1bp);
draw(curve((0,-0.1), yp, start, end), red+1bp);

limits(start,end,Crop);

xaxis("$t$",YEquals(min_y),min_x,max_x,LeftTicks());
xaxis(YEquals(max_y),min_x,max_x);
yaxis(XEquals(min_x),min_y,max_y,LeftTicks());
yaxis(XEquals(max_x),min_y,max_y);
\end{asy}
\caption*{$\{y_h\}$}
\end{subfigure}
\begin{subfigure}{1em}
+
\end{subfigure}
\begin{subfigure}{.29\linewidth}
\centering
\begin{asy}
size(103);

real min_x=-3, max_x=3;
real min_y=-3, max_y=3;

pair start=(min_x,min_y);
pair end=(max_x,max_y);

draw_grid_lines(min_x, max_x, min_y, max_y);

real yp (real t, real y) { return y+t; }

add(slopefield(yp, start, end, 10));

draw(curve((0,-1), yp, start, end), blue+1bp);

limits(start,end,Crop);

xaxis("$t$",YEquals(min_y),min_x,max_x,LeftTicks());
xaxis(YEquals(max_y),min_x,max_x);
yaxis(XEquals(min_x),min_y,max_y,LeftTicks());
yaxis(XEquals(max_x),min_y,max_y);
\end{asy}
\caption*{$y_p$}
\end{subfigure}
\begin{subfigure}{1em}
=
\end{subfigure}
\begin{subfigure}{.29\linewidth}
\centering
\begin{asy}
size(103);

real min_x=-3, max_x=3;
real min_y=-3, max_y=3;

pair start=(min_x,min_y);
pair end=(max_x,max_y);

draw_grid_lines(min_x, max_x, min_y, max_y);

real yp (real t, real y) { return y+t; }

add(slopefield(yp, start, end, 10));

draw(curve((0,-0.51), yp, start, end), heavymagenta+1bp);
draw(curve((0,0.9), yp, start, end), heavymagenta+1bp);
draw(curve((0,2.3), yp, start, end), heavymagenta+1bp);
draw(curve((-1.55,0), yp, start, end), heavymagenta+1bp);
draw(curve((-2.1,0), yp, start, end), heavymagenta+1bp);
draw(curve((-1.3,0), yp, start, end), heavymagenta+1bp);

draw(curve((0,-1), yp, start, end), blue+1bp);

limits(start,end,Crop);

xaxis("$t$",YEquals(min_y),min_x,max_x,LeftTicks());
xaxis(YEquals(max_y),min_x,max_x);
yaxis(XEquals(min_x),min_y,max_y,LeftTicks());
yaxis(XEquals(max_x),min_y,max_y);
\end{asy}
\caption*{$\{y_h\}+y_p$}
\end{subfigure}
\end{figure}
\end{overprint}
\end{example}
\end{frame}

\begin{frame}
\begin{block}{Euler-Lagrange Two-Stage Method}
We have seen that the general solution for the linear first-order differential equation
\begin{equation*}
y^\prime+p(t) y = f(t)
\end{equation*}
has the form $y=y_h+y_p$.\pause

\vspace{2mm}
We solved the corresponding homogenous equation using separation of variables, getting a one-parameter family
\begin{equation*}
y_h=c e^{-\int p(t) dt}
\end{equation*}
where $c\in\R$.\pause

\vspace{2mm}
The second step is to find a particular solution, which we will accomplish using \textbf{variation of parameters}, which was developed by French mathematician Joseph Louis Lagrange.
\end{block}
\end{frame}

\begin{frame}
\begin{block}{Variation of Parameters}
The idea of variation of parameters is to start with
\begin{equation*}
y_h(t)=c e^{-\int p(t) dt}
\end{equation*}
\onslide<2->
and change the constant $c$ to a function $v(t)$ and try a solution of the form
\begin{equation*}
y_p(t)=v(t) e^{-\int p(t) dt}
\end{equation*}
where the unknown function $v(t)$ is called the \textbf{varying parameter}.

\onslide<3->
\vspace{2mm}
Our goal is to find $v(t)$, to do so we need to substitute $y_p$ into the DE\@.

\vspace{-4mm}
\begin{overprint}
\onslide<3>
\begin{equation*}
\underbrace{\left(v^\prime(t) e^{-\int p(t)dt}-p(t)v(t) e^{-\int p(t)dt}\right)}_{y^\prime_p}
+
\underbrace{p(t)v(t)e^{-\int p(t)dt}}_{p(t)y_p} = f(t)
\end{equation*}
\onslide<4>
\vspace{4mm}
\begin{equation*}
v^\prime(t) e^{-\int p(t)dt} = f(t)
\end{equation*}
\onslide<5>
\vspace{4mm}
\begin{equation*}
v^\prime(t) = f(t) e^{\int p(t)dt}
\end{equation*}
\onslide<6->
\vspace{3mm}
\begin{equation*}
v(t) = \int f(t) e^{\int p(t)dt} dt
\end{equation*}
\end{overprint}
\onslide<7->
\vspace{-4mm}
Now that we have $v(t)$, we have determined a particular solution.
\begin{equation*}
y_p(t) = v(t) e^{-\int p(t) dt} = e^{-\int p(t) dt} \int f(t) e^{\int p(t)dt} dt 
\end{equation*}
\end{block}
\end{frame}

\begin{frame}
\begin{example}
Consider the IVP
\begin{equation*}
y^\prime + \left(\dfrac{1}{t+1}\right)y = 2,\quad y(0)=0,\quad t\geq 0
\end{equation*}
\begin{overprint}
\onslide<2-8>
\textbf{Step 1.} We start by solving the associated homogeneous equation.
\begin{equation*}
y^\prime + \left(\dfrac{1}{t+1}\right)y = 0
\end{equation*}
\visible<3->{Let us assume for the moment that $y\neq0$ and use separation of variables.
\begin{center}
\begin{tabular}{rcl}
\only<3>{$\dfrac{dy}{y}$ &$=$& $-\dfrac{dt}{t+1}$ \\}
\only<4>{$\ln \abs{y}$ &$=$& $-\ln[t+1] + c$ \\}
\only<5-6>{$\abs{y}$ & $=$ &}\only<5>{$e^{{\ln[t+1]}^{-1} + c}$}\only<6>{$e^c {\left(t+1\right)}^{-1}$}\only<5-6>{\\}
\only<7->{$y_h$ &$=$&} \only<7>{$\pm\dfrac{e^c}{t+1}$}\only<8->{$\dfrac{k}{t+1}$}
\end{tabular}
\end{center}}
\visible<8->{where $k=\pm e^c$}
\onslide<9-14>
\textbf{Step 2.} Next, using variation of parameters, we try
\begin{equation*}
y_p = \dfrac{v(t)}{t+1}
\end{equation*}
\visible<10->{which gives:
\begin{equation*}
\vphantom{\dfrac{v^\prime(t)}{t+1} = 2}
\only<10>{v^\prime(t) e^{-\int p(t) dt} = f(t)}
\only<11>{\dfrac{v^\prime(t)}{t+1} = 2}
\only<12>{v^\prime(t) = 2t+2}
\only<13->{v(t) = t^2+2t+c}
\end{equation*}}

\vspace{-5mm}
\visible<14->{But, we only need a single $v(t)$, so we can let $c=0$, giving
\begin{equation*}
y_p=\dfrac{t^2+2t}{t+1}
\end{equation*}}
\onslide<15->
\textbf{Step 3.} Thus, the general solution is:
\begin{equation*}
y(t) = y_h + y_p = \dfrac{k}{t+1}+\dfrac{t^2+2t}{t+1}
\end{equation*}
\visible<16->{\textbf{Step 4.} Substituting the initial condition into the general solution gives:
\begin{equation*}
0 = y(0) = \dfrac{k}{(0)+1}+\dfrac{{(0)}^2+2(0)}{(0)+1}
\visible<17->{\Rightarrow k=0}
\end{equation*}}
\visible<18->{Which means that the solution to the IVP is
\begin{equation*}
y(t)=\dfrac{t^2+2t}{t+1}
\end{equation*}}
\end{overprint}
\vspace{-5mm}
\end{example}
\end{frame}

\begin{frame}
\begin{block}{Euler-Lagrange Method for Solving Linear First-Order DEs}
To solve a linear differential equation
\begin{equation*}
y^\prime + p(t) y = f(t)
\end{equation*}
where $p$ and $f$ are continuous on a domain $I$, use the following steps.
\begin{description}
\item[Step 1.] Solve the corresponding homogenous equation $y^\prime+p(t)y=0$ to obtain the one-parameter family.

\vspace{-3mm}
\begin{equation*}
y_h=c e^{-\int p(t) dt}
\end{equation*}\vspace{-8mm}
\item[Step 2.] Solve
\begin{equation*}
v^\prime(t) e^{-\int p(t)dt} = f(t)
\end{equation*}
for $v(t)$ to obtain a particular solution $y_p = v(t) e^{-\int p(t)dt}$.
\item[Step 3.] Combine the results of Step 1 and Step 2 to form the general solution
\begin{equation*}
y(t)=y_h + y_p
\end{equation*}
\item[Step 4.] If you are solving an IVP, only after Step 3 can you plug in the initial condition.
\end{description}
\end{block}
\end{frame}

\begin{frame}
\begin{block}{Note}
Variation of Parameters is a very powerful method, and you will see it again in a proper differential equations course. But, for first-order (and \emph{only} first-order) equations we have a second method, called the \textbf{Integrating Factor Method} which may also be used. \pause

\vspace{2mm}
For the differential equation
\begin{equation*}
y^\prime + p(t) y = f(t)
\end{equation*}
we will break this new method down into two cases:
\begin{itemize}
\item $p(t)$ is constant.
\item $p(t)$ is variable.
\end{itemize}
\end{block}
\end{frame}

\begin{frame}
\begin{block}{Integrating Factor Method (Constant Coefficient)}
Let us look at the first-order linear differential equation
\begin{equation*}
y^\prime+ ay = f(t),\quad a\in\R
\end{equation*}
\onslide<2->
This method uses a simple observation made by Euler:
\begin{equation*}
e^{at}\left(y^\prime+ay\right) = \dfrac{d}{dt}\left(e^{at} y\right)
\end{equation*}
\begin{overprint}
\onslide<3>
Let us start with the differential equation.
\begin{equation*}
y^\prime+ ay = f(t)
\end{equation*}
\onslide<4>
We first multiply both sides of the equation by $e^{at}$.
\begin{equation*}
e^{at}\left(y^\prime+ay\right) = e^{at} f(t) 
\end{equation*}
\onslide<5>
We then apply Euler's observation to the left-hand side.
\begin{equation*}
\dfrac{d}{dt}\left(e^{at} y\right) = e^{at} f(t) 
\end{equation*}
\onslide<6>
Next we integrate both sides.
\begin{equation*}
e^{at} y = \int e^{at} f(t)  dt + c
\end{equation*}
\onslide<7->
Solving for $y$ gives:
\begin{equation*}
y(t) = e^{-at} \int e^{at} f(t)  dt + c e^{-at}
\end{equation*}
\end{overprint}
\end{block}
\onslide<8->
\begin{block}{Note}
This is the same answer we got from Variation of Parameters, though achieved through a different route. We have obtained both $y_h$ and $y_p$ at the same time.
\end{block}
\end{frame}

\begin{frame}
\begin{block}{Integrating Factor Method (Variable Coefficient)}
Now let us look at the more general first-order differential equation
\begin{equation*}
y^\prime + p(t) y = f(t)
\end{equation*}
\begin{overprint}
\onslide<2>
We seek a function $\mu(t)$ that satisfies Euler's observation, i.e.
\begin{equation*}
\mu(t)\cdot\left(y^\prime + p(t) y\right) = \dfrac{d}{dt}\left(\mu(t)\cdot y\right)
\end{equation*}
\onslide<3>
Let us carry out the differentiation on the right-hand side
\begin{equation*}
\mu(t) y^\prime + p(t)\mu(t) y = \mu^\prime (t) y + \mu(t) y^\prime
\end{equation*}
\onslide<4>
If we assume $y(t)\neq 0$, this simplifies to
\begin{equation*}
\mu^\prime (t) = p(t)\mu(t)
\end{equation*}
\onslide<5>
We can find a solution $\mu(t)>0$ by Separation of Variables.
\begin{equation*}
\dfrac{\mu^\prime (t)}{\mu(t)} = p(t)
\end{equation*}
\onslide<6>
We can find a solution $\mu(t)>0$ by Separation of Variables.
\begin{equation*}
\ln\abs{\mu(t)} = \int p(t) dt
\end{equation*}
\onslide<7->
We can find a solution $\mu(t)>0$ by Separation of Variables.
\begin{equation*}
\mu(t) = e^{\int p(t) dt}
\end{equation*}
\end{overprint}
\begin{overprint}
\onslide<8>
We now know the integrating factor, and perform the same steps as before.
\begin{equation*}
y^\prime + p(t) y = f(t)
\end{equation*}
\onslide<9>
Multiply both sides by the integrating factor.
\begin{equation*}
\mu(t)\cdot\left(y^\prime + p(t) y\right) = \mu(t)\cdot f(t)
\end{equation*}
\onslide<10>
Apply the property $\mu(t)\cdot\left(y^\prime + p(t) y\right) = {\left(\mu(t)\cdot y\right)}^\prime$ to the left-hand side.
\begin{equation*}
{\left(\mu(t) y\right)}^\prime = \mu(t) f(t)
\end{equation*}
\onslide<11>
Integrate both sides.
\begin{equation*}
\mu(t) y(t) = \int \mu(t) f(t) dt + c
\end{equation*}
\onslide<12->
Assuming $\mu(t)\neq 0$, we can solve for $y$.
\begin{equation*}
y(t) = \dfrac{1}{\mu(t)} \int \mu(t) f(t) dt + \dfrac{c}{\mu(t)}
\end{equation*}
\end{overprint}
\end{block}
\onslide<13->
\begin{block}{Note}
We have again found $y_h$ and $y_p$ at the same time.
\end{block}
\end{frame}

\begin{frame}
\begin{block}{Integrating Factor Method for First-Order Linear DEs}
\small
To solve the linear first-order DE, where $p$ and $f$ are continuous on a domain $I$.
\begin{equation*}
y^\prime + p(t) y = f(t)
\end{equation*}
\begin{description}
\item[Step 1.] Find the integrating factor $\mu(t) = e^{\int p(t) dt}$, where $\int p(t) dt$ represents \emph{any} anti-derivative of $p(t)$.
\item[Step 2.] Multiply both sides of the DE by $mu(t)$, which always simplifies to:

\vspace{-2mm}
\begin{equation*}
{\left(e^{\int p(t) dt} y(t)\right)}^\prime = e^{\int p(t) dt} f(t)
\end{equation*}
\item[Step 3.] Find the anti-derivative to get:

\vspace{-2mm}
\begin{equation*}
e^{\int p(t) dt} y(t) = \int e^{\int p(t) dt} f(t) dt +c
\end{equation*}
\item[Step 4.] Solve algebraically for $y$.

\vspace{-2mm}
\begin{equation*}
y = e^{-\int p(t) dt} \int e^{\int p(t) dt} f(t) dt + c e^{-\int p(t) dt}
\end{equation*}
\item[Step 5.] For IVPs, substitute the initial conditions in to find $c$.
\end{description}
\end{block}
\end{frame}

\begin{frame}
\begin{example}
Consider the IVP
\begin{equation*}
y^\prime - y = t,\quad y(0) = 1
\end{equation*}
Let us solve this DE using the Integrating Factor method.

\vspace{2mm}
\begin{overprint}
\onslide<2-6>
\textbf{Step 1.} Find the integrating factor:
\begin{equation*}
\mu(t) = \alt<3->{e^{\int (-1) dt}}{e^{\int p(t) dt}} \visible<4->{=e^{-t}}
\end{equation*}
\visible<5->{\textbf{Step 2.} Multiply both sides of the DE by $\mu(t)$:
\begin{equation*}
e^{-t}\left(y^\prime - y\right) = e^{-t}
\end{equation*}}
\visible<6->{Which reduces to:
\begin{equation*}
{\left(e^{-t} y\right)}^\prime = t e^{-t}
\end{equation*}}
\onslide<7->
\textbf{Step 3.} Find the antiderivative:
\begin{equation*}
e^{-t} y = \int t e^{-t} dt \visible<8->{= e^{-t}(-t-1)+c}
\end{equation*}
\visible<9->{\textbf{Step 4.} Solve for $y$:
\begin{equation*}
y(t) = e^t \left(e^{-t}\right)\left(-t-1\right) + c e^{t} \visible<10->{=-t-1+ce^{t}}
\end{equation*}}
\visible<11->{\textbf{Step 5.} Plug in the initial conditions to find the solution to the IVP\@:
\begin{equation*}
1 = y(0) = -0-1+c e^{0} \visible<12->{\Rightarrow c = 2}
\end{equation*}}
\visible<13->{Thus, the solution to the IVP is $y(t) = -t -1 + 2 e^t$}
\end{overprint}
\end{example}
\end{frame}

\end{document}
