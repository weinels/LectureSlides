\documentclass{beamer}
\usepackage[utf8]{inputenc}
\usepackage[english]{babel}
\usepackage[T1]{fontenc}
\usepackage[inline]{asymptote}
\usepackage{slide_helper}
\usepackage{asy_helper}

\begin{asydef}
import graph;
import slopefield;
import fontsize;
defaultpen(fontsize(9pt));
ngraph=1000;

int pen_pos=-1;
pen[] pens={blue, red, heavycyan, heavymagenta, lightolive};
pens.cyclic=true;
pen next_color() {return pens[++pen_pos];}
void reset_color() {pen_pos = -1;}

DefaultHead.size=new real(pen p=currentpen) {return 2.5mm;};
\end{asydef}

\newcommand*{\thead}[1]{\multicolumn{1}{c}{\bfseries #1}}

\title[MATH 1220 - Section 9.2]{Direction Fields and Euler's Method}

\begin{document}
\begin{frame}
\titlepage
\end{frame}

\begin{frame}
\begin{block}{What is a Differential Equation?}
A \textbf{differential equation} (or DE) is an equation containing derivatives. The \textbf{order} of the equation refers to the highest-order derivative that occurs.\pause

\vspace{2mm}
In this chapter we will focus on DEs that can be written as:
\begin{equation*}
\dfrac{dy}{dt}=f(t,y)
\quad\text{or}\quad
y^{\prime}=f(t,y)
\end{equation*}
Where the dependent variable $y$ is an unknown function, the \textbf{solution}.
\end{block}\pause

\begin{block}{Note}
There may be more than one solution for a given differential equation.
\end{block}\pause

\begin{block}{Analytic Definition of a Solution}
Analytically, $y(t)$ is a \textbf{solution} of a differential equation if substituting $y(t)$ for $y$ reduced the equation to an identity:

\vspace{-4mm}
\begin{equation*}
y^{\prime}(t)=f(t, y(t))
\end{equation*}

\vspace{-2mm}
on an appropriate domain for $t$.
\end{block}
\end{frame}

\begin{frame}[fragile]
\begin{example}
Verify that $y(t)$ is a solution to the DE\@.
\begin{equation*}
y^{\prime}(t)=2y,\quad y(t)=e^{2t}
\end{equation*}
\begin{overprint}
\onslide<1>
\onslide<2-6>
Substituting into the DE gives:
\begin{equation*}
\begin{aligned}
y^{\prime}(t) &= \dfrac{d}{dt}e^{2t} \\
\visible<3->{&=2e^{2t} \\}
\visible<4->{&=2y(t) \\}
\visible<5->{&=f(t, y(t))}
\end{aligned}
\end{equation*}
\visible<6->{%
Similarly, we could show that
\begin{equation*}
y(t)=2e^{2t}
\quad\text{and}\quad
y(t)=\dfrac{-3}{2}e^{2t}
\end{equation*}
are also solutions. In fact, any constant multiple of $e^{2t}$ is a solution.}
\onslide<7->
\begin{center}
\begin{asy}
size(180, IgnoreAspect);

real min_x=-2, max_x=2;
real min_y=-15, max_y=15;

pair start=(min_x,min_y);
pair end=(max_x,max_y);

real y_1(real t) {return exp(2t);}
real y_2(real t) {return 2*exp(2t);}
real y_3(real t) {return (-3/2)*exp(2t);}

draw(graph(y_1, min_x, max_x), next_color() + 1bp);
draw(graph(y_2, min_x, max_x), next_color() + 1bp);
draw(graph(y_3, min_x, max_x), next_color() + 1bp);

reset_color();

label("$y=e^{2t}$", (1.3,5), next_color());
label("$y=2 e^{2t}$", (0.45,12), next_color());
label("$y=\dfrac{3}{2} e^{2t}$", (1.5,-8), next_color());

limits(start,end,Crop);

xaxis("$t$",YEquals(0),Ticks(NoZero));
yaxis("$y$",XEquals(0),Ticks(NoZero),autorotate=false);
\end{asy}
\end{center}
\end{overprint}
\vspace{-58mm}
\end{example}
\end{frame}

\begin{frame}[fragile]
\begin{example}
Verify that $y(t)$ is a solution to the DE\@.
\begin{equation*}
y^{\prime}(t)=-\dfrac{t}{y},\quad y(t)=\sqrt{1-t^2}
\end{equation*}
\begin{overprint}
\onslide<1>
\onslide<2-6>
Substituting into the DE gives:
\begin{equation*}
\begin{aligned}
y^\prime(t) &= \dfrac{d}{dt}\left(\sqrt{1-t^2}\right) \\
\visible<3->{&= \dfrac{1}{2}{\left(1-t^2\right)}^{-\tfrac{1}{2}}\cdot(-2t)\\}
\visible<4->{&=\dfrac{-t}{\sqrt{1-t^2}}\\}
\visible<5->{&=-\dfrac{t}{y}}
\end{aligned}
\end{equation*}

\visible<6->{\vspace{-2mm}
Other solutions are of the form $y(t)=\sqrt{k-t^2}$.}
\onslide<7>
\begin{center}
\begin{asy}
size(250);

real min_x=-2, max_x=2;
real min_y=0, max_y=2;

pair start=(min_x,min_y);
pair end=(max_x,max_y);

real y_1(real t) {return sqrt(abs(1.0-t*t));}
real y_2(real t) {return sqrt(abs(0.5-t*t));}
real y_3(real t) {return sqrt(abs(3-t*t));}

draw(graph(y_1, -1, 1), next_color() + 1bp);
draw(graph(y_2, -sqrt(0.5), sqrt(0.5)), next_color() + 1bp);
draw(graph(y_3, -sqrt(3), sqrt(3)), next_color() + 1bp);

reset_color();

limits(start,end,Crop);

xaxis("$t$",YEquals(0),Ticks(NoZero));
yaxis("$y$",XEquals(0),Ticks(NoZero),autorotate=false);

label("$y=\sqrt{1-t^2}$",   (0.0,1.15), next_color(), UnFill());
label("$y=\sqrt{0.5-t^2}$", (0.0,0.25), next_color(), UnFill());
label("$y=\sqrt{3-t^2}$", (0.0,1.56), next_color(), UnFill());
\end{asy}
\end{center}
\end{overprint}
\vspace{-45mm}
\end{example}
\end{frame}

\begin{frame}
\begin{block}{Note}
It is no coincidence that the two previous examples had multiple solutions. Most differential equations have an infinite number of solutions.
\end{block}\pause
\begin{example}
Consider
\begin{equation*}
\dfrac{dy}{dt} = f(t)
\end{equation*}\pause
We can integrate both sides to get the solution to get

\vspace{-2mm}
\begin{equation*}
y=\int f(t)dt + c
\end{equation*}

\vspace{-2mm}
where $c$ is an arbitrary constant.
\end{example}\pause
\begin{block}{Family of Solutions}
In general, all solutions of a first-order DE form a \textbf{family} of solutions expressed with a single parameter $c$. Such a family is called the \textbf{general solution}. A member of the family that results from a specific value of $c$ is called a \textbf{particular solution}.
\end{block}
\end{frame}

\begin{frame}[fragile]
\begin{example}
The general solution of $y^\prime=3t^2$ is
\begin{equation*}
y=t^3+c
\end{equation*}
where $c$ may be any real value.

\begin{center}
\begin{asy}
size(150, IgnoreAspect);

real min_x=-2, max_x=2;
real min_y=-6, max_y=6;

pair start=(min_x,min_y);
pair end=(max_x,max_y);

real[] k_values = {1.0, -3.4, -1.5, 2.2, 4.0};

for (real k : k_values)
{
	real y(real t) {return t*t*t+k;}
	draw(graph(y, min_x, max_x), next_color() + 1bp, "c = " + string(k));
}

limits(start,end,Crop);

xaxis("$t$",YEquals(0),Ticks(NoZero));
yaxis("$y$",XEquals(0),Ticks(NoZero),autorotate=false);

attach(legend(), truepoint(E), 30E, UnFill());

real x_1 = 2.009;
real x_2 = 2.901;
unfill((x_1,0.3) -- (x_2,0.3) -- (x_2,-0.3) -- (x_1, -0.3) -- cycle);
\end{asy}
\end{center}
\end{example}
\end{frame}

\begin{frame}
\begin{block}{Initial-Value Problem}
The combination of a first-order differential equation and an \textbf{initial condition}
\begin{equation*}
\dfrac{dy}{dt}=f(t,y),\quad y(t_0)=y_0
\end{equation*}
is called an \textbf{initial-value problem}. It's solution will pass through the point $(t_0,y_0)$.
\end{block}\pause
\begin{block}{Note}
While a family of solutions for a DE contains multiple solutions, an IVP usually has only one solution. That is, the solution to an IVP is a particular solution to the DE\@.
\end{block}
\end{frame}

\begin{frame}[fragile]
\begin{example}
The function $y(t)=t^3+1$ is a solution to the IVP
\begin{equation*}
y^\prime = 3t^2,\quad y(0)=1
\end{equation*}\pause
Differentiating $y(t)$ confirms that 
\begin{equation*}
y^\prime(t)={(t^3+1)}^\prime=3t^2,\quad\text{and}\quad y(0)=0^3+1=1
\end{equation*}
\begin{center}
\begin{asy}
size(131, IgnoreAspect);

real min_x=-2, max_x=2;
real min_y=-6, max_y=6;

pair start=(min_x,min_y);
pair end=(max_x,max_y);

real[] k_values = {-3.4, -1.5, 2.2, 4.0};

for (real k : k_values)
{
	real y(real t) {return t*t*t+k;}
	draw(graph(y, min_x, max_x), gray + 1bp);
}

real y(real t) {return t*t*t+1.0;}
draw(graph(y, min_x, max_x), next_color() + 1bp);

limits(start,end,Crop);

xaxis("$t$",YEquals(0),Ticks(NoZero));
yaxis("$y$",XEquals(0),Ticks(NoZero),autorotate=false);
\end{asy}
\end{center}
\end{example}
\end{frame}

\begin{frame}
\begin{block}{Graphical Definition of Solutions}
A \textbf{solution} to a first-order differential equation is a function whose slope at each point is specified by the derivative.
\end{block}
\onslide<2->

\begin{block}{Direction Fields}
We can see what solution curves look like by, on regular intervals, drawing short line segments with slope determined by the DE\@ for that point. The collection of these segments are called \textbf{direction field} (or a \textbf{slope field}).
\end{block}
\end{frame}

\begin{frame}[fragile]
\begin{example}
\vspace{-2mm}
\begin{equation*}
y^\prime=-2ty+t
\end{equation*}
\begin{overprint}
\onslide<1>
\begin{center}
\begin{asy}
size(220);

real min_x=-2, max_x=4;
real min_y=-3, max_y=3;

pair start=(min_x,min_y);
pair end=(max_x,max_y);

for(real gx=min_x+1; gx<=max_x-1; ++gx)
	draw((gx,min_y)--(gx,max_y),dotted+darkgray);
    
for(real gy=min_y+1; gy<=max_y-1; ++gy)
	draw((min_x,gy)--(max_x,gy),dotted+darkgray); 
	
real yp (real t, real y) { return -2*t*y+t; }

add(slopefield(yp, start, end, 20));

limits(start,end,Crop);

xaxis("$t$",YEquals(min_y),min_x,max_x,LeftTicks());
xaxis(YEquals(max_y),min_x,max_x);
yaxis("$y$",XEquals(min_x),min_y,max_y,LeftTicks());
yaxis(XEquals(max_x),min_y,max_y);
\end{asy}
\end{center}
\onslide<2>
\begin{center}
\begin{asy}
size(220);

real min_x=-2, max_x=4;
real min_y=-3, max_y=3;

pair start=(min_x,min_y);
pair end=(max_x,max_y);

for(real gx=min_x+1; gx<=max_x-1; ++gx)
	draw((gx,min_y)--(gx,max_y),dotted+darkgray);
    
for(real gy=min_y+1; gy<=max_y-1; ++gy)
	draw((min_x,gy)--(max_x,gy),dotted+darkgray); 
	
real yp (real t, real y) { return -2*t*y+t; }

add(slopefield(yp, start, end, 20));

pair[] pts = {(0,-2), (2.5,1), (2.1,-2), (0,0.5), (0,1.75)};

for (pair p : pts)
{
	real c = (p.y - 0.5)/exp(-p.x*p.x);
	real f (real t) { return c*exp(-1*t*t) + 0.5; }
	
	pen clr = next_color();
	draw(graph(f, min_x, max_x), clr+1bp);
	dot(p, clr);
}

limits(start,end,Crop);

xaxis("$t$",YEquals(min_y),min_x,max_x,LeftTicks());
xaxis(YEquals(max_y),min_x,max_x);
yaxis("$y$",XEquals(min_x),min_y,max_y,LeftTicks());
yaxis(XEquals(max_x),min_y,max_y);
\end{asy}
\end{center}
\end{overprint}
\end{example}
\end{frame}

\begin{frame}[fragile]
\begin{example}
\vspace{-2mm}
\begin{equation*}
y^\prime=y+t^2
\end{equation*}
\begin{overprint}
\onslide<1>
\begin{center}
\begin{asy}
size(220);

real min_x=-4, max_x=4;
real min_y=-4, max_y=4;

pair start=(min_x,min_y);
pair end=(max_x,max_y);

for(real gx=min_x+1; gx<=max_x-1; ++gx)
	draw((gx,min_y)--(gx,max_y),dotted+darkgray);
    
for(real gy=min_y+1; gy<=max_y-1; ++gy)
	draw((min_x,gy)--(max_x,gy),dotted+darkgray); 
	
real yp (real t, real y) { return y+t*t; }

add(slopefield(yp, start, end, 20));

limits(start,end,Crop);

xaxis("$t$",YEquals(min_y),min_x,max_x,LeftTicks());
xaxis(YEquals(max_y),min_x,max_x);
yaxis("$y$",XEquals(min_x),min_y,max_y,LeftTicks());
yaxis(XEquals(max_x),min_y,max_y);
\end{asy}
\end{center}
\onslide<2>
\begin{center}
\begin{asy}
size(220);

real min_x=-4, max_x=4;
real min_y=-4, max_y=4;

pair start=(min_x,min_y);
pair end=(max_x,max_y);

for(real gx=min_x+1; gx<=max_x-1; ++gx)
	draw((gx,min_y)--(gx,max_y),dotted+darkgray);
    
for(real gy=min_y+1; gy<=max_y-1; ++gy)
	draw((min_x,gy)--(max_x,gy),dotted+darkgray); 
	
real yp (real t, real y) { return y+t*t; }

add(slopefield(yp, start, end, 20));

pair[] pts = {(0.0,-2.1), (-2,1), (1.9,1), (0,2), (2.5,-0.5)};

for (pair p : pts)
{
	real c = (p.y + p.x * p.x + 2 * p.x + 2)/exp(p.x);
	real f (real t) { return c*exp(t)-t*t-2*t-2; }
	
	pen clr = next_color();
	draw(graph(f, min_x, max_x), clr+1bp);
	dot(p, clr);
}

limits(start,end,Crop);

xaxis("$t$",YEquals(min_y),min_x,max_x,LeftTicks());
xaxis(YEquals(max_y),min_x,max_x);
yaxis("$y$",XEquals(min_x),min_y,max_y,LeftTicks());
yaxis(XEquals(max_x),min_y,max_y);
\end{asy}
\end{center}
\end{overprint}
\end{example}
\end{frame}

\begin{frame}[fragile]
\begin{example}
\vspace{-2mm}
\begin{equation*}
y^\prime=t-y
\only<3>{\quad\text{and}\quad y^{\prime\prime}=1-y^\prime=1-t+y}
\end{equation*}
\begin{overprint}
\onslide<1>
\begin{center}
\begin{asy}
size(220);

real min_x=-1, max_x=3;
real min_y=-1, max_y=3;

pair start=(min_x,min_y);
pair end=(max_x,max_y);

for(real gx=min_x+1; gx<=max_x-1; ++gx)
	draw((gx,min_y)--(gx,max_y),dotted+darkgray);
    
for(real gy=min_y+1; gy<=max_y-1; ++gy)
	draw((min_x,gy)--(max_x,gy),dotted+darkgray); 
	
real yp (real t, real y) { return t-y; }

add(slopefield(yp, start, end, 20));

limits(start,end,Crop);

xaxis("$t$",YEquals(min_y),min_x,max_x,LeftTicks());
xaxis(YEquals(max_y),min_x,max_x);
yaxis("$y$",XEquals(min_x),min_y,max_y,LeftTicks());
yaxis(XEquals(max_x),min_y,max_y);
\end{asy}
\end{center}
\onslide<2>
\begin{center}
\begin{asy}
size(220);

real min_x=-1, max_x=3;
real min_y=-1, max_y=3;

pair start=(min_x,min_y);
pair end=(max_x,max_y);

for(real gx=min_x+1; gx<=max_x-1; ++gx)
	draw((gx,min_y)--(gx,max_y),dotted+darkgray);
    
for(real gy=min_y+1; gy<=max_y-1; ++gy)
	draw((min_x,gy)--(max_x,gy),dotted+darkgray); 
	
real yp (real t, real y) { return t-y; }

add(slopefield(yp, start, end, 20));

pair[] pts = {(1,2), (2.1,1), (1.9,1), (0,2), (2.5,-0.5)};

for (pair p : pts)
{
	real c = exp(p.x)*(p.y-p.x+1);
	real f (real t) { return (exp(t)*t-exp(t)+c)/exp(t); }
	
	pen clr = next_color();
	draw(graph(f, min_x, max_x), clr+1bp);
	dot(p, clr);
}

limits(start,end,Crop);

xaxis("$t$",YEquals(min_y),min_x,max_x,LeftTicks());
xaxis(YEquals(max_y),min_x,max_x);
yaxis("$y$",XEquals(min_x),min_y,max_y,LeftTicks());
yaxis(XEquals(max_x),min_y,max_y);
\end{asy}
\end{center}
\onslide<3>
\begin{center}
\begin{asy}
size(220);

real min_x=-1, max_x=3;
real min_y=-1, max_y=3;

pair start=(min_x,min_y);
pair end=(max_x,max_y);

for(real gx=min_x+1; gx<=max_x-1; ++gx)
	draw((gx,min_y)--(gx,max_y),dotted+darkgray);
    
for(real gy=min_y+1; gy<=max_y-1; ++gy)
	draw((min_x,gy)--(max_x,gy),dotted+darkgray); 
	
real yp (real t, real y) { return t-y; }

add(slopefield(yp, start, end, 20));

pair[] pts = {(1,2), (2.1,1), (1.9,1), (0,2), (2.5,-0.5)};

for (pair p : pts)
{
	real c = exp(p.x)*(p.y-p.x+1);
	real f (real t) { return (exp(t)*t-exp(t)+c)/exp(t); }
	
	pen clr = next_color();
	draw(graph(f, min_x, max_x), clr+1bp);
	dot(p, clr);
}

real zed(real t) { return t-1; }
draw(graph(zed, min_x, max_x), gray+dashed+1bp);

label("Concave up", (0,0.5), black, UnFill());
label("$y^{\prime\prime}>0$", (0, 0.3), black, UnFill());
label("Concave down", (1.97, 0), black, UnFill());
label("$y^{\prime\prime}<0$", (1.97, -0.21), black, UnFill());

limits(start,end,Crop);

xaxis("$t$",YEquals(min_y),min_x,max_x,LeftTicks());
xaxis(YEquals(max_y),min_x,max_x);
yaxis("$y$",XEquals(min_x),min_y,max_y,LeftTicks());
yaxis(XEquals(max_x),min_y,max_y);
\end{asy}
\end{center}
\end{overprint}
\end{example}
\end{frame}

\begin{frame}[fragile]
\begin{block}{Equilibria}
For a differential equation, a solution that does not change over time is called an \textbf{equilibrium solution}.\pause

\vspace{2mm}
For a first-order DE $y^{\prime}=f(t,y)$, an equilibrium solution is always a horizontal line $y(t)=C$, which can be obtained by setting $y^{\prime}=0$.
\end{block}\pause

\begin{block}{Stability}
For a differential equation $y^\prime=f(t,y)$, an equilibrium solution $y(t)=C$ is called
\begin{itemize}[<+- | alert@+>]
\item \textbf{stable} if solutions \textquote{near} it tend toward it as $t\rightarrow\infty$.
\item \textbf{unstable} if solutions \textquote{near} it tend away from it as $t\rightarrow\infty$.
\end{itemize}
\end{block}

\onslide<+->
\begin{block}{Note}
A equilibrium solution is often called \textbf{semistable} if it is stable on one side and unstable on the other.
\end{block}
\end{frame}

\begin{frame}[fragile]
\begin{example}
The DE $y^\prime=-2ty+t$ has the constant solution $y(t)=\dfrac{1}{2}$.
\begin{center}
\begin{asy}
size(205);

real min_x=-2, max_x=4;
real min_y=-3, max_y=3;

pair start=(min_x,min_y);
pair end=(max_x,max_y);

for(real gx=min_x+1; gx<=max_x-1; ++gx)
	draw((gx,min_y)--(gx,max_y),dotted+darkgray);
    
for(real gy=min_y+1; gy<=max_y-1; ++gy)
	draw((min_x,gy)--(max_x,gy),dotted+darkgray); 
	
real yp (real t, real y) { return -2*t*y+t; }

add(slopefield(yp, start, end, 20));

pair[] pts = {(0,-2), (2.5,1), (2.1,-2), (0,1.75)};

for (pair p : pts)
{
	real c = (p.y - 0.5)/exp(-p.x*p.x);
	real f (real t) { return c*exp(-1*t*t) + 0.5; }
	
	pen clr = grey;
	draw(graph(f, min_x, max_x), clr+1bp);
	//dot(p, clr);
}

real f (real t) { return 0.5; }
draw(graph(f, min_x, max_x),heavymagenta+1bp); 
//dot((0, 0.5), heavymagenta);

limits(start,end,Crop);

xaxis("$t$",YEquals(min_y),min_x,max_x,LeftTicks());
xaxis(YEquals(max_y),min_x,max_x);
yaxis("$y$",XEquals(min_x),min_y,max_y,LeftTicks());
yaxis(XEquals(max_x),min_y,max_y);
\end{asy}
\end{center}
\end{example}
\end{frame}

\begin{frame}[fragile]
\begin{example}
\begin{equation*}
y^\prime=y^2-4
\end{equation*}
\begin{overprint}
\onslide<1>
\begin{center}
\begin{asy}
size(215);

real min_x=-4, max_x=4;
real min_y=-4, max_y=4;

pair start=(min_x,min_y);
pair end=(max_x,max_y);

for(real gx=min_x+1; gx<=max_x-1; ++gx)
	draw((gx,min_y)--(gx,max_y),dotted+darkgray);
    
for(real gy=min_y+1; gy<=max_y-1; ++gy)
	draw((min_x,gy)--(max_x,gy),dotted+darkgray); 
	
real yp (real t, real y) { return y*y-4; }

add(slopefield(yp, start, end, 20));

limits(start,end,Crop);

xaxis("$t$",YEquals(min_y),min_x,max_x,LeftTicks());
xaxis(YEquals(max_y),min_x,max_x);
yaxis("$y$",XEquals(min_x),min_y,max_y,LeftTicks());
yaxis(XEquals(max_x),min_y,max_y);
\end{asy}
\end{center}
\onslide<2>
\begin{center}
\begin{asy}
size(215);

real min_x=-4, max_x=4;
real min_y=-4, max_y=4;

pair start=(min_x,min_y);
pair end=(max_x,max_y);

for(real gx=min_x+1; gx<=max_x-1; ++gx)
	draw((gx,min_y)--(gx,max_y),dotted+darkgray);
    
for(real gy=min_y+1; gy<=max_y-1; ++gy)
	draw((min_x,gy)--(max_x,gy),dotted+darkgray); 
	
real yp (real t, real y) { return y*y-4; }

add(slopefield(yp, start, end, 20));

// draw middle solutions
pen eqclr = next_color();
pen clr = next_color();

for (real x=min_x-1; x <= max_x+1; x += 1)
{
	draw(curve((x,0), yp, start, end), clr+1bp);
}

// draw top solutions
pen clr = next_color();

for (real x=min_x-1; x <= max_x+1; x += 1)
{
	draw(curve((x,3.5), yp, start, end), clr+1bp);
}

// draw bottom solutions
pen clr = next_color();

for (real x=min_x-1; x <= max_x+1; x += 1)
{
	draw(curve((x,-3.5), yp, start, end), clr+1bp);
}

// draw equilibiria
pen clr = blue;

pair[] pts = {(0,-2), (0,2)};

for (pair p : pts)
{
	draw(curve(p, yp, start, end), eqclr+1.5bp);
}

limits(start,end,Crop);

xaxis("$t$",YEquals(min_y),min_x,max_x,LeftTicks());
xaxis(YEquals(max_y),min_x,max_x);
yaxis("$y$",XEquals(min_x),min_y,max_y,LeftTicks());
yaxis(XEquals(max_x),min_y,max_y);
\end{asy}
\end{center}
\end{overprint}
\end{example}
\end{frame}

\begin{frame}[fragile]
\begin{example}
We can approximate the solution of the IVP
\begin{equation*}
y^\prime = t - y,\ y(0)=1
\end{equation*}

\begin{multistepslide}
\begin{center}
\begin{asy}[height=6.3cm]
import graph;
import slopefield;
import fontsize;
defaultpen(fontsize(9pt));
real dy(real x,real y) {return x-y;}
real xmin=-.1, xmax=2.4;
real ymin=-.1, ymax=1.5;

add(slopefield(dy,(xmin,ymin),(xmax,ymax),20,grey+0.4bp,Arrow));

real h=0.5;
real EulerStep(real x,real y) {return y+h*dy(x,y);}
real t0=0.0;
real y0=1.0;
dot((t0,y0));

real t_old=t0;
real y_old=y0;
real t_new=0.0;
real y_new=0.0;

//for(int i=1; i<2; ++i)
//{
//	t_new=t_old+h;
//	y_new=EulerStep(t_old, y_old);
//	draw((t_old, y_old)--(t_new, y_new), blue+1bp);
//    dot((t_new,y_new));
//    t_old=t_new;
//    y_old=y_new;
//}

//pair P=(t0,y0);
//draw(curve(P,dy,(xmin,ymin),(xmax,ymax)),red+1bp);

limits((xmin,ymin),(xmax,ymax),Crop);

xaxis(YEquals(ymin),xmin,xmax,LeftTicks());
xaxis(YEquals(ymax),xmin,xmax);
yaxis(XEquals(xmin),ymin,ymax,RightTicks());
yaxis(XEquals(xmax),ymin,ymax);
\end{asy}
\end{center}
\nextstep
\begin{center}
\begin{asy}[height=6.3cm]
import graph;
import slopefield;
import fontsize;
defaultpen(fontsize(9pt));
real dy(real x,real y) {return x-y;}
real xmin=-.1, xmax=2.4;
real ymin=-.1, ymax=1.5;

add(slopefield(dy,(xmin,ymin),(xmax,ymax),20,grey+0.4bp,Arrow));

real h=0.5;
real EulerStep(real x,real y) {return y+h*dy(x,y);}
real t0=0.0;
real y0=1.0;
dot((t0,y0));

real t_old=t0;
real y_old=y0;
real t_new=0.0;
real y_new=0.0;

for(int i=1; i<2; ++i)
{
	t_new=t_old+h;
	y_new=EulerStep(t_old, y_old);
	draw((t_old, y_old)--(t_new, y_new), blue+1bp);
    dot((t_new,y_new));
    t_old=t_new;
    y_old=y_new;
}

//pair P=(t0,y0);
//draw(curve(P,dy,(xmin,ymin),(xmax,ymax)),red+1bp);

limits((xmin,ymin),(xmax,ymax),Crop);

xaxis(YEquals(ymin),xmin,xmax,LeftTicks());
xaxis(YEquals(ymax),xmin,xmax);
yaxis(XEquals(xmin),ymin,ymax,RightTicks());
yaxis(XEquals(xmax),ymin,ymax);
\end{asy}
\end{center}
\nextstep
\begin{center}
\begin{asy}[height=6.3cm]
import graph;
import slopefield;
import fontsize;
defaultpen(fontsize(9pt));
real dy(real x,real y) {return x-y;}
real xmin=-.1, xmax=2.4;
real ymin=-.1, ymax=1.5;

add(slopefield(dy,(xmin,ymin),(xmax,ymax),20,grey+0.4bp,Arrow));

real h=0.5;
real EulerStep(real x,real y) {return y+h*dy(x,y);}
real t0=0.0;
real y0=1.0;
dot((t0,y0));

real t_old=t0;
real y_old=y0;
real t_new=0.0;
real y_new=0.0;

for(int i=1; i<3; ++i)
{
	t_new=t_old+h;
	y_new=EulerStep(t_old, y_old);
	draw((t_old, y_old)--(t_new, y_new), blue+1bp);
    dot((t_new,y_new));
    t_old=t_new;
    y_old=y_new;
}

//pair P=(t0,y0);
//draw(curve(P,dy,(xmin,ymin),(xmax,ymax)),red+1bp);

limits((xmin,ymin),(xmax,ymax),Crop);

xaxis(YEquals(ymin),xmin,xmax,LeftTicks());
xaxis(YEquals(ymax),xmin,xmax);
yaxis(XEquals(xmin),ymin,ymax,RightTicks());
yaxis(XEquals(xmax),ymin,ymax);
\end{asy}
\end{center}
\nextstep
\begin{center}
\begin{asy}[height=6.3cm]
import graph;
import slopefield;
import fontsize;
defaultpen(fontsize(9pt));
real dy(real x,real y) {return x-y;}
real xmin=-.1, xmax=2.4;
real ymin=-.1, ymax=1.5;

add(slopefield(dy,(xmin,ymin),(xmax,ymax),20,grey+0.4bp,Arrow));

real h=0.5;
real EulerStep(real x,real y) {return y+h*dy(x,y);}
real t0=0.0;
real y0=1.0;
dot((t0,y0));

real t_old=t0;
real y_old=y0;
real t_new=0.0;
real y_new=0.0;

for(int i=1; i<4; ++i)
{
	t_new=t_old+h;
	y_new=EulerStep(t_old, y_old);
	draw((t_old, y_old)--(t_new, y_new), blue+1bp);
    dot((t_new,y_new));
    t_old=t_new;
    y_old=y_new;
}

//pair P=(t0,y0);
//draw(curve(P,dy,(xmin,ymin),(xmax,ymax)),red+1bp);

limits((xmin,ymin),(xmax,ymax),Crop);

xaxis(YEquals(ymin),xmin,xmax,LeftTicks());
xaxis(YEquals(ymax),xmin,xmax);
yaxis(XEquals(xmin),ymin,ymax,RightTicks());
yaxis(XEquals(xmax),ymin,ymax);
\end{asy}
\end{center}
\nextstep
\begin{center}
\begin{asy}[height=6.3cm]
import graph;
import slopefield;
import fontsize;
defaultpen(fontsize(9pt));
real dy(real x,real y) {return x-y;}
real xmin=-.1, xmax=2.4;
real ymin=-.1, ymax=1.5;

add(slopefield(dy,(xmin,ymin),(xmax,ymax),20,grey+0.4bp,Arrow));

real h=0.5;
real EulerStep(real x,real y) {return y+h*dy(x,y);}
real t0=0.0;
real y0=1.0;
dot((t0,y0));

real t_old=t0;
real y_old=y0;
real t_new=0.0;
real y_new=0.0;

for(int i=1; i<5; ++i)
{
	t_new=t_old+h;
	y_new=EulerStep(t_old, y_old);
	draw((t_old, y_old)--(t_new, y_new), blue+1bp);
    dot((t_new,y_new));
    t_old=t_new;
    y_old=y_new;
}

//pair P=(t0,y0);
//draw(curve(P,dy,(xmin,ymin),(xmax,ymax)),red+1bp);

limits((xmin,ymin),(xmax,ymax),Crop);

xaxis(YEquals(ymin),xmin,xmax,LeftTicks());
xaxis(YEquals(ymax),xmin,xmax);
yaxis(XEquals(xmin),ymin,ymax,RightTicks());
yaxis(XEquals(xmax),ymin,ymax);
\end{asy}
\end{center}
\nextstep
\begin{center}
\begin{asy}[height=6.3cm]
import graph;
import slopefield;
import fontsize;
defaultpen(fontsize(9pt));
real dy(real x,real y) {return x-y;}
real xmin=-.1, xmax=2.4;
real ymin=-.1, ymax=1.5;

add(slopefield(dy,(xmin,ymin),(xmax,ymax),20,grey+0.4bp,Arrow));

real h=0.5;
real EulerStep(real x,real y) {return y+h*dy(x,y);}
real t0=0.0;
real y0=1.0;
dot((t0,y0));

real t_old=t0;
real y_old=y0;
real t_new=0.0;
real y_new=0.0;

for(int i=1; i<5; ++i)
{
	t_new=t_old+h;
	y_new=EulerStep(t_old, y_old);
	draw((t_old, y_old)--(t_new, y_new), blue+1bp);
    dot((t_new,y_new));
    t_old=t_new;
    y_old=y_new;
}

pair P=(t0,y0);
draw(curve(P,dy,(xmin,ymin),(xmax,ymax)),red+1bp);

limits((xmin,ymin),(xmax,ymax),Crop);

xaxis(YEquals(ymin),xmin,xmax,LeftTicks());
xaxis(YEquals(ymax),xmin,xmax);
yaxis(XEquals(xmin),ymin,ymax,RightTicks());
yaxis(XEquals(xmax),ymin,ymax);
\end{asy}
\end{center}
\nextstep
\begin{center}
\begin{asy}[height=6.3cm]
import graph;
import slopefield;
import fontsize;
defaultpen(fontsize(9pt));
real dy(real x,real y) {return x-y;}
real xmin=-.1, xmax=2.4;
real ymin=-.1, ymax=1.5;

add(slopefield(dy,(xmin,ymin),(xmax,ymax),20,grey+0.4bp,Arrow));

real h=0.25;
real EulerStep(real x,real y) {return y+h*dy(x,y);}
real t0=0.0;
real y0=1.0;
dot((t0,y0));

real t_old=t0;
real y_old=y0;
real t_new=0.0;
real y_new=0.0;

for(int i=1; i<9; ++i)
{
	t_new=t_old+h;
	y_new=EulerStep(t_old, y_old);
	draw((t_old, y_old)--(t_new, y_new), blue+1bp);
    dot((t_new,y_new));
    t_old=t_new;
    y_old=y_new;
}

pair P=(t0,y0);
draw(curve(P,dy,(xmin,ymin),(xmax,ymax)),red+1bp);

limits((xmin,ymin),(xmax,ymax),Crop);

xaxis(YEquals(ymin),xmin,xmax,LeftTicks());
xaxis(YEquals(ymax),xmin,xmax);
yaxis(XEquals(xmin),ymin,ymax,RightTicks());
yaxis(XEquals(xmax),ymin,ymax);
\end{asy}
\end{center}
\nextstep
\begin{center}
\begin{asy}[height=6.3cm]
import graph;
import slopefield;
import fontsize;
defaultpen(fontsize(9pt));
real dy(real x,real y) {return x-y;}
real xmin=-.1, xmax=2.4;
real ymin=-.1, ymax=1.5;

add(slopefield(dy,(xmin,ymin),(xmax,ymax),20,grey+0.4bp,Arrow));

real h=0.125;
real EulerStep(real x,real y) {return y+h*dy(x,y);}
real t0=0.0;
real y0=1.0;
dot((t0,y0));

real t_old=t0;
real y_old=y0;
real t_new=0.0;
real y_new=0.0;

for(int i=1; i<17; ++i)
{
	t_new=t_old+h;
	y_new=EulerStep(t_old, y_old);
	draw((t_old, y_old)--(t_new, y_new), blue+1bp);
    dot((t_new,y_new));
    t_old=t_new;
    y_old=y_new;
}

pair P=(t0,y0);
draw(curve(P,dy,(xmin,ymin),(xmax,ymax)),red+1bp);

limits((xmin,ymin),(xmax,ymax),Crop);

xaxis(YEquals(ymin),xmin,xmax,LeftTicks());
xaxis(YEquals(ymax),xmin,xmax);
yaxis(XEquals(xmin),ymin,ymax,RightTicks());
yaxis(XEquals(xmax),ymin,ymax);
\end{asy}
\end{center}
\nextstep
\begin{center}
\begin{asy}[height=6.3cm]
import graph;
import slopefield;
import fontsize;
defaultpen(fontsize(9pt));
real dy(real x,real y) {return x-y;}
real xmin=-.1, xmax=2.4;
real ymin=-.1, ymax=1.5;

add(slopefield(dy,(xmin,ymin),(xmax,ymax),20,grey+0.4bp,Arrow));

real h=0.0625;
real EulerStep(real x,real y) {return y+h*dy(x,y);}
real t0=0.0;
real y0=1.0;
dot((t0,y0));

real t_old=t0;
real y_old=y0;
real t_new=0.0;
real y_new=0.0;

for(int i=1; i<33; ++i)
{
	t_new=t_old+h;
	y_new=EulerStep(t_old, y_old);
	draw((t_old, y_old)--(t_new, y_new), blue+1bp);
    dot((t_new,y_new));
    t_old=t_new;
    y_old=y_new;
}

pair P=(t0,y0);
draw(curve(P,dy,(xmin,ymin),(xmax,ymax)),red+1bp);

limits((xmin,ymin),(xmax,ymax),Crop);

xaxis(YEquals(ymin),xmin,xmax,LeftTicks());
xaxis(YEquals(ymax),xmin,xmax);
yaxis(XEquals(xmin),ymin,ymax,RightTicks());
yaxis(XEquals(xmax),ymin,ymax);
\end{asy}
\end{center}
\nextstep
\begin{center}
\begin{asy}[height=6.3cm]
import graph;
import slopefield;
import fontsize;
defaultpen(fontsize(9pt));
real dy(real x,real y) {return x-y;}
real xmin=-.1, xmax=2.4;
real ymin=-.1, ymax=1.5;

add(slopefield(dy,(xmin,ymin),(xmax,ymax),20,grey+0.4bp,Arrow));

real h=0.03125;
real EulerStep(real x,real y) {return y+h*dy(x,y);}
real t0=0.0;
real y0=1.0;
dot((t0,y0));

real t_old=t0;
real y_old=y0;
real t_new=0.0;
real y_new=0.0;

for(int i=1; i<65; ++i)
{
	t_new=t_old+h;
	y_new=EulerStep(t_old, y_old);
	draw((t_old, y_old)--(t_new, y_new), blue+1bp);
    dot((t_new,y_new));
    t_old=t_new;
    y_old=y_new;
}

pair P=(t0,y0);
draw(curve(P,dy,(xmin,ymin),(xmax,ymax)),red+1bp);

limits((xmin,ymin),(xmax,ymax),Crop);

xaxis(YEquals(ymin),xmin,xmax,LeftTicks());
xaxis(YEquals(ymax),xmin,xmax);
yaxis(XEquals(xmin),ymin,ymax,RightTicks());
yaxis(XEquals(xmax),ymin,ymax);
\end{asy}
\end{center}
\end{multistepslide}
\end{example}
\end{frame}

\begin{frame}
\begin{block}{Consider}
Given the IVP
\begin{equation*}
y^\prime = f(t,y),\ y(t_0)=y_0
\end{equation*}
We want to compute  approximate values for $y(t_n)$ at the (finite) set of points $t_1, t_2, t_3, \dots, t_k$.

\vspace{2mm}
\begin{overprint}
\onslide<1>
\onslide<2-4>
We can calculate the $t$-values, for $1,2,3,\dots,k$, with $t_n=t_0 + n\cdot h$.

\vspace{2mm}
Where $h$, called the \textbf{step size}, is the common difference between successive points.

\visible<3->{\vspace{2mm}
So, starting at $(t_0, y_0)$ we want to follow the tangent line determined by
\begin{equation*}
y-y_0 = (t - t_0)f(t_0, y_0)
\end{equation*}}
\visible<4->{%

\vspace{-5mm}
to find the approximate solution $\left(t_1, y(t_1)\right)$:
\begin{equation*}
y_1 = y_0 + h\cdot f(t_0, y_0)
\end{equation*}}

\onslide<5->
We can extend this process to find all $k$ points.
\begin{equation*}
\begin{aligned}
y_1 &= y_0 + h\cdot f(t_0, y_0)\\
\visible<6->{y_2 &= y_1 + h\cdot f(t_1, y_1)\\}
\visible<7->{y_3 &= y_2 + h\cdot f(t_2, y_2)\\}
\visible<8->{&\vdots\\
y_k &= y_{k-1} + h\cdot f(t_{k-1}, y_{k-1})\\}
\end{aligned}
\end{equation*}

\vspace{-5mm}
\visible<9->{The resulting piecewise-linear function (i\@.e\@. play connect-the-dots) is called the \textbf{Euler-approximate} solution.}
\end{overprint}
\vspace{-3mm}
\end{block}
\end{frame}

\begin{frame}
\begin{block}{Euler's Method}
For the Initial-value problem
\[y^\prime=f(t,y),\ y(t_0)=y_0\]
use the formulas
\begin{align*}
t_{n+1} &= t_n + h\\
y_{n+1} &= y_n+h\cdot f(t_n, y_n)
\end{align*}
to iteratively compute the points, using step size $h$, 
\[(t_1, y_1), (t_2, y_2), \dots, (t_k, y_k).\]
The piecewise-linear function connecting these points is the Euler approximtion to the solution $y(t)$ of the IVP for $t_0\leq t\leq t_k$.
\end{block}
\end{frame}

\begin{frame}[fragile]
\begin{example}
\begin{overprint}
\onslide<1-2>
Let us obtain the Euler-approximate solution of the IVP
\[y^\prime = -2ty+t,\ y(0)=-1\]
with step size $0.1$ on $[0,0.4]$.

\vspace{2mm}
\visible<2>{%
In other words:
\begin{align*}
f(t,y) &= -2ty+t = t(1-2y)\\
t_0 &= 0\\
y_0 &= -1\\
h&=0.1\\
k&=1,2,3,4
\end{align*}}
\onslide<3-6>
\begin{equation*}
\begin{aligned}
t_1&=t_0+h=0+0.1=0.1\\
y_1&=y_0+h\cdot f(t_0, y_0)= -1+(0.1)(0)(1-2(-1))=-1\\\\
\visible<4->{%
t_2&=t_1+h=0.1+0.1=0.2\\
y_2&=y_1+h\cdot f(t_1, y_1) \\ 
&= -1+(0.1)(0.1)(1-2(-1))=-0.97\\\\}
\visible<5->{%
t_3&=t_2+h=0.2+0.1=0.3\\
y_3&=y_2+h\cdot f(t_2, y_2) \\
&= -0.97+(0.1)(0.2)(1-2(-0.97))=-0.9112\\\\}
\visible<6->{%
t_4&=t_3+h=0.3+0.1=0.4\\
y_4&=y_3+h\cdot f(t_3, y_3) \\
&= -0.9112+(0.1)(0.3)(1-2(-0.9112))=-0.82652\\}
\end{aligned}
\end{equation*}

\onslide<7-8>
How does this compare to the exact solution $y(t)=0.5 - 1.5e^{-t^2}$?

\visible<8->{%
\begin{center}
\begin{tabular}{rrrrr}
\thead{$n$}&\thead{$t_n$}&\thead{$y_n$}&\thead{$y(t_n)$}&\thead{Error}\\
\hline
0&0.0&-1.000000&-1.000000&\textbf{0.000000}\\
1&0.1&-1.000000&-0.985075&\textbf{-0.014925}\\
2&0.2&-0.970000&-0.941184&\textbf{-0.028815}\\
3&0.3&-0.911200&-0.870897&\textbf{-0.040303}\\
4&0.4&-0.826528&-0.778216&\textbf{-0.048312}
\end{tabular}
\end{center}

Notice how the error grows rapidly.}

\onslide<9>
\begin{center}
\begin{asy}[height=6.3cm]
import graph;
import slopefield;
import fontsize;
defaultpen(fontsize(9pt));
size(300);
real dy(real x,real y) {return -2.0*x*y+x;}
real xmin=-.1, xmax=0.5;
real ymin=-1.1, ymax=-0.6;

add(slopefield(dy,(xmin,ymin),(xmax,ymax),20,grey+0.4bp,Arrow));

xaxis(YEquals(ymin),xmin,xmax,LeftTicks());
xaxis(YEquals(ymax),xmin,xmax);
yaxis(XEquals(xmin),ymin,ymax,RightTicks());
yaxis(XEquals(xmax),ymin,ymax);

real h=0.1;
real EulerStep(real x,real y) {return y+h*dy(x,y);}
real t0=0.0;
real y0=-1.0;

dot((t0,y0));

real t_old=t0;
real y_old=y0;
real t_new=0.0;
real y_new=0.0;

for(int i=1; i<5; ++i)
{
	t_new=t_old+h;
	y_new=EulerStep(t_old, y_old);
	draw((t_old, y_old)--(t_new, y_new), blue+1bp);
    dot((t_new,y_new));
    t_old=t_new;
    y_old=y_new;
}

pair P=(t0,y0);
draw(curve(P,dy,(xmin,ymin),(xmax,ymax)),red+1bp);
\end{asy}
\end{center}
\end{overprint}
\vspace{-69mm}
\end{example}
\end{frame}

\begin{frame}[fragile]
\begin{example}
Find the Euler-approximation of 
\begin{equation*}
y^\prime=-2ty,\ y(0)=1
\end{equation*}
using a step size of 0.2 over the range of $[0,2]$.

\begin{overprint}
\onslide<1-12>
\vspace{0mm}
\begin{center}
\begin{tabular}{rrrrr}
\thead{$n$}&\thead{$t_n$}&\thead{$y_n$}&\thead{$y(t_n)$}&\thead{Error}\\
\hline
\visible<2->{0 &0.0&1.0000000&1.0000000&\textbf{0.000000}\\}
\visible<3->{1 &0.2&1.0000000&0.9607894&\textbf{-0.039211}\\}
\visible<4->{2 &0.4&0.9200000&0.8521437&\textbf{-0.067856}\\}
\visible<5->{3 &0.6&0.7728000&0.6976763&\textbf{-0.075124}\\}
\visible<6->{4 &0.8&0.5873280&0.5272925&\textbf{-0.060036}\\}
\visible<7->{5 &1.0&0.3993830&0.3678794&\textbf{-0.031504}\\}
\visible<8->{6 &1.2&0.2396298&0.2369277&\textbf{-0.002702}\\}
\visible<9->{7 &1.4&0.1246075&0.1408584&\textbf{0.016251}\\}
\visible<10->{8 &1.6&0.0548273&0.0773047&\textbf{0.022477}\\}
\visible<11->{9 &1.8&0.0197378&0.0391639&\textbf{0.019426}\\}
\visible<12->{10&2.0&0.0055265&0.0183156&\textbf{0.012789}}
\end{tabular}
\end{center}

\onslide<13->

\vspace{1mm}
\begin{center}
\begin{asy}
import graph;
import slopefield;
import fontsize;
defaultpen(fontsize(9pt));

size(280,0);

real dy(real x,real y) {return -2.0*x*y;}
real xmin=-.1, xmax=2.1;
real ymin=-0.1, ymax=1.1;

add(slopefield(dy,(xmin,ymin),(xmax,ymax),20,grey+0.4bp,Arrow));

xaxis(YEquals(ymin),xmin,xmax,LeftTicks());
xaxis(YEquals(ymax),xmin,xmax);
yaxis(XEquals(xmin),ymin,ymax,RightTicks());
yaxis(XEquals(xmax),ymin,ymax);

real h=0.2;
real EulerStep(real x,real y) {return y+h*dy(x,y);}
real t0=0.0;
real y0=1.0;

dot((t0,y0));

real t_old=t0;
real y_old=y0;
real t_new=0.0;
real y_new=0.0;

for(int i=1; i<11; ++i)
{
	t_new=t_old+h;
	y_new=EulerStep(t_old, y_old);
	draw((t_old, y_old)--(t_new, y_new), blue+1bp);
    dot((t_new,y_new));
    t_old=t_new;
    y_old=y_new;
}

pair P=(t0,y0);
draw(curve(P,dy,(xmin,ymin),(xmax,ymax)),red+1bp);
\end{asy}
\end{center}
\end{overprint}
\vspace{-1mm}
\end{example}
\end{frame}

\begin{frame}
\begin{block}{}
There are two types of error:

\pause
\begin{itemize}
\item<2-> \textbf{Roundoff error} is the discrepancy arising from rounding numbers. This tends to snowball pretty fast when you have a great many calculations.
\item<3-> \textbf{Discretization error} is the error that results from the approximation method itself. For Euler's method this is cause by using the linear tangent lines to approximate a nonlinear curve. 
\end{itemize}
\onslide<4->
It can be shown, using Taylor series expansions, that the error is proportional to the square of the step size. (We will talk about Taylor series in chapter 11.)
\begin{equation*}
\abs{y_i-y(t_i)}\leq C\cdot h^2
\end{equation*}
Where the constant $C$ depends of the size of the second derivative of the exact solution.
\onslide<5->

\vspace{2mm}
We call this error the \textbf{local discretization error} because it estimates the error for a single step only. After $n$ steps, we have $n$ times the error. Which we call the \textbf{global discretization error}. 
\end{block}
\end{frame}
\end{document}
